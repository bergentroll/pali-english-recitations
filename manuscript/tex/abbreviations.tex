\clearpage
\thispagestyle{empty}

{\subsectionFmt{Abbreviations used in the text}}
\bigskip

{\raggedright
\ifhandbookedition
\fontsize{9}{13}\selectfont
\else
\fontsize{10}{14}\selectfont
\fi

\begin{tabular}{@{}ll@{}}
[...] & Only recited by the leader \\
̓     & Take a breath               \\
\end{tabular}

\begin{tabular}{@{}llll@{}}
  Vin  & Vinaya Piṭaka       & MJG   & Mahā-jaya-maṅgala-gāthā \\
  DN   & Dīgha Nikāya        & Thai  & Composed...             \\
  MN   & Majjhima Nikāya     & Sri L & Composed...             \\
  SN   & Saṃyutta Nikāya     &       & Tradtional...           \\
  AN   & Aṅguttara Nikāya    &       &                 \\
  Khp  & Khuddakapāṭha       &       &                 \\
  Dhp  & Dhammapada          &       &                 \\
  Ud   & Udāna               &       &                 \\
  Snp  & Sutta Nipāta        &       &                 \\
  Thag & Theragāthā          &       &                 \\
  Ja   & Jātaka              &       &                 \\
  Ps   & Paṭisambhidāmagga   &       &                 \\
  Vibh & Abhidhamma Vibhaṅga &       &                 \\
  A    & Aṭṭhakathā          &       &                 \\
  Dhs  & Dhammasaṅganī       &       &                 \\
  A    & Aṭṭhakathā          &       &                 \\

\end{tabular}

\bigskip

Wisdom Publication sources: Nikāya and sutta # (eg. DN 1)
P.T.S. sources: Nikāya, volume #, page # (eg. D i 1)

References to shorter texts consisting of verses such as the Dhammapada, Udāna,
Itivuttaka, Theragāthā, Therīgāthā or Sutta Nipāta are to the verse number or
chapter and verse number. The other longer texts are referred to by volume and
page number of the PTS edition.

}

