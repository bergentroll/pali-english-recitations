\chapter{Anumodanā}

\section{Yathā vāri-vahā pūrā}

\englishTitle{Just as Rivers}

\firstline{Yathā vāri-vahā pūrā}

Yathā vāri-vahā pūrā

\begin{cprenglish}
  Just as rivers full of water
\end{cprenglish}

Paripūrenti sāgaraṁ

\begin{cprenglish}
  Entirely fill up the sea
\end{cprenglish}

Evam-eva ito dinnaṁ petānaṁ upakappati

\begin{cprenglish}
  So will what's here been given bring blessings to departed spirits
\end{cprenglish}

Icchitaṁ patthitaṁ tumhaṁ

\begin{cprenglish}
  May all your hopes and all your longings
\end{cprenglish}

Khippam-eva samijjhatu

\begin{cprenglish}
  Come true in no long time
\end{cprenglish}

Sabbe pūrentu saṅkappā

\begin{cprenglish}
  May all your wishes be fulfilled
\end{cprenglish}

Cando paṇṇaraso yathā

\begin{cprenglish}
  Like on the fifteenth day the moon
\end{cprenglish}

Maṇi jotiraso yathā

\begin{cprenglish}
  Or like a bright and shining gem\\
  \suttaRef{DhpA.I.198}
\end{cprenglish}

\firstline{Sabb'ītiyo vivajjantu sabba-rogo vinassatu}

Sabb'ītiyo vivajjantu

\begin{cprenglish}
  May all misfortunes be avoided
\end{cprenglish}

Sabba-rogo vinassatu

\begin{cprenglish}
  May all illness be dispelled
\end{cprenglish}

Mā te bhavatv-antarāyo

\begin{cprenglish}
  May you never meet with dangers
\end{cprenglish}

Sukhī dīgh'āyuko bhava

\begin{cprenglish}
  May you be happy and live long
\end{cprenglish}

Abhivādana-sīlissa\\
Niccaṁ vuḍḍhāpacāyino\\
Cattāro dhammā vaḍḍhanti\\
Āyu vaṇṇo sukhaṁ balaṁ

\begin{cprenglish}
  For those who are respectful\\
  Who always honour the elders\\
  Four are the qualities which will increase\\
  Life, beauty, happiness and strength
\end{cprenglish}

\firstline{Bhavatu sabba-maṅgalaṁ}

Bhavatu sabba-maṅgalaṁ

\begin{cprenglish}
  May every blessing come to be
\end{cprenglish}

Rakkhantu sabba-devatā

\begin{cprenglish}
  And all good spirits guard you well
\end{cprenglish}

Sabba-buddhānubhāvena

\begin{cprenglish}
  Through the power of all Buddhas
\end{cprenglish}

Sadā sotthī bhavantu te

\begin{cprenglish}
  May you always be at ease
\end{cprenglish}

Bhavatu sabba-maṅgalaṁ

\begin{cprenglish}
  May every blessing come to be
\end{cprenglish}

Rakkhantu sabba-devatā

\begin{cprenglish}
  And all good spirits guard you well
\end{cprenglish}

Sabba-dhammānubhāvena

\begin{cprenglish}
  Through the power of all Dhammas
\end{cprenglish}

Sadā sotthī bhavantu te

\begin{cprenglish}
  May you always be at ease
\end{cprenglish}

Bhavatu sabba-maṅgalaṁ

\begin{cprenglish}
  May every blessing come to be
\end{cprenglish}

Rakkhantu sabba-devatā

\begin{cprenglish}
  And all good spirits guard you well
\end{cprenglish}

Sabba-saṅghānubhāvena

\begin{cprenglish}
  Through the power of all Saṅghas
\end{cprenglish}

Sadā sotthī bhavantu te

\begin{cprenglish}
  May you always be at ease\\
  \suttaRef{Khp 1.7 / Dhp 109 / Trad}
\end{cprenglish}



\subsubsection{Sabba-roga-vinimutto}

\instr{(This shorter form is sometimes used instead of `Yathā\ldots')}

\firstline{Sabba-roga-vinimutto}

\bigskip

\begin{paritta}
  Sabba-roga-vinimutto\\\vin sabba-santāpa-vajjito\\
  Sabba-veram-atikkanto\\\vin nibbuto ca tuvam-bhava\\
  Sabb'ītiyo vivajjantu\\\vin sabba-rogo vinassatu\\
  Mā te bhavatv-antarāyo\\\vin sukhī dīgh'āyuko bhava\\
  Abhivādana-sīlissa\\\vin niccaṁ vuḍḍhāpacāyino\\
  Cattāro dhammā vaḍḍhanti\\\vin āyu vaṇṇo sukhaṁ balaṁ \suttaRef{Dhp 109}
\end{paritta}

\bigskip

\begin{english}
May you be freed from all disease, safe from all torment, beyond all animosity
and at peace.

May all misfortunes be avoided\ldots
\end{english}

\section{Bhojana-dānānumodanā}

\firstline{Āyu-do bala-do dhīro vaṇṇa-do paṭibhāṇa-do}

\begin{twochants}
  Āyu-do bala-do dhīro & vaṇṇa-do paṭibhāṇa-do\\
  Sukhassa dātā medhāvī & sukhaṁ so adhigacchati\\
  Āyuṁ datvā balaṁ vaṇṇaṁ & sukhañ-ca paṭibhāna-do\\
  Dīgh'āyu yasavā hoti & yattha yatthūpapajjatī'ti
\end{twochants}

\bigskip

\begin{english}
  The enlightened person, having given life, strength, beauty, quick-wittedness --
  The intelligent person, a giver of happiness -- attain happiness themselves.
  Having given life, strength, beauty, happiness, and quick-wittedness,
  They have a long life and status wherever they arise.
\end{english}

\suttaRef{A.III.42}

\section{Aggappasāda-sutta-gāthā}

\firstline{Aggato ve pasannānaṁ}

% English source: Bodhivana

\begin{twochants}
  Aggato ve pasannānaṁ & aggaṁ dhammaṁ vijānataṁ\\
  Agge Buddhe pasannānaṁ & dakkhiṇeyye anuttare\\
  Agge dhamme pasannānaṁ & virāgūpasame sukhe\\
  Agge saṅghe pasannānaṁ & puññakkhette anuttare\\
  Aggasmiṁ dānaṁ dadataṁ & aggaṁ puññaṁ pavaḍḍhati\\
  Aggaṁ āyu ca vaṇṇo ca & yaso kitti sukhaṁ balaṁ\\
  Aggassa dātā medhāvī & agga-dhamma-samāhito\\
  Deva-bhūto manusso vā & aggappatto pamodatī'ti
\end{twochants}

\begin{english}
  \setlength{\parskip}{8pt}%
  For one with confidence, realising the supreme Dhamma to be supreme.
  With confidence in the Buddha, unsurpassed in deserving offerings.
  With confidence in the supreme Dhamma, the happiness of dispassion and calm.
  With confidence in the supreme Saṅgha, unsurpassed as a field of merit.

  Having given gifts to the supreme, one develops supreme merit,
  supreme long life and beauty, status, honor, happiness and strength.
  Having given to the supreme, the intelligent person, firm in the supreme Dhamma,
  Whether becoming a deva or a human being, rejoices, having attained the supreme.
\end{english}

\suttaRef{A.II.35; A.III.36}

\section{Adāsi-me ādi-gāthā}

% Alternative Pali title: Tiro-kuḍḍa-kaṇḍa

\firstline{Adāsi me akāsi me}

\begin{twochants}
Adāsi me akāsi me & ñāti-mittā sakhā ca me\\
Petānaṁ dakkhiṇaṁ dajjā & pubbe katam-anussaraṁ\\
Na hi ruṇṇaṁ vā soko vā & yā v'aññā paridevanā\\
Na taṁ petānam-atthāya & evaṁ tiṭṭhanti ñātayo\\
\end{twochants}

\bigskip

\firstline{Ayañ-ca kho dakkhiṇā dinnā}

Ayañ-ca kho dakkhiṇā dinnā\\
Saṅghamhi supatiṭṭhitā\\
Dīgha-rattaṁ hitāy'assa\\
Ṭhānaso upakappati\\
So ñāti-dhammo ca ayaṁ nidassito\\
Petāna'pūjā ca katā uḷārā\\
Balañ-ca bhikkhūnam-anuppadinnaṁ\\
Tumhehi puññaṁ pasutaṁ anappakan'ti.

% English source: Bodhivana Vol 2, p.182

\begin{english}
  \setlength{\parskip}{8pt}%
  ``He gave to me, he acted on my behalf, and he was my relative, companion,
  friend.'' Offerings should be given for the dead when one reflects thus on
  what was done in the past. For no weeping or sorrowing or any kind of
  lamentation benefit the dead whose relatives keep acting in that way.

  But when this offering is given, well-placed in the Sangha, it works for their
  long-term benefit and they profit immediately. In this way the proper duty to
  relatives has been shown and great honour has been done to the dead and the
  monks have been given strength: You have acquried merit that is not small.
\end{english}

\suttaRef{Khp.VII.v10-13}

\section{Kāla-dāna-sutta-gāthā}

\firstline{Kāle dadanti sapaññā vadaññū vīta-maccharā}

\begin{twochants}
  Kāle dadanti sapaññā & vadaññū vīta-maccharā\\
  Kālena dinnaṁ ariyesu & uju-bhūtesu tādisu\\
  Vippasanna-manā tassa & vipulā hoti dakkhiṇā\\
  Ye tattha anumodanti & veyyāvaccaṁ karonti vā\\
  Na tena dakkhiṇā onā & te pi puññassa bhāgino\\
  Tasmā dade appaṭivāna-citto & yattha dinnaṁ mahapphalaṁ\\
  Puññāni para-lokasmiṁ & patiṭṭhā honti pāṇinan'ti
\end{twochants}

\bigskip

\begin{english}
  \setlength{\parskip}{8pt}%
  Those with discernment, responsive, free from stinginess,
  give in the proper season. Having given in the proper season
  with hearts inspired by the Noble Ones straightened.
  Such -- their offering bears an abundance.

  Those who rejoice in that gift, or give assistance,
  they too have a share of the merit, and the offering is not depleted by that.
  Therefore, with an unhesitant mind, one should give where the gift bears great fruit.
  Merit is what establishes living beings in the next life.
\end{english}

\suttaRef{A.III.41}

\section{Ratanattay'ānubhāv'ādi-gāthā}

\firstline{Ratanattay'ānubhāvena ratanattaya-tejasā}

\begin{twochants}
Ratanattay'ānubhāvena & ratanattaya-tejasā\\
Dukkha-roga-bhayā verā & sokā sattu c'upaddavā\\
Anekā antarāyā pi & vinassantu asesato\\
Jaya-siddhi dhanaṁ lābhaṁ & sotthi bhāgyaṁ sukhaṁ balaṁ\\
Siri āyu ca vaṇṇo ca & bhogaṁ vuḍḍhī ca yasavā\\
Sata-vassā ca āyu ca & jīva-siddhī bhavantu te
\end{twochants}

\bigskip

\begin{english}
  \setlength{\parskip}{8pt}%
  Through the power of the Triple Gem, through the majesty of the Triple Gem,
  May suffering, disease, danger, animosity, sorrow, adversity, misfortune --
  obstacles without number -- vanish without a trace.

  Triumph, success, wealth, gain, safety, luck, happiness and strength,
  glory, long life, beauty, fortune and status increase,
  A lifespan of a hundred years, and success in your livelihood: may they be yours.
\end{english}

\section{Culla-maṅgala-cakka-vāḷa}

\firstline{Sabba-buddh'ānubhāvena}

Sabba-buddh'ānubhāvena\\
sabba-dhamm'ānubhāvena\\
sabba-saṅgh'ānubhāvena

Buddha-ratanaṁ dhamma-ratanaṁ saṅgha-ratanaṁ

Tiṇṇaṁ ratanānaṁ ānubhāvena\\
Catur-āsīti-sahassa-dhammakkhandh'ānubhāvena\\
Piṭakattay'ānubhāvena\\
Jina-sāvak'ānubhāvena

Sabbe te rogā sabbe te bhayā sabbe te antarāyā sabbe te upaddavā sabbe te
dunnimittā sabbe te avamaṅgalā vinassantu

Āyu-vaḍḍhako dhana-vaḍḍhako siri-vaḍḍhako yasa-vaḍḍhako bala-vaḍḍhako
vaṇṇa-vaḍḍhako sukha-vaḍḍhako hotu sabbadā

Dukkha-roga-bhayā verā sokā sattu c'upaddavā\\
Anekā antarāyā pi vinassantu ca tejasā\\
Jaya-siddhi dhanaṁ lābhaṁ\\
Sotthi bhāgyaṁ sukhaṁ balaṁ\\
Siri āyu ca vaṇṇo ca bhogaṁ vuḍḍhī ca yasavā\\
Sata-vassā ca āyū ca jīva-siddhī bhavantu te

Bhavatu sabba-maṅgalaṁ\ldots{}

\bigskip

\begin{english}
  \setlength{\parskip}{8pt}%
  Through the power of all the Buddhas, the power of all the Dhamma, the power of all the Saṅgha,
  the treasure of the Buddha, the treasure of the Dhamma, the treasure of the Saṅgha,
  the power of the 84,000 Dhamma groups, the power of the Tripitaka, the power
  of the Victor's disciples:

  May all your diseases, all your fears, all your obstacles,
  all your dangers, all your bad visions, all your bad omens be destroyed.

  May there be always be an increase of long life, wealth, glory, status,
  strength, beauty and happiness.

  May suffering, disease, danger, animosity, sorrow, adversity, misfortune --
  obstacles without number -- vanish through the majesty of the Triple Gem.

  Triumph, success, wealth, gain, safety, luck, happiness, strength, glory, long
  life, beauty, fortune and status increase, a lifespan of a hundred years, and
  success in your livelihood: May they be yours.

  May there be every good blessing, may all the devas protect you, through the
  power of all the Buddhas, Dhamma and Saṅgha, may you always be well.
\end{english}

\section{Mahā-maṅgala-cakka-vāḷa}

\firstline{Siri-dhiti-mati-tejo-jayasiddhi}

Siri-dhiti-mati-tejo-jayasiddhi-mahiddhi-mahāguṇā-parimita-puññādhikarassa
sabbantarāya-nivāraṇa-samatthassa bhagavato arahato sammā-sambuddhassa

Dvattiṁsa-mahā-purisa-lakkhaṇānubhāvena\\
asītyānubyañjanānubhāvena\\
aṭṭhuttara-sata-maṅgalānubhāvena\\
chabbaṇṇa-raṁsiyānubhāvena ketumālānubhāvena\\
dasa-pāramitānubhāvena\\
dasa-upapāramitānubhāvena\\
dasa-paramattha-pāramitānubhāvena\\
sīla-samādhi-paññānubhāvena\\
buddhānubhāvena\\
dhammānubhāvena\\
saṅghānubhāvena\\
tejānubhāvena\\
iddhānubhāvena\\
balānubhāvena\\
ñeyya-dhammānubhāvena\\
caturāsīti-sahassa-dhamma-kkhandhānubhāvena\\
nava-lokuttara-dhammānubhāvena\\
aṭṭhaṅgika-maggānubhāvena\\
aṭṭha-samāpattiyānubhāvena\\
chaḷabhiññānubhāvena\\
catu-sacca-ñāṇānubhāvena\\
dasa-bala-ñāṇānubhāvena\\
sabbaññuta-ñāṇānubhāvena\\
mettā-karuṇā-muditā-upekkhānubhāvena\\
sabba-parittānubhāvena\\
ratanattaya-saraṇānubhāvena\\
tuyhaṁ sabba-roga-sok'upaddava-\\ dukkha-domanass'upāyāsā vinassantu\\
sabba-antarāyā pi vinassantu\\
sabba-saṅkappā tuyhaṁ samijjhantu\\
dīghāyukā tuyhaṁ hotu sata-vassa-jīvena\\
samaṅgiko hotu sabbadā

Ākāsa-pabbata-vana-bhūmi-gaṅgā-mahāsamuddā ārakkhakā
devatā sadā tumhe anurakkhantu

% Text source: Chomtong chanting book

\bigskip

\begin{english}
  \setlength{\parskip}{8pt}%
  Through the power of the thirty-two marks of the Great Man belonging to the
  Blessed One, the Worthy One, the Rightly Self-awakened One, who through his
  accumulation of merit is endowed with glory, steadfastness of intent, majesty,
  victorious power, great might, countless great virtues, who settles all
  dangers and obstacles,

  \clearpage

  through the power of his eighty minor characteristics,\\
  of his hundred and eight blessings,\\
  of his sixfold radiance,\\
  of the aura surrounding his head,\\
  of his ten perfections, ten higher perfections and ten ultimate perfections,\\
  of his virtue, concentration and discernment,\\
  of the Buddha, Dhamma and Saṅgha,\\
  of his majesty, might and strength,\\
  of his Dhammas that can be known,\\
  of the 84,000 divisions of his Dhamma,\\
  of his nine transcendent Dhammas,\\
  of his eightfold path,\\
  of his meditative attainments,\\
  of his six cognitive skills,\\
  of his knowledge of the four noble truths,\\
  of his knowledge of the ten strengths,\\
  of his omniscience,\\
  of his goodwill, compassion, empathetic joy and equanimity,\\
  of all protective chants,\\
  of refuge in the Triple Gem,

  may all your diseases, griefs, misfortunes, pains, distresses and dispairs be destroyed,\\
  may all obstructions be destroyed, may all your resolves succeed,\\
  may you live long, always attaining a hundred years.

  May the protective devas of the sky, the mountains, the forests, the land, the
  River Ganges, and the great ocean always protect you.
\end{english}

\section{Vihāra-dāna-gāthā}

\begin{twochants}
  Sītaṁ uṇhaṁ paṭihanti & tato vāḷamigāni ca\\
  sariṁsape ca makase & sisire cāpi vuṭṭhiyo\\
  Tato vātātapo ghoro & sañjāto paṭihaññati\\
  Leṇatthañ ca sukhatthañ ca & jhāyituñ ca vipassituṁ\\
  Vihāradānaṁ saṅghassa & aggaṁ buddhehi vaṇṇitaṁ\\
  Tasmā hi paṇḍito poso & sampassaṁ attham attano\\
  Vihāre kāraye ramme & vāsayettha bahu-ssute\\
  Tesaṁ annañ ca pānañ ca & vattha-senāsanāni ca\\
  Dadeyya uju-bhūtesu & vippasannena cetasā\\
  Te tassa dhammaṁ desenti & sabbadukkhāpanūdanaṁ\\
  Yaṁ so dhammaṁ idh'aññāya & parinibbātayanāsavo'ti
\end{twochants}

% Source: Chomtong chanting book

% https://suttacentral.net/pli-tv-kd16/en/horner-brahmali

% https://www.dhammatalks.org/books/ChantingGuide/Section0089.html

\enlargethispage{\baselineskip}

\begin{english}
  \setlength{\parskip}{8pt}%
  They ward off cold and heat and beasts of prey from there\\
  And creeping things and gnats and rains in the wet season.\\
  When the dreaded hot wind arises, that is warded off.\\
  To meditate and obtain insight in a refuge and at ease:

  A dwelling-place is praised by the Awakened One\\\vin as chief gift to an Order.\\
  Therefore a wise man, looking to his own weal,\\
  Should have charming dwelling-places built\\
  So that those who have heard much can stay therein.

  To these food and drink, raiment and lodgings\\
  He should give, to the upright, with mind purified.\\
  (Then) these teach him Dhamma dispelling every ill;\\
  He, knowing that Dhamma,\\\vin here attains Nibbāna, free of taints. \suttaRef{Vin.II.147}
\end{english}

% NOTE: The original translation: He, knowing that Dhamma, here attains Nibbāna, canker-less.
%
% Using as: 'free of taints'.

\section{Saṅgaha-vatthu-gāthā}

\firstline{Dānañ-ca peyya-vajjañ-ca attha-cariyā ca yā idha}

\begin{twochants}
Dānañ-ca peyya-vajjañ-ca & attha-cariyā ca yā idha\\
Samānattatā ca dhammesu & tattha tattha yathā'rahaṁ\\
Ete kho saṅgahā loke & rathass'āṇīva yāyato\\
Ete ca saṅgahā nāssu & na mātā putta-kāraṇā\\
Labhetha mānaṁ pūjaṁ vā & pitā vā putta-kāraṇā\\
Yasmā ca saṅgahā ete & samavekkhanti paṇḍitā\\
Tasmā mahattaṁ papponti & pāsaṁsā ca bhavanti te'ti
\end{twochants}

\begin{english}
  Generosity, kind words, beneficial action,\\
  and treating all consistently, in line with what each deserves:\\
  These bonds of fellowship in the world are like the linchpin in a moving cart.\\
  Now, if these bonds of fellowship were lacking, a mother would not receive the
  honor and respect owed by her child,\\
  nor would a father receive what his child owes him.\\
  But because the wise show regard for these bonds of fellowship,\\
  they achieve greatness and are praised.
\end{english}

% AN 4.32 Saṅgha-vatthu Sutta

\suttaRef{A.II.32}

\clearpage

\section{Ādiya-sutta-gāthā}

\firstline{Bhuttā bhogā bhaṭā bhaccā vitiṇṇā āpadāsu me}

% English source: Bodhivana

\begin{twochants}
Bhuttā bhogā bhaṭā bhaccā & vitiṇṇā āpadāsu me\\
Uddhaggā dakkhiṇā dinnā & atho pañca balī katā\\
Upaṭṭhitā sīlavanto & saññatā brahma-cārino\\
Yad-atthaṁ bhogam-iccheyya & paṇḍito gharam-āvasaṁ\\
So me attho anuppatto & kataṁ ananutāpiyaṁ\\
Etaṁ anussaraṁ macco & ariya-dhamme ṭhito naro\\
Idh'eva naṁ pasaṁsanti & pecca sagge ca pamodatī'ti.
\end{twochants}

\bigskip

\begin{english}
  \setlength{\parskip}{8pt}%
  ``My wealth has been enjoyed, my dependents supported, protected from calamities by me.
  I have given lofty offerings, and performed the five oblations.
  I have provided for the virtuous, the restrained, leaders of the holy life.

  For whatever aim a wise householder would desire wealth,
  that aim have I attained. I have done what will not lead to future distress.''
  When this is recollected by a mortal, a person established in the Dhamma of the Noble Ones,
  He is praised in this life and, after death, rejoices in heaven.
\end{english}

% AN 5.41 Adiya Sutta

\suttaRef{A.III.46}

\clearpage

\section{Ariya-dhana-gāthā}

\englishTitle{Verses on the Riches of a Noble One}

\firstline{Yassa saddhā tathāgate acalā supatiṭṭhitā}

\begin{twochants}
Yassa saddhā tathāgate & acalā supatiṭṭhitā\\
Sīlañ-ca yassa kalyāṇaṁ & ariya-kantaṁ pasaṁsitaṁ\\
\end{twochants}

\begin{english}
  One whose faith in the Tathāgata\\
  Is unshaken and established well,\\
  Whose virtue is beautiful,\\
  The Noble Ones enjoy and praise;
\end{english}

\begin{twochants}
Saṅghe pasādo yass'atthi & uju-bhūtañ-ca dassanaṁ\\
Adaliddo-ti taṁ āhu & amoghaṁ tassa jīvitaṁ\\
\end{twochants}

\begin{english}
  Whose trust is in the Saṅgha,\\
  Who sees things rightly as they are,\\
  It is said that not in vain\\
  And undeluded is their life.
\end{english}

\begin{twochants}
Tasmā saddhañ-ca sīlañ-ca & pasādaṁ dhamma-dassanaṁ\\
Anuyuñjetha medhāvī & saraṁ buddhāna sāsanan'ti
\end{twochants}

\begin{english}
  To virtue and to faith,\\
  To trust to seeing truth,\\
  To these the wise devote themselves,\\
  The Buddha's teaching in their mind.
\end{english}

% AN 5.47
\suttaRef{A.III.54}

\section{Devat'ādissa-dakkhiṇā'numodanā-gāthā}

\firstline{Yasmiṁ padese kappeti vāsaṁ paṇḍita-jātiyo}

\begin{twochants}
Yasmiṁ padese kappeti & vāsaṁ paṇḍita-jātiyo\\
Sīlavant'ettha bhojetvā & saññate brahma-cārino\\
Yā tattha devatā āsuṁ & tāsaṁ dakkhiṇam-ādise\\
Tā pūjitā pūjayanti & mānitā mānayanti naṁ\\
Tato naṁ anukampanti & mātā puttaṁ va orasaṁ\\
Devatā'nukampito poso & sadā bhadrāni passati
\end{twochants}

\bigskip

\begin{english}
  \setlength{\parskip}{8pt}%
  In whatever place a wise person makes his dwelling,
  there providing food for the virtuous, the restrained, leaders of the holy life --
  He should dedicate that offering to the devas there.

  They receiving honor, will honor him; being respected, will show him respect.
  As a result, they will feel sympathy for him, like that of a mother for her child.
  A person with whom the devas sympathize always sees things go auspiciously.
\end{english}

% DN 16

\suttaRef{Vin.I.229}

