\chapterOpeningPage{funeral-chants.pdf}

\chapter{Funeral Chants}

\section{Dhamma-saṅgaṇī-mātikā}

% Dhamma-saṅgaṇī-mātikā
% Title: The List from the Dhamma Groupings

% English source: Bodhivana

\firstline{Kusalā dhammā akusalā dhammā}

Kusalā dhammā.\\
Akusalā dhammā.\\
Abyākatā dhammā.

\begin{english}
  Skillful phenomena,\\
  unskillful phenomena,\\
  undeclared phenomena.
\end{english}

Sukhāya vedanāya sampayuttā dhammā.\\
Dukkhāya vedanāya sampayuttā dhammā.\\
Adukkhamasukhāya vedanāya sampayuttā dhammā.

\begin{english}
  Phenomena conjoined with pleasant feeling,\\
  phenomena conjoined with painful feeling,\\
  phenomena conjoined with neither-painful-nor-pleasant feeling.
\end{english}

Vipākā dhammā.\\
Vipāka-dhamma-dhammā.\\
N'eva vipāka na vipāka-dhamma-dhammā.

\begin{english}
  Phenomena that are kammic results,\\
  phenomena that have kammic results,\\
  phenomena that neither are nor have kammic results.
\end{english}

\clearpage

Upādinn'upādāniyā dhammā.\\
Anupādinn'upādāniyā dhammā.\\
Anupādinnānupādāniyā dhammā.

\begin{english}
  Clung-to clingable phenomena,\\
  unclung-to clingable phenomena,\\
  unclung-to unclingable phenomena.
\end{english}

Saṅkiliṭṭha-saṅkilesikā dhammā.\\
Asaṅkiliṭṭha-saṅkilesikā dhammā.\\
Asaṅkiliṭṭhāsaṅkilesikā dhammā.

\begin{english}
  Defiled defiling phenomena,\\
  undefiled defiling phenomena,\\
  undefiled undefiling phenomena.
\end{english}

Savitakka-savicārā dhammā.\\
Avitakka-vicāra-mattā dhammā.\\
Avitakkāvicārā dhammā.

\begin{english}
  Phenomena accompanied by directed thought and evaluation,\\
  phenomena unaccompanied by directed thought but with a modicum of evaluation,\\
  phenomena unaccompanied by directed thought or evaluation.
\end{english}

Pīti-saha-gatā dhammā.\\
Sukha-saha-gatā dhammā.\\
Upekkhā-saha-gatā dhammā.

\begin{english}
  Phenomena accompanied with rapture,\\
  phenomena accompanied with pleasure,\\
  phenomena accompanied with equanimity.
\end{english}

Dassanena pahātabbā dhammā.\\
Bhāvanāya pahātabbā dhammā.\\
N'eva dassanena na bhāvanāya pahātabbā dhammā.

\begin{english}
  Phenomena to be abandoned through seeing,\\
  phenomena to be abandoned through developing,\\
  phenomena to be abandoned neither through seeing nor through developing.
\end{english}

Dassanena pahātabba-hetukā dhammā.\\
Bhāvanāya pahātabba-hetukā dhammā.\\
N'eva dassanena na bhāvanāya pahātabba-hetukā dhammā.

\begin{english}
  Phenomena connected to a cause that is to be abandoned through seeing,\\
  phenomena connected to a cause that is to be abandoned through developing,\\
  phenomena connected to a cause that is to be abandoned neither through seeing
  nor through developing.
\end{english}

Ācaya-gāmino dhammā.\\
Apacaya-gāmino dhammā.\\
N'ev'ācaya-gāmino nāpacaya-gāmino dhammā.

\begin{english}
  Phenomena leading to accumulation,\\
  phenomena leading to diminution,\\
  phenomena leading neither to accumulation nor to diminution.
\end{english}

Sekkhā dhammā.\\
Asekkhā dhammā.\\
N'eva sekkhā nāsekkhā dhammā.

\begin{english}
  Phenomena of one in training,\\
  phenomena of one beyond training,\\
  phenomena neither of one in training nor of one beyond training.
\end{english}

Parittā dhammā.\\
Mahaggatā dhammā.\\
Appamāṇā dhammā.

\begin{english}
  Limited phenomena,\\
  expanded phenomena,\\
  immeasurable phenomena.
\end{english}

Paritt'ārammaṇā dhammā.\\
Mahaggat'ārammaṇā dhammā.\\
Appamāṇ'ārammaṇā dhammā.

\begin{english}
  Limited mind-object phenomena,\\
  expanded mind-object phenomena,\\
  immeasurable mind-object phenomena.
\end{english}

Hīnā dhammā.\\
Majjhimā dhammā.\\
Paṇītā dhammā.

\begin{english}
  Lowly phenomena,\\
  middling phenomena,\\
  exquisite phenomena.
\end{english}

Micchatta-niyatā dhammā.\\
Sammatta-niyatā dhammā.\\
Aniyatā dhammā.

\clearpage

\begin{english}
  Phenomena of certain wrongness,\\
  phenomena of certain rightness,\\
  uncertain phenomena.
\end{english}

Magg'ārammaṇā dhammā.\\
Magga-hetukā dhammā.\\
Maggādhipatino dhammā.

\begin{english}
  Path mind-object phenomena,\\
  path-causing phenomena,\\
  path-dominant phenomena.
\end{english}

Uppannā dhammā.\\
Anuppannā dhammā.\\
Uppādino dhammā.

\begin{english}
  Arisen phenomena,\\
  unarisen phenomena,\\
  phenomena bound to arise.
\end{english}

Atītā dhammā.\\
Anāgatā dhammā.\\
Paccuppannā dhammā.

\begin{english}
  Past phenomena,\\
  future phenomena,\\
  present phenomena.
\end{english}

Atīt'ārammaṇā dhammā.\\
Anāgat'ārammaṇā dhammā.\\
Paccuppann'ārammaṇā dhammā.

\clearpage

\begin{english}
  Past mind-object phenomena,\\
  future mind-object phenomena,\\
  present mind-object phenomena.
\end{english}

Ajjhattā dhammā.\\
Bahiddhā dhammā.\\
Ajjhatta-bahiddhā dhammā.

\begin{english}
  Internal phenomena,\\
  external phenomena,\\
  internal-and-external phenomena.
\end{english}

Ajjhatt'ārammaṇā dhammā.\\
Bahiddh'ārammaṇā dhammā.\\
Ajjhatta-bahiddh'ārammaṇā dhammā.

\begin{english}
  Internal mind-object phenomena,\\
  external mind-object phenomena,\\
  internal-and-external mind-object phenomena.
\end{english}

Sanidassana-sappaṭighā dhammā.\\
Anidassana-sappaṭighā dhammā.\\
Anidassanāppaṭighā dhammā.

\begin{english}
  Phenomena with surface and offering resistance,\\
  phenomena without surface but offering resistance,\\
  phenomena without surface offering no resistance.
\end{english}

\suttaRef{Dhammasaṅganī 1f}

\clearpage

\section{Dhammasaṅgaṇī}

Kusalā dhammā, akusalā dhammā, abyākatā dhammā.

Katame dhammā kusalā.

Yasmiṃ samaye kāmāvacaraṃ kusalaṃ cittaṃ uppannaṃ hoti, somanassa-sahagataṃ
ñāṇa-sampayuttaṃ, rūpārammaṇaṃ vā saddārammaṇaṃ vā gandhārammaṇaṃ vā
rasārammaṇaṃ vā phoṭṭhabbārammaṇaṃ vā dhammārammaṇaṃ vā, yaṃ yaṃ vā panārabbha,
tasmiṃ samaye phasso hoti, avikkhepo hoti, ye vā pana tasmiṃ samaye aññe pi atthi paṭicca-samuppannā arūpino dhammā, ime dhammā kusalā.

\suttaRef{Dhammasaṅganī 56}

% Source: Chomtong chanting book

% NOTE: These 6.2-6.8 are extracts from the seven books (jet khampi) of the
% Abhidhamma. They are chanted usually for three days at the home of the
% deceased before a funeral takes place, depending on local custom.

\section{Vibhaṅga}

Pañcakkhandhā rūpakkhandho, vedanākkhandho, saññākkhandho, saṅkhārakkhandho,
viññāṇakkhandho.

Tattha katamo rūpakkhandho.

Yaṃ kiñci rūpaṃ atītānāgata-paccuppannaṃ ajjhattaṃ vā bahiddhā vā oḷārikaṃ vā
sukhumaṃ vā hīnaṃ vā paṇītaṃ vā yaṃ dūre santike vā, tad ekajjhaṃ
abhisaññūhitvā abhisaṅkhipitvā, ayaṃ vuccati rūpakkhandho.

\suttaRef{Vibhaṅga 1}

% Source: Chomtong chanting book

\clearpage

\section{Dhātukathā}

Saṅgaho asaṅgaho,\\
saṅgahitena asaṅgahitaṃ,\\
asaṅgahitena saṅgahitaṃ,\\
saṅgahitena saṅgahitaṃ,\\
asaṅgahitena asaṅgahitaṃ,\\
sampayogo vippayogo,\\
sampayuttena vippayuttaṃ,\\
vippayuttena sampayuttaṃ,\\
asaṅgahitaṃ.

\suttaRef{Dhātukathā 1}

% Source: Chomtong chanting book

\section{Puggalapaññatti}

Cha paññattiyo khandhapaññatti, āyatanapaññatti, dhātupaññatti, saccapaññatti,
indriyapaññatti, puggalapaññattī'ti.

Kittāvatā puggalānaṃ puggalapaññatti.

Samayavimutto, asamayavimutto,\\
kuppadhammo, akuppadhammo,\\
parihānadhammo, aparihānadhammo,\\
cetanābhabbo, anurakkhaṇābhabbo,\\
puthujjano, gotrabhū,\\
bhayūparato, abhayūparato,\\
bhabbāgamano, abhabbāgamano,\\
niyato, aniyato,\\
paṭipannako, phaleṭhito,\\
arahā, arahattāya paṭipanno. \suttaRef{Puggalapaññatti 1}

% Source: Chomtong chanting book

\section{Kathāvatthu}

Puggalo upalabbhati saccikaṭṭha-paramatthenā'ti.

Āmantā.

Yo saccikaṭṭho paramattho, tato so puggalo upalabbhati
saccikaṭṭha-paramatthenā'ti.

Na h’evaṃ vattabbe.

Ājānāhi niggahaṃ. Hañci puggalo upalabbhati
saccikaṭṭha-paramatthena, tena vata re vattabbe.

Yo saccikaṭṭho paramattho, tato so puggalo upalabbhati
saccikaṭṭha-paramatthenā'ti micchā.

\suttaRef{Kathāvatthu 1}

% Source: Chomtong chanting book

\section{Yamaka}

Ye keci kusalā dhammā, sabbe te kusalamūlā.\\ 
Ye vā pana kusalamūlā, sabbe te dhammā kusalā.\\
Ye keci kusalā dhammā, sabbe te kusalamūlena ekamūlā.\\ 
Ye vā pana kusalamūlena ekamūlā, sabbe te dhammā kusalā.

\suttaRef{Yamaka 1}

% Source: Chomtong chanting book

\section{Paṭṭhāna-mātikā-pāṭha}

\firstline{Hetu-paccayo ārammaṇa-paccayo}

% English source: Bodhivana

Hetu-paccayo, ārammaṇa-paccayo,\\
adhipati-paccayo, anantara-paccayo,\\
samanantara-paccayo, saha-jāta-paccayo,

\clearpage

\begin{english}
  Root-cause condition, support condition,\\
  dominant condition, immediate condition,\\
  quite-immediate condition, born-simultaneously condition,
\end{english}

aññam-añña-paccayo, nissaya-paccayo,\\
upanissaya-paccayo, pure-jāta-paccayo,\\
pacchā-jāta-paccayo, āsevana-paccayo,

\begin{english}
  reciprocal condition, dependence condition,\\
  immediate-dependence condition, born-before condition,\\
  born-after condition, habit condition,
\end{english}

kamma-paccayo, vipāka-paccayo,\\
āhāra-paccayo, indriya-paccayo,\\
jhāna-paccayo, magga-paccayo,

\begin{english}
  action condition, result condition,\\
  nutriment condition, faculty condition,\\
  jhāna condition, path condition,
\end{english}

sampayutta-paccayo, vippayutta-paccayo,\\
atthi-paccayo, n'atthi-paccayo,\\
vigata-paccayo, avigata-paccayo.

\begin{english}
  conjoined-with condition, disjoined-from condition,\\
  condition when existing, condition when not existing,\\
  condition when without, condition when not without.
\end{english}

\suttaRef{Tika Paṭṭhāna 1}

\clearpage

\section{Vipassanā-bhūmi-pāṭha}

% English source: Bodhivana

\firstline{Pañcakkhandhā rūpakkhandho vedanākkhandho}

Pañcakkhandhā:\\
Rūpakkhandho, vedanākkhandho, saññākkhandho, saṅkhārakkhandho, viññāṇakkhandho.

\begin{english}
  The five groups:\\
  The form group, the feeling group, the perception group, the fabrications
  group, the consciousness group.
\end{english}

Dvā-das'āyatanāni:\\
Cakkhv-āyatanaṃ rūp'āyatanaṃ,\\
Sot'āyatanaṃ sadd'āyatanaṃ,\\
Ghān'āyatanaṃ gandh'āyatanaṃ,\\
Jivh'āyatanaṃ ras'āyatanaṃ\\
Kāy'āyatanaṃ phoṭṭhabb'āyatanaṃ\\
Man'āyatanaṃ dhamm'āyatanaṃ.

\begin{english}
  The twelve spheres:\\
  The eye-sphere, the form-sphere;\\
  the ear-sphere, the sound-sphere;\\
  the nose-sphere, the smell-sphere;\\
  the tongue-sphere, the taste-sphere;\\
  the body-sphere, the touch-sphere;\\
  the intellect-sphere, the ideas-sphere.
\end{english}

Aṭṭhārasa dhātuyo:\\
Cakkhu-dhātu rūpa-dhātu cakkhu-viññāṇa-dhātu,\\
Sota-dhātu sadda-dhātu sota-viññāṇa-dhātu,\\
Ghāna-dhātu gandha-dhātu ghāna-viññāṇa-dhātu,\\
Jivhā-dhātu rasa-dhātu jivhā-viññāṇa-dhātu,\\
Kāya-dhātu phoṭṭhabba-dhātu kāya-viññāṇa-dhātu,\\
Mano-dhātu dhamma-dhātu mano-viññāṇa-dhātu.

\begin{english}
  The eighteen elements:\\
  The eye-element, form-element, eye-consciousness-element;\\
  the ear-element, sound-element, ear-consciousness-element;\\
  the nose-element, smell-element, nose-consciousness-element;\\
  the tongue-element, taste-element, tongue-consciousness-element;\\
  the body-element, touch-element, body-consciousness-element;\\
  the intellect-element, ideas-element, intellect-consciousness-element.
\end{english}

Bā-vīsat'indriyāni:\\
Cakkhu'ndriyaṃ sot'indriyaṃ ghān'indriyaṃ,\\
jivh'indriyaṃ kāy'indriyaṃ man'indriyaṃ,\\
Itth'indriyaṃ puris'indriyaṃ jīvit'indriyaṃ,\\
Sukh'indriyaṃ dukkh'indriyaṃ,\\
somanass'indriyaṃ domanass'indriyaṃ upekkh'indriyaṃ,\\
saddh'indriyaṃ viriy'indriyaṃ sat'indriyaṃ\\
samādh'indriyaṃ paññ'indriyaṃ,\\
Anaññātañ-ñassāmī-t'indriyaṃ aññ'indriyaṃ\\
aññātāv'indriyaṃ.

\begin{english}
  The twenty two facuties:\\
  The eye-faculty, ear-faculty, nose-faculty,\\
  tongue-faculty, body-faculty, intellect-faculty.\\
  Feminine-faculty, masculine-faculty, life-faculty.\\
  Bodily-pleasure-faculty, bodily-pain-faculty,\\
  mental-pleasure-facutly, mental-pain-faculty, equanimity-faculty.\\
  Faith-faculty, energy-faculty, mindfulness-faculty,\\
  concentration-faculty, wisdom-faculty.\\
  I am knowing the unknown-faculty, knowing-faculty,\\
  one who has fully known-faculty.
\end{english}

Cattāri ariya-saccāni:\\
Dukkhaṃ ariya-saccaṃ,\\
Dukkha-samudayo ariya-saccaṃ,\\
Dukkha-nirodho ariya-saccaṃ,\\
Dukkha-nirodha-gāminī paṭipadā ariya-saccaṃ.

\begin{english}
  The Four Noble Truths:\\
  The noble truth of suffering,\\
  the noble truth of the cause of suffering,\\
  the noble truth of the cessation of suffering,\\
  the noble truth of the way of practice leading to the cessation of suffering.
\end{english}

Avijjā-paccayā saṅkhārā,\\
Saṅkhāra-paccayā viññāṇaṃ,\\
Viññāṇa-paccayā nāma-rūpaṃ,\\
Nāma-rūpa-paccayā saḷ-āyatanaṃ,\\
Saḷ-āyatana-paccayā phasso,\\
Phassa-paccayā vedanā,\\
Vedanā-paccayā taṇhā,\\
Taṇhā-paccayā upādānaṃ,\\
Upādāna-paccayā bhavo,\\
Bhava-paccayā jāti,\\
Jāti-paccayā jarā-maraṇaṃ soka-parideva-dukkha-domanass'upāyāsā sambhavanti.\\
Evam-etassa kevalassa dukkhakkhandhassa samudayo hoti.

\begin{english}
  With ignorance as a condition there are fabrications.\\
  With fabrications as a condition there is consciousness.\\
  With consciousness as a condition there are name-and-form.\\
  With name-and-form as a condition there are the six sense media.\\
  With the six sense media as a condition there is contact.\\
  With contact as a condition there is feeling.\\
  With feeling as a condition there is craving.\\
  With craving as a condition there is clinging.\\
  With clinging as a condition there is becoming.\\
  With becoming as a condition there is birth.\\
  With birth as a condition, aging and death, sorrow, lamentation, pain,\\
  distress and despair are originated.
\end{english}

Avijjāya tv-eva asesa-virāga-nirodhā saṅkhāra-nirodho,\\
Saṅkhāra-nirodhā viññāṇa-nirodho,\\
Viññāṇa-nirodhā nāma-rūpa-nirodho,\\
Nāma-rūpa-nirodhā saḷ-āyatana-nirodho,\\
Saḷ-āyatana-nirodhā phassa-nirodho,\\
Phassa-nirodhā vedanā-nirodho,\\
Vedanā-nirodhā taṇhā-nirodho,\\
Taṇhā-nirodhā upādāna-nirodho,\\
Upādāna-nirodhā bhava-nirodho,\\
Bhava-nirodhā jāti-nirodho,\\
Jāti-nirodhā jarā-maraṇaṃ soka-parideva-dukkha-domanass'upāyāsā nirujjhanti.\\
Evam-etassa kevalassa dukkhakkhandhassa nirodho hoti.

\begin{english}
  From the remainderless fading and cessation of that very ignorance there is the
  cessation of fabrications.\\
From the cessation of fabrications there is the cessation of consciousness.\\
From the cessation of consciousness there is the cessation of name-and-form.
From the cessation of name-and-form there is the cessation of the six sense media.
From the cessation of the six sense media there is the cessation of contact.
From the cessation of contact there is the cessation of feeling.
From the cessation of feeling there is the cessation of craving.
From the cessation of craving there is the cessation of clinging.
From the cessation of clinging there is the cessation of becoming.
From the cessation of becoming there is the cessation of birth.
From the cessation of birth, then aging and death, sorrow, lamentation, pain,
  distress and despair all cease.

Thus is the cessation of this entire mass of suffering and stress.
\end{english}

\suttaRef{M.III.15f; M.III.280f; M.III.62; M.III.249f; S.II.1f}

\clearpage

\section{Paṃsukūla}

\instr{The following verses are often repeated three times.}

\instr{(For the dead)}

\firstline{Aniccā vata saṅkhārā}

% English source: Bodhivana

Aniccā vata saṅkhārā\\
Uppāda-vaya-dhammino\\
Uppajjitvā nirujjhanti\\
Tesaṃ vūpasamo sukho.

Sabbe sattā maranti ca\\
Mariṃsu ca marissare\\
Tath'evāhaṃ marissāmi\\
N'atthi me ettha saṃsayo.

\begin{english}
  How inconstant are fabrications!\\
  Their nature: to arise and pass away.\\
  They disband as they are arising.\\
  Their total stilling is bliss.

  All living beings are dying,\\
  have dies, and will die.\\
  In the same way, I will die:\\
  I have no doubt about this. \suttaRef{D.II.157; S.I.6}
\end{english}

\firstline{Addhuvaṃ jīvitaṃ}

Addhuvaṃ jīvitaṃ\\
Dhuvaṃ maraṇaṃ\\
Avassaṃ mayā maritabbaṃ\\
Maraṇapariyosānaṃ me jīvitaṃ\\
Jīvitaṃ me aniyataṃ\\
Maraṇaṃ me niyataṃ. \suttaRef{DhpA.III.170}

\instr{(For the living)}

\firstline{Aciraṃ vat'ayaṃ kāyo}

Aciraṃ vat'ayaṃ kāyo\\
Paṭhaviṃ adhisessati\\
Chuḍḍho apeta-viññāṇo\\
Niratthaṃ va kaliṅgaraṃ.

\begin{english}
  Not long, alas -- and it will lie\\
  this body here, upon the earth!\\
  Rejected, void of consciousness\\
  and useless as a rotten log. \suttaRef{Dhp 41}
\end{english}


