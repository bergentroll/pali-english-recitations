\chapter{Morning Chanting}

\section*{Dedication of Offerings}

[Yo so] bhagavā arahaṃ sammāsambuddho\\
Svākkhāto yena bhagavatā dhammo\\
Supaṭipanno yassa bhagavato sāvakasaṅgho\\
Tam-mayaṃ bhagavantaṃ sadhammaṃ sasaṅghaṃ\\
Imehi sakkārehi yathārahaṃ āropitehi abhipūjayāma\\
Sādhu no bhante bhagavā sucira-parinibbutopi\\
Pacchimā-janatānukampa-mānasā\\
Ime sakkāre duggata-paṇṇākāra-bhūte paṭiggaṇhātu\\
Amhākaṃ dīgharattaṃ hitāya sukhāya\\
Arahaṃ sammāsambuddho bhagavā\\
Buddhaṃ bhagavantaṃ abhivādemi\\\relax
[Svākkhāto] bhagavatā dhammo\\
Dhammaṃ namassāmi\\\relax
[Supaṭipanno] bhagavato sāvakasaṅgho\\
Saṅghaṃ namāmi

\section*{Dedication of Offerings (English)}

To the Blessed One, the Lord,\\\vin who fully attained perfect enlightenment,\\
To the Teaching which he expounded so well,\\
And to the Blessed One's disciples who have practised well,\\
To these --- the Buddha, the Dhamma, and the Saṅgha ---\\
We render with offerings our rightful homage.\\
It is well for us that the Blessed One, having attained liberation,\\
Still had compassion for later generations.\\
May these simple offerings be accepted\\
For our long-lasting benefit and for the happiness it gives us.

The Lord, the Perfectly Enlightened and Blessed One ---\\
I render homage to the Buddha, the Blessed One.

The Teaching so completely explained by him ---\\
I bow to the Dhamma.

The Blessed One's disciples who have practised well ---\\
I bow to the Saṅgha.

\section*{Preliminary Homage}

\begin{leader}
  [Handa mayaṃ buddhassa bhagavato pubbabhāga-namakāraṃ karomase]
\end{leader}

Namo tassa bhagavato arahato sammāsambuddhassa (×3)

\section*{Preliminary Homage (English)}

\begin{leader}
  [Now let us pay preliminary homage to the Buddha.]
\end{leader}

Homage to the Blessed, Noble, and Perfectly Enlightened One. (×3)

\clearpage

\section*{Homage to the Buddha}

\begin{leader}
  [Handa mayaṃ buddhābhitthutiṃ karomase]
\end{leader}

Yo so tathāgato arahaṃ sammāsambuddho\\
Vijjācaraṇa-sampanno sugato lokavidū\\
Anuttaro purisadamma-sārathi\\
Satthā deva-manussānaṃ buddho bhagavā

Yo imaṃ lokaṃ sadevakaṃ samārakaṃ sabrahmakaṃ\\
Sassamaṇa-brāhmaṇiṃ pajaṃ sadeva-manussaṃ sayaṃ abhiññā sacchikatvā pavedesi\\
Yo dhammaṃ desesi ādi-kalyāṇaṃ majjhe-kalyāṇaṃ pariyosāna-kalyāṇaṃ\\
Sātthaṃ sabyañjanaṃ kevala-paripuṇṇaṃ parisuddhaṃ brahma-cariyaṃ pakāsesi\\
Tam-ahaṃ bhagavantaṃ abhipūjayāmi\\
Tam-ahaṃ bhagavantaṃ sirasā namāmi

\section*{Homage to the Buddha (English)}

\begin{leader}
  [Now let us chant in praise of the Buddha.]
\end{leader}

The Tathāgata is the Pure One, the Perfectly Enlightened One.\\
He is impeccable in conduct and understanding,\\
The Accomplished One,\\
The Knower of the Worlds.\\
He trains perfectly those who wish to be trained.\\
He is Teacher of gods and humans.\\
He is awake and holy.\\
In this world with its gods, demons, and kind spirits,\\
Its seekers and sages, celestial and human beings, he has by \\deep insight revealed the Truth.\\
He has pointed out the Dhamma: beautiful in the beginning, \\beautiful in the middle, beautiful in the end.\\
He has explained the Spiritual Life of complete purity in its \\essence and conventions.\\
I chant my praise to the Blessed One, I bow my head to \\the Blessed One.

\section*{Homage to the Dhamma}

\begin{leader}
  [Handa mayaṃ dhammābhitthutiṃ karomase]
\end{leader}

Yo so svākkhāto bhagavatā dhammo\\
Sandiṭṭhiko akāliko ehipassiko opanayiko\\
Paccattaṃ veditabbo viññūhi\\
Tam-ahaṃ dhammaṃ abhipūjayāmi\\
Tam-ahaṃ dhammaṃ sirasā namāmi

\section*{Homage to the Dhamma (English)}

\begin{leader}
  [Now let us chant in praise of the Dhamma.]
\end{leader}

The Dhamma is well expounded by the Blessed One,\\
Apparent here and now,\\
Timeless,\\
Encouraging investigation,\\
Leading inwards,\\
To be experienced individually by the wise.\\
I chant my praise to this Teaching, I bow my head\\ to this Truth.

\section*{Homage to the Saṅgha}

\begin{leader}
  [Handa mayaṃ saṅghābhitthutiṃ karomase]
\end{leader}

Yo so supaṭipanno bhagavato sāvakasaṅgho\\
Ujupaṭipanno bhagavato sāvakasaṅgho\\
Ñāyapaṭipanno bhagavato sāvakasaṅgho\\
Sāmīcipaṭipanno bhagavato sāvakasaṅgho\\
Yadidaṃ cattāri purisayugāni aṭṭha purisapuggalā\\
Esa bhagavato sāvakasaṅgho\\
Āhuneyyo pāhuneyyo dakkhiṇeyyo añjali-karaṇīyo\\
Anuttaraṃ puññakkhettaṃ lokassa\\
Tam-ahaṃ saṅghaṃ abhipūjayāmi\\
Tam-ahaṃ saṅghaṃ sirasā namāmi

\section*{Homage to the Saṅgha (English)}

\begin{leader}
  [Now let us chant in praise of the Saṅgha.]
\end{leader}

They are the Blessed One's disciples, who have practised well,\\
Who have practised directly,\\
Who have practised insightfully,\\
Those who practise with integrity ---\\
That is the four pairs, the eight kinds of noble beings ---\\
These are the Blessed One's disciples.\\
Such ones are worthy of gifts,\\
Worthy of hospitality,\\
Worthy of offerings,\\
Worthy of respect;\\
They give occasion for incomparable goodness to arise \\in the world.\\
I chant my praise to this Saṅgha, I bow my head to\\ this Saṅgha.

\section*{Salutation to the Triple Gem}

\begin{leader}
  [Handa mayaṃ ratanattaya-paṇāma-gāthāyo c'eva\\
  saṃvega-parikittana-pāṭhañca bhaṇāmase]
\end{leader}

\firstline{Buddho susuddho karuṇā-mahaṇṇavo}

Buddho susuddho karuṇā-mahaṇṇavo\\
Yo'ccanta-suddhabbara-ñāṇa-locano\\
Lokassa pāpūpakilesa-ghātako\\
Vandāmi buddhaṃ aham-ādarena taṃ\\
Dhammo padīpo viya tassa satthuno\\
Yo magga-pākāmata-bheda-bhinnako\\
Lokuttaro yo ca tad-attha-dīpano\\
Vandāmi dhammaṃ aham-ādarena taṃ\\
Saṅgho sukhettābhyati-khetta-saññito\\
Yo diṭṭha-santo sugatānubodhako\\
Lolappahīno ariyo sumedhaso\\
Vandāmi saṅghaṃ aham-ādarena taṃ\\
Iccevam-ekantabhipūja-neyyakaṃ vatthuttayaṃ \\vandayatābhisaṅkhataṃ\\
Puññaṃ mayā yaṃ mama sabbupaddavā mā hontu ve tassa pabhāva-siddhiyā

Idha tathāgato loke uppanno arahaṃ sammāsambuddho\\
Dhammo ca desito niyyāniko upasamiko parinibbāniko sambodhagāmī sugatappavedito\\
Mayan-taṃ dhammaṃ sutvā evaṃ jānāma

Jātipi dukkhā\\
Jarāpi dukkhā\\
Maraṇampi dukkhaṃ\\
Soka-parideva-dukkha-domanass'upāyāsāpi dukkhā\\
Appiyehi sampayogo dukkho\\
Piyehi vippayogo dukkho\\
Yamp'icchaṃ na labhati tampi dukkhaṃ\\
Saṅkhittena pañcupādānakkhandhā dukkhā

Seyyathīdaṃ\\
Rūpūpādānakkhandho\\
Vedanūpādānakkhandho\\
Saññūpādānakkhandho\\
Saṅkhārūpādānakkhandho\\
Viññāṇūpādānakkhandho

Yesaṃ pariññāya\\
Dharamāno so bhagavā evaṃ bahulaṃ sāvake vineti\\
Evaṃ bhāgā ca panassa bhagavato sāvakesu anusāsanī bahulā pavattati

Rūpaṃ aniccaṃ\\
Vedanā aniccā\\
Saññā aniccā\\
Saṅkhārā aniccā\\
Viññāṇaṃ aniccaṃ\\
Rūpaṃ anattā\\
Vedanā anattā\\
Saññā anattā\\
Saṅkhārā anattā\\
Viññāṇaṃ anattā\\
Sabbe saṅkhārā aniccā\\
Sabbe dhammā anattā'ti

Te mayaṃ otiṇṇāmha jātiyā jarā-maraṇena\\
Sokehi paridevehi dukkhehi domanassehi upāyāsehi\\
Dukkhotiṇṇā dukkha-paretā\\
Appeva nāmimassa kevalassa dukkha-kkhandhassa\\
antakiriyā paññāyethā'ti

Cira-parinibbutampi taṃ bhagavantaṃ uddissa arahantaṃ sammāsambuddhaṃ\\
Saddhā agārasmā anagāriyaṃ pabbajitā\\
Tasmiṃ bhagavati brahma-cariyaṃ carāma\\
Bhikkhūnaṃ/Sīladharānaṃ sikkhāsājīva-samāpannā\\
Taṃ no brahma-cariyaṃ imassa kevalassa dukkha-kkhandhassa antakiriyāya saṃvattatu

\section*{Salutation to the Triple Gem (English)}

\begin{leader}
  [Now let us chant our salutation to the Triple Gem and a passage to arouse urgency.]
\end{leader}

The Buddha, absolutely pure, with ocean-like compassion,\\
Possessing the clear sight of wisdom,\\
Destroyer of worldly self-corruption ---\\
Devotedly indeed, that Buddha I revere.\\
The Teaching of the Lord, like a lamp,\\
Illuminating the Path and its Fruit: the Deathless,\\
That which is beyond the conditioned world ---\\
Devotedly indeed, that Dhamma I revere.\\
The Saṅgha, the most fertile ground for cultivation,\\
Those who have realized peace, awakened after the \\Accomplished One,\\
Noble and wise, all longing abandoned ---\\
Devotedly indeed, that Saṅgha I revere.

This salutation should be made to that which is worthy.\\
Through the power of such good action,\\\vin may all obstacles disappear.\\
One who knows things as they are has come into this world; and he is an Arahant, a perfectly Awakened being,\\
Purifying the way leading out of delusion, calming and directing to perfect peace, and leading to enlightenment --- this Way he has made known.

Having heard the Teaching, we know this:\\
Birth is dukkha,\\
Ageing is dukkha,\\
And death is dukkha;\\
Sorrow, lamentation, pain, grief, and despair are dukkha;\\
Association with the disliked is dukkha;\\
Separation from the liked is dukkha;\\
Not attaining one's wishes is dukkha.\\
In brief, the five focuses of identity are dukkha.

These are as follows:\\
Attachment to form,\\
Attachment to feeling,\\
Attachment to perception,\\
Attachment to mental formations,\\
Attachment to sense-consciousness.\\
For the complete understanding of this,\\
The Blessed One in his lifetime frequently instructed his disciples \\in just this way.

In addition, he further instructed:\\
Form is impermanent,\\
Feeling is impermanent,\\
Perception is impermanent,\\
Mental formations are impermanent,\\
Sense-consciousness is impermanent;

Form is not-self,\\
Feeling is not-self,\\
Perception is not-self,\\
Mental formations are not-self,\\
Sense-consciousness is not-self;\\
All conditions are transient,\\
There is no self in the created or the uncreated.\\
All of us are bound by birth, ageing, and death,\\
By sorrow, lamentation, pain, grief, and despair,\\
Bound by dukkha and obstructed by dukkha.\\
Let us all aspire to complete freedom from suffering.

\begin{instruction}
  The following is chanted only by the monks and nuns.
\end{instruction}

Remembering the Blessed One, the Noble Lord, and Perfectly Enlightened One, who long ago attained Parinibbāna,\\
We have gone forth with faith from home to homelessness,\\
And like the Blessed One, we practise the Holy Life,\\
Being fully equipped with the bhikkhus'/nuns' system of training.\\
May this Holy Life lead us to the end of this whole mass\\ of suffering.\\

\begin{instruction}
  An alternative version of the preceding section, which can be chanted by laypeople as well.
\end{instruction}

The Blessed One, who long ago attained Parinibbāna, is our refuge.\\
So too are the Dhamma and the Saṅgha.\\
Attentively we follow the pathway of that Blessed One, with all of \\our mindfulness and strength.\\
May then the cultivation of this practice\\
Lead us to the end of every kind of suffering.

\section*{Closing Homage}

[Arahaṃ] sammāsambuddho bhagavā\\
Buddhaṃ bhagavantaṃ abhivādemi

[Svākkhāto] bhagavatā dhammo\\
Dhammaṃ namassāmi

[Supaṭipanno] bhagavato sāvakasaṅgho\\
Saṅghaṃ namāmi

\section*{Closing Homage (English)}

The Lord, the Perfectly Enlightened and Blessed One ---\\
I render homage to the Buddha, the Blessed One.

The Teaching, so completely explained by him ---\\
I bow to the Dhamma.

The Blessed One's disciples, who have practised well ---\\
I bow to the Saṅgha.

