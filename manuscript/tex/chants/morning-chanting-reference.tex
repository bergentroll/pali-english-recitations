\chapter{Homage to the Triple Gem}

\section{Dedication of Offerings}

[Yo so] bhagavā arahaṁ sammāsambuddho

\begin{cprenglish}
  To the Blessed One the Worthy One who fully attained Perfect Enlightenment
\end{cprenglish}

Svākkhāto yena bhagavatā dhammo

\begin{cprenglish}
  To the Teaching which he expounded so well
\end{cprenglish}

Supaṭipanno yassa bhagavato sāvakasaṅgho

\begin{cprenglish}
  And to the Blessed One’s disciples who have practiced well
\end{cprenglish}

Tam-mayaṁ bhagavantaṁ sadhammaṁ sasaṅghaṁ

\begin{cprenglish}
  To these the Buddha the Dhamma and the Saṅgha
\end{cprenglish}

Imehi sakkārehi yathārahaṁ āropitehi abhipūjayāma

\begin{cprenglish}
  We render with offerings our rightful homage
\end{cprenglish}

Sādhu no bhante bhagavā sucira-parinibbutopi

\begin{cprenglish}
  It is well for us that the Blessed One\\
  Having attained liberation
\end{cprenglish}

Pacchimā-janatānukampa-mānasā

\begin{cprenglish}
  Still had compassion for later generations
\end{cprenglish}

Ime sakkāre duggata-paṇṇākāra-bhūte paṭiggaṇhātu

\begin{cprenglish}
  May these simple offerings be accepted
\end{cprenglish}

Amhākaṁ dīgharattaṁ hitāya sukhāya

\begin{cprenglish}
  For our long-lasting benefit and for the happiness it gives us
\end{cprenglish}

Arahaṁ sammāsambuddho bhagavā

\begin{cprenglish}
  The Worthy One the Perfectly Enlightened and Blessed One
\end{cprenglish}

Buddhaṁ bhagavantaṁ abhivādemi\relax

\begin{cprenglish}
  I render homage to the Buddha the Blessed One (Bow)
\end{cprenglish}

[Svākkhāto] bhagavatā dhammo

\begin{cprenglish}
  The Teaching so completely explained by him
\end{cprenglish}

Dhammaṁ namassāmi\relax

\begin{cprenglish}
  I bow to the Dhamma (Bow)
\end{cprenglish}

[Supaṭipanno] bhagavato sāvakasaṅgho

\begin{cprenglish}
  The Blessed One’s disciples who have practiced well
\end{cprenglish}

Saṅghaṁ namāmi

\begin{cprenglish}
  I bow to the Saṅgha (Bow)
\end{cprenglish}

\section{Preliminary Homage}

\begin{leader}
  [Handa mayaṁ buddhassa bhagavato pubbabhāga-namakāraṁ karomase]
\end{leader}
\begin{leader}
  [Now let us pay preliminary homage to the Buddha.]
\end{leader}

Namo tassa bhagavato arahato sammāsambuddhassa [3x]

\begin{cprenglish}
  Homage to the Blessed Worthy and Perfectly Enlightened One [3x]
\end{cprenglish}

\clearpage

\section{Homage to the Buddha}

\begin{leader}
  [Handa mayaṁ buddhābhitthutiṁ karomase]
\end{leader}
\begin{leader}
  [Now let us recite in praise of the Buddha]
\end{leader}

Yo so tathāgato arahaṁ sammāsambuddho

\begin{cprenglish}
  The Tathāgata is the Worthy One the Perfectly Enlightened One
\end{cprenglish}

Vijjācaraṇa-sampanno

\begin{cprenglish}
  He is impeccable in conduct and understanding
\end{cprenglish}

Sugato

\begin{cprenglish}
  The Accomplished One
\end{cprenglish}

Lokavidū

\begin{cprenglish}
  The Knower of the Worlds
\end{cprenglish}

Anuttaro purisadamma-sārathi

\begin{cprenglish}
  Unsurpassed leader of persons to be tamedi
\end{cprenglish}

Satthā deva-manussānaṁ

\begin{cprenglish}
  He is teacher of gods and humans
\end{cprenglish}

Buddho bhagavā

\begin{cprenglish}
  He is awake and holy
\end{cprenglish}

Yo imaṁ lokaṁ sadevakaṁ samārakaṁ sabrahmakaṁ

\begin{cprenglish}
  In this world with its gods ̓ demons and kind spirits
\end{cprenglish}

Sassamaṇa-brāhmaṇiṁ pajaṁ sadeva-manussaṁ sayaṁ abhiññā sacchikatvā pavedesi

\begin{cprenglish}
  Its seekers and sages  ̓  celestial and human beings\\
  He has by deep insight revealed the truth
\end{cprenglish}

Yo dhammaṁ desesi ādi-kalyāṇaṁ majjhe-kalyāṇaṁ pariyosāna-kalyāṇaṁ

\begin{cprenglish}
  He has pointed out the Dhamma\\
  Beautiful in the beginning\\
  Beautiful in the middle\\
  Beautiful in the end\\
\end{cprenglish}

Sātthaṁ sabyañjanaṁ kevala-paripuṇṇaṁ parisuddhaṁ brahma-cariyaṁ pakāsesi

\begin{cprenglish}
  He has explained the holy life of complete purity\\
  In its essence and conventions
\end{cprenglish}

Tam-ahaṁ bhagavantaṁ abhipūjayāmi tam-ahaṁ bhagavantaṁ sirasā namāmi
\begin{cprenglish}
  I chant my praise to the Blessed One\\
  I bow my head to the Blessed One (Bow)
\end{cprenglish}

\section{Homage to the Dhamma}

\begin{leader}
  [Handa mayaṁ dhammābhitthutiṁ karomase]
\end{leader}
\begin{leader}
  [Now let us recite in praise of the Dhamma]
\end{leader}

Yo so svākkhāto bhagavatā dhammo

\begin{cprenglish}
  The Dhamma is well-expounded by the Blessed One
\end{cprenglish}

Sandiṭṭhiko

\begin{cprenglish}
  Apparent here and now
\end{cprenglish}

Akāliko

\begin{cprenglish}
  Timeless
\end{cprenglish}

Ehipassiko

\begin{cprenglish}
  Encouraging investigation
\end{cprenglish}

Opanayiko

\begin{cprenglish}
  Leading inwards
\end{cprenglish}

Paccattaṁ veditabbo viññūhi

\begin{cprenglish}
  To be experienced individually by the wise
\end{cprenglish}

Tam-ahaṁ dhammaṁ abhipūjayāmi tam-ahaṁ dhammaṁ sirasā namāmi

\begin{cprenglish}
  I chant my praise to this teaching\\
  I bow my head to this truth (Bow)
\end{cprenglish}

\section{Homage to the Saṅgha}

\begin{leader}
  Handa mayaṁ saṅghābhitthutiṁ karomase
\end{leader}
\begin{leader}
  Now let us recite in praise of the Saṅgha
\end{leader}

Yo so supaṭipanno bhagavato sāvakasaṅgho

\begin{cprenglish}
  They are the Blessed One’s disciples who have practiced well
\end{cprenglish}

Ujupaṭipanno bhagavato sāvakasaṅgho

\begin{cprenglish}
  Who have practiced directly
\end{cprenglish}

Ñāyapaṭipanno bhagavato sāvakasaṅgho

\begin{cprenglish}
  Who have practiced correctlyi
\end{cprenglish}

Sāmīcipaṭipanno bhagavato sāvakasaṅgho

\begin{cprenglish}
  Who have practiced properlyi
\end{cprenglish}

Yadidaṁ cattāri purisayugāni aṭṭha purisapuggalā

\begin{cprenglish}
  That is the four pairs the eight kinds of Noble Beings
\end{cprenglish}

Esa bhagavato sāvakasaṅgho

\begin{cprenglish}
  These are the Blessed One’s disciples
\end{cprenglish}

Āhuneyyo

\begin{cprenglish}
  Such ones are worthy of gifts
\end{cprenglish}

Pāhuneyyo

\begin{cprenglish}
  Worthy of hospitality
\end{cprenglish}

Dakkhiṇeyyo

\begin{cprenglish}
  Worthy of offerings
\end{cprenglish}

Añjali-karaṇīyo

\begin{cprenglish}
  Worthy of respect
\end{cprenglish}

Anuttaraṁ puññakkhettaṁ lokassa

\begin{cprenglish}
  They give occasion for incomparable goodness to arise in the world
\end{cprenglish}

Tam-ahaṁ saṅghaṁ abhipūjayāmi tam-ahaṁ saṅghaṁ sirasā namāmi

\begin{cprenglish}
  I chant my praise to this Saṅgha\\
  I bow my head to this Saṅgha (Bow)
\end{cprenglish}


\section{Salutation to the Triple Gem}

\begin{leader}
  [Handa mayaṁ ratanattaya-paṇāma-gāthāyo c'eva saṁvega-parikittana-pāṭhañca bhaṇāmase]
\end{leader}
\begin{leader}
  [Now let us recite our salutation to the Triple Gem and a passage to arouse urgency]
\end{leader}

\firstline{Buddho susuddho karuṇā-mahaṇṇavo}

Buddho susuddho karuṇā-mahaṇṇavo

\begin{cprenglish}
  The Buddha absolutely pure with ocean-like compassion
\end{cprenglish}

Yo'ccanta-suddhabbara-ñāṇa-locano

\begin{cprenglish}
  Possessing the clear sight of wisdom
\end{cprenglish}

Lokassa pāpūpakilesa-ghātako

\begin{cprenglish}
  Destroyer of worldly self-corruption
\end{cprenglish}

Vandāmi buddhaṁ aham-ādarena taṁ

\begin{cprenglish}
  Devotedly indeed  ̓  that Buddha I revere
\end{cprenglish}

Dhammo padīpo viya tassa satthuno

\begin{cprenglish}
  The Teaching of the Lord is like a lamp
\end{cprenglish}

Yo magga-pākāmata-bheda-bhinnako

\begin{cprenglish}
  Divided into path and its fruit  ̓  the Deathless
\end{cprenglish}

Lokuttaro yo ca tad-attha-dīpano

\begin{cprenglish}
  And illuminating that goal  ̓  which is beyond the conditioned worldi
\end{cprenglish}

Vandāmi dhammaṁ aham-ādarena taṁ

\begin{cprenglish}
  Devotedly indeed  ̓  that Dhamma I revere
\end{cprenglish}

Saṅgho sukhettābhyati-khetta-saññito

\begin{cprenglish}
  The Saṅgha the most fertile ground for cultivation
\end{cprenglish}

Yo diṭṭha-santo sugatānubodhako

\begin{cprenglish}
  Those who have realised peace\\
  Awakened after the Accomplished One
\end{cprenglish}

Lolappahīno ariyo sumedhaso

\begin{cprenglish}
  Noble and wise  ̓  all longing abandoned
\end{cprenglish}

Vandāmi saṅghaṁ aham-ādarena taṁ

\begin{cprenglish}
  Devotedly indeed  ̓  that Saṅgha I revere
\end{cprenglish}

Iccevam-ekantabhipūja-neyyakaṁ vatthuttayaṁ vandayatābhisaṅkhataṁ

\begin{cprenglish}
  This salutation should be made\\
  To that triad which is worthy
\end{cprenglish}

Puññaṁ mayā yaṁ mama sabbupaddavā

\begin{cprenglish}
  Through the power of such good action
\end{cprenglish}

Mā hontu ve tassa pabhāva-siddhiyā

\begin{cprenglish}
  May all obstacles disappear
\end{cprenglish}

Idha tathāgato loke uppanno arahaṁ sammāsambuddho

\begin{cprenglish}
  One who knows things as they are  ̓  has arisen in this world\\
  And he is an Arahant  ̓  a perfectly awakened being
\end{cprenglish}

Dhammo ca desito niyyāniko upasamiko parinibbāniko sambodhagāmī sugatappavedito

\begin{cprenglish}
  Teaching the way leading out of delusion\\
  Calming and directing to perfect peace\\
  And leading to enlightenment\\
  This way he has made known\\
\end{cprenglish}

Mayan-taṁ dhammaṁ sutvā evaṁ jānāma

\begin{cprenglish}
  Having heard the Teaching we know this
\end{cprenglish}

Jātipi dukkhā

\begin{cprenglish}
  Birth is dukkha
\end{cprenglish}

Jarāpi dukkhā

\begin{cprenglish}
  Ageing is dukkha
\end{cprenglish}

Maraṇampi dukkhaṁ

\begin{cprenglish}
  And death is dukkha
\end{cprenglish}

Soka-parideva-dukkha-domanass'upāyāsāpi dukkhā

\begin{cprenglish}
  Sorrow lamentation pain displeasurei and despair are dukkha
\end{cprenglish}

Appiyehi sampayogo dukkho

\begin{cprenglish}
  Association with the disliked is dukkha
\end{cprenglish}

Piyehi vippayogo dukkho

\begin{cprenglish}
  Separation from the liked is dukkha
\end{cprenglish}

Yamp'icchaṁ na labhati tampi dukkhaṁ

\begin{cprenglish}
  Not attaining one’s wishes is dukkha
\end{cprenglish}

Saṅkhittena pañcupādānakkhandhā dukkhā

\begin{cprenglish}
  In brief  ̓  the five aggregates of clinging are dukkhai
\end{cprenglish}

Seyyathīdaṁ

\begin{cprenglish}
  These are as follows
\end{cprenglish}

Rūpūpādānakkhandho

\begin{cprenglish}
  Attachment to form
\end{cprenglish}

Vedanūpādānakkhandho

\begin{cprenglish}
  Attachment to feeling
\end{cprenglish}

Saññūpādānakkhandho

\begin{cprenglish}
  Attachment to perception
\end{cprenglish}

Saṅkhārūpādānakkhandho

\begin{cprenglish}
  Attachment to volitional formations
\end{cprenglish}

Viññāṇūpādānakkhandho

\begin{cprenglish}
  Attachment to consciousness
\end{cprenglish}

Yesaṁ pariññāya

\begin{cprenglish}
  For the complete understanding of this
\end{cprenglish}

Dharamāno so bhagavā

\begin{cprenglish}
  The Blessed One in his lifetime
\end{cprenglish}

Evaṁ bahulaṁ sāvake vineti

\begin{cprenglish}
  Frequently instructed his disciples in just this way
\end{cprenglish}

Evaṁ bhāgā ca panassa bhagavato sāvakesu anusāsanī bahulā pavattati

\begin{cprenglish}
  In addition he further instructed
\end{cprenglish}

Rūpaṁ aniccaṁ

\begin{cprenglish}
  Form is impermanent
\end{cprenglish}

Vedanā aniccā

\begin{cprenglish}
  Feeling is impermanent
\end{cprenglish}

Saññā aniccā

\begin{cprenglish}
  Perception is impermanent
\end{cprenglish}

Saṅkhārā aniccā

\begin{cprenglish}
  Volitional formations are impermanent
\end{cprenglish}

Viññāṇaṁ aniccaṁ

\begin{cprenglish}
  Consciousness is impermanent
\end{cprenglish}

Rūpaṁ anattā

\begin{cprenglish}
  Form is not-self
\end{cprenglish}

Vedanā anattā

\begin{cprenglish}
  Feeling is not-self
\end{cprenglish}

Saññā anattā

\begin{cprenglish}
  Perception is not-self
\end{cprenglish}

Saṅkhārā anattā

\begin{cprenglish}
  Volitional formations are not-self
\end{cprenglish}

Viññāṇaṁ anattā

\begin{cprenglish}
  Consciousness is not-self
\end{cprenglish}

Sabbe saṅkhārā aniccā

\begin{cprenglish}
  All conditioned things are impermanent
\end{cprenglish}

Sabbe dhammā anattā't

\begin{cprenglish}
  All things are not-self
\end{cprenglish}

Te mayaṁ otiṇṇāmha jātiyā jarā-maraṇena

\begin{cprenglish}
  All of us are affected by birth  ̓  ageing and deathi
\end{cprenglish}

Sokehi paridevehi dukkhehi domanassehi upāyāsehi

\begin{cprenglish}
  By sorrow lamentation pain displeasure and despair
\end{cprenglish}

Dukkhotiṇṇā dukkha-paretā

\begin{cprenglish}
  Affected by dukkha and afflicted by dukkha
\end{cprenglish}

Appeva nāmimassa kevalassa dukkha-kkhandhassa antakiriyā paññāyethā'ti

\begin{cprenglish}
  Let us all aspire to complete freedom from suffering
\end{cprenglish}

Cira-parinibbutampi taṁ bhagavantaṁ uddissa arahantaṁ sammāsambuddhaṁ

\begin{cprenglish}
  Remembering the Blessed One  ̓  the Worthy One  ̓  and Perfectly Enlightened One\\
  Who long ago attained Parinibbāna
\end{cprenglish}

Saddhā agārasmā anagāriyaṁ pabbajitā

\begin{cprenglish}
  We have gone forth with faith\\
  From home to homelessness
\end{cprenglish}

Tasmiṁ bhagavati brahma-cariyaṁ carāma

\begin{cprenglish}
  And like the Blessed One  ̓  we practice the holy life
\end{cprenglish}

Bhikkhūnaṁ sikkhāsājīva-samāpannā

\begin{cprenglish}
  Possessing the bhikkhus’ training and way of life
\end{cprenglish}

Taṁ no brahma-cariyaṁ imassa kevalassa dukkha-kkhandhassa antakiriyāya saṁvattatu

\begin{cprenglish}
  May this holy life  ̓  lead us to the end of this whole mass of suffering
\end{cprenglish}

