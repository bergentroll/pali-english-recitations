\chapter{Pāṭimokkha Chants}

\section{Ovāda-pāṭimokkha-gāthā}

\englishTitle{Verses on the Training Code}

\begin{leader}
  [Handa mayaṃ ovāda-pāṭimokkha-gāthāyo bhaṇāmase]
\end{leader}

\firstline{Sabba-pāpassa akaraṇaṃ}
\firstline{Khantī paramaṃ tapo tītikkhā}

Sabba-pāpassa akaraṇaṃ

\begin{cprenglish}
  Not doing any evil;
\end{cprenglish}

Kusalassūpasampadā

\begin{cprenglish}
  To be committed to the good;
\end{cprenglish}

Sacitta-pariyodapanaṃ

\begin{cprenglish}
  To purify one's mind:
\end{cprenglish}

Etaṃ buddhāna sāsanaṃ

\begin{cprenglish}
  These are the teachings of all Buddhas.
\end{cprenglish}

Khantī paramaṃ tapo tītikkhā

\begin{cprenglish}
  Patient endurance is the highest practice,\\
  burning out defilements;\\
\end{cprenglish}

Nibbānaṃ paramaṃ vadanti buddhā

\begin{cprenglish}
  The Buddhas say Nibbāna is supreme.
\end{cprenglish}

Na hi pabbajito parūpaghātī

\begin{cprenglish}
  Not a renunciant is one who injures others;
\end{cprenglish}

Samaṇo hoti paraṃ viheṭhayanto

\begin{cprenglish}
  Whoever troubles others can't be called a monk.
\end{cprenglish}

Anūpavādo anūpaghāto

\begin{cprenglish}
  Not to insult and not to injure;
\end{cprenglish}

Pāṭimokkhe ca saṃvaro

\begin{cprenglish}
  To live restrained by training rules;
\end{cprenglish}

Mattaññutā ca bhattasmiṃ

\begin{cprenglish}
  Knowing one's measure at the meal;
\end{cprenglish}

Pantañca sayan'āsanaṃ

\begin{cprenglish}
  Retreating to a lonely place;
\end{cprenglish}

Adhicitte ca āyogo

\begin{cprenglish}
  Devotion to the higher mind:
\end{cprenglish}

Etaṃ buddhāna sāsanaṃ

\begin{cprenglish}
  These are the teachings of all Buddhas. \suttaRef{Dhp 183-185}
\end{cprenglish}

\section{Sacca-kiriyā-gāthā}

\begin{leader}
  [Handa mayaṃ sacca-kiriyā-gāthāyo bhaṇāmase]
\end{leader}

\firstline{Natthi me saraṇaṃ aññaṃ}

Natthi me saraṇaṃ aññaṃ buddho me saraṇaṃ varaṃ\\
Etena sacca-vajjena sotthi me hotu sabbadā

Natthi me saraṇaṃ aññaṃ dhammo me saraṇaṃ varaṃ\\
Etena sacca-vajjena sotthi me hotu sabbadā

Natthi me saraṇaṃ aññaṃ saṅgho me saraṇaṃ varaṃ\\
Etena sacca-vajjena sotthi me hotu sabbadā

% English source: Mahamakut Patimokkha, p.138

\begin{english}
  For me there is no other Refuge, the Buddha \ldots\ Dhamma \ldots\ Sangha is
  my excellent refuge. By the utterance of this Truth, may there be blessings
  for me.
\end{english}

\section{Sīl'uddesa-pāṭha}

\begin{leader}
  [Handa mayaṃ sīl'uddesa-pāṭhaṃ bhaṇāmase]
\end{leader}

\firstline{Bhāsitam idaṃ tena bhagavatā jānatā passatā}

Bhāsitam idaṃ tena bhagavatā jānatā passatā\\\vin arahatā sammā-sambuddhena\\
Sampanna-sīlā bhikkhave viharatha\\\vin sampanna-pāṭimokkhā\\
Pāṭimokkha-saṃvara-saṃvutā viharatha\\\vin ācāra-gocara-sampannā\\
Aṇu-mattesu vajjesu bhaya-dassāvī\\\vin samādāya sikkhatha sikkhāpadesū'ti

Tasmā-tih'amhehi sikkhitabbaṃ\\
Sampanna-sīlā viharissāma sampanna-pāṭimokkhā\\
Pāṭimokkha-saṃvara-saṃvutā viharissāma\\\vin ācāra-gocara-sampannā\\
Aṇu-mattesu vajjesu bhaya-dassāvī\\\vin samādāya sikkhissāma sikkhāpadesū'ti\\
Evañ hi no sikkhitabbaṃ

% English source: Mahamakut Patimokkha, p.138

\begin{english}
  \setlength{\parskip}{8pt}%
  This has been said by the Lord, One-who-knows, One-who-sees, the Arahant, the
  Perfect Buddha enlightened by himself: `Bhikkhus, be perfect in moral
  conduct. Be perfect in the Pāṭimokkha. Dwell restrained in accordance with the
  the Pāṭimokkha. Be perfect in conduct and resort, seeing danger even in the
  slightest faults. Train yourselves by undertaking rightly the rules of training.'

  Therefore we should train ourselves thus: `We will be perfect in the
  Pāṭimokkha. We will dwell restrained in accordance with the Pāṭimokkha. We
  will be perfect in conduct and resort, seeing danger even in the slightest
  faults.' Thus indeed we should train ourselves.
\end{english}

\suttaRef{D.I.63; D.III.266f}

\section{Tāyana-gāthā}

\englishTitle{The Verses of Tāyana}

\begin{leader}
  [Handa mayaṃ tāyana-gāthāyo bhaṇāmase]
\end{leader}

\smallskip

\firstline{Chinda sotaṃ parakkamma}

Chinda sotaṃ parakkamma\\
Kāme panūda brāhmaṇa\\
Nappahāya muni kāme\\
N'ekattam-upapajjati

\begin{english}
  Exert yourself and cut the stream.\\
  Discard sense pleasures, brahmin;\\
  Not letting sensual pleasures go,\\
  A sage will not reach unity.
\end{english}

Kayirā ce kayirāthenaṃ\\
Daḷham-enaṃ parakkame\\
Sithilo hi paribbājo\\
Bhiyyo ākirate rajaṃ

\ifreferenceedition
\clearpage
\fi

\begin{english}
  Vigorously, with all one's strength,\\
  It should be done, what should be done;\\
  A lax monastic life stirs up\\
  The dust of passions all the more.
\end{english}

Akataṃ dukkaṭaṃ seyyo\\
Pacchā tappati dukkaṭaṃ\\
Katañca sukataṃ seyyo\\
Yaṃ katvā nānutappati

\begin{english}
  Better is not to do bad deeds\\
  That afterwards would bring remorse;\\
  It's rather good deeds one should do\\
  Which having done one won't regret.
\end{english}

Kuso yathā duggahito\\
Hattham-evānukantati\\
Sāmaññaṃ dupparāmaṭṭhaṃ\\
Nirayāyūpakaḍḍhati

\begin{english}
  As Kusa-grass, when wrongly grasped,\\
  Will only cut into one's hand\\
  So does the monk's life wrongly led\\
  Indeed drag one to hellish states.
\end{english}

Yaṃ kiñci sithilaṃ kammaṃ\\
Saṅkiliṭṭhañca yaṃ vataṃ\\
Saṅkassaraṃ brahma-cariyaṃ\\
Na taṃ hoti mahapphalan'ti

\ifreferenceedition
\clearpage
\fi

\begin{english}
  Whatever deed that's slackly done,\\
  Whatever vow corruptly kept,\\
  The Holy Life led in doubtful ways --\\
  All these will never bear great fruit. \suttaRef{S.I.49f}
\end{english}

\ifhandbookedition
\clearpage
\fi

\section{Sāmaṇera-sikkhā}

\firstline{Anuññāsi kho bhagavā sāmaṇerānaṃ dasa}

Anuññāsi kho bhagavā\\
Sāmaṇerānaṃ dasa sikkhā-padāni

\begin{cprenglish}
  Ten novice training rules\\
  were established by the Blessed One.
\end{cprenglish}

Tesu ca sāmaṇerehi sikkhituṃ

\begin{cprenglish}
  They are the things in which a novice should train:
\end{cprenglish}

Pāṇātipātā veramaṇī

\begin{cprenglish}
  Abstaining from killing living beings
\end{cprenglish}

Adinn'ādānā veramaṇī

\begin{cprenglish}
  Abstaining from taking what is not given
\end{cprenglish}

Abrahma-cariyā veramaṇī

\begin{cprenglish}
  Abstaining from unchastity
\end{cprenglish}

Musā-vādā veramaṇī

\begin{cprenglish}
  Abstaining from false speech
\end{cprenglish}

Surā-meraya-majja-pamādaṭṭhānā veramaṇī

\begin{cprenglish}
  Abstaining from intoxicants that dull the mind
\end{cprenglish}

Vikāla-bhojanā veramaṇī

\begin{cprenglish}
  Abstaining from eating at the wrong time
\end{cprenglish}

Nacca-gīta-vādita-visūka-dassanā veramaṇī

\begin{cprenglish}
  Abstaining from dancing, singing, music and watching shows
\end{cprenglish}

Mālā-gandha-vilepana-dhāraṇa-\\
\vin maṇḍana-vibhūsanaṭṭhānā veramaṇī

\begin{cprenglish}
  Abstaining from perfumes, beautification and adornment
\end{cprenglish}

Uccā-sayana-mahā-sayanā veramaṇī

\begin{cprenglish}
  Abstaining from lying on high or luxurious beds
\end{cprenglish}

Jāta-rūpa-rajata-paṭiggahaṇā veramaṇī'ti.

\begin{cprenglish}
  Abstaining from using gold, silver or money. \suttaRef{Vin.I.83f}
\end{cprenglish}

Anuññāsi kho Bhagavā\\
Dasahi aṅgehi samannāgataṃ sāmaṇeraṃ nāsetuṃ

\begin{cprenglish}
  Ten grounds for a novice to be dismissed\\
  were established by the Blessed One.
\end{cprenglish}

Katamehi dasahi

\begin{cprenglish}
  What are these ten?
\end{cprenglish}

Pāṇātipātī hoti

\begin{cprenglish}
  He is a killer of living beings
\end{cprenglish}

Adinn'ādāyī hoti

\begin{cprenglish}
  He is a taker of what is not given
\end{cprenglish}

Abrahma-cārī hoti

\begin{cprenglish}
  He is a practicioner of unchastity
\end{cprenglish}

Musā-vādī hoti

\begin{cprenglish}
  He is a speaker of falsity
\end{cprenglish}

Majja-pāyī hoti

\begin{cprenglish}
  He is a consumer of intoxicants
\end{cprenglish}

Buddhassa avaṇṇaṃ bhāsati

\begin{cprenglish}
  He speaks in dispraise of the Buddha
\end{cprenglish}

Dhammassa avaṇṇaṃ bhāsati

\begin{cprenglish}
  He speaks in dispraise of the Dhamma
\end{cprenglish}

Saṅghassa avaṇṇaṃ bhāsati

\begin{cprenglish}
  He speaks in dispraise of the Saṅgha
\end{cprenglish}

Micchā-diṭṭhiko hoti

\begin{cprenglish}
  He is a holder of wrong views
\end{cprenglish}

Bhikkhunī-dūsako hoti

\begin{cprenglish}
  He has corrupted a nun
\end{cprenglish}

Anuññāsi kho Bhagavā\\
Imehi dasahi aṅgehi samannāgataṃ sāmaṇeraṃ nāsetun'ti.

\begin{cprenglish}
  These are the ten grounds for a novice to be dismissed\\
  which were established by the Blessed One. \suttaRef{Vin.I.85}
\end{cprenglish}

Anuññāsi kho Bhagavā\\
Pañcahi aṅgehi samannāgatassa sāmaṇerassa daṇḍa-kammaṃ kātuṃ

\begin{cprenglish}
  Five grounds for a novice to be punished\\
  were established by the Blessed One.
\end{cprenglish}

Katamehi pañcahi

\begin{cprenglish}
  What are these five?
\end{cprenglish}

Bhikkhūnaṃ alābhāya parisakkati

\begin{cprenglish}
  He strives for the loss of the Bhikkhus
\end{cprenglish}

Bhikkhūnaṃ anatthāya parisakkati

\begin{cprenglish}
  He strives for the non-benefit of the Bhikkhus
\end{cprenglish}

Bhikkhūnaṃ anāvāsāya parisakkati

\begin{cprenglish}
  He strives for the non-residence of the Bhikkhus
\end{cprenglish}

Bhikkhū akkosati paribhāsati

\begin{cprenglish}
  He insults or abuses the Bhikkhus
\end{cprenglish}

Bhikkhū bhikkhūhi bhedeti

\begin{cprenglish}
  He causes a split between the Bhikkhus
\end{cprenglish}

Anuññāsi kho Bhagavā\\
Imehi pañcahi aṅgehi samannāgatassa\\
sāmaṇerassa daṇḍa-kammaṃ kātun'ti

\begin{cprenglish}
  These are the ten grounds for a novice to be punished\\
  that were established by the Blessed One.
\end{cprenglish}

\suttaRef{Vin.I.84}

