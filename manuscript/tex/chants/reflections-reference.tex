\chapter{Reflections}

\section{The Four Requisites}
\paliTitle{Cattaro parrikhārā}

\begin{leader}
  [Handa mayaṃ taṅkhaṇika-paccavekkhaṇa-pāṭhaṃ bhaṇāmase]
\end{leader}

Paṭisaṅkhā yoniso cīvaraṁ paṭisevāmi\\
Yāvadeva sītassa paṭighātāya\\
Uṇhassa paṭighātāya\\
Ḍaṁsa-makasa-vātātapa-siriṁsapa-samphassānaṁ paṭighātāya\\
Yāvadeva hirikopina-paṭicchādanatthaṁ

\begin{english}
  Wisely reflecting  ̓  I use the robe\\
  Only to ward off cold  ̓  to ward off heat  ̓  to ward off the touch of flies  ̓  mosquitoes wind burning and creeping things\\
  Only for the sake of modesty
\end{english}

Paṭisaṅkhā yoniso piṇḍapātaṁ paṭisevāmi\\
Neva davāya na madāya na maṇḍanāya na vibhūsanāya\\
Yāvadeva imassa kāyassa ṭhitiyā yāpanāya\\
Vihiṁsūparatiyā brahmacariyānuggahāya\\
Iti purāṇañca vedanaṁ paṭihaṅkhāmi\\
Navañca vedanaṁ na uppādessāmi\\
Yātrā ca me bhavissati anavajjatā ca phāsuvihāro cā’ti

\begin{english}
  Wisely reflecting  ̓  I use almsfood\\
  Not for fun  ̓  not for pleasure  ̓  not for fattening  ̓  not for beautification\\
  Only for the maintenance and nourishment of this body\\
  For keeping it healthy  ̓  for helping with the holy life\\
  Thinking thus: “I will allay hunger without overeating\\
  So that I may continue to live blamelessly and at ease”
\end{english}

Paṭisaṅkhā yoniso senāsanaṁ paṭisevāmi\\
Yāvadeva sītassa paṭighātāya\\
Uṇhassa paṭighātāya\\
Ḍaṁsa-makasa-vātātapa-siriṁsapa-samphassānaṁ paṭighātāya\\
Yāvadeva utuparissaya-vinodanaṁ paṭisallānārāmatthaṁ

\begin{english}
  Wisely reflecting  ̓  I use the lodging\\
  Only to ward off cold  ̓  to ward off heat  ̓  to ward off the touch of flies  ̓  mosquitoes wind burning and creeping things\\
  Only to remove the danger from weather  ̓  and for living in seclusion
\end{english}

Paṭisaṅkhā yoniso gilāna-paccaya-bhesajja-parikkhāraṁ paṭisevāmi\\
Yāvadeva uppannānaṁ veyyābādhikānaṁ vedanānaṁ paṭighātāya\\
Abyāpajjha-paramatāyā’ti

\begin{english}
  Wisely reflecting  ̓  I use supports for the sick and medicinal requisites\\
  Only to ward off painful feelings that have arisen\\
  For the maximum freedom from disease
\end{english}

\suttaRef{MN 2}

\clearpage

\section{The Repulsiveness of Food}
\paliTitle{Āhāra-paṭikūla-paccavekkhaṇa-pāṭho}

\begin{leader}
  [Handa mayaṁ āhāra-paṭikūla-paccavekkhaṇa-pāṭhaṁ bhaṇāmase]
\end{leader}

Āhāre paṭikūlasaññāparicitena bhikkhave  ̓  bhikkhuno cetasā bahulaṁ viharato

\begin{english}
  When a bhikkhu often dwells with a mind\\
  Accustomed to the perception of the repulsiveness of food
\end{english}

Rasataṇhāya cittaṁ patilīyati

\begin{english}
  His mind shrinks away from craving for tastes
\end{english}

Patikuṭati pativattati na sampasāriyati

\begin{english}
  Turns back from it\\
  Rolls away from it\\
  And is not drawn towards it
\end{english}

Upekkhā vā pāṭikulyatā vā saṇṭhāti

\begin{english}
  Either equanimity or disgust become settled in him
\end{english}

\suttaRef{AN 7.49}

Sabbo panāyaṁ piṇḍa-pāto ajigucchanīyo

\begin{english}
  None of this almsfood is innately repulsive
\end{english}

Imaṁ pūti-kāyaṁ patvā

\begin{english}
  But touching this unclean body
\end{english}

Ativiya jigucchanīyo jāyati

\begin{english}
  It becomes disgusting indeed
\end{english}

\suttaRef{Trad}

\clearpage

\section{Universal Well-Being}
\paliTitle{Mettā-pharaṇa}

\begin{leader}
  [Handa mayaṁ mettāpharaṇaṁ karomase]
\end{leader}

Ahaṁ sukhito homi\\
Niddukkho homi\\
Avero homi\\
Abyāpajjho homi\\
Anīgho homi\\
Sukhī attānaṁ pariharāmi\\
Sabbe sattā sukhitā hontu\\
Sabbe sattā averā hontu\\
Sabbe sattā abyāpajjhā hontu\\
Sabbe sattā anīghā hontu\\
Sabbe sattā sukhī attānaṁ pariharantu\\
Sabbe sattā sabbadukkhā pamuccantu\\
Sabbe sattā laddha-sampattito mā vigacchantu

Sabbe sattā kammassakā kammadāyādā kammayonī kammabandhū kammapaṭisaraṇā\\
Yaṁ kammaṁ karissanti\\
Kalyāṇaṁ vā pāpakaṁ vā\\
Tassa dāyādā bhavissanti

\begin{leader}
  [Now let us recite the reflections on universal well-being]
\end{leader}

\begin{english}
  May I abide in well-being\\
  In freedom from affliction\\
  In freedom from hostility\\
  In freedom from ill-will\\
  In freedom from anxiety\\
  And may I maintain well-being in myself\\
  May everyone abide in well-being\\
  In freedom from hostility\\
  In freedom from ill-will\\
  In freedom from anxiety\\
  And may they maintain well-being in themselves\\
  May all beings be released from all suffering\\
  And may they not be parted from the good fortune they have attainedi

  All beings are the owners of their kamma\\
  Heirs to their kamma\\
  Born of their kamma\\
  Related to their kamma\\
  Abide supported by their kamma\\
  Whatever kamma they shall do\\
  Either skillful or harmful\\
  Of such acts  ̓  they will be the heirs\\
\end{english}

\suttaRef{AN 3.65 \& 5.57}

\clearpage

\section{The Divine Abidings}
\paliTitle{Brahmavihārā}

\begin{leader}
  [Handa mayaṁ caturappamaññā obhāsanaṁ karomase]
\end{leader}

Mettā-sahagatena cetasā ekaṃ disaṃ pharitvā viharati tathā dutiyaṃ tathā tatiyaṃ tathā catutthaṃ iti uddhamadho tiriyaṃ sabbadhi sabbattatāya sabbāvantaṃ lokaṃ mettā-sahagatena cetasā vipulena mahaggatena appamāṇena averena abyāpajjhena pharitvā viharati

Karuṇā-sahagatena cetasā ekaṃ disaṃ pharitvā viharati tathā dutiyaṃ tathā tatiyaṃ tathā catutthaṃ
Iti uddhamadho tiriyaṃ sabbadhi sabbattatāya sabbāvantaṃ lokaṃ karuṇā-sahagatena cetasā vipulena mahaggatena appamāṇena averena abyāpajjhena pharitvā viharati

Muditā-sahagatena cetasā ekaṃ disaṃ pharitvā viharati tathā dutiyaṃ tathā tatiyaṃ tathā catutthaṃ iti uddhamadho tiriyaṃ sabbadhi sabbattatāya sabbāvantaṃ lokaṃ muditā-sahagatena cetasā vipulena mahaggatena appamāṇena averena abyāpajjhena pharitvā viharati

Upekkhā-sahagatena cetasā ekaṃ disaṃ pharitvā viharati tathā dutiyaṃ tathā tatiyaṃ tathā catutthaṃ iti uddhamadho tiriyaṃ sabbadhi sabbattatāya sabbāvantaṃ lokaṃ upekkhā-sahagatena cetasā vipulena mahaggatena appamāṇena averena abyāpajjhena pharitvā viharatī'ti

\begin{leader}
  [Now let us make the Four Boundless Qualities shine forth]
\end{leader}

I will abide pervading one quarter with a heart imbued with loving-kindness\\
Likewise the second likewise the third likewise the fourth\\
So above and below around and everywhere and to all as to myself\\
I will abide pervading the all-encompassing world with a heart imbued with loving-kindness\\
Abundant exalted immeasurable without hostility and without ill-will

I will abide pervading one quarter with a heart imbued with compassion\\
Likewise the second likewise the third likewise the fourth\\
So above and below around and everywhere and to all as to myself\\
I will abide pervading the all-encompassing world with a heart imbued with compassion\\
Abundant exalted immeasurable without hostility and without ill-will

I will abide pervading one quarter with a heart imbued with gladness\\
Likewise the second likewise the third likewise the fourth\\
So above and below around and everywhere and to all as to myself\\
I will abide pervading the all-encompassing world with a heart imbued with gladness\\
Abundant exalted immeasurable without hostility and without ill-will

I will abide pervading one quarter with a heart imbued with equanimity\\
Likewise the second likewise the third likewise the fourth\\
So above and below around and everywhere and to all as to myself\\
I will abide pervading the all-encompassing world with a heart imbued with equanimity\\
Abundant exalted immeasurable without hostility and without ill-will

\suttaRef{DN 13}

\clearpage

\section{Five Subjects for Frequent Recollection}

\begin{leader}
  [Handa mayaṃ abhiṇha-paccavekkhaṇa-pāṭhaṃ bhaṇāmase]
\end{leader}

Jarā-dhammomhi jaraṃ anatīto

\begin{english}
  I am of the nature to age\\
  I have not gone beyond ageing
\end{english}

Byādhi-dhammomhi byādhiṃ anatīto

\begin{english}
  I am of the nature to sicken\\
  I have not gone beyond sickness
\end{english}

Maraṇa-dhammomhi maraṇaṃ anatīto

\begin{english}
  I am of the nature to die\\
  I have not gone beyond dying
\end{english}

Sabbehi me piyehi manāpehi nānābhāvo vinābhāvo

\begin{english}
  All that is mine beloved and pleasing\\
  Will become otherwise\\
  Will become separated from me
\end{english}

Kammassakomhi kammadāyādo kammayoni kammabandhu kammapaṭisaraṇo\\
Yaṃ kammaṃ karissāmi\\
Kalyāṇaṃ vā pāpakaṃ vā\\
Tassa dāyādo bhavissāmi

\begin{english}
  I am the owner of my kamma\\
  Heir to my kamma\\
  Born of my kamma\\
  Related to my kamma\\
  Abide supported by my kamma\\
  Whatever kamma I shall do\\
  Either skillful or harmful\\
  Of such acts  ̓  I will be the heir
\end{english}

Evaṃ amhehi abhiṇhaṃ paccavekkhitabbaṃ

\begin{english}
  Thus we should frequently recollect
\end{english}

\suttaRef{AN 5.57}

\clearpage

\section{Ten Subjects for Frequent Recollection}
\paliTitle{Dasadhammā pabbajita-abhiṇha-paccavekkhaṇā}

\begin{leader}
  [Handa mayaṁ pabbajita-abhiṇha-paccavekkhaṇa-pāṭhaṁ bhaṇāmase]
\end{leader}

Dasa ime bhikkhave dhammā\\
Pabbajitena abhiṇhaṁ paccavekkhitabbā\\
Katame dasa

\begin{english}
  Bhikkhus there are these ten dhammas  ̓  which should be reflected upon again and again by one who has gone forth\\
  What are these ten?
\end{english}

Vevaṇṇiyamhi ajjhūpagato'ti pabbajitena abhiṇhaṃ paccavekkhitabbaṃ

\begin{english}
  “I have reached a state of castelessness”\\
  This should be reflected upon again and again by one who has gone forth
\end{english}

Parapaṭibaddhā me jīvikā’ti\\
Pabbajitena abhiṇhaṁ paccavekkhitabbaṁ

\begin{english}
  “My very life is sustained through the gifts of others”
  This should be reflected upon again and again by one who has gone forth
\end{english}

Añño me ākappo karaṇīyo'ti\\
Pabbajitena abhiṇhaṃ paccavekkhitabbaṃ

\begin{english}
  “Now my conduct should be different from before”\\
  This should be reflected upon again and again by one who has gone forth
\end{english}

Kacci nu kho me attā sīlato na upavadatī'ti\\
Pabbajitena abhiṇhaṃ paccavekkhitabbaṃ

\begin{english}
  “Does regret over my conduct arise in my mind?”\\
  This should be reflected upon again and again by one who has gone forth
\end{english}

Kacci nu kho maṃ anuvicca viññū sabrahmacārī sīlato na upavadantī'ti\\
Pabbajitena abhiṇhaṃ paccavekkhitabbaṃ

\begin{english}
  “Could my spiritual companions find fault with my conduct?”\\
  This should be reflected upon again and again by one who has gone forth
\end{english}

Sabbehi me piyehi manāpehi nānābhāvo vinābhāvo'ti\\
Pabbajitena abhiṇhaṃ paccavekkhitabbaṃ

\begin{english}
  “All that is mine beloved and pleasing\\
  Will become otherwise\\
  Will become separated from me”\\
  This should be reflected upon again and again by one who has gone forth
\end{english}

Kammassakomhi kammadāyādo kammayoni kammabandhu kammapaṭisaraṇo\\
Yaṃ kammaṃ karissāmi\\
Kalyāṇaṃ vā pāpakaṃ vā\\
Tassa dāyādo bhavissāmī'ti\\
Pabbajitena abhiṇhaṃ paccavekkhitabbaṃ

\begin{english}
  “I am the owner of my kamma\\
  Heir to my kamma\\
  Born of my kamma\\
  Related to my kamma\\
  Abide supported by my kamma\\
  Whatever kamma I shall do\\
  Either skillful or harmful\\
  Of such acts  ̓  I will be the heir”\\
  This should be reflected upon again and again by one who has gone forth
\end{english}

`Kathambhūtassa me rattindivā vītipatantī'ti\\
Pabbajitena abhiṇhaṃ paccavekkhitabbaṃ

\begin{english}
  “The days and nights are relentlessly passing\\
  How well am I spending my time?”\\
  This should be reflected upon again and again by one who has gone forth
\end{english}

Kacci nu kho'haṃ suññāgāre abhiramāmī'ti\\
Pabbajitena abhiṇhaṃ paccavekkhitabbaṃ

\begin{english}
  “Do I delight in solitude or not?”\\
  This should be reflected upon again and again by one who has gone forth by one who has gone forth
\end{english}

Atthi nu kho me uttari-manussa-dhammā alamariya-ñāṇa-dassana viseso adhigato\\
So’haṁ pacchime kāle sabrahmacārīhi puṭṭho na maṅku bhavissāmī’ti\\
Pabbajitena abhiṇhaṁ paccavekkhitabbaṁ

\begin{english}
  “Has my practice borne fruit with freedom or insight\\
  So that at the end of my life  ̓  I need not feel ashamed when questioned by my spiritual companions?”\\
  This should be reflected upon again and again by one who has gone forth
\end{english}

Ime kho bhikkhave dasa dhammā\\
Pabbajitena abhiṇhaṃ paccavekkhitabbā'ti

\begin{english}
  Bhikkhus these are the ten dhammas  ̓  which should be reflected upon again and again by one who has gone forth
\end{english}

\suttaRef{AN 10.48}

\clearpage

\section{Reflection on the Thirty-Two Parts}

\begin{leader}
  [Handa mayaṃ dvattiṃsākāra-pāṭhaṃ bhaṇāmase]
\end{leader}

Ayaṁ kho me kāyo uddhaṁ pādatalā adho kesamatthakā tacapariyanto pūro nānappakārassa asucino

\begin{english}
  This which is my body\\
  From the soles of the feet up\\
  And down from the crown of the head\\
  Is a sealed bag of skin\\
  Filled with unattractive things
\end{english}

Atthi imasmiṃ kāye

\begin{english}
  In this body there are
\end{english}

{\centering
  \setArrayStretch{1}

  \begin{tabular}{ r l }
    kesā            & \tr{hair of the head} \\
    lomā            & \tr{hair of the body} \\
    nakhā           & \tr{nails} \\
    dantā           & \tr{teeth} \\
    taco            & \tr{skin} \\
  \end{tabular}

  \begin{tabular}{ r l }
    maṃsaṃ          & \tr{flesh} \\
    nahārū          & \tr{sinews} \\
    aṭṭhī           & \tr{bones} \\
    aṭṭhimiñjaṃ     & \tr{bone marrow} \\
    vakkaṃ          & \tr{kidneys} \\
    hadayaṃ         & \tr{heart} \\
    yakanaṃ         & \tr{liver} \\
    kilomakaṃ       & \tr{membranes} \\
    pihakaṃ         & \tr{spleen} \\
    papphāsaṃ       & \tr{lungs} \\
    antaṃ           & \tr{bowels} \\
    antaguṇaṃ       & \tr{entrails} \\
    udariyaṃ        & \tr{undigested food} \\
    karīsaṃ         & \tr{excrement} \\
    pittaṃ          & \tr{bile} \\
    semhaṃ          & \tr{phlegm} \\
    pubbo           & \tr{pus} \\
    lohitaṃ         & \tr{blood} \\
    sedo            & \tr{sweat} \\
    medo            & \tr{fat} \\
    assu            & \tr{tears} \\
    vasā            & \tr{grease} \\
    kheḷo           & \tr{spittle} \\
    siṅghāṇikā      & \tr{mucus} \\
    lasikā          & \tr{oil of the joints} \\
    muttaṃ          & \tr{urine} \\
    matthaluṅgan'ti & \tr{brain} \\
  \end{tabular}

  \restoreArrayStretch
}

Evam-ayaṁ me kāyo uddhaṁ pādatalā adho kesamatthakā tacapariyanto pūro nānappakārassa asucino

\begin{english}
  This then which is my body from the soles of the feet up and down from the crown of the head is a sealed bag of skin filled with unattractive things 
\end{english}

\suttaRef{DN 22}

\clearpage

\section{Recollection After Using the Requisites}

\begin{leader}
  [Handa mayaṃ atīta-paccavekkhaṇa-pāṭhaṃ bhaṇāmase]
\end{leader}

\firstline{Ajja mayā apaccavekkhitvā yaṃ cīvaraṃ}

Ajja mayā apaccavekkhitvā yaṃ cīvaraṃ paribhuttaṃ taṃ yāvadeva sītassa
paṭighātāya uṇhassa paṭighātāya ḍaṃsa-makasa-vātātapa-siriṃsapa-samphassānaṃ
paṭighātāya yāvadeva hirikopina paṭicchādan'atthaṃ

\begin{english}
  Whatever robe I used today without consideration was only to ward off cold
  to ward off heat to ward off the touch of flies mosquitoes wind burning
  and creeping things only for the sake of modesty
\end{english}

Ajja mayā apaccavekkhitvā yo piṇḍapāto paribhutto so n'eva davāya na madāya
na maṇḍanāya na vibhūsanāya yāvad-eva imassa kāyassa ṭhitiyā yāpanāya
vihiṃsūparatiyā brahmacariyānuggahāya iti purāṇañca vedanaṃ paṭihaṅkhāmi
navañca vedanaṃ na uppādessāmi yātrā ca me bhavissati anavajjatā ca phāsuvihāro
cā'ti

\begin{english}
  Whatever alms-food I used today without consideration was not for fun not
  for pleasure not for fattening not for beautification only for the
  maintenance and nourishment of this body for keeping it healthy for helping
  with the Holy Life thinking thus `I will allay hunger without overeating so
  that I may continue to live blamelessly and at ease'
\end{english}

Ajja mayā apaccavekkhitvā yaṃ senāsanaṃ paribhuttaṃ taṃ yāvadeva sītassa
paṭighātāya uṇhassa paṭighātāya ḍaṃsa-makasa-vātātapa-siriṃsapa-samphassānaṃ
paṭighātāya yāvadeva utuparissaya vinodanaṃ paṭisallānārāmatthaṃ

\begin{english}
  Whatever lodging I used today without consideration was only to ward off
  cold to ward off heat to ward off the touch of flies mosquitoes wind
  burning and creeping things only to remove the danger from weather and for
  living in seclusion
\end{english}

Ajja mayā apaccavekkhitvā yo gilāna-paccayabhesajja-\\ parikkhāro paribhutto so
yāvadeva uppannānaṃ veyyābādhikānaṃ vedanānaṃ paṭighātāya
abyāpajjha-paramatāyā'ti

\begin{english}
  Whatever medicinal requisite for supporting the sick I used today without
  consideration was only to ward off painful feelings that have arisen for the
  maximum freedom from disease\\
  \suttaRef{MI10}
\end{english}

\section[Reflection on the Off-Putting Qualities]{Reflection on the Off-Putting Qualities of the Requisites}

% Pali title Dhātu-paṭikūla-paccavekkhaṇa-pāṭho

% This seems to be a modern compilation somewhat based on the MN 28 sub-commentary

\begin{leader}
  [Handa mayaṃ dhātu-paṭikūla-\\ paccavekkhaṇa-pāṭhaṃ bhaṇāmase]
\end{leader}

\firstline{Yathā paccayaṃ pavattamānaṃ dhātu-mattam}

[Yathā paccayaṃ] pavattamānaṃ dhātu-mattam-ev'etaṃ

\packedtrline{Composed of only elements according to causes and conditions}

Yad idaṃ cīvaraṃ tad upabhuñjako ca puggalo

\packedtrline{Are these robes and so is the person wearing them}

Dhātu-mattako nissatto nijjīvo suñño

\packedtrline{Merely elements not a being without a soul\\ and empty of self}

Sabbāni pana imāni cīvarāni ajigucchanīyāni

\packedtrline{None of these robes are innately repulsive}

Imaṃ pūti-kāyaṃ patvā ativiya jigucchanīyāni jāyanti

\packedtrline{But touching this unclean body they become disgusting~indeed}

Yathā paccayaṃ pavattamānaṃ dhātu-mattam-ev'etaṃ

\packedtrline{Composed of only elements according to causes and conditions}

Yad idaṃ piṇḍapāto tad upabhuñjako ca puggalo

\packedtrline{Is this almsfood and so is the person eating it}

Dhātu-mattako nissatto nijjīvo suñño

\packedtrline{Merely elements not a being without a soul\\ and empty of self}

Sabbo panāyaṃ piṇḍapāto ajigucchanīyo

\packedtrline{None of this almsfood is innately repulsive}

Imaṃ pūti-kāyaṃ patvā ativiya jigucchanīyo jāyati

\packedtrline{But touching this unclean body it becomes disgusting~indeed}

Yathā paccayaṃ pavattamānaṃ dhātu-mattam-ev'etaṃ

\packedtrline{Composed of only elements according to causes and conditions}

Yad idaṃ senāsanaṃ tad upabhuñjako ca puggalo

\packedtrline{Is this dwelling and so is the person using it}

Dhātu-mattako nissatto nijjīvo suñño

\packedtrline{Merely elements not a being without a soul\\ and empty of self}

Sabbāni pana imāni senāsanāni ajigucchanīyāni

\packedtrline{None of these dwellings are innately repulsive}

Imaṃ pūti-kāyaṃ patvā ativiya jigucchanīyāni jāyanti

\packedtrline{But touching this unclean body they become disgusting~indeed}

Yathā paccayaṃ pavattamānaṃ dhātu-mattam-ev'etaṃ

\packedtrline{Composed of only elements according to causes and conditions}

Yad idaṃ gilāna-paccaya-bhesajja-parikkhāro tad upabhuñjako ca puggalo

\packedtrline{Is this medicinal requisite and so is the person that takes it}

Dhātu-mattako nissatto nijjīvo suñño

\packedtrline{Merely elements not a being without a soul\\ and empty of self}

Sabbo panāyaṃ gilāna-paccaya-bhesajja-parikkhāro ajigucchanīyo

\packedtrline{None of this medicinal requisite is innately repulsive}

Imaṃ pūti-kāyaṃ patvā ativiya jigucchanīyo jāyati

\packedtrline{But touching this unclean body it becomes disgusting~indeed}

\section{Mettāpharaṇa}

\begin{leader}
  [Handa mayam mettāpharaṇaṃ karomase]
\end{leader}

\firstline{Ahaṃ sukhito homi niddukkho homi}

[Ahaṃ sukhito homi] niddukkho homi avero homi abyāpajjho homi anīgho homi
sukhī attānaṃ pariharāmi

Sabbe sattā sukhitā hontu sabbe sattā averā hontu sabbe sattā abyāpajjhā
hontu sabbe sattā anīghā hontu sabbe sattā sukhī attānaṃ pariharantu

Sabbe sattā sabbadukkhā pamuccantu

Sabbe sattā laddha-sampattito mā vigacchantu

Sabbe sattā kammassakā kammadāyādā kammayonī kammabandhū kammapaṭisaraṇā
yaṃ kammaṃ karissanti kalyāṇaṃ vā pāpakaṃ vā tassa dāyādā bhavissanti

\suttaRef{MI288 AV88}

\section[The Unconditioned]{Reflection on the Unconditioned}

\begin{leader}
  [Handa mayaṃ nibbāna-sutta-pāṭhaṃ bhaṇāmase]
\end{leader}

\firstline{Atthi bhikkhave ajātaṃ abhūtaṃ akataṃ}

Atthi bhikkhave ajātaṃ abhūtaṃ akataṃ asaṅkhataṃ

\begin{english}
  There is an Unborn Unoriginated Uncreated and Unformed
\end{english}

No cetaṃ bhikkhave abhavissa ajātaṃ abhūtaṃ akataṃ asaṅkhataṃ

\begin{english}
  If there was not this Unborn this Unoriginated this Uncreated this~Unformed
\end{english}

Na yidaṃ jātassa bhūtassa katassa saṅkhatassa nissaraṇaṃ paññāyetha

\begin{english}
  Freedom from the world of the born the originated the created the formed would not be possible
\end{english}

Yasmā ca kho bhikkhave atthi ajātaṃ abhūtaṃ akataṃ asaṅkhataṃ

\begin{english}
  But since there is an Unborn Unoriginated Uncreated and Unformed
\end{english}

Tasmā jātassa bhūtassa katassa saṅkhatassa nissaraṇaṃ paññāyati

\begin{english}
  Therefore is freedom possible from the world of the born the originated the created and the formed \suttaRef{Ud83}
\end{english}

\section{Reflection on the Thirty-Two Parts}

\begin{leader}
  [Handa mayaṃ dvattiṃsākāra-pāṭhaṃ bhaṇāmase]
\end{leader}

\firstline{Ayaṃ kho me kāyo uddhaṃ pādatalā}

[Ayaṃ kho] me kāyo uddhaṃ pādatalā adho kesamatthakā\\
tacapariyanto pūro nānappakārassa asucino

\begin{english}
  This which is my body from the soles of the feet up and down from the crown of the head is a sealed bag of skin filled with unattractive things
\end{english}

Atthi imasmiṃ kāye

\begin{english}
  In this body there are
\end{english}

{\centering
  \setArrayStretch{1}

  \begin{tabular}{ r l }
    kesā            & \tr{hair of the head} \\
    lomā            & \tr{hair of the body} \\
    nakhā           & \tr{nails} \\
    dantā           & \tr{teeth} \\
    taco            & \tr{skin} \\
  \end{tabular}

  \begin{tabular}{ r l }
    maṃsaṃ          & \tr{flesh} \\
    nahārū          & \tr{sinews} \\
    aṭṭhī           & \tr{bones} \\
    aṭṭhimiñjaṃ     & \tr{bone marrow} \\
    vakkaṃ          & \tr{kidneys} \\
    hadayaṃ         & \tr{heart} \\
    yakanaṃ         & \tr{liver} \\
    kilomakaṃ       & \tr{membranes} \\
    pihakaṃ         & \tr{spleen} \\
    papphāsaṃ       & \tr{lungs} \\
    antaṃ           & \tr{bowels} \\
    antaguṇaṃ       & \tr{entrails} \\
    udariyaṃ        & \tr{undigested food} \\
    karīsaṃ         & \tr{excrement} \\
    pittaṃ          & \tr{bile} \\
    semhaṃ          & \tr{phlegm} \\
    pubbo           & \tr{pus} \\
    lohitaṃ         & \tr{blood} \\
    sedo            & \tr{sweat} \\
    medo            & \tr{fat} \\
    assu            & \tr{tears} \\
    vasā            & \tr{grease} \\
    kheḷo           & \tr{spittle} \\
    siṅghāṇikā      & \tr{mucus} \\
    lasikā          & \tr{oil of the joints} \\
    muttaṃ          & \tr{urine} \\
    matthaluṅgan'ti & \tr{brain} \\
  \end{tabular}

  \restoreArrayStretch
}

Evam-ayaṃ me kāyo uddhaṃ pādatalā adho kesamatthakā\\
tacapariyanto pūro nānappakārassa asucino

\begin{english}
  This then which is my body from the soles of the feet up and down from the crown of the head is a sealed bag of skin filled with unattractive things 
\end{english}

\suttaRef{DN 22}

\clearpage

\section{Recollection of Impermanence}
\paliTitle{Aniccānussati}

\begin{leader}
  [Handa mayaṃ aniccānussati-pāṭhaṃ bhaṇāmase]
\end{leader}

Sabbe saṅkhārā anicca

\begin{english}
  All conditioned things are impermanent
\end{english}

Sabbe saṅkhārā dukkhā

\begin{english}
  All conditioned things are dukkha
\end{english}

Sabbe dhammā anattā

\begin{english}
  All things are not-self
\end{english}

\suttaRef{Dhp 277-279}

Addhuvaṁ jīvitaṁ

\begin{english}
  Life is not for sure
\end{english}

Dhuvaṁ maraṇaṁ

\begin{english}
  Death is for sure
\end{english}

Avassaṁ mayā maritabbaṁ

\begin{english}
  It is inevitable that I’ll die
\end{english}

Maraṇa-pariyosānaṁ me jīvitaṁ

\begin{english}
  Death is the culmination of my life
\end{english}

Jīvitaṁ me aniyataṁ

\begin{english}
  My life is uncertain
\end{english}

Maraṇaṁ me niyataṁ

\begin{english}
  My death is certain
\end{english}

\suttaRef{Dhp A}

Vata

\begin{english}
  Indeed
\end{english}

Ayaṁ kāyo

\begin{english}
  This body
\end{english}

Aciraṁ

\begin{english}
  Will soon
\end{english}

Apeta-viññāṇo

\begin{english}
  Be void of consciousness
\end{english}

Chuḍḍho

\begin{english}
  And cast away
\end{english}

Adhisessati

\begin{english}
  It will lie
\end{english}

Paṭhaviṁ

\begin{english}
  On the ground
\end{english}

Kaliṅgaraṁ iva

\begin{english}
  Just like a rotten log
\end{english}

Niratthaṁ

\begin{english}
  Useless
\end{english}

\suttaRef{Dhp 41}

Aniccā vata saṅkhārā

\begin{english}
  Indeed  ̓  conditioned things cannot last
\end{english}

Uppāda-vaya-dhammino

\begin{english}
  Their nature is to rise and ceasei
\end{english}

Uppajjitvā nirujjhanti

\begin{english}
  Having arisen things must cease
\end{english}

Tesaṁ vūpasamo sukho

\begin{english}
  Their stilling is true happiness
\end{english}

\suttaRef{???}

\clearpage
