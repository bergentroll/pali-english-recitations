\section{Āditta-pariyāya-sutta}

[Evaṁ me sutaṁ] ekaṁ samayaṁ bhagavā gayāyaṁ viharati gayāsīse saddhiṁ bhikkhusahassena. Tatra kho bhagavā bhikkhū āmantesi:

Sabbaṁ bhikkhave ādittaṁ!

Kiñca bhikkhave sabbaṁ ādittaṁ?

Cakkhuṁ bhikkhave ādittaṁ  ̓  rūpā ādittā  ̓  cakkhuviññāṇaṁ ādittaṁ  ̓  cakkhusamphasso āditto  ̓  yampidaṁ cakkhusamphassapaccayā uppajjati vedayitaṁ sukhaṁ vā dukkhaṁ vā adukkhamasukhaṁ vā tam pi ādittaṁ.

Kena ādittaṁ?

Ādittaṁ rāgagginā dosagginā mohagginā  ̓  ādittaṁ jātiyā jarāya maraṇena  ̓  sokehi paridevehi dukkhehi domanassehi upāyāsehi ādittan’ti vadāmi.

Sotaṁ ādittaṁ  ̓  saddā ādittā  ̓  sotaviññāṇaṁ ādittaṁ  ̓  sotasamphasso āditto  ̓  yampidaṁ sotasamphassapaccayā uppajjati vedayitaṁ sukhaṁ vā dukkhaṁ vā adukkhamasukhaṁ vā tam pi ādittaṁ.

Kena ādittaṁ?

Ādittaṁ rāgagginā dosagginā mohagginā  ̓  ādittaṁ jātiyā jarāya maraṇena  ̓  sokehi paridevehi dukkhehi domanassehi upāyāsehi ādittan’ti vadāmi.

Ghānaṁ ādittaṁ  ̓  gandhā ādittā  ̓  ghānaviññāṇaṁ ādittaṁ  ̓  ghānasamphasso āditto  ̓  yampidaṁ ghānasamphassapaccayā uppajjati vedayitaṁ sukhaṁ vā dukkhaṁ vā adukkhamasukhaṁ vā tam pi ādittaṁ.

Kena ādittaṁ?

Ādittaṁ rāgagginā dosagginā mohagginā  ̓  ādittaṁ jātiyā jarāya maraṇena  ̓  sokehi paridevehi dukkhehi domanassehi upāyāsehi ādittan’ti vadāmi.

Jivhā ādittā  ̓  rasā ādittā  ̓  jivhāviññāṇaṁ ādittaṁ  ̓  jivhāsamphasso āditto  ̓  yampidaṁ jivhāsamphassapaccayā uppajjati vedayitaṁ sukhaṁ vā dukkhaṁ vā adukkhamasukhaṁ vā tam pi ādittaṁ.

Kena ādittaṁ?

Ādittaṁ rāgagginā dosagginā mohagginā  ̓  ādittaṁ jātiyā jarāya maraṇena  ̓  sokehi paridevehi dukkhehi domanassehi upāyāsehi ādittan’ti vadāmi.

Kāyo āditto  ̓  phoṭṭhabbā ādittā  ̓  kāyaviññāṇaṁ ādittaṁ  ̓  kāyasamphasso āditto  ̓  yampidaṁ kāyasamphassapaccayā uppajjati vedayitaṁ sukhaṁ vā dukkhaṁ vā adukkhamasukhaṁ vā tam pi ādittaṁ.

Kena ādittaṁ?

Ādittaṁ rāgagginā dosagginā mohagginā  ̓  ādittaṁ jātiyā jarāya maraṇena  ̓  sokehi paridevehi dukkhehi domanassehi upāyāsehi ādittan’ti vadāmi.

Mano āditto  ̓  dhammā ādittā  ̓  manoviññāṇaṁ ādittaṁ  ̓  manosamphasso āditto  ̓  yampidaṁ manosamphassapaccayā uppajjati vedayitaṁ sukhaṁ vā dukkhaṁ vā adukkhamasukhaṁ vā tam pi ādittaṁ.

Kena ādittaṁ?

Ādittaṁ rāgagginā dosagginā mohagginā  ̓  ādittaṁ jātiyā jarāya maraṇena  ̓  sokehi paridevehi dukkhehi domanassehi upāyāsehi ādittan’ti vadāmi.

[Evaṁ passaṁ bhikkhave] sutvā ariyasāvako cakkhusmimpi nibbindati  ̓  rūpesu pi nibbindati  ̓  cakkhuviññāṇe pi nibbindati  ̓  cakkhusamphassepi nibbindati  ̓  yampidaṁ cakkhusamphassapaccayā uppajjati vedayitaṁ sukhaṁ vā dukkhaṁ vā adukkhamasukhaṁ vā tasmimpi nibbindati.

Sotasmimpi nibbindati  ̓  saddesu pi nibbindati  ̓  sotaviññāṇe pi nibbindati  ̓  sotasamphassepi nibbindati  ̓  yampidaṁ sotasamphassapaccayā uppajjati vedayitaṁ sukhaṁ vā dukkhaṁ vā adukkhamasukhaṁ vā tasmimpi nibbindati.

Ghānasmimpi nibbindati  ̓  gandhesu pi nibbindati  ̓  ghānaviññāṇe pi nibbindati  ̓  ghānasamphassepi nibbindati  ̓  yampidaṁ ghānasamphassapaccayā uppajjati vedayitaṁ sukhaṁ vā dukkhaṁ vā adukkhamasukhaṁ vā tasmimpi nibbindati.

Jivhāya pi nibbindati  ̓  rasesu pi nibbindati  ̓  jivhāviññāṇe pi nibbindati  ̓  jivhāsamphassepi nibbindati  ̓  yampidaṁ jivhāsamphassapaccayā uppajjati vedayitaṁ sukhaṁ vā dukkhaṁ vā adukkhamasukhaṁ vā tasmimpi nibbindati.

Kāyasmimpi nibbindati  ̓  phoṭṭhabbesu pi nibbindati  ̓  kāyaviññāṇe pi nibbindati  ̓  kāyasamphassepi nibbindati  ̓  yampidaṁ kāyasamphassapaccayā uppajjati vedayitaṁ sukhaṁ vā dukkhaṁ vā adukkhamasukhaṁ vā tasmimpi nibbindati.

Manasmimpi nibbindati  ̓  dhammesu pi nibbindati  ̓  manoviññāṇe pi nibbindati  ̓  manosamphasse pi nibbindati  ̓  yampidaṁ manosamphassapaccayā uppajjati vedayitaṁ sukhaṁ vā dukkhaṁ vā adukkhamasukhaṁ vā tasmimpi nibbindati.

Nibbindaṁ virajjati  ̓  virāgā vimuccati  ̓  vimuttasmiṁ ‘Vimuttam’ iti ñāṇaṁ hoti:

‘Khīṇā jāti vusitaṁ brahmacariyaṁ kataṁ karaṇīyaṁ nāparaṁ itthattāyā’ti pajānātī’ti.

Idamavoca bhagavā. Attamanā te bhikkhū bhagavato bhāsitaṁ abhinanduṁ. Imasmiñca pana veyyākaraṇasmiṁ bhaññamāne tassa bhikkhusahassassa anupādāya āsavehi cittāni vimucciṁsū’ti.

\suttaRef{SN 35.28}
