\section{The Characteristic of Not-Self}

\begin{leader}
  \soloinstr{Solo introduction}

  All beings should take pains to understand the characteristic of
  not-self, which provides matchless deliverance from self-view and
  self-perception, as taught by the supreme Buddha.

  This teaching is given so that those who meditate on experienceable
  realities may arrive at perfect comprehension;

  It is for the development of perfect understanding of these phenomena,
  and for the investigation of all defiled mind-moments.

  The consequence of this practice is total deliverance, so, desirous of
  bringing this teaching forth with its great benefit, let us now recite
  this Sutta.

\end{leader}

Thus have I heard.

At one time the Blessed One was dwelling at Benares in the deer park.
There he addressed the group of five bhikkhus:

‘Form, bhikkhus, is not-self. If, bhikkhus, form were self, then form
would not lead to affliction, and one might be able to say in regard to
form, “Let my form be thus, let my form not be thus.” But since,
bhikkhus, form is not-self, form therefore leads to affliction, and one
is not able to say in regard to form, “Let my form be thus, let my form
not be thus.”

‘Feeling is not-self. If, bhikkhus, feeling were self, feeling would
not lead to affliction, and one might be able to say in regard to
feeling, “Let my feeling be thus, let my feeling not be thus.” But
since, bhikkhus, feeling is not-self, feeling therefore leads to
affliction, and one is not able to say in regard to feeling, “Let my
feeling be thus, let my feeling not be thus.”

‘Perception is not-self. If, bhikkhus, perception were self, perception
would not lead to affliction, and one might be able to say in regard to
perception, “Let my perception be thus, let my perception not be thus.”
But since, bhikkhus, perception is not-self, perception therefore leads
to affliction, and one is not able to say in regard to perception, “Let
my perception be thus, let my perception not be thus.”

‘Mental formations are not-self. If, bhikkhus, mental formations were
self, mental formations would not lead to affliction, and one might be
able to say in regard to mental formations, “Let my mental formations be
thus, let my mental formations not be thus.” But since, bhikkhus, mental
formations are not-self, mental formations therefore lead to affliction,
and one is not able to say in regard to mental formations, “Let my
mental formations be thus, let my mental formations not be thus.”

‘Consciousness is not-self. If, bhikkhus, consciousness were self,
consciousness would not lead to affliction, and one might be able to say
in regard to consciousness, “Let my consciousness be thus, let my
consciousness not be thus.” But since, bhikkhus, consciousness is
not-self, consciousness therefore leads to affliction, and one is not
able to say in regard to consciousness, “Let my consciousness be thus,
let my consciousness not be thus.”

‘What do you think about this, bhikkhus? Is form permanent or
impermanent?’

‘Impermanent, Venerable Sir.’

‘But is that which is impermanent painful or pleasurable?’

‘Painful, Venerable Sir.’

‘But is it fit to consider that which is impermanent, painful, of a
nature to change, as “This is mine, I am this, this is my self”?’

‘It is not, Venerable Sir.’

‘What do you think about this, bhikkhus? Is feeling permanent or
impermanent?’

‘Impermanent, Venerable Sir.’

‘But is that which is impermanent painful or pleasurable?’

‘Painful, Venerable Sir.’

‘But is it fit to consider that which is impermanent, painful, of a
nature to change, as “This is mine, I am this, this is my self”?’

‘It is not, Venerable Sir.’

‘What do you think about this, bhikkhus? Is perception permanent or
impermanent?’

‘Impermanent, Venerable Sir.’

‘But is that which is impermanent painful or pleasurable?’

‘Painful, Venerable Sir.’

‘But is it fit to consider that which is impermanent, painful, of a
nature to change, as “This is mine, I am this, this is my self”?’

‘It is not, Venerable Sir.’

‘What do you think about this, bhikkhus? Are mental formations
permanent or impermanent?’

‘Impermanent, Venerable Sir.’

‘But is that which is impermanent painful or pleasurable?’

‘Painful, Venerable Sir.’

‘But is it fit to consider that which is impermanent, painful, of a
nature to change, as “This is mine, I am this, this is my self”?’

‘It is not, Venerable Sir.’

‘What do you think about this, bhikkhus? Is consciousness permanent or
impermanent?’

‘Impermanent, Venerable Sir.’

‘But is that which is impermanent painful or pleasurable?’

‘Painful, Venerable Sir.’

‘But is it fit to consider that which is impermanent, painful, of a
nature to change, as “This is mine, I am this, this is my self”?’

‘It is not, Venerable Sir.’

‘Wherefore, bhikkhus, whatever form there is, past, future, present,
internal or external, gross or subtle, inferior or superior, whether it
is far or near, all form should, by means of right wisdom, be seen as it
really is, thus: “This is not mine, I am not this, this is not my self.”

‘Whatever feeling there is, past, future, present, internal or
external, gross or subtle, inferior or superior, whether it is far or
near, all feeling should, by means of right wisdom, be seen as it really
is, thus: “This is not mine, I am not this, this is not my self.”

‘Whatever perception there is, past, future, present, internal or
external, gross or subtle, inferior or superior, whether it is far or
near, all perception should, by means of right wisdom, be seen as it really
is, thus: “This is not mine, I am not this, this is not my self.”

‘Whatever mental formations there are, past, future, present, internal
or external, gross or subtle, inferior or superior, whether they are far
or near, all mental formations should, by means of right wisdom, be seen
as they really are, thus: “This is not mine, I am not this, this is not
my self.”

‘Whatever consciousness there is, past, future, present, internal or
external, gross or subtle, inferior or superior, whether far or near,
all consciousness should, by means of right wisdom, be seen as it really
is, thus: “This is not mine, I am not this, this is not my self.”

‘Seeing in this way, bhikkhus, the wise noble disciple becomes
disenchanted with form, becomes disenchanted with feeling, becomes
disenchanted with perception, becomes disenchanted with mental
formations, becomes disenchanted with consciousness. Becoming
disenchanted, their passions fade away; with the fading of passion the
heart is liberated; with liberation there comes the knowledge: “It is
liberated,” and they know: “Destroyed is birth, the Holy Life has been
lived out, done is what had to be done, there is no more coming into any
state of being.”\thinspace ’

Thus spoke the Blessed One. Delighted, the group of five bhikkhus
rejoiced in what the Blessed One had said. Moreover, while this discourse was
being delivered, the minds of the five bhikkhus were freed from the
defilements, through clinging no more.

Thus ends the discourse on The Characteristic of Not-self.

