\section{The Discourse on the Characteristic of Not-Self}

[Thus have I heard] on one occasion the Blessed One was dwelling at Benares  ̓  in the Deer Park at Isipatana. There he addressed the bhikkhus of the group of five: “Bhikkhus” – “Venerable Sir” they replied. The Blessed One said this:i

Bhikkhus form is not-self. If form were self  ̓  then form would not lead to affliction  ̓  and one could command to form: “Let my form be thus  ̓  let my form not be thus.”

But since form is not-self  ̓  it leads to affliction  ̓  and none can command to form: “Let my form be thus  ̓  let my form not be thus.”

Feeling is not-self. If feeling were self  ̓  then feeling would not lead to affliction  ̓  and one could command to feeling: “Let my feeling be thus  ̓  let my feeling not be thus.”

But since feeling is not-self  ̓  it leads to affliction  ̓  and none can command to feeling: “Let my feeling be thus  ̓  let my feeling not be thus.”

Perception is not-self. If perception were self  ̓  then perception would not lead to affliction  ̓  and one could command to perception: “Let my perception be thus  ̓  let my perception not be thus.”

But since perception is not-self  ̓  it leads to affliction  ̓  and none can command to perception: “Let my perception be thus  ̓  let my perception not be thus.”

Volitional formations are not-self. If volitional formations were self  ̓  then volitional formations would not lead to affliction  ̓  and one could command to volitional formations: “Let my volitional formations be thus  ̓  let my volitional formations not be thus.”

But since volitional formations are not-self  ̓  they lead to affliction  ̓  and none can command to volitional formations: “Let my volitional formations be thus  ̓  let my volitional formations not be thus.”

Consciousness is not-self. If consciousness were self  ̓  then consciousness would not lead to affliction  ̓  and one could command to consciousness: “Let my consciousness be thus  ̓  let my consciousness not be thus.”

But since consciousness is not-self  ̓  it leads to affliction  ̓  and none can command to consciousness: “Let my consciousness be thus  ̓  let my consciousness not be thus.”

[Bhikkhus what do you think:] “Is form permanent or impermanent?”

“Impermanent venerable Sir.”

“Is what is impermanent satisfactory or unsatisfactory?” “Unsatisfactory venerable Sir.”

“Is what is impermanent unsatisfactory and subject to change fit to be regarded thus: 'This is mine  ̓  this I am  ̓  this is my self?'”

“No venerable Sir.”

Bhikkhus what do you think: “Is feeling permanent or impermanent?”

“Impermanent venerable Sir.”

“Is what is impermanent satisfactory or unsatisfactory?” “Unsatisfactory venerable Sir.”

“Is what is impermanent unsatisfactory and subject to change fit to be regarded thus: 'This is mine  ̓  this I am  ̓  this is my self?'”

“No venerable Sir.”

Bhikkhus what do you think: “Is perception permanent or impermanent?” “Impermanent venerable Sir.”

“Is what is impermanent satisfactory or unsatisfactory?” “Unsatisfactory venerable Sir.”

“Is what is impermanent unsatisfactory and subject to change fit to be regarded thus: 'This is mine  ̓  this I am  ̓  this is my self?'

“No venerable Sir.”

Bhikkhus what do you think: “Are volitional formations permanent or impermanent?”

“Impermanent venerable Sir.”

“Is what is impermanent satisfactory or unsatisfactory?” “Unsatisfactory venerable Sir.”

“Is what is impermanent unsatisfactory and subject to change fit to be regarded thus: 'This is mine  ̓  this I am  ̓  this is my self?'”

“No venerable Sir.”

Bhikkhus what do you think: “Is consciousness permanent or impermanent?”

“Impermanent venerable Sir.”

“Is what is impermanent satisfactory or unsatisfactory?” “Unsatisfactory venerable Sir.”

“Is what is impermanent unsatisfactory and subject to change fit to be regarded thus: 'This is mine  ̓  this I am  ̓  this is my self?'

“No venerable Sir.”

[Therefore bhikkhus] any kind of form whatsoever  ̓  whether past future or present  ̓  internal or external  ̓  gross or subtle  ̓  inferior or superior  ̓  far or near  ̓  must be seen with right wisdom as it really is: “This is not mine  ̓  this I am not  ̓  this is not my self.”

Any kind of feeling whatsoever  ̓  whether past future or present  ̓  internal or external  ̓  gross or subtle  ̓  inferior or superior  ̓  far or near  ̓  must be seen with right wisdom as it really is: “This is not mine  ̓  this I am not  ̓  this is not my self.”

Any kind of perception whatsoever  ̓  whether past future or present  ̓  internal or external  ̓  gross or subtle  ̓  inferior or superior  ̓  far or near  ̓  must be seen with right wisdom as it really is: “This is not mine  ̓  this I am not  ̓  this is not my self.”

Any kind of volitional formation whatsoever  ̓  whether past future or present  ̓  internal or external  ̓  gross or subtle  ̓  inferior or superior  ̓  far or near  ̓  must be seen with right wisdom as it really is: “This is not mine  ̓  this I am not  ̓  this is not my self.”

Any kind of consciousness whatsoever  ̓  whether past future or present  ̓  internal or external  ̓  gross or subtle  ̓  inferior or superior  ̓  far or near  ̓  must be seen with right wisdom as it really is: “This is not mine  ̓  this I am not  ̓  this is not my self.”

[Bhikkhus when a noble discipleii]  ̓  who has heard the teachingiii sees thus  ̓  he becomes disenchanted with form  ̓  becomes disenchanted with feeling  ̓  becomes disenchanted with perception  ̓  becomes disenchanted with volitional formations  ̓  becomes disenchanted with consciousness.

When he is disenchanted passion fades away. With the fading of passion he is liberated. When liberated there is knowledge that he is liberated. He understands: “Birth is exhausted  ̓  the holy life is fulfilled  ̓  what has to be done is done  ̓  there is nothing else to do for the sake of liberation.”iv

That is what the Blessed One said. The bhikkhus of the group of five were glad and they approved of his words. Now during this utterance  ̓  the hearts of the bhikkhus of the group of five  ̓  were liberated from the taints through the cessation of clinging.

\suttaRef{SN 22.59}
