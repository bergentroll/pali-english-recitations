\chapterOpeningPage{anumodana.pdf}

\chapter{Anumodanā}

\section{Just as Rivers}

\paliTitle{Yathā vāri-vahā pūrā}

[Yathā vāri-vahā pūrā]

\begin{cprenglish}
  Just as rivers full of water
\end{cprenglish}

Paripūrenti sāgaraṁ

\begin{cprenglish}
  Entirely fill up the sea
\end{cprenglish}

Evam-eva ito dinnaṁ petānaṁ upakappati

\begin{cprenglish}
  Likewise what’s been given here
\end{cprenglish}

Petānaṁ upakappati

\begin{cprenglish}
  Benefits the departed ones
\end{cprenglish}

Icchitaṁ patthitaṁ tumhaṁ

\begin{cprenglish}
  May all your hopes and all your longings
\end{cprenglish}

Khippam-eva samijjhatu

\begin{cprenglish}
  Come true in no long time
\end{cprenglish}

Sabbe pūrentu saṅkappā

\begin{cprenglish}
  May all your wishes be fulfilled
\end{cprenglish}

Cando paṇṇaraso yathā

\begin{cprenglish}
  Like on the fifteenth day the moon
\end{cprenglish}

Maṇi jotiraso yathā

\begin{cprenglish}
  Or like a bright and shining gem\\
  \suttaRef{[DhpA.I.198]}
\end{cprenglish}

Sabb'ītiyo vivajjantu

\begin{cprenglish}
  May all misfortunes be avoided
\end{cprenglish}

Sabba-rogo vinassatu

\begin{cprenglish}
  May all illness be dispelled
\end{cprenglish}

Mā te bhavatv-antarāyo

\begin{cprenglish}
  May you never meet with dangers
\end{cprenglish}

Sukhī dīgh'āyuko bhava

\begin{cprenglish}
  May you be happy and live long
\end{cprenglish}

Abhivādana-sīlissa

\begin{cprenglish}
  For one who often pays homage
\end{cprenglish}

Niccaṁ vuḍḍhāpacāyino

\begin{cprenglish}
  And always respects elders
\end{cprenglish}

Cattāro dhammā vaḍḍhanti

\begin{cprenglish}
  Four things increase
\end{cprenglish}

Āyu vaṇṇo sukhaṁ balaṁ

\begin{cprenglish}
  Long-life beauty  ̓  happiness and strength
\end{cprenglish}

Bhavatu sabba-maṅgalaṁ

\begin{cprenglish}
  May every blessing come to be
\end{cprenglish}

Rakkhantu sabba-devatā

\begin{cprenglish}
  And all good spirits guard you well
\end{cprenglish}

Sabba-buddhānubhāvena

\begin{cprenglish}
  Through the power of all Buddhas
\end{cprenglish}

Sadā sotthī bhavantu te

\begin{cprenglish}
  May you always be at ease
\end{cprenglish}

Bhavatu sabba-maṅgalaṁ

\begin{cprenglish}
  May every blessing come to be
\end{cprenglish}

Rakkhantu sabba-devatā

\begin{cprenglish}
  And all good spirits guard you well
\end{cprenglish}

Sabba-dhammānubhāvena

\begin{cprenglish}
  Through the power of all Dhammas
\end{cprenglish}

Sadā sotthī bhavantu te

\begin{cprenglish}
  May you always be at ease
\end{cprenglish}

Bhavatu sabba-maṅgalaṁ

\begin{cprenglish}
  May every blessing come to be
\end{cprenglish}

Rakkhantu sabba-devatā

\begin{cprenglish}
  And all good spirits guard you well
\end{cprenglish}

Sabba-saṅghānubhāvena

\begin{cprenglish}
  Through the power of all Saṅghas
\end{cprenglish}

Sadā sotthī bhavantu te

\begin{cprenglish}
  May you always be at ease\\
\end{cprenglish}

\suttaRef{[Khp 1.7 / Dhp 109 / Trad]}

\clearpage

\section{Yathā vāri-vahā pūrā}

\begin{twochants}
  Yathā vāri-vahā pūrā – Paripūrenti sāgaraṁ\\
  Evam-eva ito dinnaṁ – Petānaṁ upakappati\\
\end{twochants}

\begin{english}
  Just as rivers full of water\\
  Entirely fill up the sea\\
  Likewise what’s been given here\\
  Benefits the departed onesi
\end{english}

\suttaRef{[Khp 1.7]}

\begin{twochants}
  Icchitaṁ patthitaṁ tumhaṁ\\
  Khippam-eva samijjhatu\\
  Sabbe pūrentu saṅkappā\\
  Cando paṇṇa-raso yathā\\
  Maṇi joti-raso yathāi
\end{twochants}

\begin{english}
  May all your hopes and all your longings\\
  Come true in no long time\\
  May all your wishes be fulfilled\\
  Like on the fifteenth day the moon\\
  Or like a bright and shining gem
\end{english}

\begin{twochants}
  Sabb’ītiyo vivajjantui – Sabba-rogo vinassatu\\
  Mā te bhavatv-antarāyo – Sukhī dīgh’āyuko bhava
\end{twochants}

\begin{english}
  May all misfortunes be avoided\\
  May all illness be dispelled\\
  May you never meet with dangers\\
  May you be happy and live long
\end{english}

\suttaRef{[Khp A]}

\begin{twochants}
  Abhivādana-sīlissa – Niccaṁ vuḍḍhāpacāyino\\
  Cattāro dhammā vaḍḍhanti – Āyu vaṇṇo sukhaṁ balaṁi
\end{twochants}

\begin{english}
  For one who often pays homage\\
  And always respects elders\\
  Four things increase\\
  Long-life beauty  ̓  happiness and strength
\end{english}

\suttaRef{[Dhp 109]}

\section{Ratanattay'ānubhāv'ādi-gāthā}

\begin{twochants}
  Ratanattay'ānubhāvena – Ratanattaya-tejasā\\
  Dukkha-roga-bhayā verā – Sokā sattu c'upaddavā\\
  Anekā antarāyā pi – Vinassantu asesato\\
  Jaya-siddhi dhanaṁ lābhaṁ – Sotthi bhāgyaṁ sukhaṁ balaṁ\\
  Siri āyu ca vaṇṇo ca – Bhogaṁ vuḍḍhī ca yasavā\\
  Sata-vassā ca āyu ca – Jīva-siddhī bhavantu te
\end{twochants}

\begin{english}
  Through the power \& through the radiant energy of the (Triple) Gem,\\
  May suffering, disease, fear, animosity,\\
  Sorrow, adversity, misfortune\\
  Obstacles without number vanish without a trace.\\
  Triumph, success, wealth, \& gain,\\
  Safety, luck, happiness, strength,\\
  Glory, long life, \& beauty, fortune, increase, \& status,\\
  A lifespan of 100 years, and success in your livelihood:\\
  May they be yours.
\end{english}

\suttaRef{[Thai]}

\section{Bhojana-dānānumodanā}

\begin{twochants}
  [Yo yassa bhojanaṁ deti]\\
  So tassa deti pañcapi\\
  Āyuṁ balaṁ sukhaṁ vaṇṇaṁ\\
  Paṭibhānañca pañcamaṁi\\
  Āyu-do bala-do dhīro\\
  Vaṇṇa-do paṭibhāṇa-do\\
  Sukhassa dātā medhāvī\\
  Sukhaṁ so adhigacchati\\
  Āyuṁ datvā balaṁ vaṇṇaṁ\\
  Sukhañ-ca paṭibhāna-kaṁii\\
  Dīgh’āyu yasavā hoti\\
  Yattha yatthūpapajjati\\
  Abhivādanasīlissa\\
  Niccaṁ vuḍḍhāpacāyino\\
  Cattāro dhammā vaḍḍhanti\\
  Āyu vaṇṇo sukhaṁ balaṁ\\
  Padakkhiṇaṁ kāyakammaṁ\\
  Vācākammaṁ padakkhiṇaṁ\\
  Padakkhiṇaṁ manokammaṁ\\
  Paṇīdhi te padakkhiṇe\\
  Padakkhiṇāni katvāna\\
  Labhantatthe padakkhiṇe\\
  Te atthaladdhā sukhitā\\
  Virūḷhā Buddhasāsane\\
  Arogā sukhitā hotha\\
  Saha sabbehi ñātibhī
\end{twochants}

\bigskip

\begin{english}
  [One who gives food to another]\\
  Gives to the other five things too\\
  Long-life, strength, happiness, beauty\\
  And intelligence as the fifth.\\
  The wise life-giver, strength-giver\\
  Beauty-giver, wit-giver\\
  Wise giver of happiness\\
  Attains happiness.\\
  Having given life, strength and beauty\\
  Happiness and wit\\
  One is long-lived and glorious\\
  Wherever one is reborn.\\
  For one who often pays homage\\
  And always respects elders\\
  Four things increase:\\
  Long-life, beauty\\
  Happiness and strength.\\
  Felicitous is bodily kamma\\
  Verbal kamma is felicitous\\
  Felicitous is mental kamma\\
  When aspiring for felicity.\\
  Having done the felicitous\\
  They get felicitous rewards\\
  They are happy who get such rewards\\
  And grow in the Buddhasāsana.\\
  May you all be healthy and happy\\
  Together with all your relatives
\end{english}

\suttaRef{[AN 5.37 / Dhp 109 / AN 3.155]}

\section{Culla-maṅgala-cakka-vāḷa}

[Sabba-buddh'ānubhāvena]\\
Sabba-dhamm'ānubhāvena\\
Sabba-saṅgh'ānubhāvena\\
Buddha-ratanaṁ dhamma-ratanaṁ saṅgha-ratanaṁ\\
Tiṇṇaṁ ratanānaṁ ānubhāvena\\
Catur-āsīti-sahassa-dhammakkhandh'ānubhāvena\\
Piṭakattay'ānubhāvena\\
Jina-sāvak'ānubhāvena\\
Sabbe te rogā sabbe te bhayā sabbe te antarāyā sabbe te upaddavā sabbe te\\
Dunnimittā sabbe te avamaṅgalā vinassantu\\
Āyu-vaḍḍhako dhana-vaḍḍhako siri-vaḍḍhako yasa-vaḍḍhako bala-vaḍḍhako\\
Vaṇṇa-vaḍḍhako sukha-vaḍḍhako hotu sabbadā\\
Dukkha-roga-bhayā verā sokā sattu c'upaddavā\\
Anekā antarāyā pi vinassantu ca tejasā\\
Jaya-siddhi dhanaṁ lābhaṁ\\
Sotthi bhāgyaṁ sukhaṁ balaṁ\\
Siri āyu ca vaṇṇo ca bhogaṁ vuḍḍhī ca yasavā\\
Sata-vassā ca āyū ca jīva-siddhī bhavantu te

\begin{english}
  Through the power of all the Buddhas, the power of all the Dhamma, the power of all the Saṅgha, the gem of the Buddha, the gem of the Dhamma, the gem of the Saṅgha, the power of the Triple Gem: the power of the 84,000 Dhamma aggregates, the power of the Tripitaka, the power of the Victor’s disciples: May all your diseases, all your fears, all your obstacles, all your dangers, all your bad visions, all your bad omens be destroyed. May there always be an increase of long life, wealth, glory, status, strength, beauty and happiness.
\end{english}

\suttaRef{[MJG]}

\section{Aggappasāda-sutta-gāthā}

\begin{twochants}
  Aggato ve pasannānaṁ – Aggaṁ dhammaṁ vijānataṁ\\
  Agge Buddhe pasannānaṁ – Dakkhiṇeyye anuttare\\
\end{twochants}

\begin{english}
  For one with confidence,\\
  Realizing the supreme Dhamma to be supreme,\\
  With confidence in the supreme Buddha,\\
  Unsurpassed in deserving offerings,
\end{english}

\begin{twochants}
  Agge dhamme pasannānaṁ – Virāgūpasame sukhe\\
  Agge saṅghe pasannānaṁ – Puññakkhette anuttare\\
\end{twochants}

\begin{english}
  With confidence in the supreme Dhamma,\\
  The happiness of dispassion \& calm,\\
  With confidence in the supreme Saṅgha,\\
  Unsurpassed as a field of merit,
\end{english}

\begin{twochants}
  Aggasmiṁ dānaṁ dadataṁ – Aggaṁ puññaṁ pavaḍḍhati\\
  Aggaṁ āyu ca vaṇṇo ca – Yaso kitti sukhaṁ balaṁ
\end{twochants}

\begin{english}
  Having given gifts to the supreme,\\
  One develops supreme merit,\\
  Supreme long life \& beauty,\\
  Status, honour, happiness, strength.
\end{english}

\begin{twochants}
  Aggassa dātā medhāvī – Agga-dhamma-samāhito\\
  Deva-bhūto manusso vā – Aggappatto pamodatī'ti
\end{twochants}

\begin{english}
  Having given to the supreme, the intelligent person,\\
  Firm in the supreme Dhamma,\\
  Whether becoming a deva or a human being,\\
  Rejoices, having attained the supreme.
\end{english}

\suttaRef{[AN 5.32]}

\section{Kāla-dāna-sutta-gāthā}

\begin{twochants}
  Kāle dadanti sapaññā – vadaññū vīta-maccharā\\
  Kālena dinnaṁ ariyesu – uju-bhūtesu tādisu\\
  Vippasanna-manā tassa – vipulā hoti dakkhiṇā\\
\end{twochants}

\begin{english}
  Those with discernment, responsive, free from stinginess, give in the proper season.\\
  Having given in the proper season\\
  With hearts inspired by the Noble Ones,\\
  Straightened,\\
  Their offering bears an abundance.
\end{english}

\begin{twochants}
  Ye tattha anumodanti – veyyāvaccaṁ karonti vā\\
  Na tena dakkhiṇā onā – te pi puññassa bhāgino\\
\end{twochants}

\begin{english}
  Those who rejoice in that gift,\\
  Or give assistance,\\
  They too have a share of the merit,\\
  And the offering is not depleted by that.
\end{english}

\begin{twochants}
  Tasmā dade appaṭivāna-citto – yattha dinnaṁ mahapphalaṁ\\
  Puññāni para-lokasmiṁ – patiṭṭhā honti pāṇinan'ti
\end{twochants}

\begin{english}
  Therefore, with an unhesitant mind, one should give\\
  Where the gift bears great fruit.\\
  Merit is what establishes\\
  Living beings in the next life.
\end{english}

\suttaRef{[AN 5.36]}

\section{So attha-laddho}

\begin{twochants}
  So attha-laddho sukhito – Viruḷho buddha-sāsane\\
  Arogo sukhito hohi – Saha sabbehi ñātibhi
\end{twochants}

\begin{english}
  May he gain benefits and happiness\\
  And grow in Buddha’s religion,\\
  Without disease and happy\\
  May he be together with all his relatives.
\end{english}

\begin{twochants}
  Sā attha-laddhā sukhitā – Viruḷhā buddha-sāsane\\
  Arogā sukhitā hohi – Saha sabbehi ñātibhi
\end{twochants}

\begin{english}
  May she gain benefits and happiness\\
  And grow in Buddha’s religion,\\
  Without disease and happy\\
  May she be together with all her relatives.
\end{english}

\begin{twochants}
  Te attha-laddhā sukhitā – Viruḷhā buddha-sāsane\\
  Arogā sukhitā hotha – Saha sabbehi ñātibhi
\end{twochants}

\begin{english}
  May they gain benefits and happiness\\
  And grow in Buddha’s religion,\\
  Without disease and happy\\
  May they be together with all their relatives.
\end{english}

\suttaRef{[AN 3.155]}
