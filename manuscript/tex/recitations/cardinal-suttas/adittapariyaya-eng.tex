\section{The Fire Sermon}

[Thus have I heard] on one occasion the Blessed One was dwelling at Gayā  ̓  at Gayā’s Head  ̓  together with a thousand bhikkhus. There he addressed the bhikkhus:

Bhikkhus all is burning!

And what bhikkhus is the all that is burning?

Bhikkhus the eye is burning  ̓  forms are burning  ̓  eye-consciousness is burning  ̓  eye-contact is burning  ̓  and what is felt as pleasant or painful  ̓  or neither-painful-nor-pleasant  ̓  that arises from eye-contact as its condition  ̓  that too is burning.

Burning with what?

Burning with the fire of lust  ̓  with the fire of hate  ̓  with the fire of delusion.

I say it is burning with birth  ̓  ageing and death  ̓  with sorrows  ̓  with lamentations  ̓  with pains  ̓  with displeasures  ̓  with despairs.

The ear is burning  ̓  sounds are burning  ̓  ear-consciousness is burning  ̓  ear-contact is burning  ̓  and what is felt as pleasant or painful  ̓  or neither-painful-nor-pleasant  ̓  that arises from ear-contact as its condition  ̓  that too is burning.

Burning with what?

Burning with the fire of lust  ̓  with the fire of hate  ̓  with the fire of delusion.

I say it is burning with birth  ̓  ageing and death  ̓  with sorrows  ̓  with lamentations  ̓  with pains  ̓  with displeasures  ̓  with despairs.

The nose is burning  ̓  odours are burning  ̓  nose-consciousness is burning  ̓  nose-contact is burning  ̓  and what is felt as pleasant or painful  ̓  or neither-painful-nor-pleasant  ̓  that arises from nose-contact as its condition  ̓  that too is burning.

Burning with what?

Burning with the fire of lust  ̓  with the fire of hate  ̓  with the fire of delusion.

I say it is burning with birth  ̓  ageing and death  ̓  with sorrows  ̓  with lamentations  ̓  with pains  ̓  with displeasures  ̓  with despairs.

The tongue is burning  ̓  flavours are burning  ̓  tongue-consciousness is burning  ̓  tongue-contact is burning  ̓  and what is felt as pleasant or painful  ̓  or neither-painful-nor-pleasant  ̓  that arises from tongue-contact as its condition  ̓  that too is burning.

Burning with what?

Burning with the fire of lust  ̓  with the fire of hate  ̓  with the fire of delusion.

I say it is burning with birth  ̓  ageing and death  ̓  with sorrows  ̓  with lamentations  ̓  with pains  ̓  with displeasures  ̓  with despairs.

The body is burning  ̓  tangibles are burning  ̓  body-consciousness is burning  ̓  body-contact is burning  ̓  and what is felt as pleasant or painful  ̓  or neither-painful-nor-pleasant  ̓  that arises from body-contact as its condition  ̓  that too is burning.

Burning with what?

Burning with the fire of lust  ̓  with the fire of hate  ̓  with the fire of delusion.

I say it is burning with birth  ̓  ageing and death  ̓  with sorrows  ̓  with lamentations  ̓  with pains  ̓  with displeasures  ̓  with despairs.

The mind is burning  ̓  mind-objects are burning  ̓  mind-consciousness is burning  ̓  mind-contact is burning  ̓  and what is felt as pleasant or painful  ̓  or neither-painful-nor-pleasant  ̓  that arises from mind-contact as its condition  ̓  that too is burning.

Burning with what?

Burning with the fire of lust  ̓  with the fire of hate  ̓  with the fire of delusion.

I say it is burning with birth  ̓  ageing and death  ̓  with sorrows  ̓  with lamentations  ̓  with pains  ̓  with displeasures  ̓  with despairs.

[Bhikkhus when a noble disciple]  ̓  who has heard the teaching sees thus  ̓  he becomes disenchanted with the eye  ̓  becomes disenchanted with forms  ̓  becomes disenchanted with eye-consciousness  ̓  becomes disenchanted with eye-contact  ̓  and what is felt as pleasant or painful  ̓  or neither-painful-nor-pleasant  ̓  that arises from eye-contact as its condition  ̓  with that too he becomes disenchanted.

He becomes disenchanted with the ear  ̓  becomes disenchanted with sounds  ̓  becomes disenchanted with ear-consciousness  ̓  becomes disenchanted with ear-contact  ̓  and what is felt as pleasant or painful  ̓  or neither-painful-nor-pleasant  ̓  that arises from ear-contact as its condition  ̓  with that too be becomes disenchanted.

He becomes disenchanted with the nose  ̓  becomes disenchanted with odours  ̓  becomes disenchanted with nose-consciousness  ̓  becomes disenchanted with nose-contact  ̓  and what is felt as pleasant or painful  ̓  or neither-painful-nor-pleasant  ̓  that arises from nose-contact as its condition  ̓  with that too he becomes disenchanted.

He becomes disenchanted with the tongue  ̓  becomes disenchanted with flavours  ̓  becomes disenchanted with tongue-consciousness  ̓  becomes disenchanted with tongue-contact  ̓  and what is felt as pleasant or painful  ̓  or neither-painful-nor-pleasant  ̓  that arises from tongue-contact as its condition  ̓  with that too he becomes disenchanted.

He becomes disenchanted with the body  ̓  becomes disenchanted with tangibles  ̓  becomes disenchanted with body-consciousness  ̓  becomes disenchanted with body-contact  ̓  and what is felt as pleasant or painful  ̓  or neither-painful-nor-pleasant  ̓  that arises from body-contact as its condition  ̓  with that too he becomes disenchanted.

He becomes disenchanted with the mind  ̓  becomes disenchanted with mind-objects  ̓  becomes disenchanted with mind-consciousness  ̓  becomes disenchanted with mind-contact  ̓  and what is felt as pleasant or painful  ̓  or neither-painful-nor-pleasant  ̓  that arises from mind-contact as its condition  ̓  with that too he becomes disenchanted.

When he is disenchanted passion fades away. With the fading of passion he is liberated. When liberated there is knowledge that he is liberated. He understands: “Birth is exhausted  ̓  the holy life is fulfilled  ̓  what has to be done is done  ̓  there is nothing else to do for the sake of liberation.”

That is what the Blessed One said. The bhikkhus were glad and they approved of his words. Now during this utterance  ̓  the hearts of those thousand bhikkhus  ̓  were liberated from the taints through the cessation of clinging.

\suttaRef{[SN 35.28]}
