\section{The Discourse on the Characteristic of Not-Self}
\label{characteristic-of-not-self}

\begin{leader-english}
  \anglebracketleft\ \hspace{-0.5mm}Thus have I heard \hspace{-0.5mm}\anglebracketright\
\end{leader-english}

\linkdest{endnote108-body}
\begin{english-only-hang}
  On one occasion\makeatletter\hyperlink{endnote108-appendix}\Hy@raisedlink{\hypertarget{endnote108-body}{}{\pagenote{%
        \hypertarget{endnote108-appendix}{\hyperlink{endnote108-body}{The entire original version of the English translation has been replaced.}}}}}\makeatother
\linkdest{endnote109-body}
  the Blessed One was dwelling at Benares \breathmark\ in the Deer Park at Isipatana. There he addressed the bhikkhus of the group of five: ``Bhikkhus'' – ``Venerable Sir'' they replied. The Blessed One said this:\makeatletter\hyperlink{endnote109-appendix}\Hy@raisedlink{\hypertarget{endnote109-body}{}{\pagenote{%
        \hypertarget{endnote109-appendix}{\hyperlink{endnote109-body}{The following passage is absent in the Thai edition of the \textit{Tipiṭaka}: “'Bhikkhus' – 'Venerable Sir', they replied. The Blessed One said this:”}}}}}\makeatother
\end{english-only-hang}

\begin{english-only-hang}
  Bhikkhus form is not-self. If form were self \breathmark\ then form would not lead to affliction \breathmark\ and one could command to form: ``Let my form be thus \breathmark\ let my form not be thus.'' But since form is not-self \breathmark\ it leads to affliction \breathmark\ and none can command to form: ``Let my form be thus \breathmark\ let my form not be thus.''
\end{english-only-hang}

\begin{english-only-hang}
  Feeling is not-self. If feeling were self \breathmark\ then feeling would not lead to affliction \breathmark\ and one could command to feeling: ``Let my feeling be thus \breathmark\ let my feeling not be thus.''\\
  But since feeling is not-self \breathmark\ it leads to affliction \breathmark\ and none can command to feeling: ``Let my feeling be thus \breathmark\ let my feeling not be thus.''
\end{english-only-hang}

\begin{english-only-hang}
  Perception is not-self. If perception were self \breathmark\ then perception would not lead to affliction \breathmark\ and one could command to perception: ``Let my perception be thus \breathmark\ let my perception not be thus.''\\
  But since perception is not-self \breathmark\ it leads to affliction \breathmark\ and none can command to perception: ``Let my perception be thus \breathmark\ let my perception not be thus.''
\end{english-only-hang}

\begin{english-only-hang}
  Volitional formations are not-self. If volitional formations were self \breathmark\ then volitional formations would not lead to affliction \breathmark\ and one could command to volitional formations: ``Let my volitional formations be thus \breathmark\ let my volitional formations not be thus.''\\
  But since volitional formations are not-self \breathmark\ they lead to \mbox{affliction}~\breathmark\ and none can command to volitional formations: ``Let my volitional formations be thus \breathmark\ let my volitional formations not be thus.''
\end{english-only-hang}

\begin{english-only-hang}
  Consciousness is not-self. If consciousness were self \breathmark\ then consciousness would not lead to affliction \breathmark\ and one could command to consciousness: ``Let my consciousness be thus \breathmark\ let my consciousness not be thus.''\\
  But since consciousness is not-self \breathmark\ it leads to affliction \breathmark\ and none can command to consciousness: ``Let my consciousness be thus \breathmark\ let my consciousness not be thus.''
\end{english-only-hang}

\begin{leader-english-only}
  \anglebracketleft\ \hspace{-0.5mm}\textit{Bhikkhus what do you think:} \hspace{-0.5mm}\anglebracketright\
\end{leader-english-only}

\vspace{-0.99em} % TODO too close

\begin{english-only-nohang}
  \begin{english-only-hang}
    ``Is form permanent or impermanent?''
  \end{english-only-hang}
  ``Impermanent venerable Sir.''\\
  ``Is what is impermanent satisfactory or unsatisfactory?''\\
  ``Unsatisfactory venerable Sir.''\\
  \begin{english-hangtogether}
    ``Is what is impermanent unsatisfactory and subject to change fit to be regarded thus: 'This is mine \breathmark\ this I am \breathmark\ this is my self?'
  \end{english-hangtogether}
  ``No venerable Sir.''
\end{english-only-nohang}

\begin{english-only-nohang}
  \begin{english-only-hang}
    Bhikkhus what do you think: ``Is feeling permanent or impermanent?''
  \end{english-only-hang}
  ``Impermanent venerable Sir.''\\
  ``Is what is impermanent satisfactory or unsatisfactory?''\\
  ``Unsatisfactory venerable Sir.''\\
  \begin{english-hangtogether}
    ``Is what is impermanent unsatisfactory and subject to change fit to be regarded thus: 'This is mine \breathmark\ this I am \breathmark\ this is my self?'
  \end{english-hangtogether}
  ``No venerable Sir.''
\end{english-only-nohang}

\begin{english-only-nohang}
  \begin{english-only-hang}
    Bhikkhus what do you think: ``Is perception permanent or impermanent?''
  \end{english-only-hang}
  ``Impermanent venerable Sir.''\\
  ``Is what is impermanent satisfactory or unsatisfactory?''\\
  ``Unsatisfactory venerable Sir.''\\
  \begin{english-hangtogether}
    ``Is what is impermanent unsatisfactory and subject to change fit to be regarded thus: 'This is mine \breathmark\ this I am \breathmark\ this is my self?'
  \end{english-hangtogether}
  ``No venerable Sir.''
\end{english-only-nohang}

\begin{english-only-nohang}
  \begin{english-only-hang}
    Bhikkhus what do you think: ``Are volitional formations permanent or impermanent?''
  \end{english-only-hang}
  ``Impermanent venerable Sir.''\\
  ``Is what is impermanent satisfactory or unsatisfactory?''\\
  ``Unsatisfactory venerable Sir.''
  \begin{english-hangtogether}
    ``Is what is impermanent unsatisfactory and subject to change fit to be regarded thus: 'This is mine \breathmark\ this I am \breathmark\ this is my self?'''
  \end{english-hangtogether}
  ``No venerable Sir.''
\end{english-only-nohang}

\begin{english-only-nohang}
  \begin{english-only-hang}
    Bhikkhus what do you think: ``Is consciousness permanent or impermanent?''
  \end{english-only-hang}
  ``Impermanent venerable Sir.''\\
  ``Is what is impermanent satisfactory or unsatisfactory?''\\
  ``Unsatisfactory venerable Sir.''\\
  \begin{english-hangtogether}
    ``Is what is impermanent unsatisfactory and subject to change fit to be regarded thus: 'This is mine \breathmark\ this I am \breathmark\ this is my self?'
  \end{english-hangtogether}
  ``No venerable Sir.''
\end{english-only-nohang}

\begin{leader-english-only}
  \anglebracketleft\ \hspace{-0.5mm}\textit{Therefore bhikkhus} \hspace{-0.5mm}\anglebracketright\
\end{leader-english-only}

\vspace{-0.99em} % TODO too close

\begin{english-only-hang}
  Any kind of form whatsoever \breathmark\ whether past future or present \breathmark\ internal or external \breathmark\ gross or subtle \breathmark\ inferior or superior \breathmark\ far or near \breathmark\ must be seen with right wisdom as it really is: ``This is not mine \breathmark\ this I am not \breathmark\ this is not my self.''
\end{english-only-hang}

\begin{english-only-hang}
  Any kind of feeling whatsoever \breathmark\ whether past future or present \breathmark\ internal or external \breathmark\ gross or subtle \breathmark\ inferior or superior \breathmark\ far or near \breathmark\ must be seen with right wisdom as it really is: ``This is not mine \breathmark\ this I am not \breathmark\ this is not my self.''
\end{english-only-hang}

\begin{english-only-hang}
  Any kind of perception whatsoever \breathmark\ whether past future or present \breathmark\ internal or external \breathmark\ gross or subtle \breathmark\ inferior or \mbox{superior}~\breathmark\ far or near \breathmark\ must be seen with right wisdom as it really is: ``This is not mine \breathmark\ this I am not \breathmark\ this is not my self.''
\end{english-only-hang}

\begin{english-only-hang}
  Any kind of volitional formation whatsoever \breathmark\ whether past future or present \breathmark\ internal or external \breathmark\ gross or subtle \breathmark\ inferior or superior \breathmark\ far or near \breathmark\ must be seen with right wisdom as it really is: ``This is not mine \breathmark\ this I am not \breathmark\ this is not my self.''
\end{english-only-hang}

\begin{english-only-hang}
  Any kind of consciousness whatsoever \breathmark\ whether past future or present \breathmark\ internal or external \breathmark\ gross or subtle \breathmark\ inferior or \mbox{superior}~\breathmark\ far or near \breathmark\ must be seen with right wisdom as it really is: ``This is not mine \breathmark\ this I am not \breathmark\ this is not my self.''
\end{english-only-hang}

\linkdest{endnote110-body}
\begin{leader-english-only}
  \anglebracketleft\ \hspace{-0.5mm}\textit{Bhikkhus when a noble disciple}\makeatletter\hyperlink{endnote110-appendix}\Hy@raisedlink{\hypertarget{endnote110-body}{}{\pagenote{%
        \hypertarget{endnote110-appendix}{\hyperlink{endnote110-body}{\textit{Ariyasāvaka} can be translated as “noble disciple” or “disciple of the noble one” (\textit{ariyassa}+\textit{sāvaka}=\textit{ariyasāvaka}). I have opted for the first option here because the discourse addresses the group of five monks who all were \textit{Ariyas} already at that time, but it needs to be kept in mind that the term \textit{ariyasāvaka} does not always refer to individuals who have already attained one of the four paths or fruits. This can be seen from MN 27, where a person is referred to as \textit{ariyasāvaka} without reference to any form of awakening-attainment. It is only at the end of the discourse that this \textit{ariyasāvaka} attains awakening.}}}}}\makeatother \thinspace\hspace{-0.5mm}\anglebracketright\
\end{leader-english-only}

\vspace{-0.99em} % TODO too close

\linkdest{endnote111-body}
\begin{english-only-hang}
  Who has heard the teaching\makeatletter\hyperlink{endnote111-appendix}\Hy@raisedlink{\hypertarget{endnote111-body}{}{\pagenote{%
        \hypertarget{endnote111-appendix}{\hyperlink{endnote111-body}{The word “the teaching” is not explicitly expressed in the Pāḷi, but was inserted for comprehension.}}}}}\makeatother
  sees thus \breathmark\ he becomes disenchanted with form \breathmark\ becomes disenchanted with feeling \breathmark\ becomes disenchanted with \mbox{perception}~\breathmark\ becomes disenchanted with volitional formations \breathmark\ becomes disenchanted with consciousness.
\end{english-only-hang}

\linkdest{endnote112-body}
\begin{english-only-hang}
  When he is disenchanted passion fades away. With the fading of passion he is liberated. When liberated there is knowledge that he is liberated. He understands: ``Birth is exhausted \breathmark\ the holy life is fulfilled \breathmark\ what has to be done is done \breathmark\ there is nothing else to do for the sake of liberation.''\makeatletter\hyperlink{endnote112-appendix}\Hy@raisedlink{\hypertarget{endnote112-body}{}{\pagenote{%
        \hypertarget{endnote112-appendix}{\hyperlink{endnote112-body}{\textit{Kataṁ karaṇīyaṁ, nāparaṁ itthattāyā’ti} literally means “What has to be done is done. There is nothing else (to so) for the sake of such a (liberated) state.” For a discussion of this passage see Bhikkhu Bodhi, Middle Length Discourses, \textit{Bhayabheravasutta}, MN 4, footnote 76.}}}}}\makeatother
\end{english-only-hang}

\clearpage

\begin{english-only-hang}
  That is what the Blessed One said. The bhikkhus of the group of five were glad and they approved of his words. Now during this utterance \breathmark\ the hearts of the bhikkhus of the group of five \breathmark\ were liberated from the taints through the cessation of clinging.
\end{english-only-hang}

\suttaRef{[SN 22.59]}

\bottomNav{fire-sermon}
