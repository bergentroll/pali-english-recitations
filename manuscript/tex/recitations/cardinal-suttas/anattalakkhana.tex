\section{Anatta-lakkhaṇa-sutta}
\label{anatta-lakkhana}

[Evaṁ me sutaṁ] ekaṁ samayaṁ bhagavā bārāṇasiyaṁ viharati isipatane migadāye. Tatra kho bhagavā pañcavaggiye bhikkhū āmantesi: “Bhikkhavo”ti \breathmark\ “Bhadante”ti te bhikkhū bhagavato paccassosuṁ. Bhagavā etadavoca.

% % TODO: endnote 114-120 note working in cardinal suttas section

Rūpaṁ bhikkhave anattā \breathmark\ rūpañca hidaṁ bhikkhave attā abhavissa \breathmark\ nayidaṁ rūpaṁ ābādhāya saṁvatteyya \breathmark\ labbhetha ca rūpe \breathmark\ “Evaṁ me rūpaṁ hotu \breathmark\ evaṁ me rūpaṁ mā ahosī”ti.

Yasmā ca kho bhikkhave rūpaṁ anattā \breathmark\ tasmā rūpaṁ ābādhāya saṁvattati \breathmark\ na ca labbhati rūpe \breathmark\ “Evaṁ me rūpaṁ hotu \breathmark\ evaṁ me rūpaṁ mā ahosī”ti.

Vedanā anattā \breathmark\ vedanā ca hidaṁ bhikkhave attā abhavissa \breathmark\ nayidaṁ vedanā ābādhāya saṁvatteyya \breathmark\ labbhetha ca vedanāya \breathmark\ “Evaṁ me vedanā hotu \breathmark\ evaṁ me vedanā mā ahosī”ti.

Yasmā ca kho bhikkhave vedanā anattā \breathmark\ tasmā vedanā ābādhāya saṁvattati \breathmark\ na ca labbhati vedanāya \breathmark\ “Evaṁ me vedanā hotu \breathmark\ evaṁ me vedanā mā ahosī”ti.

Saññā anattā \breathmark\ saññā ca hidaṁ bhikkhave attā abhavissa \breathmark\ nayidaṁ saññā ābādhāya saṁvatteyya \breathmark\ labbhetha ca saññāya \breathmark\ “Evaṁ me saññā hotu \breathmark\ evaṁ me saññā mā ahosī”ti.
>>>>>>> main

Yasmā ca kho bhikkhave saññā anattā \breathmark\ tasmā saññā ābādhāya saṁvattati \breathmark\ na ca labbhati saññāya \breathmark\ “Evaṁ me saññā hotu \breathmark\ evaṁ me saññā mā ahosī”ti.

Saṅkhārā anattā \breathmark\ saṅkhārā ca hidaṁ bhikkhave attā abhavissaṁsu \breathmark\ nayidaṁ saṅkhārā ābādhāya saṁvatteyyuṁ \breathmark\ labbhetha ca saṅkhāresu \breathmark\ “Evaṁ me saṅkhārā hontu \breathmark\ evaṁ me saṅkhārā mā ahesun”ti.

Yasmā ca kho bhikkhave saṅkhārā anattā \breathmark\ tasmā saṅkhārā ābādhāya saṁvattanti \breathmark\ na ca labbhati saṅkhāresu \breathmark\ “Evaṁ me saṅkhārā hontu \breathmark\ evaṁ me saṅkhārā mā ahesun”ti.

Viññāṇaṁ anattā \breathmark\ viññāṇañca hidaṁ bhikkhave attā abhavissa \breathmark\ nayidaṁ viññāṇam ābādhāya saṁvatteyya \breathmark\ labbhetha ca viññāṇe \breathmark\ “Evaṁ me viññāṇaṁ hotu \breathmark\ evaṁ me viññāṇaṁ mā ahosī”ti.

Yasmā ca kho bhikkhave viññāṇaṁ anattā \breathmark\ tasmā viññāṇaṁ ābādhāya saṁvattati \breathmark\ na ca labbhati viññāṇe \breathmark\ “Evaṁ me viññāṇaṁ hotu \breathmark\ evaṁ me viññāṇaṁ mā ahosī”ti.

[Taṁ kiṁ maññatha bhikkhave] rūpaṁ niccaṁ vā aniccaṁ vāti? Aniccaṁ bhante.

Yam panāniccaṁ dukkhaṁ vā taṁ sukhaṁ vāti?

Dukkhaṁ bhante.

Yam panāniccaṁ dukkhaṁ viparināmadhammaṁ kallaṁ nu taṁ samanupassituṁ \breathmark\ “Etaṁ mama esohamasmi eso me attā”ti?

No hetaṁ bhante.

Taṁ kiṁ maññatha bhikkhave vedanā niccā vā aniccā vāti?

Aniccā bhante.

Yam panāniccaṁ dukkhaṁ vā taṁ sukhaṁ vāti?

Dukkhaṁ bhante.

Yam panāniccaṁ dukkhaṁ viparināmadhammaṁ kallaṁ nu taṁ samanupassituṁ \breathmark\ “Etaṁ mama esohamasmi eso me attā”ti?

No hetaṁ bhante.

Taṁ kiṁ maññatha bhikkhave saññā niccā vā aniccā vāti?

Aniccā bhante.

Yam panāniccaṁ dukkhaṁ vā taṁ sukhaṁ vāti?

Dukkhaṁ bhante.

Yam panāniccaṁ dukkhaṁ viparināmadhammaṁ kallaṁ nu taṁ samanupassituṁ \breathmark\ “Etaṁ mama esohamasmi eso me attā”ti?

No hetaṁ bhante.

Taṁ kiṁ maññatha bhikkhave saṅkhārā niccā vā aniccā vāti?

Aniccā bhante.

Yam panāniccaṁ dukkhaṁ vā taṁ sukhaṁ vāti?

Dukkhaṁ bhante.

Yam panāniccaṁ dukkhaṁ viparināmadhammaṁ kallaṁ nu taṁ samanupassituṁ \breathmark\ “Etaṁ mama esohamasmi eso me attā”ti?

No hetaṁ bhante.

Taṁ kiṁ maññatha bhikkhave viññāṇaṁ niccaṁ vā aniccaṁ vāti?

Aniccaṁ bhante.

Yam panāniccaṁ dukkhaṁ vā taṁ sukhaṁ vāti?

Dukkhaṁ bhante.

Yam panāniccaṁ dukkhaṁ viparināmadhammaṁ kallaṁ nu taṁ samanupassituṁ \breathmark\ “Etaṁ mama esohamasmi eso me attā”ti?

No hetaṁ bhante.

[Tasmātiha bhikkhave] yaṅkiñci rūpaṁ atītānāgata-paccuppannaṁ \breathmark\ ajjhattaṁ vā bahiddhā vā \breathmark\ oḷārikaṁ vā sukhumaṁ vā \breathmark\ hīnaṁ vā paṇītaṁ vā \breathmark\ yaṁ dūre santike vā \breathmark\ sabbaṁ rūpaṁ: \breathmark\ “Netaṁ mama nesohamasmi na m’eso attā”ti \breathmark\ evametaṁ yathābhūtaṁ sammappaññāya daṭṭhabbaṁ.

Yā kāci vedanā atītānāgata-paccuppannā \breathmark\ ajjhattā vā bahiddhā vā \breathmark\ oḷārikā vā sukhumā vā \breathmark\ hīnā vā paṇītā vā \breathmark\ yā dūre santike vā \breathmark\ sabbā vedanā: \breathmark\ “Netaṁ mama nesohamasmi na m’eso attā”ti \breathmark\ evametaṁ yathābhūtaṁ sammappaññāya daṭṭhabbaṁ.

Yā kāci saññā atītānāgata-paccuppannā \breathmark\ ajjhattā vā bahiddhā vā \breathmark\ oḷārikā vā sukhumā vā \breathmark\ hīnā vā paṇītā vā \breathmark\ yā dūre santike vā \breathmark\ sabbā saññā: \breathmark\ “Netaṁ mama nesohamasmi na m’eso attā”ti \breathmark\ evametaṁ yathābhūtaṁ sammappaññāya daṭṭhabbaṁ.

Ye keci saṅkhārā atītānāgata-paccuppannā \breathmark\ ajjhattā vā bahiddhā vā \breathmark\ oḷārikā vā sukhumā vā \breathmark\ hīnā vā paṇītā vā \breathmark\ yā dūre santike vā \breathmark\ sabbe saṅkhārā: \breathmark\ “Netaṁ mama nesohamasmi na m’eso attā”ti \breathmark\ evametaṁ yathābhūtaṁ sammappaññāya daṭṭhabbaṁ.

Yaṅkiñci viññāṇaṁ atītānāgata-paccuppannaṁ \breathmark\ ajjhattaṁ vā bahiddhā vā \breathmark\ oḷārikaṁ vā sukhumaṁ vā \breathmark\ hīnaṁ vā paṇītaṁ vā \breathmark\ yaṁ dūre santike vā \breathmark\ sabbaṁ viññāṇaṁ: \breathmark\ “Netaṁ mama nesohamasmi na m’eso attā”ti \breathmark\ evametaṁ yathābhūtaṁ sammappaññāya daṭṭhabbaṁ.

[Evaṁ passaṁ bhikkhave] sutvā ariyasāvako rūpasmim-pi nibbindati \breathmark\ vedanāya-pi nibbindati \breathmark\ saññāya-pi nibbindati \breathmark\ saṅkhāresu-pi nibbindati \breathmark\ viññāṇasmim-pi nibbindati \breathmark\ nibbindaṁ virajjati \breathmark\ virāgā vimuccati \breathmark\ vimuttasmiṁ “Vimuttam” iti ñāṇaṁ hoti \breathmark\ “Khīṇā jāti vusitaṁ brahmacariyaṁ kataṁ karaṇīyaṁ nāparaṁ itthattāyā”ti pajānātī’ti.

Idamavoca bhagavā. Attamanā pañcavaggiyā bhikkhū bhagavato bhāsitaṁ abhinanduṁ. Imasmiñca pana veyyākaraṇasmiṁ bhaññamāne pañcavaggiyānaṁ bhikkhūnaṁ anupādāya āsavehi cittāni vimucciṁsū’ti.

\suttaRef{[SN 22.59]}

\bottomNav{}
