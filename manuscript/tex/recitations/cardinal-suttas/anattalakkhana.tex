\section{Anatta-lakkhaṇa-sutta}
\label{anatta-lakkhana}

\begin{leader}
  〈 Evaṁ me sutaṁ 〉
\end{leader}

\begin{pali-hang}
  Ekaṁ samayaṁ bhagavā bārāṇasiyaṁ viharati isipatane migadāye. Tatra kho bhagavā pañcavaggiye bhikkhū āmantesi: ``Bhikkhavo''ti \breathmark\ ``Bhadante''ti te bhikkhū bhagavato paccassosuṁ. Bhagavā etadavoca.\hyperlink{endnote1-appendix}{\hypertarget{endnote1-body}{\pagenote{%
        \hypertarget{endnote1-appendix}{\hyperlink{endnote1-body}{The following passage is absent in the Thai edition of the \textit{Tipiṭaka}: “Bhikkhavo” ti; “Bhadante” ti te bhikkhū Bhagavato paccassosuṁ. Bhagavā etadavoca.}}}}}
\end{pali-hang}


\begin{pali-hang}
  Rūpaṁ bhikkhave anattā \breathmark\ rūpañca hidaṁ bhikkhave attā abhavissa \breathmark\ nayidaṁ rūpaṁ ābādhāya saṁvatteyya \breathmark\ labbhetha ca rūpe \breathmark\ ``Evaṁ me rūpaṁ hotu \breathmark\ evaṁ me rūpaṁ mā ahosī''ti.
\end{pali-hang}

\begin{pali-hang}
  Yasmā ca kho bhikkhave rūpaṁ anattā \breathmark\ tasmā rūpaṁ ābādhāya saṁvattati \breathmark\ na ca labbhati rūpe \breathmark\ ``Evaṁ me rūpaṁ hotu \breathmark\ evaṁ me rūpaṁ mā ahosī''ti.
\end{pali-hang}

\begin{pali-hang}
  Vedanā anattā \breathmark\ vedanā ca hidaṁ bhikkhave attā abhavissa \breathmark\ nayidaṁ vedanā ābādhāya saṁvatteyya \breathmark\ labbhetha ca vedanāya \breathmark\ ``Evaṁ me vedanā hotu \breathmark\ evaṁ me vedanā mā ahosī''ti.
\end{pali-hang}

\begin{pali-hang}
  Yasmā ca kho bhikkhave vedanā anattā \breathmark\ tasmā vedanā ābādhāya saṁvattati \breathmark\ na ca labbhati vedanāya \breathmark\ ``Evaṁ me vedanā hotu \breathmark\ evaṁ me vedanā mā ahosī''ti.
\end{pali-hang}

\begin{pali-hang}
  Saññā anattā \breathmark\ saññā ca hidaṁ bhikkhave attā abhavissa \breathmark\ nayidaṁ saññā ābādhāya saṁvatteyya \breathmark\ labbhetha ca saññāya \breathmark\ ``Evaṁ me saññā hotu \breathmark\ evaṁ me saññā mā ahosī''ti.
\end{pali-hang}

\begin{pali-hang}
  Yasmā ca kho bhikkhave saññā anattā \breathmark\ tasmā saññā ābādhāya saṁvattati \breathmark\ na ca labbhati saññāya \breathmark\ ``Evaṁ me saññā hotu \breathmark\ evaṁ me saññā mā ahosī''ti.
\end{pali-hang}

\begin{pali-hang}
  Saṅkhārā anattā \breathmark\ saṅkhārā ca hidaṁ bhikkhave attā \mbox{abhavissaṁsu}~\breathmark\ nayidaṁ saṅkhārā ābādhāya saṁvatteyyuṁ \breathmark\ labbhetha ca saṅkhāresu \breathmark\ ``Evaṁ me saṅkhārā hontu \breathmark\ evaṁ me saṅkhārā mā ahesun''ti.
\end{pali-hang}

\begin{pali-hang}
  Yasmā ca kho bhikkhave saṅkhārā anattā \breathmark\ tasmā saṅkhārā ābādhāya saṁvattanti \breathmark\ na ca labbhati saṅkhāresu \breathmark\ ``Evaṁ me saṅkhārā hontu \breathmark\ evaṁ me saṅkhārā mā ahesun''ti.
\end{pali-hang}

\begin{pali-hang}
  Viññāṇaṁ anattā \breathmark\ viññāṇañca hidaṁ bhikkhave attā abhavissa \breathmark\ nayidaṁ viññāṇam ābādhāya saṁvatteyya \breathmark\ labbhetha ca viññāṇe \breathmark\ ``Evaṁ me viññāṇaṁ hotu \breathmark\ evaṁ me viññāṇaṁ mā ahosī''ti.
\end{pali-hang}

\begin{pali-hang}
  Yasmā ca kho bhikkhave viññāṇaṁ anattā \breathmark\ tasmā viññāṇaṁ ābādhāya saṁvattati \breathmark\ na ca labbhati \mbox{viññāṇe}~\breathmark\ ``Evaṁ me viññāṇaṁ hotu \breathmark\ evaṁ me viññāṇaṁ mā ahosī''ti.
\end{pali-hang}

\begin{pali-hang}
  [Taṁ kiṁ maññatha bhikkhave] rūpaṁ niccaṁ vā aniccaṁ vāti? Aniccaṁ bhante.
\end{pali-hang}

\begin{pali-hang}
  Yam panāniccaṁ dukkhaṁ vā taṁ sukhaṁ vāti?
\end{pali-hang}

\begin{pali-hang}
  Dukkhaṁ bhante.
\end{pali-hang}

\begin{pali-hang}
  Yam panāniccaṁ dukkhaṁ viparināmadhammaṁ kallaṁ nu taṁ samanupassituṁ \breathmark\ ``Etaṁ mama esohamasmi eso me attā''ti?
\end{pali-hang}

\begin{pali-hang}
  No hetaṁ bhante.
\end{pali-hang}

\begin{pali-hang}
  Taṁ kiṁ maññatha bhikkhave vedanā niccā vā aniccā vāti?
\end{pali-hang}

\begin{pali-hang}
  Aniccā bhante.
\end{pali-hang}

\begin{pali-hang}
  Yam panāniccaṁ dukkhaṁ vā taṁ sukhaṁ vāti?
\end{pali-hang}

\begin{pali-hang}
  Dukkhaṁ bhante.
\end{pali-hang}

\begin{pali-hang}
  Yam panāniccaṁ dukkhaṁ viparināmadhammaṁ kallaṁ nu taṁ samanupassituṁ \breathmark\ ``Etaṁ mama esohamasmi eso me attā''ti?
\end{pali-hang}

\begin{pali-hang}
  No hetaṁ bhante.
\end{pali-hang}

\begin{pali-hang}
  Taṁ kiṁ maññatha bhikkhave saññā niccā vā aniccā vāti?
\end{pali-hang}

\begin{pali-hang}
  Aniccā bhante.
\end{pali-hang}

\begin{pali-hang}
  Yam panāniccaṁ dukkhaṁ vā taṁ sukhaṁ vāti?
\end{pali-hang}

\begin{pali-hang}
  Dukkhaṁ bhante.
\end{pali-hang}

\begin{pali-hang}
  Yam panāniccaṁ dukkhaṁ viparināmadhammaṁ kallaṁ nu taṁ samanupassituṁ \breathmark\ ``Etaṁ mama esohamasmi eso me attā''ti?
\end{pali-hang}

\begin{pali-hang}
  No hetaṁ bhante.
\end{pali-hang}

\begin{pali-hang}
  Taṁ kiṁ maññatha bhikkhave saṅkhārā niccā vā aniccā vāti?
\end{pali-hang}

\begin{pali-hang}
  Aniccā bhante.
\end{pali-hang}

\begin{pali-hang}
  Yam panāniccaṁ dukkhaṁ vā taṁ sukhaṁ vāti?
\end{pali-hang}

\begin{pali-hang}
  Dukkhaṁ bhante.
\end{pali-hang}

\begin{pali-hang}
  Yam panāniccaṁ dukkhaṁ viparināmadhammaṁ kallaṁ nu taṁ samanupassituṁ \breathmark\ ``Etaṁ mama esohamasmi eso me attā''ti?
\end{pali-hang}

\begin{pali-hang}
  No hetaṁ bhante.
\end{pali-hang}

\begin{pali-hang}
  Taṁ kiṁ maññatha bhikkhave viññāṇaṁ niccaṁ vā aniccaṁ vāti?
\end{pali-hang}

\begin{pali-hang}
  Aniccaṁ bhante.
\end{pali-hang}

\begin{pali-hang}
  Yam panāniccaṁ dukkhaṁ vā taṁ sukhaṁ vāti?
\end{pali-hang}

\begin{pali-hang}
  Dukkhaṁ bhante.
\end{pali-hang}

\begin{pali-hang}
  Yam panāniccaṁ dukkhaṁ viparināmadhammaṁ kallaṁ nu taṁ samanupassituṁ \breathmark\ ``Etaṁ mama esohamasmi eso me attā''ti?
\end{pali-hang}

\begin{pali-hang}
  No hetaṁ bhante.
\end{pali-hang}

\begin{pali-hang}
  [Tasmātiha bhikkhave] yaṅkiñci rūpaṁ atītānāgata-paccuppannaṁ \breathmark\ ajjhattaṁ vā bahiddhā vā \breathmark\ oḷārikaṁ vā sukhumaṁ vā \breathmark\ hīnaṁ vā paṇītaṁ vā \breathmark\ yaṁ dūre santike vā \breathmark\ sabbaṁ rūpaṁ: \breathmark\ ``Netaṁ mama nesohamasmi na m'eso attā''ti \breathmark\ evametaṁ yathābhūtaṁ sammappaññāya daṭṭhabbaṁ.
\end{pali-hang}

\begin{pali-hang}
  Yā kāci vedanā atītānāgata-paccuppannā \breathmark\ ajjhattā vā bahiddhā \mbox{vā}~\breathmark\ oḷārikā vā sukhumā vā\breathmark\ hīnā vā paṇītā vā \breathmark\ yā dūre santike vā \breathmark\ sabbā vedanā: \breathmark\ ``Netaṁ mama nesohamasmi na m'eso attā''ti \breathmark\ evametaṁ yathābhūtaṁ sammappaññāya daṭṭhabbaṁ.
\end{pali-hang}

\begin{pali-hang}
  Yā kāci saññā atītānāgata-paccuppannā \breathmark\ ajjhattā vā bahiddhā vā \breathmark\ oḷārikā vā sukhumā \breathmark\ hīnā vā paṇītā vā \breathmark\ yā dūre santike vā \breathmark\ sabbā saññā: \breathmark\ ``Netaṁ mama nesohamasmi na m'eso attā''ti \breathmark\ evametaṁ yathābhūtaṁ sammappaññāya daṭṭhabbaṁ.
\end{pali-hang}

\begin{pali-hang}
  Ye keci saṅkhārā atītānāgata-paccuppannā \breathmark\ ajjhattā vā bahiddhā vā \breathmark\ oḷārikā vā sukhumā vā \breathmark\ hīnā vā paṇītā vā \breathmark\ yā dūre santike \mbox{vā}~\breathmark\ sabbe saṅkhārā: \breathmark\ ``Netaṁ mama nesohamasmi na m'eso attā''ti \breathmark\ evametaṁ yathābhūtaṁ sammappaññāya daṭṭhabbaṁ.
\end{pali-hang}

\begin{pali-hang}
  Yaṅkiñci viññāṇaṁ atītānāgata-paccuppannaṁ \breathmark\ ajjhattaṁ vā bahiddhā vā \breathmark\ oḷārikaṁ vā sukhumaṁ vā \breathmark\ hīnaṁ vā paṇītaṁ vā \breathmark\ yaṁ dūre santike vā \breathmark\ sabbaṁ viññāṇaṁ: \breathmark\ ``Netaṁ mama nesohamasmi na m'eso attā''ti \breathmark\ evametaṁ yathābhūtaṁ sammappaññāya daṭṭhabbaṁ.
\end{pali-hang}

\begin{pali-hang}
  [Evaṁ passaṁ bhikkhave] sutvā ariyasāvako rūpasmim-pi nibbindati \breathmark\ vedanāya-pi nibbindati \breathmark\ saññāya-pi nibbindati \breathmark\ saṅkhāresu-pi nibbindati \breathmark\ viññāṇasmim-pi nibbindati \breathmark\ nibbindaṁ virajjati \breathmark\ virāgā vimuccati \breathmark\ vimuttasmiṁ ``Vimuttam'' iti ñāṇaṁ hoti \breathmark\ ``Khīṇā jāti vusitaṁ brahmacariyaṁ kataṁ karaṇīyaṁ nāparaṁ itthattāyā''ti pajānātī'ti.
\end{pali-hang}

\begin{pali-hang}
  Idamavoca bhagavā. Attamanā pañcavaggiyā bhikkhū bhagavato bhāsitaṁ abhinanduṁ. Imasmiñca pana veyyākaraṇasmiṁ bhaññamāne pañcavaggiyānaṁ bhikkhūnaṁ anupādāya āsavehi cittāni vimucciṁsū'ti.
\end{pali-hang}

\suttaRef{[SN 22.59]}

\bottomNav{striving-according-to-dhamma}
