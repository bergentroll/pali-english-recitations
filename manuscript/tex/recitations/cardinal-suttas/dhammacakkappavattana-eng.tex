\section{The Discourse on Setting in Motion the Wheel of Dhamma}
\label{wheel-of-dhamma-full}

[Thus have I heard]

On one occasion the Blessed One was dwelling at Bārāṇasī \breathmark\ in the Deer Park at Isipatana. There the Blessed One addressed the bhikkhus of the group of five thus:

Bhikkhus these two extremes should not be followed by one who has gone forth into homelessness. What two? The pursuit of sensual happiness in sensual pleasures \breathmark\ which is low \breathmark\ vulgar \breathmark\ the way of worldlings \breathmark\ ignoble \breathmark\ unbeneficial \breathmark\ and the pursuit of self-mortification \breathmark\ which is painful \breathmark\ ignoble \breathmark\ unbeneficial.

Without veering towards either of these extremes \breathmark\ the Tathāgata has awakened to the middle way \breathmark\ which gives rise to vision \breathmark\ which gives rise to knowledge \breathmark\ which leads to peace \breathmark\ to direct knowledge \breathmark\ to enlightenment \breathmark\ to Nibbāna.

And what \breathmark\ bhikkhus \breathmark\ is the Middle Way awakened to by the Tathāgata which gives rise to vision \breathmark\ which gives rise to knowledge \breathmark\ which leads to peace \breathmark\ to direct knowledge \breathmark\ to enlightenment \breathmark\ to Nibbāna.

It is this Noble Eightfold Path:

That Is \breathmark\ Right View \breathmark\ Right Intention \breathmark\ Right Speech \breathmark\ Right Action \breathmark\ Right Livelihood \breathmark\ Right Effort \breathmark\ Right Mindfulness \breathmark\ Right Concentration.

This \breathmark\ bhikkhus \breathmark\ is that middle way awakened to by the Tathāgata \breathmark\ which gives rise to vision \breathmark\ which gives rise to knowledge \breathmark\ which leads to peace \breathmark\ to direct knowledge \breathmark\ to enlightenment \breathmark\ to Nibbāna.

Now this \breathmark\ bhikkhus \breathmark\ is the noble truth of suffering: birth is suffering \breathmark\ ageing is suffering \breathmark\ illness is suffering \breathmark\ death is suffering \breathmark\ union with what is displeasing is suffering \breathmark\ separation from what is pleasing is suffering \breathmark\ not to get what one wants is suffering \breathmark\ in brief \breathmark\ the five aggregates subject to clinging are suffering.

Now this \breathmark\ bhikkhus \breathmark\ is the noble truth of the origin of suffering: it is this craving which leads to renewed existence \breathmark\ accompanied by delight and lust \breathmark\ seeking delight here and there \breathmark\ that is \breathmark\ craving for sensual pleasures \breathmark\ craving for existence \breathmark\ craving for extermination.

Now this \breathmark\ bhikkhus \breathmark\ is the noble truth of the cessation of suffering: it is the remainderless fading away and cessation of that same craving \breathmark\ the giving up and relinquishing of it \breathmark\ freedom from it \breathmark\ nonreliance on it.

Now this \breathmark\ bhikkhus \breathmark\ is the noble truth of the way leading to the cessation of suffering: It is this Noble Eightfold Path;

That Is \breathmark\ Right View \breathmark\ Right Intention \breathmark\ Right Speech \breathmark\ Right Action \breathmark\ Right Livelihood \breathmark\ Right Effort \breathmark\ Right Mindfulness \breathmark\ Right Concentration.

This is the noble truth of suffering’: thus \breathmark\ bhikkhus \breathmark\ in regard to things unheard before \breathmark\ there arose in me vision \breathmark\ knowledge \breathmark\ wisdom \breathmark\ true knowledge \breathmark\ and light.

This noble truth of suffering is to be fully understood’: thus \breathmark\ bhikkhus \breathmark\ in regard to things unheard before \breathmark\ there arose in me vision \breathmark\ knowledge \breathmark\ wisdom \breathmark\ true knowledge \breathmark\ and light.

This noble truth of suffering has been fully understood’: thus \breathmark\ bhikkhus \breathmark\ in regard to things unheard before \breathmark\ there arose in me vision \breathmark\ knowledge \breathmark\ wisdom \breathmark\ true knowledge \breathmark\ and light.

This is the noble truth of the origin of suffering’: thus \breathmark\ bhikkhus \breathmark\ in regard to things unheard before \breathmark\ there arose in me vision \breathmark\ knowledge \breathmark\ wisdom \breathmark\ true knowledge \breathmark\ and light.

This noble truth of the origin of suffering is to be abandoned’: thus \breathmark\ bhikkhus \breathmark\ in regard to things unheard before \breathmark\ there arose in me vision \breathmark\ knowledge \breathmark\ wisdom \breathmark\ true knowledge \breathmark\ and light.

This noble truth of the origin of suffering has been abandoned’: thus \breathmark\ bhikkhus \breathmark\ in regard to things unheard before \breathmark\ there arose in me vision \breathmark\ knowledge \breathmark\ wisdom \breathmark\ true knowledge \breathmark\ and light.

This is the noble truth of the cessation of suffering’: thus \breathmark\ bhikkhus \breathmark\ in regard to things unheard before \breathmark\ there arose in me vision \breathmark\ knowledge \breathmark\ wisdom \breathmark\ true knowledge \breathmark\ and light.

This noble truth of the cessation of suffering is to be realized’: thus \breathmark\ bhikkhus \breathmark\ in regard to things unheard before \breathmark\ there arose in me vision \breathmark\ knowledge \breathmark\ wisdom \breathmark\ true knowledge \breathmark\ and light.

This noble truth of the cessation of suffering has been realized’: thus \breathmark\ bhikkhus \breathmark\ in regard to things unheard before \breathmark\ there arose in me vision \breathmark\ knowledge \breathmark\ wisdom \breathmark\ true knowledge \breathmark\ and light.

This is the noble truth of the way leading to the cessation of suffering’: thus \breathmark\ bhikkhus \breathmark\ in regard to things unheard before \breathmark\ there arose in me vision \breathmark\ knowledge \breathmark\ wisdom \breathmark\ true knowledge \breathmark\ and light.

This noble truth of the way leading to the cessation of suffering is to be developed’: thus \breathmark\ bhikkhus \breathmark\ in regard to things unheard before \breathmark\ there arose in me vision \breathmark\ knowledge \breathmark\ wisdom \breathmark\ true knowledge \breathmark\ and light.

This noble truth of the way leading to the cessation of suffering has been developed’: thus \breathmark\ bhikkhus \breathmark\ in regard to things unheard before \breathmark\ there arose in me vision \breathmark\ knowledge \breathmark\ wisdom \breathmark\ true knowledge \breathmark\ and light.

So long \breathmark\ bhikkhus \breathmark\ as my knowledge and vision of these Four Noble Truths as they really are in their three phases and twelve aspects was not thoroughly purified in this way \breathmark\ I did not claim to have awakened to the unsurpassed perfect enlightenment in this world with its devas \breathmark\ Māra \breathmark\ and Brahmā \breathmark\ in this generation with its ascetics and brahmins \breathmark\ its devas and humans.

But when my knowledge and vision of these Four Noble Truths as they really are in their three phases and twelve aspects was thoroughly purified in this way \breathmark\ then I claimed to have awakened to the unsurpassed perfect enlightenment in this world with its devas \breathmark\ Māra \breathmark\ and Brahmā \breathmark\ in this generation with its ascetics and brahmins \breathmark\ its devas and humans.

This is what the Blessed One said. Elated \breathmark\ the bhikkhus of the group of five delighted in the Blessed One’s statement.

And while this discourse was being spoken \breathmark\ there arose in the Venerable Kondañña the dust-free \breathmark\ stainless vision of the Dhamma:

“Whatever is subject to origination is all subject to cessation.”

And when the Wheel of the Dhamma had been set in motion by the Blessed One, the earth-dwelling devas raised a cry:

“At Bārāṇasī in the Deer Park at Isipatana \breathmark\ this unsurpassed Wheel of the Dhamma has been set in motion by the Blessed One \breathmark\ which cannot be stopped by any ascetic or brahmin or deva or Māra or Brahmā or by anyone in the world.”

Having heard the cry of the earth-dwelling devas \breathmark\ the devas of the realm of the Four Great Kings raised a cry …

Having heard the cry of the devas of the realm of the Four Great Kings \breathmark\ the Tāvatiṁsa devas raised a cry …

Having heard the cry of the Tāvatiṁsa devas \breathmark\ the Yāma devas raised a cry …

Having heard the cry of the Yāma devas \breathmark\ the Tusita devas raised a cry …

Having heard the cry of the Tusita devas \breathmark\ the Nimmānaratī devas raised a cry …

Having heard the cry of the Nimmānaratī devas \breathmark\ the Paranimmitavasavattī devas raised a cry …

Having heard the cry of the Paranimmitavasavattī devas \breathmark\ the devas of Brahmā’s company raised a cry:

“At Bārāṇasī in the Deer Park at Isipatana \breathmark\ this unsurpassed Wheel of the Dhamma has been set in motion by the Blessed One \breathmark\ which cannot be stopped by any ascetic or brahmin or deva or Māra or Brahmā or by anyone in the world.”

Thus at that moment \breathmark\ at that instant \breathmark\ at that second \breathmark\ the cry spread as far as the brahmā world \breathmark\ and this ten thousandfold world system shook \breathmark\ quaked \breathmark\ and trembled \breathmark\ and an immeasurable glorious radiance appeared in the world surpassing the divine majesty of the devas.

Then the Blessed One uttered this inspired utterance:

“Koṇḍañña has indeed understood! Koṇḍañña has indeed understood!”

In this way the Venerable Koṇḍañña acquired the name “Aññā Koṇḍañña—Koṇḍañña Who Has Understood.”

\suttaRef{[SN 56.11]}
