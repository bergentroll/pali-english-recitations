\section{Dhammacakkappavattana-sutta}

\begin{leader}
  〈 Evaṁ me sutaṁ 〉
\end{leader}

\begin{pali-hang}
  Ekaṁ samayaṁ bhagavā bārāṇasiyaṁ viharati isipatane migadāye. Tatra kho bhagavā pañcavaggiye bhikkhū āmantesi:
\end{pali-hang}

\begin{pali-hang}
  Dveme bhikkhave antā pabbajitena na sevitabbā: yo cāyaṁ kāmesu kāmasukhallikānuyogo hīno gammo pothujjaniko anariyo anatthasañhito yo cāyaṁ attakilamathānuyogo dukkho anariyo anatthasañhito.
\end{pali-hang}

\begin{pali-hang}
  Ete te bhikkhave ubho ante anupagamma majjhimā paṭipadā tathāgatena abhisambuddhā cakkhukaraṇī ñāṇakaraṇī upasamāya abhiññāya sambodhāya nibbānāya saṁvattati.
\end{pali-hang}

\begin{pali-hang}
  Katamā ca sā bhikkhave majjhimā paṭipadā tathāgatena abhisambuddhā cakkhukaraṇī ñāṇakaraṇī upasamāya abhiññāya sambodhāya nibbānāya saṁvattati?
\end{pali-hang}

\begin{pali-hang}
  Ayameva ariyo aṭṭhaṅgiko maggo seyyathīdaṁ:
\end{pali-hang}

\begin{pali-hang}
  Sammā-diṭṭhi sammā-saṅkappo sammā-vācā sammā-kammanto sammā-ājīvo sammā-vāyāmo sammā-sati sammā-samādhi.
\end{pali-hang}

\begin{pali-hang}
  Ayaṁ kho sā bhikkhave majjhimā paṭipadā Tathāgatena abhisambuddhā cakkhukaraṇī ñāṇakaraṇī upasamāya abhiññāya sambodhāya nibbānāya saṁvattati.
\end{pali-hang}

\begin{pali-hang}
  Idaṁ kho pana bhikkhave dukkhaṁ ariyasaccaṁ:
\end{pali-hang}

\begin{pali-hang}
  Jātipi dukkhā jarāpi dukkhā byādhipi dukkho maraṇampi dukkhaṁ soka-parideva-dukkha-domanassupāyāsāpi dukkhā appiyehi sampayogo dukkho piyehi vippayogo dukkho yam-picchaṁ na labhati tampi dukkhaṁ saṅkhittena pañcupādānakkhandā dukkhā.
\end{pali-hang}

\begin{pali-hang}
  Idaṁ kho pana bhikkhave dukkhasamudayo ariyasaccaṁ:
\end{pali-hang}

\begin{pali-hang}
  Yāyaṁ taṇhā ponobbhavikā nandirāgasahagatā tatra tatrābhinandinī seyyathīdaṁ: kāmataṇhā bhavataṇhā vibhavataṇhā.
\end{pali-hang}

\begin{pali-hang}
  Idaṁ kho pana bhikkhave dukkhanirodho ariyasaccaṁ:
\end{pali-hang}

\begin{pali-hang}
  Yo tassā yeva taṇhāya asesavirāganirodho cāgo paṭinissaggo mutti anālayo.
\end{pali-hang}

\begin{pali-hang}
  Idaṁ kho pana bhikkhave dukkhanirodhagāminī paṭipadā ariyasaccaṁ:
\end{pali-hang}

\begin{pali-hang}
  Ayameva ariyo aṭṭhaṅgiko maggo seyyathīdam: Sammā-diṭṭhi sammā-saṅkappo sammā-vācā sammā-kammanto sammā-ājīvo sammā-vāyāmo sammā-sati sammā-samādhi.
\end{pali-hang}

\begin{pali-hang}
  [Idaṁ dukkhaṁ] ariyasaccanti me bhikkhave pubbe ananussutesu dhammesu cakkhuṁ udapādi ñāṇaṁ udapādi paññā udapādi vijjā udapādi āloko udapādi.
\end{pali-hang}

\begin{pali-hang}
  Taṁ kho panidaṁ dukkhaṁ ariyasaccaṁ pariññeyyanti me bhikkhave pubbe ananussutesu dhammesu cakkhuṁ udapādi ñāṇaṁ udapādi paññā udapādi vijjā udapādi āloko udapādi.
\end{pali-hang}

\begin{pali-hang}
  Taṁ kho panidaṁ dukkhaṁ ariyasaccaṁ pariññātanti me bhikkhave pubbe ananussutesu dhammesu cakkhuṁ udapādi ñāṇaṁ udapādi paññā udapādi vijjā udapādi āloko udapādi.
\end{pali-hang}

\begin{pali-hang}
  Idaṁ dukkhasamudayo ariyasaccanti me bhikkhave pubbe ananussutesu dhammesu cakkhuṁ udapādi ñāṇaṁ udapādi paññā udapādi vijjā udapādi āloko udapādi.
\end{pali-hang}

\begin{pali-hang}
  Taṁ kho panidaṁ dukkhasamudayo ariyasaccaṁ pahātabbanti me bhikkhave pubbe ananussutesu dhammesu cakkhuṁ udapādi ñāṇaṁ udapādi paññā udapādi vijjā udapādi āloko udapādi.
\end{pali-hang}

\begin{pali-hang}
  Taṁ kho panidaṁ dukkhasamudayo ariyasaccaṁ pahīnanti me bhikkhave pubbe ananussutesu dhammesu cakkhuṁ udapādi ñāṇaṁ udapādi paññā udapādi vijjā udapādi āloko udapādi.
\end{pali-hang}

\begin{pali-hang}
  Idaṁ dukkhanirodho ariyasaccanti me bhikkhave pubbe ananussutesu dhammesu cakkhuṁ udapādi ñāṇaṁ udapādi paññā udapādi vijjā udapādi āloko udapādi.
\end{pali-hang}

\begin{pali-hang}
  Taṁ kho panidaṁ dukkhanirodho ariyasaccaṁ sacchikātabbanti me bhikkhave pubbe ananussutesu dhammesu cakkhuṁ udapādi ñāṇaṁ udapādi paññā udapādi vijjā udapādi āloko udapādi.
\end{pali-hang}

\begin{pali-hang}
  Taṁ kho panidaṁ dukkhanirodho ariyasaccaṁ sacchikatanti me bhikkhave pubbe ananussutesu dhammesu cakkhuṁ udapādi ñāṇaṁ udapādi paññā udapādi vijjā udapādi āloko udapādi.
\end{pali-hang}

\begin{pali-hang}
  Idaṁ dukkhanirodhagāminī paṭipadā ariyasaccanti me bhikkhave pubbe ananussutesu dhammesu cakkhuṁ udapādi ñāṇaṁ udapādi paññā udapādi vijjā udapādi āloko udapādi.
\end{pali-hang}

\begin{pali-hang}
  Taṁ kho panidaṁ dukkhanirodhagāminī paṭipadā ariyasaccaṁ bhāvetabbanti me bhikkhave pubbe ananussutesu dhammesu cakkhuṁ udapādi ñāṇaṁ udapādi paññā udapādi vijjā udapādi āloko udapādi.
\end{pali-hang}

\begin{pali-hang}
  Taṁ kho panidaṁ dukkhanirodhagāminī paṭipadā ariyasaccaṁ bhāvitanti me bhikkhave pubbe ananussutesu dhammesu cakkhuṁ udapādi ñāṇaṁ udapādi paññā udapādi vijjā udapādi āloko udapādi.
\end{pali-hang}

\begin{pali-hang}
  [Yāva kīvañca me] bhikkhave imesu catūsu ariyasaccesu evantiparivaṭṭaṁ dvādasākāraṁ yathābhūtaṁ ñāṇadassanaṁ na suvisuddhaṁ ahosi neva tāvāhaṁ bhikkhave sadevake loke samārake sabrahmake sassamaṇabrāhmaṇiyā pajāya sadevamanussāya anuttaraṁ sammāsambodhiṁ abhisambuddho paccaññāsiṁ.
\end{pali-hang}

\begin{pali-hang}
  Yato ca kho me bhikkhave imesu catūsu ariyasaccesu evantiparivaṭṭaṁ dvādasākāraṁ yathābhūtaṁ ñāṇadassanaṁ suvisuddham ahosi athāham bhikkhave sadevake loke samārake sabrahmake sassamaṇabrāhmaṇiyā pajāya sadevamanussāya anuttaraṁ sammāsambodhiṁ abhisambuddho paccaññāsiṁ.
\end{pali-hang}

Ñāṇañca pana me dassanaṁ udapādi.

\begin{pali-hang}
  ``Akuppā me vimutti ayamantimā jāti natthidāni punabbhavo''ti.
\end{pali-hang}

\begin{pali-hang}
  Idam avoca bhagavā. Attamanā pañcavaggiyā bhikkhu bhagavato bhāsitaṁ abhinanduṁ.
\end{pali-hang}

\begin{pali-hang}
  Imasmiñca pana veyyākaraṇasmiṁ bhaññamāne āyasmato koṇḍaññassa virajaṁ vītamalaṁ dhammacakkhuṁ udapādi:
\end{pali-hang}

``Yaṅkinci samudayadhammaṁ sabbantaṁ nirodhadhamman''ti.

\begin{pali-hang}
  [Pavattite ca bhagavatā] dhammacakke bhummā devā saddamanussāvesuṁ:
\end{pali-hang}

\begin{pali-hang}
  ``Etaṁ bhagavatā bārāṇasiyaṁ isipatane migadāye anuttaraṁ dhammacakkaṁ pavattitaṁ appaṭivattiyaṁ samaṇena vā brāhmaṇena vā devena vā mārena vā brahmunā vā kenaci vā lokasmin''ti.
\end{pali-hang}

\begin{pali-hang}
  Bhummānaṁ devānaṁ saddaṁ sutvā cātummahārājikā devā saddamanussāvesuṁ...
\end{pali-hang}

\begin{pali-hang}
  Cātummahārājikanaṁ devānaṁ saddaṁ sutvā tāvatiṁsā devā saddamanussāvesuṁ...
\end{pali-hang}

\begin{pali-hang}
  Tāvatiṁsānaṁ devānaṁ saddaṁ sutvā yāmā devā saddamanussāvesuṁ...
\end{pali-hang}

\begin{pali-hang}
  Yāmānaṁ devānaṁ saddaṁ sutvā tusitā devā saddamanussāvesuṁ...
\end{pali-hang}

\begin{pali-hang}
  Tusitānaṁ devānaṁ saddaṁ sutvā Nimmānaratī devā saddamanussavesum...
\end{pali-hang}

\begin{pali-hang}
  Nimmānaratīnaṁ devānaṁ saddaṁ sutvā Paranimmitavasavattī devā saddamanussāvesuṁ...
\end{pali-hang}

\begin{pali-hang}
  Paranimmitavasavattīnaṁ devānaṁ saddaṁ sutvā Brahmakāyikā devā saddamanussāvesuṁ:
\end{pali-hang}

\begin{pali-hang}
  ``Etaṁ bhagavatā bārāṇasiyaṁ isipatane migadāye anuttaraṁ dhammacakkaṁ pavattitaṁ appaṭivattiyaṁ samaṇena vā brāhmaṇena vā devena vā mārena vā brahmunā vā kenaci vā lokasmin''ti.
\end{pali-hang}

\begin{pali-hang}
  Itiha tena khaṇena tena muhuttena yāva brahmalokā saddo abbhuggacchi. Ayañca dasasahassī lokadhātu saṅkampi sampakampi sampavedhi appamāṇo ca oḷāro obhāso loke pāturahosi atikkammeva devānaṁ devānubhāvaṁ.
\end{pali-hang}

Atha kho bhagavā udānaṁ udānesi:

\begin{pali-hang}
  ``Aññāsi vata bho koṇḍañño aññāsi vata bho koṇḍañño''ti. Itihidaṁ āyasmato koṇḍaññassa aññākoṇḍañño tveva nāmaṁ ahosī'ti.
\end{pali-hang}

\suttaRef{[SN 56.11]}
