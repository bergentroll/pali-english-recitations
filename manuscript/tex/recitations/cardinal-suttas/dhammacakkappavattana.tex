\section{Dhammacakkappavattana-sutta}

[Evaṁ me sutaṁ]

Ekaṁ samayaṁ bhagavā bārāṇasiyaṁ viharati isipatane migadāye. Tatra kho bhagavā pañcavaggiye bhikkhū āmantesi:

Dveme bhikkhave antā pabbajitena na sevitabbā: yo cāyaṁ kāmesu kāmasukhallikānuyogo hīno gammo pothujjaniko anariyo anatthasañhito yo cāyaṁ attakilamathānuyogo dukkho anariyo anatthasañhito.

Ete te bhikkhave ubho ante anupagamma majjhimā paṭipadā Tathāgatena abhisambuddhā cakkhukaraṇī ñāṇakaraṇī upasamāya abhiññāya sambodhāya nibbānāya saṁvattati.

Katamā ca sā bhikkhave majjhimā paṭipadā Tathāgatena abhisambuddhā cakkhukaraṇī ñāṇakaraṇī upasamāya abhiññāya sambodhāya nibbānāya saṁvattati?

Ayameva ariyo aṭṭhaṅgiko maggo seyyathīdaṁ:

Sammā-diṭṭhi sammā-saṅkappo sammā-vācā sammā-kammanto sammā-ājīvo sammā-vāyāmo sammā-sati sammā-samādhi.
Ayaṁ kho sā bhikkhave majjhimā paṭipadā Tathāgatena abhisambuddhā cakkhukaraṇī ñāṇakaraṇī upasamāya abhiññāya sambodhāya nibbānāya saṁvattati.

Idaṁ kho pana bhikkhave dukkhaṁ ariyasaccaṁ:

Jātipi dukkhā jarāpi dukkhā byādhipi dukkho maraṇampi dukkhaṁ soka-parideva-dukkha-domanassupāyāsāpi dukkhā appiyehi sampayogo dukkho piyehi vippayogo dukkho yam-picchaṁ na labhati tampi dukkhaṁ saṅkhittena pañcupādānakkhandā dukkhā.

Idaṁ kho pana bhikkhave dukkhasamudayo ariyasaccaṁ:

Yāyaṁ taṇhā ponobbhavikā nandirāgasahagatā tatra tatrābhinandinī seyyathīdaṁ: kāmataṇhā bhavataṇhā vibhavataṇhā.

Idaṁ kho pana bhikkhave dukkhanirodho ariyasaccaṁ:

Yo tassā yeva taṇhāya asesavirāganirodho cāgo paṭinissaggo mutti anālayo.

Idaṁ kho pana bhikkhave dukkhanirodhagāminī paṭipadā ariyasaccaṁ:

Ayameva ariyo aṭṭhaṅgiko maggo seyyathīdam: Sammā-diṭṭhi sammā-saṅkappo sammā-vācā sammā-kammanto sammā-ājīvo sammā-vāyāmo sammā-sati sammā-samādhi.

[Idaṁ dukkhaṁ] ariyasaccanti me bhikkhave pubbe ananussutesu dhammesu cakkhuṁ udapādi ñāṇaṁ udapādi paññā udapādi vijjā udapādi āloko udapādi.

Taṁ kho panidaṁ dukkhaṁ ariyasaccaṁ pariññeyyanti me bhikkhave pubbe ananussutesu dhammesu cakkhuṁ udapādi ñāṇaṁ udapādi paññā udapādi vijjā udapādi āloko udapādi.

Taṁ kho panidaṁ dukkhaṁ ariyasaccaṁ pariññātanti me bhikkhave pubbe ananussutesu dhammesu cakkhuṁ udapādi ñāṇaṁ udapādi paññā udapādi vijjā udapādi āloko udapādi.

Idaṁ dukkhasamudayo ariyasaccanti me bhikkhave pubbe ananussutesu dhammesu cakkhuṁ udapādi ñāṇaṁ udapādi paññā udapādi vijjā udapādi āloko udapādi.

Taṁ kho panidaṁ dukkhasamudayo ariyasaccaṁ pahātabbanti me bhikkhave pubbe ananussutesu dhammesu cakkhuṁ udapādi ñāṇaṁ udapādi paññā udapādi vijjā udapādi āloko udapādi.

Taṁ kho panidaṁ dukkhasamudayo ariyasaccaṁ pahīnanti me bhikkhave pubbe ananussutesu dhammesu cakkhuṁ udapādi ñāṇaṁ udapādi paññā udapādi vijjā udapādi āloko udapādi.

Idaṁ dukkhanirodho ariyasaccanti me bhikkhave pubbe ananussutesu dhammesu cakkhuṁ udapādi ñāṇaṁ udapādi paññā udapādi vijjā udapādi āloko udapādi.

Taṁ kho panidaṁ dukkhanirodho ariyasaccaṁ sacchikātabbanti me bhikkhave pubbe ananussutesu dhammesu cakkhuṁ udapādi ñāṇaṁ udapādi paññā udapādi vijjā udapādi āloko udapādi.

Taṁ kho panidaṁ dukkhanirodho ariyasaccaṁ sacchikatanti me bhikkhave pubbe ananussutesu dhammesu cakkhuṁ udapādi ñāṇaṁ udapādi paññā udapādi vijjā udapādi āloko udapādi.

Idaṁ dukkhanirodhagāminī paṭipadā ariyasaccanti me bhikkhave pubbe ananussutesu dhammesu cakkhuṁ udapādi ñāṇaṁ udapādi paññā udapādi vijjā udapādi āloko udapādi.

Taṁ kho panidaṁ dukkhanirodhagāminī paṭipadā ariyasaccaṁ bhāvetabbanti me bhikkhave pubbe ananussutesu dhammesu cakkhuṁ udapādi ñāṇaṁ udapādi paññā udapādi vijjā udapādi āloko udapādi.

Taṁ kho panidaṁ dukkhanirodhagāminī paṭipadā ariyasaccaṁ bhāvitanti me bhikkhave pubbe ananussutesu dhammesu cakkhuṁ udapādi ñāṇaṁ udapādi paññā udapādi vijjā udapādi āloko udapādi.

[Yāva kīvañca me] bhikkhave imesu catūsu ariyasaccesu evantiparivaṭṭaṁ dvādasākāraṁ yathābhūtaṁ ñāṇadassanaṁ na suvisuddhaṁ ahosi neva tāvāhaṁ bhikkhave sadevake loke samārake sabrahmake sassamaṇabrāhmaṇiyā pajāya sadevamanussāya anuttaraṁ sammāsambodhiṁ abhisambuddho paccaññāsiṁ.

Yato ca kho me bhikkhave imesu catūsu ariyasaccesu evantiparivaṭṭaṁ dvādasākāraṁ yathābhūtaṁ ñāṇadassanaṁ suvisuddham ahosi athāham bhikkhave sadevake loke samārake sabrahmake sassamaṇabrāhmaṇiyā pajāya sadevamanussāya anuttaraṁ sammāsambodhiṁ abhisambuddho paccaññāsiṁ.

Ñāṇañca pana me dassanaṁ udapādi. % % TODO: period missing here in original

“Akuppā me vimutti ayamantimā jāti natthidāni punabbhavo”ti.

Idam avoca bhagavā. Attamanā pañcavaggiyā bhikkhu bhagavato bhāsitaṁ abhinanduṁ.

Imasmiñca pana veyyākaraṇasmiṁ bhaññamāne āyasmato koṇḍaññassa virajaṁ vītamalaṁ dhammacakkhuṁ udapādi:

“Yaṅkinci samudayadhammaṁ sabbantaṁ nirodhadhamman”ti.

[Pavattite ca bhagavatā] dhammacakke bhummā devā saddamanussāvesuṁ:

“Etaṁ bhagavatā bārāṇasiyaṁ isipatane migadāye anuttaraṁ dhammacakkaṁ pavattitaṁ appaṭivattiyaṁ samaṇena vā brāhmaṇena vā devena vā mārena vā brahmunā vā kenaci vā lokasmin”ti.

Bhummānaṁ devānaṁ saddaṁ sutvā cātummahārājikā devā saddamanussāvesuṁ...

Cātummahārājikanaṁ devānaṁ saddaṁ sutvā tāvatiṁsā devā saddamanussāvesuṁ...

Tāvatiṁsānaṁ devānaṁ saddaṁ sutvā yāmā devā saddamanussāvesuṁ...

Yāmānaṁ devānaṁ saddaṁ sutvā tusitā devā saddamanussāvesuṁ...

Tusitānaṁ devānaṁ saddaṁ sutvā Nimmānaratī devā saddamanussavesum...

Nimmānaratīnaṁ devānaṁ saddaṁ sutvā Paranimmitavasavattī devā saddamanussāvesuṁ...

Paranimmitavasavattīnaṁ devānaṁ saddaṁ sutvā Brahmakāyikā devā saddamanussāvesuṁ:

“Etaṁ bhagavatā bārāṇasiyaṁ isipatane migadāye anuttaraṁ dhammacakkaṁ pavattitaṁ appaṭivattiyaṁ samaṇena vā brāhmaṇena vā devena vā mārena vā brahmunā vā kenaci vā lokasmin”ti.

Itiha tena khaṇena tena muhuttena yāva brahmalokā saddo abbhuggacchi. Ayañca dasasahassī lokadhātu saṅkampi sampakampi sampavedhi appamāṇo ca oḷāro obhāso loke pāturahosi atikkammeva devānaṁ devānubhāvaṁ.

Atha kho bhagavā udānaṁ udānesi:

“Aññāsi vata bho koṇḍañño aññāsi vata bho koṇḍañño”ti. Itihidaṁ āyasmato koṇḍaññassa aññākoṇḍañño tveva nāmaṁ ahosī’ti.

\suttaRef{[SN 56.11]}
