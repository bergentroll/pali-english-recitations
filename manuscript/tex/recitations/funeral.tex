[Namo tassa] bhagavato arahato sammā-sambuddhassa [3x]

\begin{english}
Homage to the Blessed, Worthy, and Perfectly Enlightened One
\end{english}

[Kusalā dhammā]
Akusalā dhammā
Abyākatā dhammā

\begin{english}
  [Wholesome dhammas]
  Unwholesome dhammas
  Undetermined dhammas
\end{english}

Sukhāya vedanāya sampayuttā dhammā
Dukkhāya vedanāya sampayuttā dhammā
Adukkhamasukhāya vedanāya sampayuttā dhammā

\begin{english}
  Dhammas associated with pleasant feeling
  Dhammas associated with painful feeling
  Dhammas associated with neither-painful-nor-pleasant
\end{english}

Vipākā dhammā
Vipāka-dhamma-dhammā
N’eva vipāka na vipāka-dhamma-dhammā

\begin{english}
  Consequential dhamma
  Subject to consequential dhamma
  Neither consequential nor subject to consequential dhamma
\end{english}

Upādinn’upādāniyā dhammā
Anupādinn’upādāniyā dhammā
Anupādinnānupādāniyā dhammā

\begin{english}
  Clung dhammas which can be clung to
  Unclung dhammas which can be clung to
  Unclung dhammas which cannot be clung to
\end{english}

Saṅkiliṭṭha-saṅkilesikā dhammā
Asaṅkiliṭṭha-saṅkilesikā dhammā
Asaṅkiliṭṭhāsaṅkilesikā dhammā

\begin{english}
  Dhammas defiled and subject to defilements
  Dhammas undefiled but subject to defilements
  Dhammas neither defiled nor subject to defilements
\end{english}

Savitakka-savicārā dhammā
Avitakka-vicāra-mattā dhammā
Avitakkāvicārā dhammā

\begin{english}
  Dhammas with thought and examination
  Dhammas without thought but with examination
  Dhammas with neither thought nor examination
\end{english}

Pīti-saha-gatā dhammā
Sukha-saha-gatā dhammā
Upekkhā-saha-gatā dhammā

\begin{english}
  Dhammas accompanied by rapture
  Dhammas accompanied by pleasure
  Dhammas accompanied by equanimity
\end{english}

Dassanena pahātabbā dhammā
Bhāvanāya pahātabbā dhammā
N’eva dassanena na bhāvanāya pahātabbā dhammā

\begin{english}
  Dhammas abandoned by seeing
  Dhammas abandoned by development
  Dhammas abandoned by neither seeing nor development
\end{english}

Dassanena pahātabba-hetukā dhammā
Bhāvanāya pahātabba-hetukā dhammā
N’eva dassanena na bhāvanāya pahātabba-hetukā dhammā

\begin{english}
  Conditioned dhammas abandoned by seeing
  Conditioned dhammas abandoned by development
  Conditioned dhammas abandoned by neither seeing nor development
\end{english}

Ācaya-gāmino dhammā
Apacaya-gāmino dhammā
N’ev’ācaya-gāmino nāpacaya-gāmino dhammā

\begin{english}
  Dhammas leading to building up
  Dhamma leading to dismantling
  Dhammas leading to neither building up nor dismantling
\end{english}

Sekkhā dhammā
Asekkhā dhammā
N’eva sekkhā nāsekkhā dhammā

\begin{english}
  Dhammas of one in training
  Dhammas of one beyond training
  Dhammas of neither one in training nor one beyond training
\end{english}

Parittā dhammā
Mahaggatā dhammā
Appamāṇā dhammā

\begin{english}
  Limited dhammas
  Exhalted dhammas
  Immeasurable dhammas
\end{english}

Paritt’ārammaṇā dhammā
Mahaggat’ārammaṇā dhammā
Appamāṇ’ārammaṇā dhammā

\begin{english}
  Dhammas from limited sense-obejcts
  Dhammas from exhalted sense-objects
  Dhammas from immeasurable sense-objects
\end{english}

Hīnā dhammā
Majjhimā dhammā
Paṇītā dhammā

\begin{english}
  Inferior dhammas
  Moderate dhammas
  Superior dhammas
\end{english}

Micchatta-niyatā dhammā
Sammatta-niyatā dhammā
Aniyatā dhammā

\begin{english}
  Certain wrong dhammas
  Certain correct dhammas
  Uncertain dhammas
\end{english}

Magg’ārammaṇā dhammā
Magga-hetukā dhammā
Maggādhipatino dhammā

\begin{english}
  Dhammas with the path as object
  Dhammas with the path as cause
  Dhammas with the path as predominant factor
\end{english}

Uppannā dhammā
Anuppannā dhammā
Uppādino dhammā

\begin{english}
  Arisen dhammas
  Unarisen dhammas
  Bound to arise dhammas
\end{english}

Atītā dhammā
Anāgatā dhammā
Paccuppannā dhammā

\begin{english}
  Past dhammas
  Future dhammas
  Present dhammas
\end{english}

Atīt’ārammaṇā dhammā
Anāgat’ārammaṇā dhammā
Paccuppann’ārammaṇā dhammā

\begin{english}
  Dhammas with past sense-objects
  Dhammas with future sense-objects
  Dhammas with present sense-objects
\end{english}

Ajjhattā dhammā
Bahiddhā dhammā
Ajjhatta-bahiddhā dhammā

\begin{english}
  Internal dhammas
  External dhammas
  Internal and external dhamams
\end{english}

Ajjhatt’ārammaṇā dhammā
Bahiddh’ārammaṇā dhammā
Ajjhatta-bahiddh’ārammaṇā dhammā

\begin{english}
  Dhammas with internal sense-objects
  Dhammas with external sense-objects
  Dhammas with internal and external sense-objects
\end{english}

Sanidassana-sappaṭighā dhammā
Anidassana-sappaṭighā dhammā
Anidassanāppaṭighā dhammā

\begin{english}
  Visible and reactive dhammas
  Non-visible and reactive dhammas
  Non-visible and unobstructive dhammas
\end{english}

\suttaRef{[Dhs 1]}

[Pañcakkhandhā]:
Rūpakkhandho, vedanākkhandho,
saññākkhandho, saṅkhārakkhandho,
viññāṇakkhandho.

\begin{english}
  [The five aggregates]: The aggregate of form, the aggregateof feeling, the aggregate of perception, the aggregate of volitional formations, the aggregate of consciousness.
\end{english}

\suttaRef{[MN 109]}

Dvā-das’āyatanāni:
Cakkhv-āyatanaṁ rūp’āyatanaṁ,
Sot’āyatanaṁ sadd’āyatanaṁ,
Ghān’āyatanaṁ gandh’āyatanaṁ,
Jivh’āyatanaṁ ras’āyatanaṁ,
Kāy’āyatanaṁ phoṭṭhabb’āyatanaṁ,
Man’āyatanaṁ dhamm’āyatanaṁ

\begin{english}
  The eye-base, the form base
  The ear-base, the sound-base
  The nose-base, the odour-base
  The tongue-base, the flavour-base
  The body-base, the tangible-base
  The mind-base, the mind-object base.
\end{english}

\suttaRef{[MN 148]}

Aṭṭhārasa dhātuyo:
Cakkhu-dhātu rūpa-dhātu cakkhu-viññāṇa-dhātu,
Sota-dhātu sadda-dhātu sota-viññāṇa-dhātu,
Ghāna-dhātu gandha-dhātu ghāna-viññāṇa-dhātu,
Jivhā-dhātu rasa-dhātu jivhā-viññāṇa-dhātu,
Kāya-dhātu phoṭṭhabba-dhātu kāya-viññāṇa-dhātu,
Mano-dhātu dhamma-dhātu mano-viññāṇa-dhātu

\begin{english}
  Eighteen elements: The eye element, the form element, the eye-consciousness
  element;
  The ear element, the sound element, the ear-consciousness element;
  The nose element, the odour element, the nose-consciousness element;
  The tongue element, the flavour element, the tongue-consciousness element;
  The body element, the tangible element, the body-consciousness element;
  The mind element, the mind-object element, the mind-consciousness element.
\end{english}

\suttaRef{[MN 115]}

Bā-vīsat’indriyāni:i
Cakkhu’ndriyaṁ sot’indriyaṁ ghān’indriyaṁ
jivh’indriyaṁ kāy’indriyaṁ man’indriyaṁ,
itth’indriyaṁ puris’indriyaṁ jīvit’indriyaṁ,
sukh’indriyaṁ dukkh’indriyaṁ
somanass’indriyaṁ domanass’indriyaṁ upekkh’indriyaṁ,
saddh’indriyaṁ viriy’indriyaṁ sat’indriyaṁ
samādh’indriyaṁ paññ’indriyaṁ,
anaññātañ-ñassāmīt’indriyaṁ aññ’indriyaṁ aññātāv’indriyaṁ.

\begin{english}
  Twenty-two faculties: The eye faculty, ear faculty, nose faculty, tongue faculty, body faculty, mind faculty, faculty of feminity, faculty of masculinity, life faculty, pleasure faculty, pain faculty, happiness faculty, displeasure faculty, equanimity faculty, conviction faculty, energy faculty, mindfulness faculty, concentration faculty, wisdom faculty, the ‘I am knowing the unknown’ faculty, knowledge faculty, the faculty of one with complete knowledge.
\end{english}

\suttaRef{[Vibh]}

Cattāri ariya-saccāni:
Dukkhaṁ ariya-saccaṁ,
Dukkha-samudayo ariya-saccaṁ,
Dukkha-nirodho ariya-saccaṁ,
Dukkha-nirodha-gāminī paṭipadā ariya-saccaṁ.

\begin{english}
  The four noble truths:
  The noble truth of dukkha
  The noble truth of the origin of dukkha
  The noble truth of the cessation of dukkha
  The noble truth of the way leading to the cessation of dukkhā
\end{english}

\suttaRef{[SN 56.24]}

Avijjā-paccayā saṅkhārā,
Saṅkhāra-paccayā viññāṇaṁ,
Viññāṇa-paccayā nāma-rūpaṁ,
Nāma-rūpa-paccayā saḷāyatanaṁ,
Saḷāyatana-paccayā phasso,
Phassa-paccayā vedanā,
Vedanā-paccayā taṇhā,
Taṇhā-paccayā upādānaṁ,
Upādāna-paccayā bhavo,
Bhava-paccayā jāti,
Jāti-paccayā jarā-maraṇaṁ soka-parideva-dukkha-domanass’upāyāsā sambhavanti.
Evam-etassa kevalassa dukkhakkhandhassa samudayo hoti.

Avijjāya tv-eva asesa-virāga-nirodhā, saṅkhāra-nirodho,
Saṅkhāra-nirodhā, viññāṇa-nirodho,
Viññāṇa-nirodhā, nāma-rūpa-nirodho,
Nāma-rūpa-nirodhā, saḷāyatana-nirodho,
Saḷāyatana-nirodhā, phassa-nirodho,
Phassa-nirodhā, vedanā-nirodho,
Vedanā-nirodhā, taṇhā-nirodho,
Taṇhā-nirodhā, upādāna-nirodho,
Upādāna-nirodhā, bhava-nirodho,
Bhava-nirodhā, jāti-nirodho,
Jāti-nirodhā, jarā-maraṇaṁ soka-parideva-dukkha-domanass’upāyāsā nirujjhanti.
Evam-etassa kevalassa dukkhakkhandhassa nirodho hoti.

\begin{english}
  With ignorance as condition, volitional formations;
  With volitional formations as condition, consciousness;
  With consciousness as condition, name-and-form;
  With name-and-form as condition, the six sense bases;
  With the six sense bases as condition, contact;
  With contact as condition, feeling;
  With feeling as condition, craving;
  With craving as condition, clinging;
  With clinging as condition, existence;
  With existence as condition, birth;
  With birth as condition, ageing-and-death, sorrow, lamentation, pain, displeasure, and despair come to be.
  Such is the origin of this whole mass of suffering.

  But with the remainderless fading away and cessation of ignorance comes cessation of volitional formations;
  With the cessation of volitional formations, cessation of consciousness;
  With the cessation of consciousness, cessation of name-and-form;
  With the cessation of name-and-form, cessation of the six sense bases;
  With the cessation of the six sense bases, cessation of contact;
  With the cessation of contact, cessation of feeling;
  With the cessation of feeling, cessation of craving;
  With the cessation of craving, cessation of clinging;
  With the cessation of clinging, cessation of existence;
  With the cessation of existence, cessation of birth;
  With the cessation of birth, ageing-and-death, sorrow, lamentation, pain, displeasure, and despair cease.
  Such is the cessation of this whole mass of suffering.
\end{english}

\suttaRef{[SN 12.1]}

[Hetu-paccayo], ārammaṇa-paccayo,
adhipati-paccayo, anantara-paccayo,
samanantara-paccayo, saha-jāta-paccayo,
aññam-añña-paccayo, nissaya-paccayo,
upanissaya-paccayo, pure-jāta-paccayo,
pacchā-jāta-paccayo, āsevana-paccayo,
kamma-paccayo, vipāka-paccayo,
āhāra-paccayo, indriya-paccayo,
jhāna-paccayo, magga-paccayo,
sampayutta-paccayo, vippayutta-paccayo,
atthi-paccayo, n’atthi-paccayo,
vigata-paccayo, avigata-paccayo.

\begin{english}
  [Root condition], sense-object condition,
  predominant condition, immediate condition,
  directly immediate condition, coexistent condition,
  reciprocity condition, dependence condition,
  sufficing condition, pre-existent condition,
  post-existent condition, repetition condition,
  action condition, result condition,
  nutriment condition, faculty condition,
  jhāna condition, path condition,
  associated condition, separated condition,
  existence condition, non-existence condition,
  disappeared condition, non-dissappeared condition.
\end{english}

\suttaRef{[Dhs A]}

[Adāsi me akāsi me]
Ñāti-mittā sakhā ca me
Petānaṁ dakkhiṇaṁ dajjā
Pubbe katam-anussaraṁ

\begin{english}
  “He gave to me [gifts], he did [things] for me.
  They're my relatives, friends and pals.”
  To the deceased one should give offerings,
  Remembering what was done before.
\end{english}

Na hi ruṇṇaṁ vā soko vā
Yā v’aññā paridevanā
Na taṁ petānam-atthāya
Evaṁ tiṭṭhanti ñātayo

\begin{english}
  For neither weeping nor sorrow,
  Nor any form of lamentation
  Benefits the departed ones.
  Such is how the relatives remain.
\end{english}

Ayañ-ca kho dakkhinā dinnā
Saṅghamhi supatiṭṭhitā
Dīgharattaṁ hitāy’assa
Ṭhānaso upakappati

\begin{english}
  And this offering that has been given
  And firmly established in the Saṅgha,
  Would be for their long-term welfare
  And arrives there immediately.
\end{english}

So ñāti-dhammo ca ayaṁ nidassito
Petāna’pūjā ca katā uḷārā
Balañ-ca bhikkhūnam-anuppadinnaṁ
Tumhehi puññaṁ pasutaṁ anappakan’ti

\begin{english}
  And the duty of relatives has been shown,
  And lofty honouring of the departed done;
  Strength has also been given to the bhikkhus,
  And much merit accumulated by you all.
\end{english}

\suttaRef{[Khp 7]}

[Aniccā vata saṅkhārā]
Uppāda-vaya-dhammino
Uppajjitvā nirujjhanti
Tesaṁ vūpasamo sukho [3x]

\begin{english}
  [Indeed, conditioned things cannot last]
  Their nature is to rise and cease; i
  Having arisen things must cease;
  Their stilling is true happiness.
\end{english}

\suttaRef{[DN 16]}

[Sabbe sattā] maranti ca
Mariṁsu ca marissare
Tath’evāhaṁ marissāmi
N’atthi me ettha saṁsayo [3x]

\begin{english}
  [All living beings] are dying,
  Have died, and will die.
  In the same way, I will die,
  I have no doubt about this.
\end{english}

\suttaRef{[Thai]}

[Aciraṁ vat’ayaṁ kāyo]
Paṭhaviṁ adhisessati
Chuḍḍho apeta-viññāṇo
Niratthaṁ va kaliṅgaraṁ [3x]

\begin{english}
  [All too soon, this body]
  Will lie on the ground cast off,
  Bereft of consciousness,
  Like a useless scrap of wood.
\end{english}

\suttaRef{[Dhp 41]}

[Bhavatu sabba-maṅgalaṁ]
Rakkhantu sabba-devatā
Sabba-buddhānubhāvena
Sadā sotthī bhavantu te

\begin{english}
May every blessing come to be
And all good spirits guard you well
Through the power of all Buddhas
May you always be at ease
\end{english}

Bhavatu sabba-maṅgalaṁ
Rakkhantu sabba-devatā
Sabba-dhammānubhāvena
Sadā sotthī bhavantu te

\begin{english}
May every blessing come to be
And all good spirits guard you well
Through the power of all Dhammas
May you always be at ease
\end{english}

Bhavatu sabba-maṅgalaṁ
Rakkhantu sabba-devatā
Sabba-saṅghānubhāvena
Sadā sotthī  ̓  bhavantu te

\begin{english}
May every blessing come to be
And all good spirits guard you well
Through the power of all Saṅghas
May you always be at ease
\end{english}
