\ifdesktopversion
\chapterOpeningPage{homage-to-the-triple-gem-compressed.jpg}
\else
\chapterOpeningPage{homage-to-the-triple-gem.jpg}
\fi

\chapter{Homage to the Triple Gem}

\begingroup
\setsechook{%
  % New page for each section.
  \clearpage%
  % Empty the default section number printing, so that we can handle it.
  \setsecnumformat{}%
}

\sectionPaliTitle{Buddh'ādi-āmisa-pūjā}
\section{Dedication of Offerings}
\label{dedication-of-offerings}

\begin{pali-leader}
  \anglebracketleft\ \hspace{-0.5mm}Yo so \hspace{-0.5mm}\anglebracketright\
\end{pali-leader}

\begin{pali-hangtogether}
Bhagavā arahaṁ sammāsambuddho
\end{pali-hangtogether}

\linkdest{endnote2-body}
\begin{english-hang}
  To the Blessed One the Worthy One\makeatletter\hyperlink{endnote2-appendix}\Hy@raisedlink{{\pagenote{%\linebreak
        \hypertarget{endnote2-appendix}{\hyperlink{endnote2-body}{WPN: ``The Lord''. The underlying Pāli term is ``\textit{Arahant}''. ``Lord'', however, has connotations that do not fit well to the way the Buddha is portrayed in the discourses. In dictionaries ``lord'' is commonly defined as: ``an appellation for a person or deity who has authority, control, or power over others, acting like a master, a chief, or a ruler''. The ``Worthy One'' seems a better choice of terms, since it is also how ``\textit{Arahant}'' was used in pre-Buddhist era. PTS explains: ``[Vedic \textit{arhant}, ppr. of \textit{arhati} (see \textit{arahati}), meaning deserving, worthy]. Before Buddhism used as honourific title of high officials like the English ``His Worship'' ; at the rise of Buddhism applied popularly to all ascetics (\textit{Dial. III.3–6})''. Throughout this chanting book, all occurrences of ``\textit{Arahant}'' have therefore been consistently translated as ``Worthy One'', thus replacing previous translations, such as ``The Lord'', ``Noble One'' etc.}}}}}\makeatother\thinspace
  who fully attained Perfect Enlightenment
\end{english-hang}

Svākkhāto yena bhagavatā dhammo

\begin{english}
  To the Teaching which he expounded so well
\end{english}

Supaṭipanno yassa bhagavato sāvaka-saṅgho

\begin{english}
  And to the Blessed One's disciples who have practiced well
\end{english}

Tam'mayaṁ bhagavantaṁ sadhammaṁ sasaṅghaṁ

\begin{english}
  To these the Buddha the Dhamma and the Saṅgha
\end{english}

Imehi sakkārehi yath'ārahaṁ āropitehi abhipūjayāma

\begin{english}
  We render with offerings our rightful homage
\end{english}

Sādhu no bhante bhagavā sucira-parinibbuto'pi

\begin{english}
  It is well for us that the Blessed One\\
  Having attained liberation
\end{english}

Pacchimā-janat'ānukampa-mānasā

\begin{english}
  Still had compassion for later generations
\end{english}

Ime sakkāre duggata-paṇṇākāra-bhūte paṭiggaṇhātu

\begin{english}
  May these simple offerings be accepted
\end{english}

Amhākaṁ dīgharattaṁ hitāya sukhāya

\begin{english}
  For our long-lasting benefit and for the happiness it gives us
\end{english}

\clearpage

\begin{leader}
  \anglebracketleft\ \hspace{-0.5mm}Arahaṁ \hspace{-0.5mm}\anglebracketright\
\end{leader}

\vspace{-0.5cm}

Sammāsambuddho bhagavā

\begin{english}
  The Worthy One the Perfectly Enlightened and Blessed One
\end{english}

Buddhaṁ bhagavantaṁ abhivādemi\relax

\begin{english}
  I render homage to the Buddha the Blessed One \hfill{(Bow)}
\end{english}

\begin{leader}
  \anglebracketleft\ \hspace{-0.5mm}Svākkhāto \hspace{-0.5mm}\anglebracketright\
\end{leader}

\vspace{-0.5cm}

Bhagavatā dhammo

\begin{english}
  The Teaching so completely explained by him
\end{english}

Dhammaṁ namassāmi\relax

\begin{english}
  I bow to the Dhamma \hfill{(Bow)}
\end{english}

\begin{leader}
  \anglebracketleft\ \hspace{-0.5mm}Supaṭipanno \hspace{-0.5mm}\anglebracketright\
\end{leader}

\vspace{-0.5cm}

Bhagavato sāvaka-saṅgho

\begin{english}
  The Blessed One's disciples who have practiced well
\end{english}

Saṅghaṁ namāmi

\begin{english}
  I bow to the Saṅgha \hfill{(Bow)}
\end{english}

\suttaRef{[Trad]}

\sectionPaliTitle{Pubbabhāga-namakāra}
\section{Preliminary Homage}
\label{preliminary-homage}

\begin{leader}
  \anglebracketleft\ \hspace{-0.5mm}Handa mayaṁ buddhassa bhagavato pubbabhāga-namakāraṁ karomase \hspace{-0.5mm}\anglebracketright\
\end{leader}

\begin{leader-english-belowpali}
  \anglebracketleft\ \hspace{-0.5mm}Now let us pay preliminary homage to the Buddha \hspace{-0.5mm}\anglebracketright\
\end{leader-english-belowpali}

Namo tassa bhagavato arahato sammāsambuddhassa \hfill{[3x]}

\begin{english}
  Homage to the Blessed Worthy and Perfectly Enlightened One \hfill{[3x]}
\end{english}

\suttaRef{[DN 21]}

\sectionPaliTitle{Buddh'ābhitthuti}
\section{Homage to the Buddha}
\label{homage-buddha}

\begin{leader}
  \anglebracketleft\ \hspace{-0.5mm}Handa mayaṁ buddh'ābhitthutiṁ karomase \hspace{-0.5mm}\anglebracketright\
\end{leader}
\begin{leader-english-belowpali}
  \anglebracketleft\ \hspace{-0.5mm}Now let us recite in praise of the Buddha \hspace{-0.5mm}\anglebracketright\
\end{leader-english-belowpali}

Yo so tathāgato arahaṁ sammāsambuddho

\begin{english}
  The Tathāgata is the Worthy One the Perfectly Enlightened One
\end{english}

Vijjācaraṇa-sampanno

\begin{english}
  He is impeccable in conduct and understanding
\end{english}

Sugato

\begin{english}
  The Accomplished One
\end{english}

Lokavidū

\begin{english}
  The Knower of the Worlds
\end{english}

Anuttaro purisadamma-sārathi

\linkdest{endnote3-body}
\begin{english}
  Unsurpassed leader of persons to be tamed\makeatletter\hyperlink{endnote3-appendix}\Hy@raisedlink{{\pagenote{%
        \hypertarget{endnote3-appendix}{\hyperlink{endnote3-body}{WPN: ``He trains perfectly those who wish to be trained''. The aspect of wishing to be trained is not found in the \textit{Pāli}.}}}}}\makeatother
\end{english}

Satthā deva-manussānaṁ

\begin{english}
  He is teacher of gods and humans
\end{english}

Buddho bhagavā

\begin{english}
  He is awake and holy
\end{english}

Yo imaṁ lokaṁ sadevakaṁ samārakaṁ sabrahmakaṁ

\begin{english}
  In this world with its gods \breathmark\ demons and kind spirits
\end{english}

\begin{pali-hang}
  Sassamaṇa-brāhmaṇiṁ pajaṁ sadeva-manussaṁ sayaṁ abhiññā sacchikatvā pavedesi
\end{pali-hang}

\begin{english}
  Its seekers and sages \breathmark\ celestial and human beings\\
  He has by deep insight revealed the truth
\end{english}

\begin{pali-hang}
  Yo dhammaṁ desesi ādi-kalyāṇaṁ majjhe-kalyāṇaṁ pariyosāna-kalyāṇaṁ
\end{pali-hang}

\begin{english-verses}
  He has pointed out the Dhamma\\
  Beautiful in the beginning\\
  Beautiful in the middle\\
  Beautiful in the end\\
\end{english-verses}

\begin{pali-hang}
  Sātthaṁ sabyañjanaṁ kevala-paripuṇṇaṁ parisuddhaṁ brahmacariyaṁ pakāsesi
\end{pali-hang}

\linkdest{endnote4-body}
\begin{english}
  He has explained the holy life of complete purity\makeatletter\hyperlink{endnote4-appendix}\Hy@raisedlink{{\pagenote{%
        \hypertarget{endnote4-appendix}{\hyperlink{endnote4-body}{WPN: ``He has explained the spiritual life of complete purity''. While ``spiritual life'' is not a bad translation, for the sake of consistency with the rest of the chanting book, this occurrence was changed to ``holy life''.}}}}}\makeatother\\
  In its essence and conventions
\end{english}

\suttaRef{[SN 55.7]}

\begin{pali-hang}
  Tam'ahaṁ bhagavantaṁ abhipūjayāmi tam'ahaṁ bhagavantaṁ sirasā namāmi
\end{pali-hang}

\begin{english}
  I chant my praise to the Blessed One\\
  I bow my head to the Blessed One \hfill{(Bow)}
\end{english}

\suttaRef{[Thai]}

\sectionPaliTitle{Dhamm'ābhitthuti}
\section{Homage to the Dhamma}
\label{homage-dhamma}

\begin{leader}
  \anglebracketleft\ \hspace{-0.5mm}Handa mayaṁ dhamm'ābhitthutiṁ karomase \hspace{-0.5mm}\anglebracketright\
\end{leader}
\begin{leader-english-belowpali}
  \anglebracketleft\ \hspace{-0.5mm}Now let us recite in praise of the Dhamma \hspace{-0.5mm}\anglebracketright\
\end{leader-english-belowpali}

Yo so svākkhāto bhagavatā dhammo

\begin{english}
  The Dhamma is well-expounded by the Blessed One
\end{english}

Sandiṭṭhiko

\begin{english}
  Apparent here and now
\end{english}

Akāliko

\begin{english}
  Timeless
\end{english}

Ehipassiko

\begin{english}
  Encouraging investigation
\end{english}

Opanayiko

\begin{english}
  Leading inwards
\end{english}

Paccattaṁ veditabbo viññūhi

\begin{english}
  To be experienced individually by the wise
\end{english}

\suttaRef{[SN 12.41]}

\begin{pali-hang}
  Tam'ahaṁ dhammaṁ abhipūjayāmi tam'ahaṁ dhammaṁ sirasā namāmi
\end{pali-hang}

\begin{english}
  I chant my praise to this teaching\\
  I bow my head to this truth \hfill{(Bow)}
\end{english}

\suttaRef{[Trad]}

\sectionPaliTitle{Saṅgh'ābhitthuti}
\section{Homage to the Saṅgha}
\label{homage-sangha}

\begin{leader}
  \anglebracketleft\ \hspace{-0.5mm}Handa mayaṁ saṅgh'ābhitthutiṁ karomase \hspace{-0.5mm}\anglebracketright\
\end{leader}
\begin{leader-english-belowpali}
  \anglebracketleft\ \hspace{-0.5mm}Now let us recite in praise of the Saṅgha \hspace{-0.5mm}\anglebracketright\
\end{leader-english-belowpali}

Yo so supaṭipanno bhagavato sāvaka-saṅgho

\begin{english}
  They are the Blessed One's disciples who have practiced well
\end{english}

Uju-paṭipanno bhagavato sāvaka-saṅgho

\linkdest{endnote5-body}
\begin{english}
    Who have practiced directly\makeatletter\hyperlink{endnote5-appendix}\Hy@raisedlink{{\pagenote{%
        \hypertarget{endnote5-appendix}{\hyperlink{endnote5-body}{To practice ``directly''(Pāli: \textit{uju}) means, to practice the most direct way to \textit{nibbāna}; the straight way; no detours.}}}}}\makeatother

\end{english}

Ñāya-paṭipanno bhagavato sāvaka-saṅgho

\linkdest{endnote6-body}
\begin{english}
    Who have practiced correctly\makeatletter\hyperlink{endnote6-appendix}\Hy@raisedlink{{\pagenote{%
      \hypertarget{endnote6-appendix}{\hyperlink{endnote6-body}{WPN: ``Who have practiced insightfully''}}}}}\makeatother

\end{english}

Sāmīci-paṭipanno bhagavato sāvaka-saṅgho

\linkdest{endnote7-body}
\begin{english}
  Who have practiced properly\makeatletter\hyperlink{endnote7-appendix}\Hy@raisedlink{{\pagenote{%
        \hypertarget{endnote7-appendix}{\hyperlink{endnote7-body}{WPN: ``Those who practice with integrity''}}}}}\makeatother

\end{english}

Yad'idaṁ cattāri purisa-yugāni aṭṭha purisa-puggalā

\begin{english}
  That is the four pairs the eight kinds of Noble Beings
\end{english}

Esa bhagavato sāvaka-saṅgho

\begin{english}
  These are the Blessed One's disciples
\end{english}

Āhuneyyo

\begin{english}
  Such ones are worthy of gifts
\end{english}

Pāhuneyyo

\begin{english}
  Worthy of hospitality
\end{english}

Dakkhiṇeyyo

\begin{english}
  Worthy of offerings
\end{english}

Añjali-karaṇīyo

\begin{english}
  Worthy of respect
\end{english}

Anuttaraṁ puñña-kkhettaṁ lokassa

\linkdest{endnote149-body}
\begin{english}
 The unsurpassed field of merit for the world\makeatletter\hyperlink{endnote149-appendix}\Hy@raisedlink{{\pagenote{%
        \hypertarget{endnote149-appendix}{\hyperlink{endnote149-body}{WPN: ``They give occasion for incomparable goodness to arise in the world''}}}}}\makeatother
\end{english}

\suttaRef{[SN 12.41]}

\begin{pali-hang}
  Tam'ahaṁ saṅghaṁ abhipūjayāmi tam'ahaṁ saṅghaṁ sirasā namāmi
\end{pali-hang}

\begin{english}
  I chant my praise to this Saṅgha\\
  I bow my head to this Saṅgha \hfill{(Bow)}
\end{english}

\sectionPaliTitle{Ratanattaya-paṇāma}
\section{Salutation to the Triple Gem}
\label{salutation}

\begin{leader}
  \anglebracketleft\ \hspace{-0.5mm}Handa mayaṁ ratanattaya-paṇāma-gāthāyo c'eva saṁvega-parikittana-pāṭhañ'ca bhaṇāmase \hspace{-0.5mm}\anglebracketright\
\end{leader}
\begin{leader-english-belowpali}
  \anglebracketleft\ \hspace{-0.5mm}Now let us recite our salutation to the Triple Gem and a passage to arouse urgency \hspace{-0.5mm}\anglebracketright\
\end{leader-english-belowpali}

Buddho susuddho karuṇā-mah'aṇṇavo

\begin{english}
  The Buddha absolutely pure with ocean-like compassion
\end{english}

Yo'ccanta-suddhabbara-ñāṇa-locano

\begin{english}
  Possessing the clear sight of wisdom
\end{english}

Lokassa pāp'ūpakilesa-ghātako

\begin{english}
  Destroyer of worldly self-corruption
\end{english}

Vandāmi buddhaṁ aham'ādarena taṁ

\begin{english}
  Devotedly indeed \breathmark\ that Buddha I revere
\end{english}

Dhammo padīpo viya tassa satthuno

\linkdest{endnote8-body}
\begin{english}
  The Teaching of the Lord is like a lamp\makeatletter\hyperlink{endnote8-appendix}\Hy@raisedlink{{\pagenote{%
        \hypertarget{endnote8-appendix}{\hyperlink{endnote8-body}{WPN: ``The teaching of the Lord like a lamp''}}}}}\makeatother
\end{english}

Yo magga-pāk'āmata-bheda-bhinnako

\linkdest{endnote9-body}
\begin{english}
  Divided into path and its fruit \breathmark\ the Deathless\makeatletter\hyperlink{endnote9-appendix}\Hy@raisedlink{{\pagenote{%
        \hypertarget{endnote9-appendix}{\hyperlink{endnote9-body}{WPN: ``Illuminating the path and its fruit, the Deathless''}}}}}\makeatother
\end{english}

Lok'uttaro yo ca tad'attha-dīpano

\linkdest{endnote10-body}
\begin{english}
  And illuminating that goal \breathmark\ which is beyond the conditioned world\makeatletter\hyperlink{endnote10-appendix}\Hy@raisedlink{{\pagenote{%
        \hypertarget{endnote10-appendix}{\hyperlink{endnote10-body}{WPN: ``That which is beyond the conditioned world''}}}}}\makeatother
\end{english}

Vandāmi dhammaṁ aham'ādarena taṁ

\begin{english}
  Devotedly indeed \breathmark\ that Dhamma I revere
\end{english}

Saṅgho sukhett'ābhyati-khetta-saññito

\begin{english}
  The Saṅgha the most fertile ground for cultivation
\end{english}

Yo diṭṭha-santo sugat'ānubodhako

\begin{english}
  Those who have realised peace\\
  Awakened after the Accomplished One
\end{english}

Lolappahīno ariyo sumedhaso

\begin{english}
  Noble and wise \breathmark\ all longing abandoned
\end{english}

Vandāmi saṅghaṁ aham'ādarena taṁ

\begin{english}
  Devotedly indeed \breathmark\ that Saṅgha I revere
\end{english}

Icc'evam'ekant'abhipūjaneyyakaṁ\\
vatthu-ttayaṁ vandayat'ābhisaṅkhataṁ

\linkdest{endnote11-body}
\begin{english}
  This salutation should be made\\
  To that triad\makeatletter\hyperlink{endnote11-appendix}\Hy@raisedlink{{\pagenote{%
        \hypertarget{endnote11-appendix}{\hyperlink{endnote11-body}{WPN: ``To that which is worthy''. This passage refers to the triple (\textit{taya}) gems and not just to the \textit{Saṅgha}.}}}}}\makeatother
  which is worthy
\end{english}

Puññaṁ mayā yaṁ mama sabb'upaddavā

\begin{english}
  Through the power of such good action
\end{english}

Mā hontu ve tassa pabhāva-siddhiyā

\begin{english}
  May all obstacles disappear
\end{english}

Idha tathāgato loke uppanno arahaṁ sammāsambuddho

\linkdest{endnote12-body}
\begin{english}
  One who knows things as they are \breathmark\ has arisen in this world\makeatletter\hyperlink{endnote12-appendix}\Hy@raisedlink{{\pagenote{%
        \hypertarget{endnote12-appendix}{\hyperlink{endnote12-body}{``One who knows things as they are'' is an unusual translation for \textit{Tathāgata}. Also ``arisen in'' is better than ``has come into'', otherwise one might think that he has come from somewhere, already being a \textit{Tathāgata}.}}}}}\makeatother\\

  And he is an \textit{Arahant} \breathmark\ a perfectly awakened being
\end{english}

\begin{pali-hang}
  Dhammo ca desito niyyāniko upasamiko parinibbāniko sambodhagāmī sugata-ppavedito
\end{pali-hang}

\linkdest{endnote13-body}
\begin{english-verses}
  Teaching the way leading out of delusion\makeatletter\hyperlink{endnote13-appendix}\Hy@raisedlink{{\pagenote{%
        \hypertarget{endnote13-appendix}{\hyperlink{endnote13-body}{No mention of ``delusion'' in the Pāli. It could also refer to \textit{samsāra} or \textit{dukkha}.}}}}}\makeatother\\
  Calming and directing to perfect peace\\
  And leading to enlightenment\\
  This way he has made known\\
\end{english-verses}

Mayan'taṁ dhammaṁ sutvā evaṁ jānāma

\begin{english}
  Having heard the Teaching we know this
\end{english}

\suttaRef{[Thai]}

Jāti'pi dukkhā

\begin{english}
  Birth is dukkha
\end{english}

Jarā'pi dukkhā

\begin{english}
  Ageing is dukkha
\end{english}

Maraṇam'pi dukkhaṁ

\begin{english}
  And death is dukkha
\end{english}

Soka-parideva-dukkha-domanass'upāyāsā'pi dukkhā

\linkdest{endnote14-body}
\begin{english}
  Sorrow lamentation pain displeasure\makeatletter\hyperlink{endnote14-appendix}\Hy@raisedlink{{\pagenote{%
        \hypertarget{endnote14-appendix}{\hyperlink{endnote14-body}{WPN: ``grief''}}}}}\makeatother
  and despair are dukkha
\end{english}

Appiyehi sampayogo dukkho

\begin{english}
  Association with the disliked is dukkha
\end{english}

Piyehi vippayogo dukkho

\begin{english}
  Separation from the liked is dukkha
\end{english}

Yam'p'icchaṁ na labhati tam'pi dukkhaṁ

\begin{english}
  Not attaining one's wishes is dukkha
\end{english}

\suttaRef{[SN 56.11]}

Saṅkhittena pañc'upādāna-kkhandhā dukkhā

\linkdest{endnote15-body}
\begin{english}
  In brief \breathmark\ the five aggregates of clinging are dukkha\makeatletter\hyperlink{endnote15-appendix}\Hy@raisedlink{{\pagenote{%
        \hypertarget{endnote15-appendix}{\hyperlink{endnote15-body}{WPN: ``In brief the five focuses of identity are dukkha''}}}}}\makeatother
\end{english}

Seyyath'īdaṁ

\begin{english}
  These are as follows
\end{english}

Rūp'ūpādāna-kkhandho

\begin{english}
  Attachment to form
\end{english}

Vedan'ūpādāna-kkhandho

\begin{english}
  Attachment to feeling
\end{english}

Saññ'ūpādāna-kkhandho

\begin{english}
  Attachment to perception
\end{english}

Saṅkhār'ūpādāna-kkhandho

\linkdest{endnote16-body}
\begin{english}
  Attachment to volitional formations\makeatletter\hyperlink{endnote16-appendix}\Hy@raisedlink{{\pagenote{%
        \hypertarget{endnote16-appendix}{\hyperlink{endnote16-body}{WPN: ``Attachment to mental formations''. While the Pāli term ``saṅkhārā'' only means ``formations'', Bhikkhu Bodhi’s rendering as ``volitional formations'' captures well the intentional/volitional forces behind the formative nature of the mind.}}}}}\makeatother
\end{english}

Viññāṇ'ūpādāna-kkhandho

\linkdest{endnote17-body}
\begin{english}
  Attachment to consciousness\makeatletter\hyperlink{endnote17-appendix}\Hy@raisedlink{{\pagenote{%
        \hypertarget{endnote17-appendix}{\hyperlink{endnote17-body}{WPN: ``Attachment to sense-consciousness''}}}}}\makeatother
\end{english}

\suttaRef{[DN 22]}

Yesaṁ pariññāya

\begin{english}
  For the complete understanding of this
\end{english}

Dharamāno so bhagavā

\begin{english}
  The Blessed One in his lifetime
\end{english}

Evaṁ bahulaṁ sāvake vineti

\begin{english}
  Frequently instructed his disciples in just this way
\end{english}

\begin{pali-hang}
  Evaṁ bhāgā ca panassa bhagavato sāvakesu anusāsanī bahulā pavattati
\end{pali-hang}

\begin{english}
  In addition he further instructed
\end{english}

\suttaRef{[Thai]}

Rūpaṁ aniccaṁ

\begin{english}
  Form is impermanent
\end{english}

Vedanā aniccā

\begin{english}
  Feeling is impermanent
\end{english}

Saññā aniccā

\begin{english}
  Perception is impermanent
\end{english}

Saṅkhārā aniccā

\linkdest{endnote18-body}
\begin{english}
  Volitional formations are impermanent\makeatletter\hyperlink{endnote18-appendix}\Hy@raisedlink{{\pagenote{%
        \hypertarget{endnote18-appendix}{\hyperlink{endnote18-body}{WPN: ``Mental formations are impermanent''}}}}}\makeatother
\end{english}

Viññāṇaṁ aniccaṁ

\linkdest{endnote19-body}
\begin{english}
  Consciousness is impermanent\makeatletter\hyperlink{endnote19-appendix}\Hy@raisedlink{{\pagenote{%
        \hypertarget{endnote19-appendix}{\hyperlink{endnote19-body}{WPN: ``Sense-consciousness is impermanent''}}}}}\makeatother
\end{english}

Rūpaṁ anattā

\begin{english}
  Form is not-self
\end{english}

Vedanā anattā

\begin{english}
  Feeling is not-self
\end{english}

Saññā anattā

\begin{english}
  Perception is not-self
\end{english}

Saṅkhārā anattā

\linkdest{endnote20-body}
\begin{english}
  Volitional formations are not-self\makeatletter\hyperlink{endnote20-appendix}\Hy@raisedlink{{\pagenote{%
        \hypertarget{endnote20-appendix}{\hyperlink{endnote20-body}{WPN: ``Mental formations are not-self''}}}}}\makeatother
\end{english}

Viññāṇaṁ anattā

\linkdest{endnote21-body}
\begin{english}
  Consciousness is not-self\thinspace\makeatletter\hyperlink{endnote21-appendix}\Hy@raisedlink{{\pagenote{%
        \hypertarget{endnote21-appendix}{\hyperlink{endnote21-body}{WPN: ``Sense-consciousness is not-self''}}}}}\makeatother
\end{english}

Sabbe saṅkhārā aniccā

\linkdest{endnote22-body}
\begin{english}
  All conditioned things are impermanent\makeatletter\hyperlink{endnote22-appendix}\Hy@raisedlink{{\pagenote{%
        \hypertarget{endnote22-appendix}{\hyperlink{endnote22-body}{WPN: ``All conditions are transient''}}}}}\makeatother
\end{english}

Sabbe dhammā anattā'ti

\linkdest{endnote23-body}
\begin{english}
  All things are not-self\thinspace\makeatletter\hyperlink{endnote23-appendix}\Hy@raisedlink{{\pagenote{%
        \hypertarget{endnote23-appendix}{\hyperlink{endnote23-body}{WPN: ``There is no self in the created or the uncreated''. While this is not a very accurate translation, it is indeed the case that the term ``\textit{sabbe dhammā}'' includes the uncreated, \textit{nibbāna} (see AN 5.32).}}}}}\makeatother
\end{english}

\suttaRef{[MN 35]}

Te mayaṁ otiṇṇ'āmha jātiyā jarā-maraṇena

\linkdest{endnote24-body}
\begin{english}
  All of us are affected by birth \breathmark\ ageing and death\makeatletter\hyperlink{endnote24-appendix}\Hy@raisedlink{{\pagenote{%
        \hypertarget{endnote24-appendix}{\hyperlink{endnote24-body}{WPN: ``All of us are bound by birth ageing and death''}}}}}\makeatother
\end{english}

Sokehi paridevehi dukkhehi domanassehi upāyāsehi

\linkdest{endnote25-body}
\linkdest{endnote26-body}
\begin{english}
  By sorrow lamentation pain displeasure\makeatletter\hyperlink{endnote25-appendix}\Hy@raisedlink{{\pagenote{%
        \hypertarget{endnote25-appendix}{\hyperlink{endnote25-body}{WPN: ``grief''}}}}}\makeatother
  and despair\makeatletter\hyperlink{endnote26-appendix}\Hy@raisedlink{{\pagenote{%
        \hypertarget{endnote26-appendix}{\hyperlink{endnote26-body}{In Pāli, these terms are in plural form, however, for the sake recitation they are kept singular.}}}}}\makeatother
\end{english}

Dukkh'otiṇṇā dukkha-paretā

\linkdest{endnote27-body}
\begin{english}
  Affected by dukkha and afflicted by dukkha\makeatletter\hyperlink{endnote27-appendix}\Hy@raisedlink{{\pagenote{%
        \hypertarget{endnote27-appendix}{\hyperlink{endnote27-body}{WPN: ``All of us are bound by birth ageing and death''}}}}}\makeatother
\end{english}

\begin{pali-hang}
  App'evanām'imassa kevalassa dukkha-kkhandhassa antakiriyā paññāyethā'ti
\end{pali-hang}

\begin{english}
  Let us all aspire to complete freedom from suffering
\end{english}

\suttaRef{[SN 22.80]}

\clearpage

\begin{center}
  \textit{\textbf{(The following is recited only by the bhikkhus)}}
\end{center}

\begin{pali-hang}
  Cira-parinibbutam'pi taṁ bhagavantaṁ uddissa arahantaṁ sammāsambuddhaṁ
\end{pali-hang}

\begin{english-hang}
  Remembering the Blessed One \breathmark\ the Worthy One \breathmark\ and Perfectly Enlightened One\\
\end{english-hang}

\begin{english}
  Who long ago attained Parinibbāna
\end{english}

Saddhā agārasmā anagāriyaṁ pabbajitā

\begin{english}
  We have gone forth with faith\\
  From home to homelessness
\end{english}

Tasmiṁ bhagavati brahmacariyaṁ carāma

\begin{english}
  And like the Blessed One \breathmark\ we practice the holy life
\end{english}

Bhikkhūnaṁ sikkhā-sājīva-samāpannā

\linkdest{endnote28-body}
\begin{english}
  Possessing the bhikkhus' training and way of life\makeatletter\hyperlink{endnote28-appendix}\Hy@raisedlink{{\pagenote{%
        \hypertarget{endnote28-appendix}{\hyperlink{endnote28-body}{WPN: ``Being fully equipped with the bhikkhus'system of training''}}}}}\makeatother
\end{english}

\begin{pali-hang}
  Taṁ no brahmacariyaṁ imassa kevalassa dukkha-kkhandhassa antakiriyāya saṁvattatu
\end{pali-hang}

\begin{english}
  May this holy life \breathmark\ lead us to the end of this whole mass of suffering
\end{english}

\suttaRef{[Thai]}

\bottomNav{universal-well-being}

\sectionPaliTitle{Pacchima-vandanā}
\section{Closing Homage}
\label{closing-homage}

\vspace{5pt}

\begin{leader}
  \anglebracketleft\ \hspace{-0.5mm}Arahaṁ \hspace{-0.5mm}\anglebracketright\
\end{leader}

\vspace{-0.5cm}

Sammāsambuddho bhagavā

\begin{english}
  The Worthy One the Perfectly Enlightened and Blessed One
\end{english}

Buddhaṁ bhagavantaṁ abhivādemi

\begin{english}
  I render homage to the Buddha the Blessed One \hfill{(Bow)}
\end{english}

\begin{leader}
  \anglebracketleft\ \hspace{-0.5mm}Svākkhāto \hspace{-0.5mm}\anglebracketright\
\end{leader}

\vspace{-0.5cm}

Bhagavatā dhammo

\begin{english}
  The Teaching so completely explained by him
\end{english}

Dhammaṁ namassāmi

\begin{english}
  I bow to the Dhamma \hfill{(Bow)}
\end{english}

\begin{leader}
  \anglebracketleft\ \hspace{-0.5mm}Supaṭipanno \hspace{-0.5mm}\anglebracketright\
\end{leader}

\vspace{-0.5cm}

Bhagavato sāvaka-saṅgho

\begin{english}
  The Blessed One's disciples who have practiced well
\end{english}

Saṅghaṁ namāmi

\begin{english}
  I bow to the Saṅgha \hfill{(Bow)}\\
\end{english}

\suttaRef{[Thai]}

\null
\vfill

\ifdesktopversion
\begin{minipage}[b][25pt][c]{\linewidth}
  \begin{leader}
    \textbf{\vspace{0.2em}\textsc{\hyperref[schedule]{Schedule}\\
        % \rule{\linewidth}{0.8pt}
        {\centering\pgfornament[color=sbs-brown,width=4cm]{88}}\\
        \vspace{0.8em}
        \hyperref[buddhas-first-exclamation]{Set 1} \hspace{0.02cm} — \hspace{0.02cm} \hyperref[characteristic-of-not-self]{Set 2} \hspace{0.02cm} — \hspace{0.02cm} \hyperref[noble-eightfold-path]{Set 3} \hspace{0.02cm} — \hspace{0.02cm} \hyperref[dedication-of-offerings]{Set 4} \hspace{0.02cm} — \hspace{0.02cm} \hyperref[mindfulness-of-breathing]{Set 5}\\
        \vspace{0.5em}
        \hyperref[anatta-lakkhana]{Set 6} — \hyperref[dependent-origination]{Set 7} — \hyperref[aditta-pariyaya]{Set 8} — \hyperref[deva-aradhana]{Set 9} — \hyperref[pubba-bhaga-nama-kara-patho-funeral]{Set 10}}}
  \end{leader}
\end{minipage}
\fi
