\chapterOpeningPage{morning-chanting.pdf}

\chapter{Homage to the Triple Gem}

\begingroup
\setsechook{%
  % New page for each section.
  \clearpage%
  % Empty the default section number printing, so that we can handle it.
  \setsecnumformat{}%
}

\section{Dedication of Offerings}
\label{dedication-of-offerings}

[Yo so] bhagavā arahaṁ sammāsambuddho

\begin{english-hang}
  To the Blessed One the Worthy One\hyperlink{endnote2-appendix}{\hypertarget{endnote2-body}{\pagenote{%
        \hypertarget{endnote2-appendix}{\hyperlink{endnote2-body}{Orig: ``The Lord''. The underlying Pāli term is ``\textit{Arahant}''. ``Lord'', however, has connotations that do not fit well to the way the Buddha is portrayed in the discourses. In dictionaries ``lord'' is commonly defined as: ``\textit{an appellation for a person or deity who has authority, control, or power over others, acting like a master, a chief, or a ruler.'' The ``Worthy One'' seems a better choice of terms, since it is also how \textit``Arahant'' was used in pre-Buddhist era. PTS explains: ``[Vedic arhant, ppr. of arhati (see arahati), meaning deserving, worthy]. Before Buddhism used as honourific title of high officials like the English "His Worship" ; at the rise of Buddhism applied popularly to all ascetics (Dial. III.3–6).''} Throughout this chanting book, all occurrences of ``\textit{Arahant}'' have therefore been consistently translated as ``Worthy One'', thus substituting previous translations as ``The Lord'', ``Noble One'' etc.}}}}}
  who fully attained Perfect Enlightenment
\end{english-hang}

Svākkhāto yena bhagavatā dhammo

\begin{english}
  To the Teaching which he expounded so well
\end{english}

Supaṭipanno yassa bhagavato sāvakasaṅgho

\begin{english}
  And to the Blessed One's disciples who have practiced well
\end{english}

Tam-mayaṁ bhagavantaṁ sadhammaṁ sasaṅghaṁ

\begin{english}
  To these the Buddha the Dhamma and the Saṅgha
\end{english}

Imehi sakkārehi yathārahaṁ āropitehi abhipūjayāma

\begin{english}
  We render with offerings our rightful homage
\end{english}

Sādhu no bhante bhagavā sucira-parinibbutopi

\begin{english}
  It is well for us that the Blessed One\\
  Having attained liberation
\end{english}

Pacchimā-janatānukampa-mānasā

\begin{english}
  Still had compassion for later generations
\end{english}

Ime sakkāre duggata-paṇṇākāra-bhūte paṭiggaṇhātu

\begin{english}
  May these simple offerings be accepted
\end{english}

Amhākaṁ dīgharattaṁ hitāya sukhāya

\begin{english}
  For our long-lasting benefit and for the happiness it gives us
\end{english}

\clearpage

Arahaṁ sammāsambuddho bhagavā

\begin{english}
  The Worthy One the Perfectly Enlightened and Blessed One
\end{english}

Buddhaṁ bhagavantaṁ abhivādemi\relax

\begin{english}
  I render homage to the Buddha the Blessed One \hfill{(Bow)}
\end{english}

[Svākkhāto] bhagavatā dhammo

\begin{english}
  The Teaching so completely explained by him
\end{english}

Dhammaṁ namassāmi\relax

\begin{english}
  I bow to the Dhamma \hfill{(Bow)}
\end{english}

[Supaṭipanno] bhagavato sāvakasaṅgho

\begin{english}
  The Blessed One's disciples who have practiced well
\end{english}

Saṅghaṁ namāmi

\begin{english}
  I bow to the Saṅgha \hfill{(Bow)}
\end{english}

\section{Preliminary Homage}
\label{preliminary-homage}

\begin{leader}
  〈 Handa mayaṁ buddhassa bhagavato pubbabhāga-namakāraṁ karomase 〉
\end{leader}

\begin{leader-english}
  〈 Now let us pay preliminary homage to the Buddha 〉
\end{leader-english}

Namo tassa bhagavato arahato sammāsambuddhassa \hfill{[3x]}

\begin{english}
  Homage to the Blessed Worthy and Perfectly Enlightened One \hfill{[3x]}
\end{english}

\section{Homage to the Buddha}
\label{homage-buddha}

\begin{leader}
  〈 Handa mayaṁ buddhābhitthutiṁ karomase 〉
\end{leader}
\begin{leader-english}
  〈 Now let us recite in praise of the Buddha 〉
\end{leader-english}

Yo so tathāgato arahaṁ sammāsambuddho

\begin{english}
  The Tathāgata is the Worthy One the Perfectly Enlightened One
\end{english}

Vijjācaraṇa-sampanno

\begin{english}
  He is impeccable in conduct and understanding
\end{english}

Sugato

\begin{english}
  The Accomplished One
\end{english}

Lokavidū

\begin{english}
  The Knower of the Worlds
\end{english}

Anuttaro purisadamma-sārathi

\begin{english}
  Unsurpassed leader of persons to be tamed\hyperlink{endnote3-appendix}{\hypertarget{endnote3-body}{\pagenote{%
    \hypertarget{endnote3-appendix}{\hyperlink{endnote3-body}{Orig: ``He trains perfectly those who wish to be trained''. The aspect of wishing to be trained is not found in the Pāli.}}}}}
\end{english}

Satthā deva-manussānaṁ

\begin{english}
  He is teacher of gods and humans
\end{english}

Buddho bhagavā

\begin{english}
  He is awake and holy
\end{english}

Yo imaṁ lokaṁ sadevakaṁ samārakaṁ sabrahmakaṁ

\begin{english}
  In this world with its gods ̓ demons and kind spirits
\end{english}

\begin{pali-hang}
  Sassamaṇa-brāhmaṇiṁ pajaṁ sadeva-manussaṁ sayaṁ abhiññā sacchikatvā pavedesi
\end{pali-hang}

\begin{english}
  Its seekers and sages \breathmark\ celestial and human beings\\
  He has by deep insight revealed the truth
\end{english}

\begin{pali-hang}
Yo dhammaṁ desesi ādi-kalyāṇaṁ majjhe-kalyāṇaṁ pariyosāna-kalyāṇaṁ
\end{pali-hang}

\begin{english}
  He has pointed out the Dhamma\\
  Beautiful in the beginning\\
  Beautiful in the middle\\
  Beautiful in the end\\
\end{english}

\begin{pali-hang}
Sātthaṁ sabyañjanaṁ kevala-paripuṇṇaṁ parisuddhaṁ brahma-cariyaṁ pakāsesi
\end{pali-hang}

\begin{english}
  He has explained the holy life of complete purity\hyperlink{endnote4-appendix}{\hypertarget{endnote4-body}{\pagenote{%
    \hypertarget{endnote4-appendix}{\hyperlink{endnote4-body}{Orig: ``He has explained the spiritual life of complete purity''. While ``spiritual life'' is not a bad translation, for the sake of consistency with the rest of the chanting book, this occurrence was changed to ``holy life''}}}}}\\
  In its essence and conventions
\end{english}

\begin{pali-hang}
Tam-ahaṁ bhagavantaṁ abhipūjayāmi tam-ahaṁ bhagavantaṁ sirasā namāmi
\end{pali-hang}

\begin{english}
  I chant my praise to the Blessed One\\
  I bow my head to the Blessed One \hfill{(Bow)}
\end{english}

\section{Homage to the Dhamma}
\label{homage-dhamma}

\begin{leader}
  〈 Handa mayaṁ dhammābhitthutiṁ karomase 〉
\end{leader}
\begin{leader-english}
  〈 Now let us recite in praise of the Dhamma 〉
\end{leader-english}

Yo so svākkhāto bhagavatā dhammo

\begin{english}
  The Dhamma is well-expounded by the Blessed One
\end{english}

Sandiṭṭhiko

\begin{english}
  Apparent here and now
\end{english}

Akāliko

\begin{english}
  Timeless
\end{english}

Ehipassiko

\begin{english}
  Encouraging investigation
\end{english}

Opanayiko

\begin{english}
  Leading inwards
\end{english}

Paccattaṁ veditabbo viññūhi

\begin{english}
  To be experienced individually by the wise
\end{english}

\begin{pali-hang}
Tam-ahaṁ dhammaṁ abhipūjayāmi tam-ahaṁ dhammaṁ sirasā namāmi
\end{pali-hang}

\begin{english}
  I chant my praise to this teaching\\
  I bow my head to this truth \hfill{(Bow)}
\end{english}

\section{Homage to the Saṅgha}
\label{homage-sangha}

\begin{leader}
  〈 Handa mayaṁ saṅghābhitthutiṁ karomase 〉
\end{leader}
\begin{leader-english}
  〈 Now let us recite in praise of the Saṅgha 〉
\end{leader-english}

Yo so supaṭipanno bhagavato sāvakasaṅgho

\begin{english}
  They are the Blessed One's disciples who have practiced well
\end{english}

Ujupaṭipanno bhagavato sāvakasaṅgho

\begin{english}
  Who have practiced directly\hyperlink{endnote5-appendix}{\hypertarget{endnote5-body}{\pagenote{%
    \hypertarget{endnote5-appendix}{\hyperlink{endnote5-body}{To practice ``directly''(Pāli: \textit{uju}) means, to practice the most direct way to \textit{nibbāna}; the straight way; no B-tours.}}}}}
\end{english}

Ñāyapaṭipanno bhagavato sāvakasaṅgho\hyperlink{endnote6-appendix}{\hypertarget{endnote6-body}{\pagenote{%
  \hypertarget{endnote6-appendix}{\hyperlink{endnote6-body}{Orig: ``Who have practiced insightfully''}}}}}

\begin{english}
  Who have practiced correctly\hyperlink{endnote7-appendix}{\hypertarget{endnote7-body}{\pagenote{%
    \hypertarget{endnote7-appendix}{\hyperlink{endnote7-body}{Orig: ``Those who practice with integrity''}}}}}
\end{english}

Sāmīcipaṭipanno bhagavato sāvakasaṅgho

\begin{english}
  Who have practiced properlyi
\end{english}

Yadidaṁ cattāri purisayugāni aṭṭha purisapuggalā

\begin{english}
  That is the four pairs the eight kinds of Noble Beings
\end{english}

Esa bhagavato sāvakasaṅgho

\begin{english}
  These are the Blessed One's disciples
\end{english}

Āhuneyyo

\begin{english}
  Such ones are worthy of gifts
\end{english}

Pāhuneyyo

\begin{english}
  Worthy of hospitality
\end{english}

Dakkhiṇeyyo

\begin{english}
  Worthy of offerings
\end{english}

Añjali-karaṇīyo

\begin{english}
  Worthy of respect
\end{english}

Anuttaraṁ puññakkhettaṁ lokassa

\begin{english}
  They give occasion for incomparable goodness to arise in the world
\end{english}

\begin{pali-hang}
  Tam-ahaṁ saṅghaṁ abhipūjayāmi tam-ahaṁ saṅghaṁ sirasā namāmi
\end{pali-hang}

\begin{english}
  I chant my praise to this Saṅgha\\
  I bow my head to this Saṅgha \hfill{(Bow)}
\end{english}

\section{Salutation to the Triple Gem}
\label{salutation}

\begin{leader}
  〈 Handa mayaṁ ratanattaya-paṇāma-gāthāyo c'eva saṁvega-parikittana-pāṭhañca bhaṇāmase 〉
\end{leader}
\begin{leader-english}
  〈 Now let us recite our salutation to the Triple Gem and a passage to arouse urgency 〉
\end{leader-english}

Buddho susuddho karuṇā-mahaṇṇavo

\begin{english}
  The Buddha absolutely pure with ocean-like compassion
\end{english}

Yo'ccanta-suddhabbara-ñāṇa-locano

\begin{english}
  Possessing the clear sight of wisdom
\end{english}

Lokassa pāpūpakilesa-ghātako

\begin{english}
  Destroyer of worldly self-corruption
\end{english}

Vandāmi buddhaṁ aham-ādarena taṁ

\begin{english}
  Devotedly indeed \breathmark\ that Buddha I revere
\end{english}

Dhammo padīpo viya tassa satthuno

\begin{english}
  The Teaching of the Lord is like a lamp\hyperlink{endnote8-appendix}{\hypertarget{endnote8-body}{\pagenote{%
    \hypertarget{endnote8-appendix}{\hyperlink{endnote8-body}{Orig: ``The teaching of the Lord like a lamp''}}}}}
\end{english}

Yo magga-pākāmata-bheda-bhinnako

\begin{english}
  Divided into path and its fruit \breathmark\ the Deathless\hyperlink{endnote9-appendix}{\hypertarget{endnote9-body}{\pagenote{%
    \hypertarget{endnote9-appendix}{\hyperlink{endnote2-body}{Orig: ``Illuminating the path and its fruit, the Deathless''}}}}}
\end{english}

Lokuttaro yo ca tad-attha-dīpano

\begin{english-hang}
  And illuminating that goal \breathmark\ which is beyond the conditioned world\hyperlink{endnote10-appendix}{\hypertarget{endnote10-body}{\pagenote{%
    \hypertarget{endnote10-appendix}{\hyperlink{endnote10-body}{Orig: ``That which is beyond the conditioned world''}}}}}
\end{english-hang}

Vandāmi dhammaṁ aham-ādarena taṁ

\begin{english}
  Devotedly indeed \breathmark\ that Dhamma I revere
\end{english}

Saṅgho sukhettābhyati-khetta-saññito

\begin{english}
  The Saṅgha the most fertile ground for cultivation
\end{english}

Yo diṭṭha-santo sugatānubodhako

\begin{english}
  Those who have realised peace\\
  Awakened after the Accomplished One
\end{english}

Lolappahīno ariyo sumedhaso

\begin{english}
  Noble and wise \breathmark\ all longing abandoned
\end{english}

Vandāmi saṅghaṁ aham-ādarena taṁ

\begin{english}
  Devotedly indeed \breathmark\ that Saṅgha I revere
\end{english}

\begin{pali-hang}
  Iccevam-ekantabhipūja-neyyakaṁ vatthuttayaṁ vandayatābhisaṅkhataṁ
\end{pali-hang}

\begin{english}
  This salutation should be made\\
  To that triad\hyperlink{endnote11-appendix}{\hypertarget{endnote11-body}{\pagenote{%
    \hypertarget{endnote11-appendix}{\hyperlink{endnote11-body}{Orig: ``To that which is worthy''. This passage refers to the triple (\textit{taya}) gems and not just to the Saṅgha.}}}}}
  which is worthy
\end{english}

Puññaṁ mayā yaṁ mama sabbupaddavā

\begin{english}
  Through the power of such good action
\end{english}

Mā hontu ve tassa pabhāva-siddhiyā

\begin{english}
  May all obstacles disappear
\end{english}

Idha tathāgato loke uppanno arahaṁ sammāsambuddho

\begin{english}
  One who knows things as they are \breathmark\ has arisen in this world\hyperlink{endnote12-appendix}{\hypertarget{endnote12-body}{\pagenote{%
    \hypertarget{endnote12-appendix}{\hyperlink{endnote12-body}{``One who knows things as they are'' is an unusual translation for \textit{Tathāgata}. Also ``arisen in'' is better than ``has come into'', otherwise one might think that he has come from somewhere, already being a \textit{Tathāgata}.}}}}}\\
  And he is an \textit{Arahant} \breathmark\ a perfectly awakened being
\end{english}

\begin{pali-hang}
  Dhammo ca desito niyyāniko upasamiko parinibbāniko sambodhagāmī sugatappavedito
\end{pali-hang}

\begin{english}
  Teaching the way leading out of delusion\hyperlink{endnote13-appendix}{\hypertarget{endnote13-body}{\pagenote{%
    \hypertarget{endnote13-appendix}{\hyperlink{endnote13-body}{No mention of ``delusion'' in the Pāli. It could also refer to \textit{samsāra} or \textit{dukkha}.}}}}}\\
  Calming and directing to perfect peace\\
  And leading to enlightenment\\
  This way he has made known\\
\end{english}

Mayan-taṁ dhammaṁ sutvā evaṁ jānāma

\begin{english}
  Having heard the Teaching we know this
\end{english}

Jātipi dukkhā

\begin{english}
  Birth is dukkha
\end{english}

Jarāpi dukkhā

\begin{english}
  Ageing is dukkha
\end{english}

Maraṇampi dukkhaṁ

\begin{english}
  And death is dukkha
\end{english}

Soka-parideva-dukkha-domanass'upāyāsāpi dukkhā

\begin{english}
  Sorrow lamentation pain displeasure\hyperlink{endnote14-appendix}{\hypertarget{endnote14-body}{\pagenote{%
    \hypertarget{endnote14-appendix}{\hyperlink{endnote14-body}{Orig: ``grief''}}}}}
  and despair are dukkha
\end{english}

Appiyehi sampayogo dukkho

\begin{english}
  Association with the disliked is dukkha
\end{english}

Piyehi vippayogo dukkho

\begin{english}
  Separation from the liked is dukkha
\end{english}

Yamp'icchaṁ na labhati tampi dukkhaṁ

\begin{english}
  Not attaining one's wishes is dukkha
\end{english}

Saṅkhittena pañcupādānakkhandhā dukkhā

\begin{english}
  In brief \breathmark\ the five aggregates of clinging are dukkha\hyperlink{endnote15-appendix}{\hypertarget{endnote15-body}{\pagenote{%
    \hyperlink{endnote15-appendix}{\hypertarget{endnote15-body}{Orig: ``In brief the five focuses of identity are dukkha''}}}}}
\end{english}

Seyyathīdaṁ

\begin{english}
  These are as follows
\end{english}

Rūpūpādānakkhandho

\begin{english}
  Attachment to form
\end{english}

Vedanūpādānakkhandho

\begin{english}
  Attachment to feeling
\end{english}

Saññūpādānakkhandho

\begin{english}
  Attachment to perception
\end{english}

Saṅkhārūpādānakkhandho

\begin{english}
  Attachment to volitional formations\hyperlink{endnote16-appendix}{\hypertarget{endnote16-body}{\pagenote{%
    \hyperlink{endnote16-appendix}{\hypertarget{endnote16-body}{Orig: ``Attachment to mental formations''}}}}}
\end{english}

Viññāṇūpādānakkhandho

\begin{english}
  Attachment to consciousness\hyperlink{endnote17-appendix}{\hypertarget{endnote17-body}{\pagenote{%
    \hyperlink{endnote17-appendix}{\hypertarget{endnote17-body}{Orig: ``Attachment to sense-consciousness''}}}}}
\end{english}

Yesaṁ pariññāya

\begin{english}
  For the complete understanding of this
\end{english}

Dharamāno so bhagavā

\begin{english}
  The Blessed One in his lifetime
\end{english}

Evaṁ bahulaṁ sāvake vineti

\begin{english}
  Frequently instructed his disciples in just this way
\end{english}

\begin{pali-hang}
  Evaṁ bhāgā ca panassa bhagavato sāvakesu anusāsanī bahulā pavattati
\end{pali-hang}

\begin{english}
  In addition he further instructed
\end{english}

Rūpaṁ aniccaṁ

\begin{english}
  Form is impermanent
\end{english}

Vedanā aniccā

\begin{english}
  Feeling is impermanent
\end{english}

Saññā aniccā

\begin{english}
  Perception is impermanent
\end{english}

Saṅkhārā aniccā

\begin{english}
  Volitional formations are impermanent\hyperlink{endnote18-appendix}{\hypertarget{endnote18-body}{\pagenote{%
    \hyperlink{endnote18-appendix}{\hypertarget{endnote18-body}{Orig: ``Mental formations are impermanent''}}}}}
\end{english}

Viññāṇaṁ aniccaṁ

\begin{english}
  Consciousness is impermanent\hyperlink{endnote19-appendix}{\hypertarget{endnote19-body}{\pagenote{%
    \hyperlink{endnote19-appendix}{\hypertarget{endnote19-body}{Orig: ``Sense-consciousness is impermanent''}}}}}
\end{english}

Rūpaṁ anattā

\begin{english}
  Form is not-self
\end{english}

Vedanā anattā

\begin{english}
  Feeling is not-self
\end{english}

Saññā anattā

\begin{english}
  Perception is not-self
\end{english}

Saṅkhārā anattā

\begin{english}
  Volitional formations are not-self\hyperlink{endnote20-appendix}{\hypertarget{endnote20-body}{\pagenote{%
    \hyperlink{endnote20-appendix}{\hypertarget{endnote20-body}{Orig: ``Mental formations are not-self''}}}}}
\end{english}

Viññāṇaṁ anattā

\begin{english}
  Consciousness is not-self\hyperlink{endnote21-appendix}{\hypertarget{endnote21-body}{\pagenote{%
    \hyperlink{endnote21-appendix}{\hypertarget{endnote21-body}{Orig: ``Sense-consciousness is not-self''}}}}}
\end{english}

Sabbe saṅkhārā aniccā

\begin{english}
  All conditioned things are impermanent\hyperlink{endnote22-appendix}{\hypertarget{endnote22-body}{\pagenote{%
    \hyperlink{endnote22-appendix}{\hypertarget{endnote22-body}{Orig: ``All conditions are transient''}}}}}
\end{english}

Sabbe dhammā anattā't

\begin{english}
  All things are not-self\hyperlink{endnote23-appendix}{\hypertarget{endnote23-body}{\pagenote{%
    \hyperlink{endnote23-appendix}{\hypertarget{endnote23-body}{Orig: ``There is no self in the created or the uncreated''. While this is not a very accurate translation, it is indeed the case that the term ``sabbe dhammā'' includes the uncreated, \textit{nibbāna} (see AN 5.32).}}}}}
\end{english}

Te mayaṁ otiṇṇāmha jātiyā jarā-maraṇena

\begin{english}
  All of us are affected by birth \breathmark\ ageing and death\hyperlink{endnote24-appendix}{\hypertarget{endnote24-body}{\pagenote{%
    \hyperlink{endnote24-appendix}{\hypertarget{endnote24-body}{Orig: ``All of us are bound by birth ageing and death''}}}}}
\end{english}

Sokehi paridevehi dukkhehi domanassehi upāyāsehi

\begin{english}
  By sorrow lamentation pain displeasure\hyperlink{endnote25-appendix}{\hypertarget{endnote25-body}{\pagenote{%
    \hyperlink{endnote25-appendix}{\hypertarget{endnote25-body}{Orig: ``grief''}}}}}
  and despair\hyperlink{endnote26-appendix}{\hypertarget{endnote26-body}{\pagenote{%
    \hyperlink{endnote26-appendix}{\hypertarget{endnote26-body}{In Pāli, these terms are in plural form, however, for the sake recitation they are kept singular.}}}}}
\end{english}

Dukkhotiṇṇā dukkha-paretā

\begin{english}
  Affected by dukkha and afflicted by dukkha\hyperlink{endnote27-appendix}{\hypertarget{endnote27-body}{\pagenote{%
    \hyperlink{endnote27-appendix}{\hypertarget{endnote27-body}{Orig: ``All of us are bound by birth ageing and death''}}}}}
\end{english}

\begin{pali-hang}
  Appeva nāmimassa kevalassa dukkha-kkhandhassa antakiriyā paññāyethā'ti
\end{pali-hang}

\begin{english}
  Let us all aspire to complete freedom from suffering
\end{english}

\begin{center}
  \textit{\textbf{(The following is recited only by the bhikkhus)}}
\end{center}

\begin{pali-hang}
  Cira-parinibbutampi taṁ bhagavantaṁ uddissa arahantaṁ sammāsambuddhaṁ
\end{pali-hang}

\begin{english-hang}
  Remembering the Blessed One \breathmark\ the Worthy One \breathmark\ and Perfectly Enlightened One\\
\end{english-hang}

\begin{english}
  Who long ago attained Parinibbāna
\end{english}

Saddhā agārasmā anagāriyaṁ pabbajitā

\begin{english}
  We have gone forth with faith\\
  From home to homelessness
\end{english}

Tasmiṁ bhagavati brahma-cariyaṁ carāma

\begin{english}
  And like the Blessed One \breathmark\ we practice the holy life
\end{english}

Bhikkhūnaṁ sikkhāsājīva-samāpannā

\begin{english}
  Possessing the bhikkhus'training and way of life\hyperlink{endnote28-appendix}{\hypertarget{endnote28-body}{\pagenote{%
    \hyperlink{endnote28-appendix}{\hypertarget{endnote28-body}{Orig: ``Being fully equipped with the bhikkhus'system of training''}}}}}
\end{english}

\begin{pali-hang}
  Taṁ no brahma-cariyaṁ imassa kevalassa dukkha-kkhandhassa antakiriyāya saṁvattatu
\end{pali-hang}

\begin{english}
  May this holy life \breathmark\ lead us to the end of this whole mass of suffering
\end{english}

\bottomNav{universal-well-being}

\section{Closing Homage}
\label{closing-homage}

[Arahaṁ] sammāsambuddho bhagavā

\begin{english}
  The Worthy One the Perfectly Enlightened and Blessed One
\end{english}

Buddhaṁ bhagavantaṁ abhivādemi

\begin{english}
  I render homage to the Buddha the Blessed One \hfill{(Bow)}
\end{english}

[Svākkhāto] bhagavatā dhammo

\begin{english}
  The Teaching so completely explained by him
\end{english}

Dhammaṁ namassāmi

\begin{english}
  I bow to the Dhamma \hfill{(Bow)}
\end{english}

[Supaṭipanno] bhagavato sāvakasaṅgho

\begin{english}
  The Blessed One's disciples who have practiced well
\end{english}

Saṅghaṁ namāmi

\begin{english}
  I bow to the Saṅgha \hfill{(Bow)}\\
\end{english}

\null
\vfill

\begin{minipage}[b][25pt][c]{1.0\linewidth}
  \begin{leader}
    \textbf{\textsc{\hyperref[schedule]{Content}\\
        \rule{\linewidth}{0.8pt}
        \hyperref[buddhas-first-exclamation]{Set 1} \hspace{0.01cm} — \hspace{0.01cm} \hyperref[characteristic-of-not-self]{Set 2} \hspace{0.01cm} — \hspace{0.01cm} \hyperref[noble-eightfold-path]{Set 3} \hspace{0.01cm} — \hspace{0.01cm} \hyperref[dedication-of-offerings]{Set 4} \hspace{0.01cm} — \hspace{0.01cm} \hyperref[mindfulness-of-breathing]{Set 5}\\
        \hyperref[anatta-lakkhana]{Set 6} — \hyperref[dependent-origination]{Set 7} — \hyperref[aditta-pariyaya]{Set 8} — \hyperref[deva-aradhana]{Set 9} — \hyperref[pubba-bhaga-nama-kara-patho]{Set 10}}}
  \end{leader}
\end{minipage}
