\chapterOpeningPage{reflections.pdf}

\chapter{Reflections}

\section{The Four Requisites}
\paliTitle{Cattaro parrikhārā}

\begin{leader}
  [Handa mayaṁ taṅkhaṇika-paccavekkhaṇa-pāṭhaṁ bhaṇāmase]
\end{leader}

Paṭisaṅkhā yoniso cīvaraṁ paṭisevāmi\\
Yāvadeva sītassa paṭighātāya\\
Uṇhassa paṭighātāya\\
Ḍaṁsa-makasa-vātātapa-siriṁsapa-samphassānaṁ paṭighātāya\\
Yāvadeva hirikopina-paṭicchādanatthaṁ

\begin{english}
  Wisely reflecting  ̓  I use the robe\\
  Only to ward off cold  ̓  to ward off heat  ̓  to ward off the touch of flies  ̓  mosquitoes wind burning and creeping things\\
  Only for the sake of modesty
\end{english}

Paṭisaṅkhā yoniso piṇḍapātaṁ paṭisevāmi\\
Neva davāya na madāya na maṇḍanāya na vibhūsanāya\\
Yāvadeva imassa kāyassa ṭhitiyā yāpanāya\\
Vihiṁsūparatiyā brahmacariyānuggahāya\\
Iti purāṇañca vedanaṁ paṭihaṅkhāmi\\
Navañca vedanaṁ na uppādessāmi\\
Yātrā ca me bhavissati anavajjatā ca phāsuvihāro cā’ti

\begin{english}
  Wisely reflecting  ̓  I use almsfood\\
  Not for fun  ̓  not for pleasure  ̓  not for fattening  ̓  not for beautification\\
  Only for the maintenance and nourishment of this body\\
  For keeping it healthy  ̓  for helping with the holy life\\
  Thinking thus: “I will allay hunger without overeating\\
  So that I may continue to live blamelessly and at ease”
\end{english}

Paṭisaṅkhā yoniso senāsanaṁ paṭisevāmi\\
Yāvadeva sītassa paṭighātāya\\
Uṇhassa paṭighātāya\\
Ḍaṁsa-makasa-vātātapa-siriṁsapa-samphassānaṁ paṭighātāya\\
Yāvadeva utuparissaya-vinodanaṁ paṭisallānārāmatthaṁ

\begin{english}
  Wisely reflecting  ̓  I use the lodging\\
  Only to ward off cold  ̓  to ward off heat  ̓  to ward off the touch of flies  ̓  mosquitoes wind burning and creeping things\\
  Only to remove the danger from weather  ̓  and for living in seclusion
\end{english}

Paṭisaṅkhā yoniso gilāna-paccaya-bhesajja-parikkhāraṁ paṭisevāmi\\
Yāvadeva uppannānaṁ veyyābādhikānaṁ vedanānaṁ paṭighātāya\\
Abyāpajjha-paramatāyā’ti

\begin{english}
  Wisely reflecting  ̓  I use supports for the sick and medicinal requisites\\
  Only to ward off painful feelings that have arisen\\
  For the maximum freedom from disease
\end{english}

\suttaRef{[MN 2]}

\clearpage

\section{The Repulsiveness of Food}
\paliTitle{Āhāra-paṭikūla-paccavekkhaṇa-pāṭho}

\begin{leader}
  [Handa mayaṁ āhāra-paṭikūla-paccavekkhaṇa-pāṭhaṁ bhaṇāmase]
\end{leader}

Āhāre paṭikūlasaññāparicitena bhikkhave  ̓  bhikkhuno cetasā bahulaṁ viharato

\begin{english}
  When a bhikkhu often dwells with a mind\\
  Accustomed to the perception of the repulsiveness of food
\end{english}

Rasataṇhāya cittaṁ patilīyati

\begin{english}
  His mind shrinks away from craving for tastes
\end{english}

Patikuṭati pativattati na sampasāriyati

\begin{english}
  Turns back from it\\
  Rolls away from it\\
  And is not drawn towards it
\end{english}

Upekkhā vā pāṭikulyatā vā saṇṭhāti

\begin{english}
  Either equanimity or disgust become settled in him
\end{english}

\suttaRef{[AN 7.49]}

Sabbo panāyaṁ piṇḍa-pāto ajigucchanīyo

\begin{english}
  None of this almsfood is innately repulsive
\end{english}

Imaṁ pūti-kāyaṁ patvā

\begin{english}
  But touching this unclean body
\end{english}

Ativiya jigucchanīyo jāyati

\begin{english}
  It becomes disgusting indeed
\end{english}

\suttaRef{[Trad]}

\clearpage

\section{Universal Well-Being}
\label{universal-well-being}
\paliTitle{Mettā-pharaṇa}

\begin{leader}
  [Handa mayaṁ mettāpharaṇaṁ karomase]
\end{leader}

Ahaṁ sukhito homi\\
Niddukkho homi\\
Avero homi\\
Abyāpajjho homi\\
Anīgho homi\\
Sukhī attānaṁ pariharāmi\\
Sabbe sattā sukhitā hontu\\
Sabbe sattā averā hontu\\
Sabbe sattā abyāpajjhā hontu\\
Sabbe sattā anīghā hontu\\
Sabbe sattā sukhī attānaṁ pariharantu\\
Sabbe sattā sabbadukkhā pamuccantu\\
Sabbe sattā laddha-sampattito mā vigacchantu

Sabbe sattā kammassakā kammadāyādā kammayonī kammabandhū kammapaṭisaraṇā\\
Yaṁ kammaṁ karissanti\\
Kalyāṇaṁ vā pāpakaṁ vā\\
Tassa dāyādā bhavissanti

\begin{leader}
  [Now let us recite the reflections on universal well-being]
\end{leader}

\begin{english}
  May I abide in well-being\\
  In freedom from affliction\\
  In freedom from hostility\\
  In freedom from ill-will\\
  In freedom from anxiety\\
  And may I maintain well-being in myself\\
  May everyone abide in well-being\\
  In freedom from hostility\\
  In freedom from ill-will\\
  In freedom from anxiety\\
  And may they maintain well-being in themselves\\
  May all beings be released from all suffering\\
  And may they not be parted from the good fortune they have attainedi

  All beings are the owners of their kamma\\
  Heirs to their kamma\\
  Born of their kamma\\
  Related to their kamma\\
  Abide supported by their kamma\\
  Whatever kamma they shall do\\
  Either skillful or harmful\\
  Of such acts  ̓  they will be the heirs\\
\end{english}

\suttaRef{[AN 3.65 \& 5.57]}

\clearpage

\section{The Divine Abidings}
\paliTitle{Brahmavihārā}

\begin{leader}
  [Handa mayaṁ caturappamaññā obhāsanaṁ karomase]
\end{leader}

Mettā-sahagatena cetasā ekaṁ disaṁ pharitvā viharati tathā dutiyaṁ tathā tatiyaṁ tathā catutthaṁ iti uddhamadho tiriyaṁ sabbadhi sabbattatāya sabbāvantaṁ lokaṁ mettā-sahagatena cetasā vipulena mahaggatena appamāṇena averena abyāpajjhena pharitvā viharati

Karuṇā-sahagatena cetasā ekaṁ disaṁ pharitvā viharati tathā dutiyaṁ tathā tatiyaṁ tathā catutthaṁ
Iti uddhamadho tiriyaṁ sabbadhi sabbattatāya sabbāvantaṁ lokaṁ karuṇā-sahagatena cetasā vipulena mahaggatena appamāṇena averena abyāpajjhena pharitvā viharati

Muditā-sahagatena cetasā ekaṁ disaṁ pharitvā viharati tathā dutiyaṁ tathā tatiyaṁ tathā catutthaṁ iti uddhamadho tiriyaṁ sabbadhi sabbattatāya sabbāvantaṁ lokaṁ muditā-sahagatena cetasā vipulena mahaggatena appamāṇena averena abyāpajjhena pharitvā viharati

Upekkhā-sahagatena cetasā ekaṁ disaṁ pharitvā viharati tathā dutiyaṁ tathā tatiyaṁ tathā catutthaṁ iti uddhamadho tiriyaṁ sabbadhi sabbattatāya sabbāvantaṁ lokaṁ upekkhā-sahagatena cetasā vipulena mahaggatena appamāṇena averena abyāpajjhena pharitvā viharatī'ti

\begin{leader}
  [Now let us make the Four Boundless Qualities shine forth]
\end{leader}

I will abide pervading one quarter with a heart imbued with loving-kindness\\
Likewise the second likewise the third likewise the fourth\\
So above and below around and everywhere and to all as to myself\\
I will abide pervading the all-encompassing world with a heart imbued with loving-kindness\\
Abundant exalted immeasurable without hostility and without ill-will

I will abide pervading one quarter with a heart imbued with compassion\\
Likewise the second likewise the third likewise the fourth\\
So above and below around and everywhere and to all as to myself\\
I will abide pervading the all-encompassing world with a heart imbued with compassion\\
Abundant exalted immeasurable without hostility and without ill-will

I will abide pervading one quarter with a heart imbued with gladness\\
Likewise the second likewise the third likewise the fourth\\
So above and below around and everywhere and to all as to myself\\
I will abide pervading the all-encompassing world with a heart imbued with gladness\\
Abundant exalted immeasurable without hostility and without ill-will

I will abide pervading one quarter with a heart imbued with equanimity\\
Likewise the second likewise the third likewise the fourth\\
So above and below around and everywhere and to all as to myself\\
I will abide pervading the all-encompassing world with a heart imbued with equanimity\\
Abundant exalted immeasurable without hostility and without ill-will

\suttaRef{[DN 13]}

\clearpage

\section{Five Subjects for Frequent Recollection}

\begin{leader}
  [Handa mayaṁ abhiṇha-paccavekkhaṇa-pāṭhaṁ bhaṇāmase]
\end{leader}

Jarā-dhammomhi jaraṁ anatīto

\begin{english}
  I am of the nature to age\\
  I have not gone beyond ageing
\end{english}

Byādhi-dhammomhi byādhiṁ anatīto

\begin{english}
  I am of the nature to sicken\\
  I have not gone beyond sickness
\end{english}

Maraṇa-dhammomhi maraṇaṁ anatīto

\begin{english}
  I am of the nature to die\\
  I have not gone beyond dying
\end{english}

Sabbehi me piyehi manāpehi nānābhāvo vinābhāvo

\begin{english}
  All that is mine beloved and pleasing\\
  Will become otherwise\\
  Will become separated from me
\end{english}

Kammassakomhi kammadāyādo kammayoni kammabandhu kammapaṭisaraṇo\\
Yaṁ kammaṁ karissāmi\\
Kalyāṇaṁ vā pāpakaṁ vā\\
Tassa dāyādo bhavissāmi

\begin{english}
  I am the owner of my kamma\\
  Heir to my kamma\\
  Born of my kamma\\
  Related to my kamma\\
  Abide supported by my kamma\\
  Whatever kamma I shall do\\
  Either skillful or harmful\\
  Of such acts  ̓  I will be the heir
\end{english}

Evaṁ amhehi abhiṇhaṁ paccavekkhitabbaṁ

\begin{english}
  Thus we should frequently recollect
\end{english}

\suttaRef{[AN 5.57]}

\clearpage

\section{Ten Subjects for Frequent Recollection}
\paliTitle{Dasadhammā pabbajita-abhiṇha-paccavekkhaṇā}

\begin{leader}
  [Handa mayaṁ pabbajita-abhiṇha-paccavekkhaṇa-pāṭhaṁ bhaṇāmase]
\end{leader}

Dasa ime bhikkhave dhammā\\
Pabbajitena abhiṇhaṁ paccavekkhitabbā\\
Katame dasa

\begin{english}
  Bhikkhus there are these ten dhammas  ̓  which should be reflected upon again and again by one who has gone forth\\
  What are these ten?
\end{english}

Vevaṇṇiyamhi ajjhūpagato'ti pabbajitena abhiṇhaṁ paccavekkhitabbaṁ

\begin{english}
  “I have reached a state of castelessness”\\
  This should be reflected upon again and again by one who has gone forth
\end{english}

Parapaṭibaddhā me jīvikā’ti\\
Pabbajitena abhiṇhaṁ paccavekkhitabbaṁ

\begin{english}
  “My very life is sustained through the gifts of others”
  This should be reflected upon again and again by one who has gone forth
\end{english}

Añño me ākappo karaṇīyo'ti\\
Pabbajitena abhiṇhaṁ paccavekkhitabbaṁ

\begin{english}
  “Now my conduct should be different from before”\\
  This should be reflected upon again and again by one who has gone forth
\end{english}

Kacci nu kho me attā sīlato na upavadatī'ti\\
Pabbajitena abhiṇhaṁ paccavekkhitabbaṁ

\begin{english}
  “Does regret over my conduct arise in my mind?”\\
  This should be reflected upon again and again by one who has gone forth
\end{english}

Kacci nu kho maṁ anuvicca viññū sabrahmacārī sīlato na upavadantī'ti\\
Pabbajitena abhiṇhaṁ paccavekkhitabbaṁ

\begin{english}
  “Could my spiritual companions find fault with my conduct?”\\
  This should be reflected upon again and again by one who has gone forth
\end{english}

Sabbehi me piyehi manāpehi nānābhāvo vinābhāvo'ti\\
Pabbajitena abhiṇhaṁ paccavekkhitabbaṁ

\begin{english}
  “All that is mine beloved and pleasing\\
  Will become otherwise\\
  Will become separated from me”\\
  This should be reflected upon again and again by one who has gone forth
\end{english}

Kammassakomhi kammadāyādo kammayoni kammabandhu kammapaṭisaraṇo\\
Yaṁ kammaṁ karissāmi\\
Kalyāṇaṁ vā pāpakaṁ vā\\
Tassa dāyādo bhavissāmī'ti\\
Pabbajitena abhiṇhaṁ paccavekkhitabbaṁ

\begin{english}
  “I am the owner of my kamma\\
  Heir to my kamma\\
  Born of my kamma\\
  Related to my kamma\\
  Abide supported by my kamma\\
  Whatever kamma I shall do\\
  Either skillful or harmful\\
  Of such acts  ̓  I will be the heir”\\
  This should be reflected upon again and again by one who has gone forth
\end{english}

`Kathambhūtassa me rattindivā vītipatantī'ti\\
Pabbajitena abhiṇhaṁ paccavekkhitabbaṁ

\begin{english}
  “The days and nights are relentlessly passing\\
  How well am I spending my time?”\\
  This should be reflected upon again and again by one who has gone forth
\end{english}

Kacci nu kho'haṁ suññāgāre abhiramāmī'ti\\
Pabbajitena abhiṇhaṁ paccavekkhitabbaṁ

\begin{english}
  “Do I delight in solitude or not?”\\
  This should be reflected upon again and again by one who has gone forth by one who has gone forth
\end{english}

Atthi nu kho me uttari-manussa-dhammā alamariya-ñāṇa-dassana viseso adhigato\\
So’haṁ pacchime kāle sabrahmacārīhi puṭṭho na maṅku bhavissāmī’ti\\
Pabbajitena abhiṇhaṁ paccavekkhitabbaṁ

\begin{english}
  “Has my practice borne fruit with freedom or insight\\
  So that at the end of my life  ̓  I need not feel ashamed when questioned by my spiritual companions?”\\
  This should be reflected upon again and again by one who has gone forth
\end{english}

Ime kho bhikkhave dasa dhammā\\
Pabbajitena abhiṇhaṁ paccavekkhitabbā'ti

\begin{english}
  Bhikkhus these are the ten dhammas  ̓  which should be reflected upon again and again by one who has gone forth
\end{english}

\suttaRef{[AN 10.48]}

\clearpage

\section{The Thirty-Two Parts}

\begin{leader}
  [Handa mayaṁ dvattiṁsākāra-pāṭhaṁ bhaṇāmase]
\end{leader}

Ayaṁ kho me kāyo uddhaṁ pādatalā adho kesamatthakā tacapariyanto pūro nānappakārassa asucino

\begin{english}
  This which is my body\\
  From the soles of the feet up\\
  And down from the crown of the head\\
  Is a sealed bag of skin\\
  Filled with unattractive things
\end{english}

Atthi imasmiṁ kāye

\begin{english}
  In this body there are
\end{english}

{\centering
  \setArrayStretch{1}

  \begin{tabular}{ r l }
    kesā            & \tr{hair of the head} \\
    lomā            & \tr{hair of the body} \\
    nakhā           & \tr{nails} \\
    dantā           & \tr{teeth} \\
    taco            & \tr{skin} \\
  \end{tabular}

  \begin{tabular}{ r l }
    maṁsaṁ          & \tr{flesh} \\
    nahārū          & \tr{sinews} \\
    aṭṭhī           & \tr{bones} \\
    aṭṭhimiñjaṁ     & \tr{bone marrow} \\
    vakkaṁ          & \tr{kidneys} \\
    hadayaṁ         & \tr{heart} \\
    yakanaṁ         & \tr{liver} \\
    kilomakaṁ       & \tr{membranes} \\
    pihakaṁ         & \tr{spleen} \\
    papphāsaṁ       & \tr{lungs} \\
    antaṁ           & \tr{bowels} \\
    antaguṇaṁ       & \tr{entrails} \\
    udariyaṁ        & \tr{undigested food} \\
    karīsaṁ         & \tr{excrement} \\
    pittaṁ          & \tr{bile} \\
    semhaṁ          & \tr{phlegm} \\
    pubbo           & \tr{pus} \\
    lohitaṁ         & \tr{blood} \\
    sedo            & \tr{sweat} \\
    medo            & \tr{fat} \\
    assu            & \tr{tears} \\
    vasā            & \tr{grease} \\
    kheḷo           & \tr{spittle} \\
    siṅghāṇikā      & \tr{mucus} \\
    lasikā          & \tr{oil of the joints} \\
    muttaṁ          & \tr{urine} \\
    matthaluṅgan'ti & \tr{brain} \\
  \end{tabular}

  \restoreArrayStretch
}

Evam-ayaṁ me kāyo uddhaṁ pādatalā adho kesamatthakā tacapariyanto pūro nānappakārassa asucino

\begin{english}
  This then which is my body from the soles of the feet up and down from the crown of the head is a sealed bag of skin filled with unattractive things
\end{english}

\suttaRef{[DN 22]}

\clearpage

\section{Recollection of Impermanence}
\paliTitle{Aniccānussati}

\begin{leader}
  [Handa mayaṁ aniccānussati-pāṭhaṁ bhaṇāmase]
\end{leader}

Sabbe saṅkhārā anicca

\begin{english}
  All conditioned things are impermanent
\end{english}

Sabbe saṅkhārā dukkhā

\begin{english}
  All conditioned things are dukkha
\end{english}

Sabbe dhammā anattā

\begin{english}
  All things are not-self
\end{english}

\suttaRef{[Dhp 277-279]}

Addhuvaṁ jīvitaṁ

\begin{english}
  Life is not for sure
\end{english}

Dhuvaṁ maraṇaṁ

\begin{english}
  Death is for sure
\end{english}

Avassaṁ mayā maritabbaṁ

\begin{english}
  It is inevitable that I’ll die
\end{english}

Maraṇa-pariyosānaṁ me jīvitaṁ

\begin{english}
  Death is the culmination of my life
\end{english}

Jīvitaṁ me aniyataṁ

\begin{english}
  My life is uncertain
\end{english}

Maraṇaṁ me niyataṁ

\begin{english}
  My death is certain
\end{english}

\suttaRef{[Dhp A]}

Vata

\begin{english}
  Indeed
\end{english}

Ayaṁ kāyo

\begin{english}
  This body
\end{english}

Aciraṁ

\begin{english}
  Will soon
\end{english}

Apeta-viññāṇo

\begin{english}
  Be void of consciousness
\end{english}

Chuḍḍho

\begin{english}
  And cast away
\end{english}

Adhisessati

\begin{english}
  It will lie
\end{english}

Paṭhaviṁ

\begin{english}
  On the ground
\end{english}

Kaliṅgaraṁ iva

\begin{english}
  Just like a rotten log
\end{english}

Niratthaṁ

\begin{english}
  Useless
\end{english}

\suttaRef{[Dhp 41]}

Aniccā vata saṅkhārā

\begin{english}
  Indeed  ̓  conditioned things cannot last
\end{english}

Uppāda-vaya-dhammino

\begin{english}
  Their nature is to rise and ceasei
\end{english}

Uppajjitvā nirujjhanti

\begin{english}
  Having arisen things must cease
\end{english}

Tesaṁ vūpasamo sukho

\begin{english}
  Their stilling is true happiness
\end{english}

\suttaRef{[Trad]}

\clearpage
