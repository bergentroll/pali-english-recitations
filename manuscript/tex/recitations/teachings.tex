\chapterOpeningPage{teachings.pdf}

\chapter{Teachings}

\sectionPaliTitle{Dhamma-cakkappavattana}
\section{Setting in Motion the Wheel of Dhamma}
\label{wheel-of-dhamma-abridged}

\begin{center}
  [Handa mayaṁ dhamma-cakkappavattana sutta-pāṭhaṁ bhaṇāmase]
\end{center}

Dveme bhikkhave antā

\begin{english}
  Bhikkhus there are these two extremes
\end{english}

Pabbajitena na sevitabbā

\begin{english}
  That should not be pursued \breathmark\ by one who has gone forth
\end{english}

Yo cāyaṁ kāmesu kāma-sukh’allikānuyogo

\begin{english}
  That is whatever is tied up to sense pleasures\\
  Within the realm of sensuality
\end{english}

Hīno

\begin{english}
  Which is low
\end{english}

Gammo

\begin{english}
  Common
\end{english}

Pothujjaniko

\begin{english}
  The way of the common folk
\end{english}

Anariyo

\begin{english}
  Not the way of the Noble Ones
\end{english}

Anattha-sañhito

\begin{english}
  And pointless
\end{english}

Yo cāyaṁ atta-kilamathānuyogo

\begin{english}
  Then there is whatever is tied up\\
  With self-deprivation
\end{english}

Dukkho

\begin{english}
  Which is painful
\end{english}

Anariyo

\begin{english}
  Not the way of the Noble Ones
\end{english}

Anattha-sañhito

\begin{english}
  And pointless
\end{english}

Ete te bhikkhave ubho ante anupagamma majjhimā paṭipadā tathāgatena abhisambuddhā

\begin{english}
  Bhikkhus without going to either of these extremes\\
  The Tathāgata has ultimately awakened\\
  To a middle way of practice
\end{english}

Cakkhu-karaṇī

\begin{english}
  Giving rise to vision
\end{english}

Ñāṇa-karaṇī

\begin{english}
  Making for insight
\end{english}

Upasamāya

\begin{english}
  Leading to calm
\end{english}

Abhiññāya

\begin{english}
  To heightened knowing
\end{english}

Sambodhāya

\begin{english}
  Awakening
\end{english}

Nibbānāya saṁvattati

\begin{english}
  And to Nibbāna
\end{english}

Katamā ca sā bhikkhave majjhimā paṭipadā

\begin{english}
  And what bhikkhus is that middle way of practice?
\end{english}

Ayam-eva ariyo aṭṭhaṅgiko maggo

\begin{english}
  It is just this Noble Eightfold Path\pagenote{%
    Orig: “It is this Noble Eightfold Path”}
\end{english}

Seyyathīdaṁ

\begin{english}
  Which is as follows
\end{english}

Sammā-diṭṭhi

\begin{english}
  Right View
\end{english}

Sammā-saṅkappo

\begin{english}
  Right Intention
\end{english}

Sammā-vācā

\begin{english}
  Right Speech
\end{english}

Sammā-kammanto

\begin{english}
  Right Action
\end{english}

Sammā-ājīvo

\begin{english}
  Right Livelihood
\end{english}

Sammā-vāyāmo

\begin{english}
  Right Effort
\end{english}

Sammā-sati

\begin{english}
  Right Mindfulness
\end{english}

Sammā-samādhi

\begin{english}
  Right Concentration
\end{english}

Ayaṁ kho sā bhikkhave majjhimā paṭipadā tathāgatena abhisambuddhā

\begin{english}
  This bhikkhus is the middle way of practice\\
  That the Tathāgata has ultimately awakened to
\end{english}

Cakkhu-karaṇī

\begin{english}
  Giving rise to vision
\end{english}

Ñāṇa-karaṇī

\begin{english}
  Making for insight
\end{english}

Upasamāya

\begin{english}
  Leading to calm
\end{english}

Abhiññāya

\begin{english}
  To heightened knowing
\end{english}

Sambodhāya

\begin{english}
  Awakening
\end{english}

Nibbānāya saṁvattati

\begin{english}
  And to Nibbāna
\end{english}

Idaṁ kho pana bhikkhave dukkhaṁ ariya-saccaṁ

\begin{english}
  This bhikkhus is the Noble Truth of dukkha
\end{english}

Jātipi dukkhā

\begin{english}
  Birth is dukkha
\end{english}

Jarāpi dukkhā

\begin{english}
  Ageing is dukkha
\end{english}

Byādhipi dukkho

\begin{english}
  Sickness is dukkha\pagenote{%
    For some reason “byādhipi dukkho” is missing from the WPN version of this chanting book. The reason for this omission is unclear, because even the Thai edition of this discourse contains “byādhipi dukkho”. However, the definition of the four noble truths in the context of the Satipaṭṭhānasuttas in DN and MN do not contain “byādhipi dukkho”. It may be that the compilers of this chanting book copy/pasted this passage from the wrong discourse, thinking it is the same anayway, whereas in reality there is this small difference.}
\end{english}

Maraṇampi dukkhaṁ

\begin{english}
  And death is dukkha
\end{english}

Soka-parideva-dukkha-domanassupāyāsāpi dukkhā

\begin{english}
  Sorrow lamentation pain displeasure\pagenote{%
    Orig: “grief”}
  and despair are dukkha
\end{english}

Appiyehi sampayogo dukkho

\begin{english}
  Association with the disliked is dukkha
\end{english}

Piyehi vippayogo dukkho

\begin{english}
  Separation from the liked is dukkha
\end{english}

Yampicchaṁ na labhati tampi dukkhaṁ

\begin{english}
  Not attaining one’s wishes is dukkha
\end{english}

Saṅkhittena pañcupādānakkhandhā dukkhā

\begin{english}
  In brief \breathmark\ the five aggregates of clinging are dukkha\pagenote{%
    Orig: “In brief the five focuses of identity are dukkha”}
\end{english}

Idaṁ kho pana bhikkhave dukkha-samudayo ariya-saccaṁ

\begin{english}
  This bhikkhus is the Noble Truth of the origin of dukkha
\end{english}

Yā’yaṁ taṇhā

\begin{english}
  It is this craving
\end{english}

Ponobbhavikā

\begin{english}
  Which leads to rebirth
\end{english}

Nandi-rāga-sahagatā

\begin{english}
  Accompanied by delight and lust
\end{english}

Tatra-tatrābhinandinī

\begin{english}
  Delighting now here now there
\end{english}

Seyyathīdaṁ

\begin{english}
  Which is as follows
\end{english}

Kāma-taṇhā

\begin{english}
  Craving for sensuality
\end{english}

Bhava-taṇhā

\begin{english}
  Craving to become
\end{english}

Vibhava-taṇhā

\begin{english}
  Craving not to become
\end{english}

Idaṁ kho pana bhikkhave dukkha-nirodho ariya-saccaṁ

\begin{english}
  This bhikkhus is the Noble Truth of the cessation of dukkha
\end{english}

Yo tassāy’eva taṇhāya asesa-virāga-nirodho

\begin{english}
  It is the remainderless fading away and cessation\\
  Of that very craving
\end{english}

Cāgo

\begin{english}
  Its relinquishment
\end{english}

Paṭinissaggo

\begin{english}
  Letting go
\end{english}

Mutti

\begin{english}
  Release
\end{english}

Anālayo

\begin{english}
  Without any attachment
\end{english}

Idaṁ kho pana bhikkhave dukkha-nirodha-gāminī-paṭipadā ariya-saccaṁ

\begin{english}
  This bhikkhus is the Noble Truth of the way of practice\\
  Leading to the cessation of dukkha
\end{english}

Ayam-eva ariyo aṭṭh’aṅgiko maggo

\begin{english}
  It is just this Noble Eightfold Path
\end{english}

Seyyathīdaṁ

\begin{english}
  Which is as follows
\end{english}

Sammā-diṭṭhi

\begin{english}
  Right View
\end{english}

Sammā-saṅkappo

\begin{english}
  Right Intention
\end{english}

Sammā-vācā

\begin{english}
  Right Speech
\end{english}

Sammā-kammanto

\begin{english}
  Right Action
\end{english}

Sammā-ājīvo

\begin{english}
  Right Livelihood
\end{english}

Sammā-vāyāmo

\begin{english}
  Right Effort
\end{english}

Sammā-sati

\begin{english}
  Right Mindfulness
\end{english}

Sammā-samādhi

\begin{english}
  Right Concentration
\end{english}

Idaṁ dukkhaṁ ariya-saccan’ti me bhikkhave\\
Pubbe ananussutesu dhammesu\\
Cakkhuṁ udapādi\\
Ñāṇaṁ udapādi\\
Paññā udapādi\\
Vijjā udapādi\\
Āloko udapādi

\begin{english}
  Bhikkhus in regard to things unheard of before\\
  Vision arose\\
  Insight arose\\
  Discernment arose\\
  Knowledge arose\\
  Light arose in me\pagenote{%
    Orig: “Light arose”}\\
  “This is the Noble Truth of dukkha”
\end{english}

Taṁ kho pan’idaṁ dukkhaṁ ariya-saccaṁ pariññeyyan’ti

\begin{english}
  This Noble Truth of dukkha\pagenote{%
    Orig: “Now this Noble Truth of dukkha”}\\
  Should be completely understood
\end{english}

Taṁ kho pan’idaṁ dukkhaṁ ariya-saccaṁ pariññātan’ti

\begin{english}
  This Noble Truth of dukkha\\
  Has been completely understood
\end{english}

Idaṁ dukkha-samudayo ariya-saccan’ti me bhikkhave\\
Pubbe ananussutesu dhammesu\\
Cakkhuṁ udapādi\\
Ñāṇaṁ udapādi\\
Paññā udapādi\\
Vijjā udapādi\\
Āloko udapādi

\begin{english}
  Bhikkhus in regard to things unheard of before\\
  Vision arose\\
  Insight arose\\
  Discernment arose\\
  Knowledge arose\\
  Light arose in me\\
  “This is the Noble Truth of the origin of dukkha”
\end{english}

Taṁ kho pan’idaṁ dukkha-samudayo ariya-saccaṁ pahātabban’ti

\begin{english}
  This origin of dukkha\\
  Should be abandoned
\end{english}

Taṁ kho pan’idaṁ dukkha-samudayo ariya-saccaṁ pahīnan’ti

\begin{english}
  This origin of dukkha\\
  Has been abandoned
\end{english}

Idaṁ dukkha-nirodho ariya-saccan’ti me bhikkhave\\
Pubbe ananussutesu dhammesu\\
Cakkhuṁ udapādi\\
Ñāṇaṁ udapādi\\
Paññā udapādi\\
Vijjā udapādi\\
Āloko udapādi

\begin{english}
  Bhikkhus in regard to things unheard of before\\
  Vision arose\\
  Insight arose\\
  Discernment arose\\
  Knowledge arose\\
  Light arose in me\\
  “This is the Noble Truth of the cessation of dukkha”
\end{english}

Taṁ kho pan’idaṁ dukkha-nirodho ariya-saccaṁ sacchi-kātabban’ti

\begin{english}
  This cessation of dukkha\\
  Should be experienced directly
\end{english}

Taṁ kho pan’idaṁ dukkha-nirodho ariya-saccaṁ sacchikatan’ti

\begin{english}
  This cessation of dukkha\\
  Has been experienced directly
\end{english}

Idaṁ dukkha-nirodha-gāminī-paṭipadā ariya-saccan’ti me bhikkhave
Pubbe ananussutesu dhammesu\\
Cakkhuṁ udapādi\\
Ñāṇaṁ udapādi\\
Paññā udapādi\\
Vijjā udapādi\\
Āloko udapādi

\begin{english}
  Bhikkhus in regard to things unheard of before\\
  Vision arose\\
  Insight arose\\
  Discernment arose\\
  Knowledge arose\\
  Light arose in me\\
  “This is the Noble Truth of the way of practice\\
  Leading to the cessation of dukkha”
\end{english}

Taṁ kho pan’idaṁ dukkha-nirodha-gāminī-paṭipadā ariya-saccaṁ bhāvetabban’ti

\begin{english}
  This way of practice \breathmark\ leading to the cessation of dukkha\\
  Should be developed
\end{english}

Taṁ kho pan’idaṁ dukkha-nirodha-gāminī-paṭipadā ariya-saccaṁ bhāvitan’ti

\begin{english}
  This way of practice \breathmark\ leading to the cessation of dukkha\\
  Has been developed
\end{english}

Yāva-kīvañ-ca me bhikkhave imesu catūsu ariya-saccesu\\
Evan-ti-parivaṭṭaṁ dvādas’ākāraṁ yathā-bhūtaṁ ñāṇa-dassanaṁ na suvisuddhaṁ ahosi

\begin{english}
  As long bhikkhus as my knowledge and understanding\\
  As it actually is\\
  Of these Four Noble Truths\\
  With their three phases and twelve aspects\pagenote{%
    The three phases are comprised of the statement of the noble truth itself, followed by what is the task that is to be performed in relation to this noble truth, followed by the statement that the task has been performed. Four truths multiplied by three phases, results in 12 aspects.}\\
  Was not entirely pure
\end{english}

N’eva tāvāhaṁ bhikkhave sadevake loke samārake sabrahmake\\
Sassamaṇa-brāhmaṇiyā pajāya sadeva-manussāya\\
Anuttaraṁ sammā-sambodhiṁ abhisambuddho\pagenote{%
  Instead of the nominative “abhisambuddo” (Thai and Buddha Jayanti Pāli edition), one would usually expect the accusative or a quotation here, which is indeed what is found in the PTS and Chatta Saṅgāyana editions, giving “abhisambuddho’ti”. However, the anoumolous reading “abhisambuddho” finds support from Mahāvasutu: “yāvac cāhaṁ bhikṣavaḥ imāni catvāry āryasatyāni evaṁ triparivartaṁ dvādaśākāraṁ yathābhūtaṁ samyakprajñayā nābhyajñāsiṣaṁ na tāvad ahaṁ anuttarāṁ samyaksaṁbodhim abhisaṁbuddho pratijānehaṁ”, which according to Ven. Ānandajoti probably indicates that “abhisambuddho” is an old/original reading.}
paccaññāsiṁ

\begin{english}
  I did not claim bhikkhus\\
  In this world of devas\\
  Māra and Brahmā\\
  Amongst mankind with its priests and renunciants\\
  Kings and commoners\\
  An ultimate awakening\\
  To unsurpassed perfect enlightenment
\end{english}

Yato ca kho me bhikkhave imesu catūsu ariya-saccesu\\
Evan-ti-parivaṭṭaṁ dvādas’ākāraṁ yathā-bhūtaṁ ñāṇa-dassanaṁ suvisuddhaṁ ahosi

\begin{english}
  But when bhikkhus my knowledge and understanding\\
  As it actually is\\
  Of these Four Noble Truths\\
  With their three phases and twelve aspects\\
  Was indeed entirely pure
\end{english}

Athāhaṁ bhikkhave sadevake loke samārake sabrahmake\\
Sassamaṇa-brāhmaṇiyā pajāya sadeva-manussāya\\
Anuttaraṁ sammā-sambodhiṁ abhisambuddho paccaññāsiṁ

\begin{english}
  Then indeed did I claim bhikkhus\\
  In this world of devas\\
  Māra and Brahmā\\
  Amongst mankind with its priests and renunciants\\
  Kings and commoners\\
  An ultimate awakening\\
  To unsurpassed perfect enlightenment
\end{english}

Ñāṇañ-ca pana me dassanaṁ udapādi

\begin{english}
  Now knowledge and understanding arose in me
\end{english}

Akuppā me vimutti

\begin{english}
  My release is unshakeable
\end{english}

Ayam-antimā jāti

\begin{english}
  This is my last birth
\end{english}

N’atthidāni punabbhavo’ti

\begin{english}
  There won’t be any further becoming
\end{english}

\suttaRef{[SN 56.11]}

\bottomNav{true-false-refuges}

\sectionPaliTitle{Satta-sambojjhaṅgā}
\section{Anupubba-sikkhā}

\begin{center}
  [Handa mayaṁ anupubba-sikkha-pāṭhaṁ bhaṇāmase]
\end{center}

Taṁ dhammaṁ suṇāti gahapati vā gahapatiputto vā aññatarasmiṁ vā kule paccājāto. So taṁ dhammaṁ sutvā tathāgate saddhaṁ paṭilabhati. So tena saddhāpaṭilābhena samannāgato iti paṭisañcikkhati: ‘sambādho gharāvāso rajopatho abbhokāso pabbajjā. Nayidaṁ sukaraṁ agāraṁ ajjhāvasatā ekantaparipuṇṇaṁ ekantaparisuddhaṁ sankhalikhitaṁ brahmacariyaṁ carituṁ. Yannūnāhaṁ kesamassuṁ ohāretvā kāsāyāni vatthāni acchādetvā agārasmā anagāriyaṁ pabbajeyyan’ti. So aparena samayena appaṁ vā bhogakkhandhaṁ pahāya mahantaṁ vā bhogakkhandhaṁ pahāya appaṁ vā ñātiparivaṭṭaṁ pahāya mahantaṁ vā ñātiparivaṭṭaṁ pahāya kesamassuṁ ohāretvā kāsāyāni vatthāni acchādetvā agārasmā anagāriyaṁ pabbajati.

\suttaRef{[MN 27 / 38 / 51]}

Sakkā nu kho bho gotama imasmimpi dhammavinaye evameva anupubbasikkhā anupubbakiriyā anupubbapaṭipadā paññapetun"ti?

"Sakkā imasmimpi dhammavinaye anupubbasikkhā anupubbakiriyā anupubbapaṭipadā paññapetuṁ. Tathāgato purisadammaṁ labhitvā paṭhamaṁ evaṁ vineti: ‘ehi tvaṁ bhikkhu sīlavā hohi pātimokkhasaṁvarasaṁvuto viharāhi ācāragocarasampanno aṇumattesu vajjesu bhayadassāvī samādāya sikkhassu sikkhāpadesū’ti.

\suttaRef{[MN 107]}

So evaṁ pabbajito samāno bhikkhūnaṁ sikkhāsājīva-samāpanno pāṇātipātaṁ pahāya pāṇātipātā paṭivirato hoti nihitadaṇḍo nihitasattho lajjī dayāpanno sabbapāṇabhūta-hitānukampī viharati. Adinnādānaṁ pahāya adinnādānā paṭivirato hoti dinnādāyī dinnapāṭikankhī athenena sucibhūtena attanā viharati. Abrahmacariyaṁ pahāya brahmacārī hoti ārācārī virato methunā gāmadhammā.

Musāvādaṁ pahāya musāvādā paṭivirato hoti saccavādī saccasandho theto paccayiko avisaṁvādako lokassa. Pisuṇaṁ vācaṁ pahāya pisuṇāya vācāya paṭivirato hoti ito sutvā na amutra akkhātā imesaṁ bhedāya amutra vā sutvā na imesaṁ akkhātā amūsaṁ bhedāya—iti bhinnānaṁ vā sandhātā sahitānaṁ vā anuppadātā samaggārāmo samaggarato samagganandī samaggakaraṇiṁ vācaṁ bhāsitā hoti. Pharusaṁ vācaṁ pahāya pharusāya vācāya paṭivirato hoti yā sā vācā nelā kaṇṇasukhā pemanīyā hadayaṅgamā porī bahujanakantā bahujanamanāpā tathārūpiṁ vācaṁ bhāsitā hoti. Samphappalāpaṁ pahāya samphappalāpā paṭivirato hoti kālavādī bhūtavādī atthavādī dhammavādī vinayavādī nidhānavatiṁ vācaṁ bhāsitā kālena sāpadesaṁ pariyantavatiṁ atthasaṁhitaṁ.

So bījagāma-bhūtagāma-samārambhā paṭivirato hoti ekabhattiko hoti rattūparato virato vikālabhojanā \breathmark\ nacca-gītavāditavisūkadassanā paṭivirato hoti \breathmark\ mālā­gandha­ vilepana­dhāraṇa­maṇḍana­vibhūsanaṭṭhānā paṭivirato hoti \breathmark\ uccāsayana-mahāsayanā paṭivirato hoti \breathmark\ jātarūparajata-paṭiggahaṇā paṭivirato hoti \breathmark\ āmaka-dhañña-paṭiggahaṇā paṭivirato hoti \breathmark\ āmaka-maṁsa-paṭiggahaṇā paṭivirato hoti \breathmark\ itthikumārika-paṭiggahaṇā paṭivirato hoti \breathmark\ dāsidāsa-paṭiggahaṇā paṭivirato hoti \breathmark\ ajeḷaka-paṭiggahaṇā paṭivirato hoti \breathmark\ kukkuṭasūkara-paṭiggahaṇā paṭivirato hoti \breathmark\ hatthi-gavassa-vaḷava-paṭiggahaṇā paṭivirato hoti \breathmark\ khettavatthu-paṭiggahaṇā paṭivirato hoti \breathmark\ dūteyya-pahiṇa-gamanānuyogā paṭivirato hoti \breathmark\ kayavikkayā paṭivirato hoti \breathmark\ tulākūṭa-kaṁsa-kūṭa-mānakūṭā paṭivirato hoti \breathmark\ ukkoṭana-vañcana-nikati-sāciyogā paṭivirato hoti \breathmark\ chedana­vadha-bandhana­ viparāmosa-ālopa-sahasākārā paṭivirato hoti.

So santuṭṭho hoti kāyaparihārikena cīvarena kucchi-parihārikena piṇḍapātena. So yena yeneva pakkamati samādāyeva pakkamati. Seyyathāpi nāma pakkhī sakuṇo yena yeneva ḍeti sapattabhārova ḍeti \breathmark\ evameva bhikkhu santuṭṭho hoti kāyaparihārikena cīvarena kucchiparihārikena piṇḍapātena. So yena yeneva pakkamati samādāyeva pakkamati. So iminā ariyena sīlakkhandhena samannāgato ajjhattaṁ anavajjasukhaṁ paṭisaṁvedeti.

\suttaRef{[MN 51]}

Tamenaṁ tathāgato uttariṁ vineti: ‘ehi tvaṁ bhikkhu indriyesu guttadvāro hohi cakkhunā rūpaṁ disvā mā nimittaggāhī hohi mānubyañjanaggāhī. Yatvādhikaraṇamenaṁ cakkhundriyaṁ asaṁvutaṁ viharantaṁ abhijjhādomanassā pāpakā akusalā dhammā anvāssaveyyuṁ tassa saṁvarāya paṭipajjāhi \breathmark\ rakkhāhi cakkhundriyaṁ cakkhundriye saṁvaraṁ āpajjāhi. Sotena saddaṁ sutvā. Ghānena gandhaṁ ghāyitvā. Jivhāya rasaṁ sāyitvā. Kāyena phoṭṭhabbaṁ phusitvā. Manasā dhammaṁ viññāya mā nimittaggāhī hohi mānubyañjanaggāhī. Yatvādhikaraṇamenaṁ manindriyaṁ asaṁvutaṁ viharantaṁ abhijjhādomanassā pāpakā akusalā dhammā anvāssaveyyuṁ tassa saṁvarāya paṭipajjāhi \breathmark\ rakkhāhi manindriyaṁ manindriye saṁvaraṁ āpajjāhī’ti.

‘Ehi tvaṁ bhikkhu bhojane mattaññū hohi. Paṭisankhā yoniso āhāraṁ āhāreyyāsi—neva davāya na madāya na maṇḍanāya na vibhūsanāya yāvadeva imassa kāyassa ṭhitiyā yāpanāya vihiṁsūparatiyā brahmacariyānuggahāya—iti purāṇañca vedanaṁ paṭihankhāmi navañca vedanaṁ na uppādessāmi yātrā ca me bhavissati anavajjatā ca phāsuvihāro cā’ti.

‘Ehi tvaṁ bhikkhu jāgariyaṁ anuyutto viharāhi divasaṁ cankamena nisajjāya āvaraṇīyehi dhammehi cittaṁ parisodhehi rattiyā paṭhamaṁ yāmaṁ cankamena nisajjāya āvaraṇīyehi dhammehi cittaṁ parisodhehi rattiyā majjhimaṁ yāmaṁ dakkhiṇena passena sīhaseyyaṁ kappeyyāsi pāde pādaṁ accādhāya sato sampajāno uṭṭhānasaññaṁ manasikaritvā rattiyā pacchimaṁ yāmaṁ paccuṭṭhāya cankamena nisajjāya āvaraṇīyehi dhammehi cittaṁ parisodhehī’ti.

‘Ehi tvaṁ bhikkhu satisampajaññena samannāgato hohi abhikkante paṭikkante sampajānakārī ālokite vilokite sampajānakārī samiñjite pasārite sampajānakārī sanghāṭipattacīvaradhāraṇe sampajānakārī asite pīte khāyite sāyite sampajānakārī uccārapassāvakamme sampajānakārī gate ṭhite nisinne sutte jāgarite bhāsite tuṇhībhāve sampajānakārī’ti.

‘Ehi tvaṁ bhikkhu vivittaṁ senāsanaṁ bhajāhi araññaṁ rukkhamūlaṁ pabbataṁ kandaraṁ giriguhaṁ susānaṁ vanapatthaṁ abbhokāsaṁ palālapuñjan’ti.

So pacchābhattaṁ piṇḍapātapaṭikkanto nisīdati pallankaṁ ābhujitvā ujuṁ kāyaṁ paṇidhāya parimukhaṁ satiṁ upaṭṭhapetvā. So abhijjhaṁ loke pahāya vigatābhijjhena cetasā viharati abhijjhāya cittaṁ parisodheti \breathmark\ byāpāda-padosaṁ pahāya abyāpannacitto viharati sabbapāṇabhūta-hitānukampī byāpādapadosā cittaṁ parisodheti \breathmark\ thina-middhaṁ pahāya vigatathinamiddho viharati ālokasaññī sato sampajāno thinamiddhā cittaṁ parisodheti \breathmark\ uddhacca-kukkuccaṁ pahāya anuddhato viharati ajjhattaṁ vūpasanta-citto uddhaccakukkuccā cittaṁ parisodheti \breathmark\ vicikicchaṁ pahāya tiṇṇavicikiccho viharati akathaṁkathī kusalesu dhammesu vicikicchāya cittaṁ parisodheti.

\suttaRef{[MN 107]}

Bhikkhu yathā iṇaṁ yathā rogaṁ yathā bandhanāgāraṁ yathā dāsabyaṁ yathā kantāraddhāna-maggaṁ ime pañca nīvaraṇe appahīne attani samanupassati. Seyyathāpi bhikkhave āṇaṇyaṁ yathā ārogyaṁ yathā bandhanāmokkhaṁ yathā bhujissaṁ yathā khemantabhūmiṁ \breathmark\ evameva bhikkhu ime pañca nīvaraṇe pahīne attani samanupassati. So tatonidānaṁ labhetha pāmojjaṁ adhigaccheyya somanassaṁ.

So ime pañca nīvaraṇe pahāya cetaso upakkilese paññāya dubbalīkaraṇe paṭhamaṁ jhānaṁ dutiyaṁ jhānaṁ tatiyaṁ jhānaṁ catutthaṁ jhānaṁ upasampajja viharati.

So evaṁ samāhite citte parisuddhe pariyodāte anangaṇe vigatūpakkilese mudubhūte kammaniye ṭhite āneñjappatte pubbe-nivāsānussati-ñāṇāya sattānaṁ cutūpapāta-ñāṇāya āsavānaṁ khaya-ñāṇāya cittaṁ abhininnāmeti.

So ‘idaṁ dukkhan’ti yathābhūtaṁ pajānāti ‘ayaṁ dukkha- samudayo’ti yathābhūtaṁ pajānāti ‘ayaṁ dukkhanirodho’ti yathābhūtaṁ pajānāti ‘ayaṁ dukkhanirodhagāminī paṭipadā’ti yathābhūtaṁ pajānāti.

‘Ime āsavā’ti yathābhūtaṁ pajānāti ‘ayaṁ āsavasamudayo’ti yathābhūtaṁ pajānāti ‘ayaṁ āsavanirodho’ti yathābhūtaṁ pajānāti ‘ayaṁ āsavanirodhagāminī paṭipadā’ti yathābhūtaṁ pajānāti.

Tassa evaṁ jānato evaṁ passato kāmāsavāpi cittaṁ vimuccati bhavāsavāpi cittaṁ vimuccati avijjāsavāpi cittaṁ vimuccati. Vimuttasmiṁ vimuttamiti ñāṇaṁ hoti: ‘khīṇā jāti vusitaṁ brahmacariyaṁ kataṁ karaṇīyaṁ nāparaṁ itthattāyā’ti pajānātī’ti.

\suttaRef{[MN 39]}

Ye kho te bhikkhū sekkhā apattamānasā anuttaraṁ yogakkhemaṁ patthayamānā viharanti tesu me ayaṁ evarūpī anusāsanī hoti.

\suttaRef{[MN 107]}

\section{The Gradual Training}
\label{gradual-training}

\begin{center}
  [Now let us recite the Gradual Training]
\end{center}
% % % TODO: justify?

\begin{justify}
“A householder or householder’s son \breathmark\ or one born in some other clan hears the Dhamma. On hearing the Dhamma he acquires faith in the Tathāgata. Possessing that faith he considers thus: ‘Household life is crowded and dusty \breathmark\ life gone forth is wide open. It is not easy while living at home \breathmark\ to lead the holy life utterly perfect and pure \breathmark\ as a polished shell. Suppose I shave off my hair and beard \breathmark\ put on the ochre robe \breathmark\ and go forth from home to homelessness.’ On a later occasion \breathmark\ abandoning a small or large fortune \breathmark\ abandoning a small or large circle of relatives \breathmark\ he shaves off his hair and beard \breathmark\ puts on the ochre robe \breathmark\ and goes forth from home to homelessness.”
\end{justify}

\suttaRef{[MN 27 / 38 / 51]}

\begin{justify}
“Now is it possible Master Gotama \breathmark\ to describe a gradual training \breathmark\ gradual practice \breathmark\ and gradual progress in this Dhamma and Vinaya?”
\end{justify}

\begin{justify}
“It is possible to describe a gradual training \breathmark\ gradual practice \breathmark\ and gradual progress in this Dhamma and Vinaya. When the Tathāgata obtains a person to be tamed \breathmark\ he first disciplines him thus: ‘Come bhikkhu be virtuous \breathmark\ dwell restrained with the restraint of the Pātimokkha \breathmark\ perfect in conduct and resort \breathmark\ and seeing danger in the slightest faults \breathmark\ train by undertaking the training rules.”
\end{justify}

\suttaRef{[MN 107]}

\begin{justify}
“Having thus gone forth \breathmark\ and possessing the bhikkhus’ training and way of life \breathmark\ abandoning the destruction of life \breathmark\ he abstains from destroying life \breathmark\ with rod and weapon laid aside \breathmark\ conscientious \breathmark\ merciful \breathmark\ he abides compassionate to all living beings. Abandoning the taking of what is not given \breathmark\ he abstains from taking what is not given \breathmark\ taking only what is given \breathmark\ expecting only what is given \breathmark\ by not stealing he abides in purity.
\end{justify}

\begin{justify}
Abandoning the household life \breathmark\ he observes the holy life \breathmark\ living apart \breathmark\ abstaining from the vulgar practice of sexual intercourse.”
\end{justify}

\begin{justify}
“Abandoning false speech \breathmark\ he abstains from false speech \breathmark\ he speaks truth \breathmark\ adheres to truth \breathmark\ is trustworthy and reliable \breathmark\ one who is no deceiver of the world. Abandoning malicious speech \breathmark\ he abstains from malicious speech \breathmark\ he does not repeat elsewhere what he has heard here \breathmark\ in order to divide those people\pagenote{%
  Orig: [those people]. Removed brackets for the sake of recitability.}
from these \breathmark\ nor does he repeat here what he has heard elsewhere \breathmark\ in order to divide these people from those \breathmark\ thus he is one who reunites those who are divided \breathmark\ a promoter of friendships \breathmark\ who enjoys unity \breathmark\ rejoices in unity \breathmark\ delights in unity \breathmark\ a speaker of words that promote unity. Abandoning harsh speech \breathmark\ he abstains from harsh speech \breathmark\ he speaks words that are gentle \breathmark\ pleasing to the ear \breathmark\ and loveable \breathmark\ that go to the heart \breathmark\ are courteous \breathmark\ desired by many and agreeable to many. Abandoning idle chatter \breathmark\ he abstains from idle chatter \breathmark\ he speaks at the right time \breathmark\ speaks what is fact \breathmark\ speaks on what is good \breathmark\ speaks on the Dhamma and Vinaya \breathmark\ at the right time \breathmark\ he speaks words that are worth recording \breathmark\ reasonable \breathmark\ moderate \breathmark\ and beneficial.”
\end{justify}

\begin{justify}
“He abstains from harming seeds and plants. He practices eating one meal a day \breathmark\ abstaining from eating at night and outside the proper time. He abstains from dancing \breathmark\ singing \breathmark\ music and entertainments. He abstains from wearing garlands \breathmark\ smartening himself with scent \breathmark\ and embellishing himself with unguents. He abstains from high and luxurious sleeping places. He abstains from accepting gold and silver. He abstains from accepting raw grain. He abstains from accepting raw meat. He abstains from accepting women and girls. He abstains from accepting men and women slaves. He abstains from accepting goats and sheep. He abstains from accepting fowl and pigs. He abstains from accepting elephants \breathmark\ cattle \breathmark\ horses and mares. He abstains from accepting fields and land.
\end{justify}

\begin{justify}
“He abstains from running errands and messages. He abstains from buying and selling. He abstains from false weights \breathmark\ false metals \breathmark\ and false measures. He abstains from accepting bribes \breathmark\ deceiving \breathmark\ defrauding \breathmark\ and trickery. He abstains fromwounding \breathmark\ murdering \breathmark\ binding \breathmark\ robbery \breathmark\ plunder and violence.”
\end{justify}

\begin{justify}
“He becomes content with robes to protect his body \breathmark\ and almsfood to maintain his stomach \breathmark\ and wherever he goes he sets out taking only these with him. Just as a bird \breathmark\ wherever it goes \breathmark\ flies with its wings as its only burden \breathmark\ so too the bhikkhu becomes content \breathmark\ with robes to protect his body \breathmark\ and almsfood to maintain his stomach \breathmark\ and wherever he goes \breathmark\ he sets out taking only these with him. Possessing this aggregate of noble virtue \breathmark\ he experiences within himself a bliss that is blameless.”
\end{justify}

\suttaRef{[MN 51]}

\begin{justify}
“Then the Tathāgata disciplines him further: ‘Come bhikkhu \breathmark\ guard the doors of your sense faculties. On seeing a form with the eye \breathmark\ do not grasp at its signs and features. Since if you were to leave the eye faculty unguarded \breathmark\ evil unwholesome states of longing and grief might invade you \breathmark\ practice the way of its restraint \breathmark\ guard the eye faculty \breathmark\ undertake the restraint of the eye faculty. On hearing a sound with the ear. On smelling an odour with the nose. On tasting a flavour with the tongue. On touching a tangible with the body. On cognizing a mind-object with the mind \breathmark\ do not grasp at its signs and features. Since if you were to leave the mind faculty unguarded \breathmark\ evil unwholesome states of longing and grief might invade you \breathmark\ practice the way of its restraint \breathmark\ guard the mind faculty \breathmark\ undertake the restraint of the mind faculty.”
\end{justify}

\begin{justify}
“Come bhikkhu \breathmark\ be moderate in eating. Wisely reflecting I use almsfood \breathmark\ not for fun \breathmark\ not for pleasure \breathmark\ not for fattening \breathmark\ not for beautification \breathmark\ only for the maintenance and nourishment of this body \breathmark\ for keeping it healthy \breathmark\ for helping with the holy life \breathmark\ thinking thus: I will allay hunger without overeating \breathmark\ so that I may continue to live blamelessly and at ease.”
\end{justify}

\begin{justify}
“Come bhikkhu \breathmark\ be devoted to wakefulness. During the day \breathmark\ while walking back and forth and sitting \breathmark\ purify your mind of obstructive states. In the first watch of the night \breathmark\ while walking back and forth and sitting \breathmark\ purify your mind of obstructive states. In the middle watch of the night \breathmark\ you should lie down on the right side \breathmark\ in the lion’s pose \breathmark\ with one foot overlapping the other \breathmark\ mindful and fully aware \breathmark\ after noting in your mind the time for rising. After rising in the third watch of the night \breathmark\ while walking back and forth and sitting \breathmark\ purify your mind of obstructive states.”
\end{justify}

\begin{justify}
“Come bhikkhu \breathmark\ be possessed of mindfulness and full awareness. Act in full awareness when going forward and returning \breathmark\ act in full awareness when looking ahead and looking away \breathmark\ act in full awareness when flexing and extending your limbs \breathmark\ act in full awareness when wearing your robes \breathmark\ and carrying your outer robe and bowl \breathmark\ act in full awareness when eating \breathmark\ drinking \breathmark\ consuming food \breathmark\ and tasting \breathmark\ act in full awareness when defecating and urinating \breathmark\ act in full awareness when walking \breathmark\ standing \breathmark\ sitting \breathmark\ falling asleep \breathmark\ waking up\pagenote{%
  Contrary to popular belief, mindfulness and full awareness is not meant to be practiced while sleeping, but only before falling asleep and after waking up.}
\breathmark\ talking \breathmark\ and keeping silent.”
\end{justify}

\begin{justify}
“Come bhikkhu \breathmark\ resort to a secluded resting place: a forest \breathmark\ the foot of a tree \breathmark\ a mountain \breathmark\ a ravine \breathmark\ a hillside cave \breathmark\ a charnel ground \breathmark\ a jungle thicket \breathmark\ an open space \breathmark\ a heap of straw.”
\end{justify}

\begin{justify}
“After his meal \breathmark\ on returning from almsround he sits down \breathmark\ having crossed his legs \breathmark\ sets his body erect \breathmark\ having established mindfulness in front of him. Abandoning longing\pagenote{%
  Orig: “covetousness”}
for the world \breathmark\ he abides with a mind free from longing \breathmark\ he purifies his mind from longing. Abandoning ill will and hatred \breathmark\ he abides with a mind free from ill will \breathmark\ compassionate for the welfare of all living beings \breathmark\ he purifies his mind from ill will and hatred. Abandoning sloth and torpor \breathmark\ he abides free from sloth and torpor \breathmark\ percipient of light \breathmark\ mindful and fully aware \breathmark\ he purifies his mind from sloth and torpor. Abandoning restlessness and remorse \breathmark\ he abides unagitated \breathmark\ with a mind inwardly peaceful \breathmark\ he purifies his mind from restlessness and remorse. Abandoning doubt \breathmark\ he abides having gone beyond doubt \breathmark\ unperplexed about wholesome states \breathmark\ he purifies his mind from doubt.”
\end{justify}

\suttaRef{[MN 107]}

\begin{justify}
“When these five hindrances are unabandoned in himself \breathmark\ he sees them respectively as a debt \breathmark\ a disease \breathmark\ a prison \breathmark\ slavery \breathmark\ and a road across a desert. But when these five hindrances have been abandoned in himself \breathmark\ he sees that as freedom from debt \breathmark\ freedom from disease \breathmark\ release from prison \breathmark\ freedom from slavery \breathmark\ and a land of safety. Considering thus \breathmark\ he would be glad and full of joy.”
\end{justify}

\begin{justify}
“Having abandoned these five hindrances \breathmark\ imperfections of the mind that weaken wisdom \breathmark\ he enters upon and abides in the first jhāna \breathmark\ the second jhāna \breathmark\ the third jhāna \breathmark\ the fourth jhāna.”
\end{justify}

\begin{justify}
“When his concentrated mind is thus purified \breathmark\ bright \breathmark\ unblemished \breathmark\ rid of imperfection \breathmark\ malleable \breathmark\ wieldy \breathmark\ steady \breathmark\ and attained to imperturbability \breathmark\ he directs it to knowledge of the recollection of past lives \breathmark\ to knowledge of the passing away and reappearance of beings \breathmark\ and to knowledge of the destruction of the taints.”
\end{justify}

\begin{justify}
“He understands as it actually is: This is suffering \breathmark\ This is the origin of suffering \breathmark\ This is the cessation of suffering \breathmark\ This is the way of practice \breathmark\ leading to the cessation of suffering \breathmark\ These are the taints \breathmark\ This is the origin of the taints \breathmark\ This is the cessation of the taints \breathmark\ This is the way of practice \breathmark\ leading to the cessation of the taints.”
\end{justify}

\begin{justify}
“When he knows and sees thus \breathmark\ his mind is liberated from the taint of sensual desire \breathmark\ from the taint of being \breathmark\ and from the taint of ignorance. When liberated there is knowledge that he is liberated. He understands: ‘Birth is exhausted \breathmark\ the holy life is fulfilled \breathmark\ what has to be done is done \breathmark\ there is nothing else to do for the sake of liberation.’”
\end{justify}

\suttaRef{[MN 39]}

\begin{justify}
“This is my instruction to those bhikkhus who are in higher training \breathmark\ whose minds have not yet attained the goal \breathmark\ who abide aspiring to the supreme security from bondage.”
\end{justify}

\suttaRef{[MN 107]}

\bottomNav{sharing-aspirations}

\sectionPaliTitle{Bodhipakkihya-dhammā}
\section{Requisites for Awakening}
\label{requisites-for-awakening}

\begin{center}
  [Handa mayaṁ bodhipakkhiya-dhamma-pāṭhaṁ bhaṇāmase]
\end{center}

Bhikkhave ye te mayā dhammā abhiññā desitā

\begin{english}
  Bhikkhus those things I have taught you from my direct knowledge
\end{english}

Te vo sādhukaṁ uggahetvā

\begin{english}
  Having been thoroughly learned by you
\end{english}

Āsevitabbā bhāvetabbā bahulīkātabbā

\begin{english}
  Should be practiced developed and made much of
\end{english}

Yathayidaṁ brahmacariyaṁ addhaniyaṁ assa ciraṭṭhitikaṁ

\begin{english}
  So that this holy life may last for a long time
\end{english}

Tadassa bahujana-hitāya bahujana-sukhāya

\begin{english}
  That would be for the welfare and happiness of many people
\end{english}

Lokānukampāya

\begin{english}
  Out of compassion for the world
\end{english}

Atthāya hitāya sukhāya devamanussānaṁ

\begin{english}
  For the benefit welfare and happiness of gods and humans
\end{english}

Katame ca te bhikkhave dhammā mayā abhiññā desitā

\begin{english}
  And what bhikkhus are those things I have taught you from my direct knowledge?
\end{english}

Seyyathīdaṁ

\begin{english}
  They are as follows:
\end{english}

Cattāro satipaṭṭhānā

\begin{english}
  The Four Foundations of Mindfulness
\end{english}

Cattāro sammappadhānā

\begin{english}
  The Four Right Strivings
\end{english}

Cattāro iddhipādā

\begin{english}
  The Four Bases of Spiritual Power
\end{english}

Pañcindriyāni

\begin{english}
  The Five Faculties
\end{english}

Pañca balāni

\begin{english}
  The Five Powers
\end{english}

Satta bojjhaṅgā

\begin{english}
  The Seven Factors of Awakening
\end{english}

Ariyo aṭṭhaṅgiko maggo

\begin{english}
  The Noble Eightfold Path
\end{english}

\suttaRef{[DN 16]}

\bottomNav{principles-of-non-decline}

\sectionPaliTitle{Satta-sambojjhaṅgā}
\section{The Seven Factors of Awakening}
\label{seven-factors-of-awakening}

\begin{center}
  [Handa mayaṁ satta-sambojjhaṅga-pāṭhaṁ bhaṇāmase]
\end{center}

Sattime bhikkhave bojjhaṅgā bhāvitā bahulīkatā

\begin{english}
  Bhikkhus when the Seven Factors of Awakening are developed and cultivated
\end{english}

Ariyā niyyānikā

\begin{english}
  They are noble and emancipating
\end{english}

Nīyanti takkarassa sammā dukkhakkhayāya

\begin{english}
  Acting them out \breathmark\ leads to the complete destruction of suffering
\end{english}

/suttaRef{[SN 46.19]}

Ye te bhikkhave bhikkhū

\begin{english}
  Bhikkhus those bhikkhus
\end{english}

Sīlasampannā

\begin{english}
  Who are accomplished in virtue
\end{english}

Samādhisampannā

\begin{english}
  Accomplished in concentration
\end{english}

Ñāṇasampannā

\begin{english}
  Accomplished in wisdom
\end{english}

Vimuttisampannā

\begin{english}
  Accomplished in liberation
\end{english}

Vimuttiñāṇadassanasampannā

\begin{english}
  Accomplished in the knowledge and vision of liberation:
\end{english}

Dassanam-pāhaṁ bhikkhave tesaṁ bhikkhūnaṁ bahukāraṁ vadāmi

\begin{english}
  I say even the sight of those bhikkhus is helpful
\end{english}

Savanam-pāhaṁ

\begin{english}
  Even listening to them
\end{english}

Upasaṅkamanam-pāhaṁ

\begin{english}
  Even approaching them
\end{english}

Payirupāsanam-pāhaṁ

\begin{english}
  Even attending on them
\end{english}

Anussatim-pāhaṁ

\begin{english}
  Even recollecting them
\end{english}

Anupabbajjam-pāhaṁ

\begin{english}
  Even going forth after them is helpful
\end{english}

Taṁ kissa hetu

\begin{english}
  For what reason?
\end{english}

Tathārūpānaṁ bhikkhave bhikkhūnaṁ dhammaṁ sutvā

\begin{english}
  Because when one has heard the Dhamma from such bhikkhus
\end{english}

Dvayena vūpakāsena vūpakaṭṭho viharati

\begin{english}
  One dwells withdrawn by way of two kinds of withdrawal
\end{english}

Kāyavūpakāsena ca cittavūpakāsena ca

\begin{english}
  Withdrawal of body and withdrawal of mind
\end{english}

So tathā vūpakaṭṭho viharanto

\begin{english}
  Dwelling thus withdrawn
\end{english}

Taṁ dhammaṁ anussarati anuvitakketi

\begin{english}
  One recollects that Dhamma and thinks it over
\end{english}

So tathā sato viharanto

\begin{english}
  Dwelling thus mindfully
\end{english}

Taṁ dhammaṁ paññāya pavicinati

\begin{english}
  One discriminates that Dhamma with wisdom
\end{english}

Pavicarati parivīmaṁsam-āpajjati

\begin{english}
  Examines it \breathmark\ makes an investigation of it
\end{english}

Tassa taṁ dhammaṁ paññāya pavicinato

\begin{english}
  For one who discriminates that Dhamma with wisdom
\end{english}

Pavicarato parivīmaṁsam-āpajjato

\begin{english}
  Examines it \breathmark\ makes an investigation of it
\end{english}

Āraddhaṁ hoti vīriyaṁ asallīnaṁ

\begin{english}
  One’s energy is aroused without slackening
\end{english}

Āraddhavīriyassa uppajjati pīti nirāmisā

\begin{english}
  For one who is energetic\\
  Spiritual rapture arises
\end{english}

Pītimanassa kāyopi passambhati

\begin{english}
  For one whose mind is uplifted by rapture\\
  The body becomes tranquil
\end{english}

Cittampi passambhati

\begin{english}
  And the mind becomes tranquil
\end{english}

Passaddhakāyassa sukhino

\begin{english}
  For one whose body is tranquil and who is happy
\end{english}

Cittaṁ samādhiyati

\begin{english}
  The mind becomes concentrated
\end{english}

So tathāsamāhitaṁ cittaṁ sādhukaṁ ajjhupekkhitā hoti

\begin{english}
  One closely looks on with equanimity
  At the mind thus concentrated
\end{english}

\suttaRef{[SN 46.3]}

Ime kho bhikkhave satta bojjhaṅgā’ti

\begin{english}
  Bhikkhus these are the Seven Factors of Awakening
\end{english}

\suttaRef{[SN 46.22]}

\bottomNav{words-on-loving-kindness}

\sectionPaliTitle{Ariy’aṭṭhaṅgika-magga}
\section{The Noble Eightfold Path}
\label{noble-eightfold-path}

\begin{center}
  [Handa mayaṁ ariyaṭṭhaṅgika-magga-pāṭhaṁ bhaṇāmase]
\end{center}

Ayam-eva ariyo aṭṭh'aṅgiko maggo

\begin{english}
  This is the Noble Eightfold Path
\end{english}

Seyyathīdaṁ

\begin{english}
  Which is as follows
\end{english}

Sammā-diṭṭhi

\begin{english}
  Right View
\end{english}

Sammā-saṅkappo

\begin{english}
  Right Intention
\end{english}

Sammā-vācā

\begin{english}
  Right Speech
\end{english}

Sammā-kammanto

\begin{english}
  Right Action
\end{english}

Sammā-ājīvo

\begin{english}
  Right Livelihood
\end{english}

Sammā-vāyāmo

\begin{english}
  Right Effort
\end{english}

Sammā-sati

\begin{english}
  Right Mindfulness
\end{english}

Sammā-samādhi

\begin{english}
  Right Concentration
\end{english}

Katamā ca bhikkhave sammā-diṭṭhi

\begin{english}
  And what bhikkhus is Right View?
\end{english}

Yaṁ kho bhikkhave dukkhe ñāṇaṁ

\begin{english}
  Knowledge of suffering
\end{english}

Dukkha-samudaye ñāṇaṁ

\begin{english}
  Knowledge of the origin of suffering
\end{english}

Dukkha-nirodhe ñāṇaṁ

\begin{english}
  Knowledge of the cessation of suffering
\end{english}

Dukkha-nirodha-gāminiyā paṭipadāya ñāṇaṁ

\begin{english}
  Knowledge of the way of practice\\
  Leading to the cessation of suffering
\end{english}

Ayaṁ vuccati bhikkhave sammā-diṭṭhi

\begin{english}
  This bhikkhus is called Right View
\end{english}

Katamo ca bhikkhave sammā-saṅkappo

\begin{english}
  And what bhikkhus is Right Intention?
\end{english}

Nekkhamma-saṅkappo

\begin{english}
  The intention of renunciation
\end{english}

Abyāpāda-saṅkappo

\begin{english}
  The intention of non-ill-will
\end{english}

Avihiṁsā-saṅkappo

\begin{english}
  The intention of non-cruelty
\end{english}

Ayaṁ vuccati bhikkhave sammā-saṅkappo

\begin{english}
  This bhikkhus is called Right Intention
\end{english}

Katamā ca bhikkhave sammā-vācā

\begin{english}
  And what bhikkhus is Right Speech?
\end{english}

Musā-vādā veramaṇī

\begin{english}
  Abstaining from false speech
\end{english}

Pisuṇāya vācāya veramaṇī

\begin{english}
  Abstaining from malicious speech
\end{english}

Pharusāya vācāya veramaṇī

\begin{english}
  Abstaining from harsh speech
\end{english}

Samphappalāpā veramaṇī

\begin{english}
  Abstaining from idle chatter
\end{english}

Ayaṁ vuccati bhikkhave sammā-vācā

\begin{english}
  This bhikkhus is called Right Speech
\end{english}

Katamo ca bhikkhave sammā-kammanto

\begin{english}
  And what bhikkhus is Right Action?
\end{english}

Pāṇātipātā veramaṇī

\begin{english}
  Abstaining from killing living beings
\end{english}

Adinnādānā veramaṇī

\begin{english}
  Abstaining from taking what is not given
\end{english}

Kāmesu-micchācārā veramaṇī

\begin{english}
  Abstaining from sexual misconduct
\end{english}

Ayaṁ vuccati bhikkhave sammā-kammanto

\begin{english}
  This bhikkhus is called Right Action
\end{english}

Katamo ca bhikkhave sammā-ājīvo

\begin{english}
  And what bhikkhus is Right Livelihood?
\end{english}

Idha bhikkhave ariya-sāvako\\
Micchā-ājīvaṁ pahāya\\
Sammā-ājīvena jīvitaṁ kappeti

\begin{english}
  Here bhikkhus a noble disciple\\
  Having abandoned wrong livelihood\\
  Earns his living by right livelihood
\end{english}

Ayaṁ vuccati bhikkhave sammā-ājīvo

\begin{english}
  This bhikkhus is called Right Livelihood
\end{english}

Katamo ca bhikkhave sammā-vāyāmo

\begin{english}
  And what bhikkhus is Right Effort?
\end{english}

Idha bhikkhave bhikkhu\\
Anuppannānaṁ pāpakānaṁ akusalānaṁ dhammānaṁ anuppādāya\\
Chandaṁ janeti\\
Vāyamati\\
Vīriyaṁ ārabhati\\
Cittaṁ paggaṇhāti padahati

\begin{english}
  Here bhikkhus a bhikkhu awakens zeal\\
  For the non-arising of unarisen evil unwholesome states\\
  He puts forth effort\\
  Arouses energy\\
  Exerts his mind\\
  And strives
\end{english}

Uppannānaṁ pāpakānaṁ akusalānaṁ dhammānaṁ pahānāya\\
Chandaṁ janeti\\
Vāyamati\\
Vīriyaṁ ārabhati\\
Cittaṁ paggaṇhāti padahati

\begin{english}
  He awakens zeal for the abandoning of arisen evil unwholesome states\\
  He puts forth effort\\
  Arouses energy\\
  Exerts his mind\\
  And strives
\end{english}

Anuppannānaṁ kusalānaṁ dhammānaṁ uppādāya\\
Chandaṁ janeti\\
Vāyamati\\
Vīriyaṁ ārabhati\\
Cittaṁ paggaṇhāti padahati

\begin{english}
  He awakens zeal for the arising of unarisen wholesome states\\
  He puts forth effort\\
  Arouses energy\\
  Exerts his mind\\
  And strives
\end{english}

Uppannānaṁ kusalānaṁ dhammānaṁ ṭhitiyā\\
Asammosāya\\
Bhiyyobhāvāya\\
Vepullāya\\
Bhāvanāya pāripūriyā\\
Chandaṁ janeti\\
Vāyamati\\
Vīriyaṁ ārabhati\\
Cittaṁ paggaṇhāti padahati

\begin{english}
  He awakens zeal for the continuance\\
  Non-disappearance\\
  Strengthening\\
  Increase and fulfillment by development\\
  Of arisen wholesome states\\
  He puts forth effort\\
  Arouses energy\\
  Exerts his mind\\
  And strives
\end{english}

Ayaṁ vuccati bhikkhave sammā-vāyāmo

\begin{english}
  This bhikkhus is called Right Effort
\end{english}

Katamā ca bhikkhave sammā-sati

\begin{english}
  And what bhikkhus is Right Mindfulness?
\end{english}

Idha bhikkhave bhikkhu kāye kāyānupassī viharati

\begin{english}
  Here bhikkhus a bhikkhu abides\\
  Contemplating the body as a body
\end{english}

Ātāpī sampajāno satimā

\begin{english}
  Ardent \breathmark\ fully aware \breathmark\ and mindful
\end{english}

Vineyya loke abhijjhā-domanassaṁ

\begin{english}
  Having put away
  Longing and grief for the world\pagenote{%
    Orig: “covetousness and grief”}
\end{english}

Vedanāsu vedanānupassī viharati

\begin{english}
  He abides contemplating feelings as feelings
\end{english}

Ātāpī sampajāno satimā

\begin{english}
  Ardent \breathmark\ fully aware \breathmark\ and mindful
\end{english}

Vineyya loke abhijjhā-domanassaṁ

\begin{english}
  Having put away\\
  Longing and grief for the world
\end{english}

Citte cittānupassī viharati

\begin{english}
  He abides contemplating mind as mind
\end{english}

Ātāpī sampajāno satimā

\begin{english}
  Ardent \breathmark\ fully aware \breathmark\ and mindful
\end{english}

Vineyya loke abhijjhā-domanassaṁ

\begin{english}
  Having put away\\
  Longing and grief for the world
\end{english}

Dhammesu dhammānupassī viharati

\begin{english}
  He abides contemplating dhammas as dhammas\pagenote{%
    Orig: “He abides contemplating mind-objects as mind-objects”. Since “mind-object” is not an ideal translation for “dhamma” in this context, it is preferable to leave “dhamma” untranslated here.}
\end{english}

Ātāpī sampajāno satimā

\begin{english}
  Ardent \breathmark\ fully aware \breathmark\ and mindful
\end{english}

Vineyya loke abhijjhā-domanassaṁ

\begin{english}
  Having put away\\
  Longing and grief for the world
\end{english}

Ayaṁ vuccati bhikkhave sammā-sati

\begin{english}
  This bhikkhus is called Right Mindfulness
\end{english}

Katamo ca bhikkhave sammā-samādhi

\begin{english}
  And what bhikkhus is Right Concentration?
\end{english}

Idha bhikkhave bhikkhu

\begin{english}
  Here bhikkhus a bhikkhu
\end{english}

Vivicc’eva\pagenote{%
  “Quite secluded from sense pleasures” means being completely and entirely secluded; not just somewhat/moderately secluded. The Pāli term ‘eva’, which has been translated as ‘quite’ is an emphatic particle, intensifying the adjective it qualifies. Acc. to Oxford English Dictionary, the English term ‘quite’ has two connotations: 1. to the utmost or most absolute extent or degree; absolutely; completely; 2. To a certain or fairly significant extent or degree; fairly. It is the first connotation in which ‘quite’ is used in the phrase “quite secluded from sense pleasures”.}
kāmehi

\begin{english}
  Quite secluded from sense pleasures
\end{english}

Vivicca akusalehi dhammehi

\begin{english}
  Secluded from unwholesome states
\end{english}

Savitakkaṁ savicāraṁ viveka-jaṁ pīti-sukhaṁ paṭhamaṁ jhānaṁ upasampajja viharati

\begin{english}
  Enters upon and abides \breathmark\ in the first Jhāna\\
  Accompanied by thought and examination\\
  With rapture and pleasure \breathmark\ born of seclusion
\end{english}

Vitakka-vicārānaṁ vūpasamā

\begin{english}
  With the stilling of thought and examination
\end{english}

Ajjhattaṁ sampasādanaṁ\\
Cetaso ekodibhāvaṁ\\
Avitakkaṁ avicāraṁ samādhi-jaṁ pīti-sukhaṁ dutiyaṁ jhānaṁ upasampajja viharati

\begin{english}
  He enters upon and abides \breathmark\ in the second Jhāna\\
  Accompanied by self-confidence \breathmark\ and singleness of mind\\
  Without thought and examination\\
  With rapture and pleasure \breathmark\ born of concentration
\end{english}

Pītiyā ca virāgā

\begin{english}
  With the fading away as well of rapture
\end{english}

Upekkhako ca viharati

\begin{english}
  He abides in equanimity
\end{english}

Sato ca sampajāno

\begin{english}
  Mindful \breathmark\ and fully aware
\end{english}

Sukhañ-ca kāyena paṭisaṁvedeti

\begin{english}
  And experiencing pleasure with the body\pagenote{%
    Orig: “Still feeling pleasure with the body”. The Pāli doesn't say “still”, but more importantly SN 48.40 states that physical pleasure (sukha) has ceased in 3rd Jhāna, and mental pleasure (somanassa) has ceased in 4th Jhāna. Therefore, according to SN 48.40 kāya in the context of the 3rd Jhāna stock formula cannot refer to the physical body. The Pāli commentaries agree, and explain that “kāya” here refers to the ‘mental body’, in particular to mental pleasure (somanassa). However, a comparative study by Prof. Kuan Tse-Fu titled “Clarification on Feelings in Buddhist Dhyāna/Jhāna Meditation” brought to light that the Chinese parallel to SN 48.40 has a different sequence of disappearing types of vedanā in the sequence of the four Jhānas. While it would go beyond the scope of this footnote to discuss the matter in detail, it is noteworthy that the parallel to SN 48.40, the Aviparātaka Sūtra of the Āgamas, indeed mentions that bodily pleasant feeling (sukha) disappears only in the 4th Jhāna, whereas mental pleasant feeling (somanassa) has disappeared already in the 3rd Jhāna, alongside with the disappearance of pīti. In the context of this chanting book, we have therefore chosen to stay with the literal translation of the word “kāya” as “body”, and invite the reader to draw his own conclusions.}
\end{english}

Yaṁ taṁ ariyā ācikkhanti
‘Upekkhako satimā sukha-vihārī’ti\pagenote{%
  Orig: “viharatī’ti“}
tatiyaṁ jhānaṁ upasampajja viharat

\begin{english}
  He enters upon and abides \breathmark\ in the third Jhāna\\
  On account of which the Noble Ones announce\\
  ‘He has a pleasant abiding\\
  With equanimity and is mindful’
\end{english}

Sukhassa ca pahānā

\begin{english}
  With the abandoning of pleasure
\end{english}

Dukkhassa ca pahānā

\begin{english}
  And the abandoning of pain
\end{english}

Pubb’eva somanassa domanassānaṁ atthaṅgamā

\begin{english}
  With the previous disappearance of joy and displeasure\pagenote{%
    Orig: “grief”}
\end{english}

Adukkhamasukhaṁ upekkhā-sati-pārisuddhiṁ catutthaṁ jhānaṁ upasampajja viharati

\begin{english}
  He enters upon and abides \breathmark\ in the fourth Jhāna\\
  Accompanied by neither pain nor pleasure\\
  And purity of mindfulness\\
  Due to equanimity
\end{english}

Ayaṁ vuccati bhikkhave sammā-samādhi

\begin{english}
  This bhikkhus is called Right Concentration
\end{english}

Ayam-eva ariyo aṭṭh'aṅgiko maggo

\begin{english}
  This is the Noble Eightfold Path
\end{english}

\suttaRef{[SN 45.8]}

\bottomNav{repulsiveness-of-food}

\sectionPaliTitle{Ānāpānassati}
\section{Mindfulness of Breathing}
\label{mindfulness-of-breathing}

\begin{center}
  [Handa mayaṁ ānāpānassati-sutta-pāṭhaṁ bhaṇāmase]
\end{center}

Ānāpānassati bhikkhave bhāvitā bahulī-katā

\begin{english}
  Bhikkhus when mindfulness of breathing is developed and cultivated
\end{english}

Mahapphalā hoti mahā-nisaṁsā

\begin{english}
  It is of great fruit and great benefit
\end{english}

Ānāpānassati bhikkhave bhāvitā bahulī-katā

\begin{english}
  When mindfulness of breathing is developed and cultivated
\end{english}

Cattāro satipaṭṭhāne paripūreti

\begin{english}
  It fulfills the Four Foundations of Mindfulness
\end{english}

Cattāro satipaṭṭhānā bhāvitā bahulī-katā

\begin{english}
  When the Four Foundations of Mindfulness are developed and cultivated
\end{english}

Satta-bojjhaṅge paripūrenti

\begin{english}
  They fulfill the Seven Factors of Awakening
\end{english}

Satta-bojjhaṅgā bhāvitā bahulī-katā

\begin{english}
  When the Seven Factors of Awakening are developed and cultivated
\end{english}

Vijjā-vimuttiṁ paripūrenti

\begin{english}
  They fulfill true knowledge and deliverance
\end{english}

Kathaṁ bhāvitā ca bhikkhave ānāpānassati kathaṁ bahulī-katā

\begin{english}
  And how bhikkhus is mindfulness of breathing developed and cultivated
\end{english}

Mahapphalā hoti mahā-nisaṁsā

\begin{english}
  So that it is of great fruit and great benefit?
\end{english}

Idha bhikkhave bhikkhu

\begin{english}
  Here bhikkhus a bhikkhu
\end{english}

Arañña-gato vā

\begin{english}
  Gone to the forest
\end{english}

Rukkha-mūla-gato vā

\begin{english}
  To the foot of a tree
\end{english}

Suññāgāra-gato vā

\begin{english}
  Or to an empty hut
\end{english}

Nisīdati pallaṅkaṁ ābhujitvā

\begin{english}
  Sits down \breathmark\ having crossed his legs
\end{english}

Ujuṁ kāyaṁ paṇidhāya parimukhaṁ satiṁ upaṭṭhapetvā

\begin{english}
  Sets his body erect\\
  Having established mindfulness in front of him
\end{english}

So sato'va assasati sato'va passasati

\begin{english}
  Ever mindful he breathes in\\
  Mindful he breathes out
\end{english}

Dīghaṁ vā assasanto dīghaṁ assasāmī'ti pajānāti

\begin{english}
  Breathing in long he knows `I breathe in long'
\end{english}

Dīghaṁ vā passasanto dīghaṁ passasāmī'ti pajānāti

\begin{english}
  Breathing out long he knows `I breathe out long'
\end{english}

Rassaṁ vā assasanto rassaṁ assasāmī'ti pajānāti

\begin{english}
  Breathing in short he knows `I breathe in short'
\end{english}

Rassaṁ vā passasanto rassaṁ passasāmī'ti pajānāti

\begin{english}
  Breathing out short he knows `I breathe out short'
\end{english}

Sabba-kāya-paṭisaṁvedī assasissāmī'ti sikkhati

\begin{english}
  He trains thus:\\
  `I shall breathe in experiencing the whole body'
\end{english}

Sabba-kāya-paṭisaṁvedī passasissāmī'ti sikkhati

\begin{english}
  He trains thus:\\
  `I shall breathe out experiencing the whole body'
\end{english}

Passambhayaṁ kāya-saṅkhāraṁ assasissāmī'ti sikkhati

\begin{english}
  He trains thus:\\
  `I shall breathe in tranquillizing the bodily formation'\pagenote{%
    Orig: “I shall breathe in tranquillising the bodily formations”. Kāyasaṅkhāraṁ is singular, not plural.}
\end{english}

Passambhayaṁ kāya-saṅkhāraṁ passasissāmī'ti sikkhati

\begin{english}
  He trains thus:\\
  `I shall breathe out tranquillizing the bodily formation'
\end{english}

Pīti-paṭisaṁvedī assasissāmī'ti sikkhati

\begin{english}
  He trains thus:\\
  `I shall breathe in experiencing rapture'
\end{english}

Pīti-paṭisaṁvedī passasissāmī'ti sikkhati

\begin{english}
  He trains thus:\\
  `I shall breathe out experiencing rapture'
\end{english}

Sukha-paṭisaṁvedī assasissāmī'ti sikkhati

\begin{english}
  He trains thus:\\
  `I shall breathe in experiencing pleasure'
\end{english}

Sukha-paṭisaṁvedī passasissāmī'ti sikkhati

\begin{english}
  He trains thus:\\
  `I shall breathe out experiencing pleasure’
\end{english}

% % TODO: choose cprenglish or english and also find difference between quotes

\begin{english}
  He trains thus:\\
  `I shall breathe in experiencing pleasure'
\end{english}

Sukha-paṭisaṁvedī passasissāmī'ti sikkhati

\begin{english}
  He trains thus:\\
  `I shall breathe out experiencing pleasure'
\end{english}

Citta-saṅkhāra-paṭisaṁvedī assasissāmī'ti sikkhati

\begin{english}
  He trains thus:\\
  `I shall breathe in experiencing the mental formations'\pagenote{%
    Orig: “I shall breathe in experiencing the mental formations”. Cittasaṅkhāraṁ is probably meant to be singular, not plural. This is not clear when looking at compounds, however, considering that the subsequent practice explicitly uses mental formation in singular cittasaṅkhāraṁ”, this suggests that it is probably used in the same way here.}
\end{english}

Citta-saṅkhāra-paṭisaṁvedī passasissāmī'ti sikkhati

\begin{english}
  He trains thus:\\
  `I shall breathe out experiencing the mental formations'.
\end{english}

Passambhayaṁ citta-saṅkhāraṁ assasissāmī'ti sikkhati

\begin{english}
  He trains thus:\\
  `I shall breathe in tranquillizing the mental formations'\pagenote{%
    Orig: “I shall breathe in tranquillising the mental formations”. Cittasaṅkhāraṁ is singular, not plural.}
\end{english}

Passambhayaṁ citta-saṅkhāraṁ passasissāmī'ti sikkhati

\begin{english}
  He trains thus:\\
  `I shall breathe out tranquillizing the mental formations'
\end{english}

Citta-paṭisaṁvedī assasissāmī'ti sikkhati

\begin{english}
  He trains thus:\\
  `I shall breathe in experiencing the mind'
\end{english}

Citta-paṭisaṁvedī passasissāmī'ti sikkhati

\begin{english}
  He trains thus:\\
  `I shall breathe out experiencing the mind'
\end{english}

Abhippamodayaṁ cittaṁ assasissāmī'ti sikkhati

\begin{english}
  He trains thus:\\
  `I shall breathe in gladdening the mind'
\end{english}

Abhippamodayaṁ cittaṁ passasissāmī'ti sikkhati

\begin{english}
  He trains thus:\\
  `I shall breathe out gladdening the mind'
\end{english}

Samādahaṁ cittaṁ assasissāmī'ti sikkhati

\begin{english}
  He trains thus:\\
  `I shall breathe in concentrating the mind'
\end{english}

Samādahaṁ cittaṁ passasissāmī'ti sikkhati

\begin{english}
  He trains thus:\\
  `I shall breathe out concentrating the mind'
\end{english}

Vimocayaṁ cittaṁ assasissāmī'ti sikkhati

\begin{english}
  He trains thus:\\
  `I shall breathe in liberating the mind'
\end{english}

Vimocayaṁ cittaṁ passasissāmī'ti sikkhati

\begin{english}
  He trains thus:\\
  `I shall breathe out liberating the mind'
\end{english}

Aniccānupassī assasissāmī'ti sikkhati

\begin{english}
  He trains thus:\\
  `I shall breathe in contemplating impermanence'
\end{english}

Aniccānupassī passasissāmī'ti sikkhati

\begin{english}
  He trains thus:\\
  `I shall breathe out contemplating impermanence'
\end{english}

Virāgānupassī assasissāmī'ti sikkhati

\begin{english}
  He trains thus:\\
  `I shall breathe in contemplating the fading away of passions'
\end{english}

Virāgānupassī passasissāmī'ti sikkhati

\begin{english}
  He trains thus:\\
  `I shall breathe out contemplating the fading away of passions'
\end{english}

Nirodhānupassī assasissāmī'ti sikkhati

\begin{english}
  He trains thus:\\
  `I shall breathe in contemplating cessation'
\end{english}

Nirodhānupassī passasissāmī'ti sikkhati

\begin{english}
  He trains thus:\\
  `I shall breathe out contemplating cessation'
\end{english}

Paṭinissaggānupassī assasissāmī'ti sikkhati

\begin{english}
  He trains thus:\\
  `I shall breathe in contemplating relinquishment'
\end{english}

Paṭinissaggānupassī passasissāmī'ti sikkhati

\begin{english}
  He trains thus:\\
  `I shall breathe out contemplating relinquishment'
\end{english}

Evaṁ bhāvitā kho bhikkhave ānāpānassati evaṁ bahulīkatā

\begin{english}
  Bhikkhus, that is how mindfulness of breathing is developed and cultivated
\end{english}

Mahapphalā hoti mahānisaṁsā'ti

\begin{english}
  So that it is of great fruit and great benefit.
\end{english}

\suttaRef{[MN 118]}

\bottomNav{highest-blessings}

\sectionPaliTitle{Paṭicca samuppāda}
\section{Dependent Origination}
\label{dependent-origination}

\begin{center}
  [Handa mayaṁ paṭicca samuppāda-vibhaṅgaṁ bhaṇāmase]
\end{center}
\begin{center}
  [Now let us recite the Analysis of Dependent Origination\pagenote{%
    Lit.: “the 'Discourse Analysis' from the 'Analysis of Dependent Origination'”  The “Discourse Analysis” is a sub-chapter (Abh.Vibh.130f) from the “Analysis of Dependent Origination”, which is part of the 2nd book of the Abhidhammapiṭaka called the “Vibhaṅga”. Apart from minor variations, there is great similarity between this analysis and the analysis found in SN 12.2, as part of the Suttapiṭaka.}]
\end{center}

Avijjā-paccayā saṅkhārā

\begin{english}
  From ignorance as a\pagenote{%
    Here and at other places of the English translation, the term “a condition” is used. The indefinite article “a” indicates, that there could be other conditions as well (e.g. all previous conditions in the sequence are a condition for all subsequent ones), but the directly aforementioned condition is the predominant one (adhipati-paccaya) for the subsequent conditioned thing to arise.}
  condition arise\pagenote{%
    Here and at other places of the English translation, the term “arises” is inserted, because the term “sambhavati” from “Jāti-paccayā jarāmaraṇaṁ; soka parideva dukkha domanassupāyāsā sambhavanti” applies to all 11 links (12 minus ignorance) and not only to ageing-and-death etc.}
  formations
\end{english}

Saṅkhāra-paccayā viññāṇaṁ

\begin{english}
  From formations as a condition arises consciousness
\end{english}

Viññāṇa-paccayā nāmarūpaṁ

\begin{english}
  From consciousness as a condition arises name-and-form\pagenote{%
    In the context of dependent origination, the compound “nāmarūpa” is translated as “mind-and-body” in order to cover the entirety of what is conventionally called “a being, a person”. In other contexts, particularly if used separately, these terms may have other connotations.}
\end{english}

Nāmarūpa-paccayā saḷāyatanaṁ

\begin{english}
  From name-and-form as a condition arises the sixfold-sense-base
\end{english}

Saḷāyatana-paccayā phasso

\begin{english}
  From the sixfold-sense-base as a condition arises contact
\end{english}

Phassa-paccayā vedanā

\begin{english}
  From contact as a condition arises feeling
\end{english}

Vedanā-paccayā taṇhā

\begin{english}
  From feeling as a condition arises craving
\end{english}

Taṇhā-paccayā upādānaṁ

\begin{english}
  From craving as a condition arises clinging
\end{english}

Upādāna-paccayā bhavo

\begin{english}
  From clinging as a condition arises becoming
\end{english}

Bhava-paccayā jāti

\begin{english}
  From becoming as a condition arises birth
\end{english}

Jāti-paccayā jarāmaraṇaṁ soka parideva dukkha domanssupāyāsā sambhavanti

\begin{english}
  From birth as a condition arise ageing-and-death\\
  Sorrow lamentation pain displeasure and despair
\end{english}

Evametassa kevalassa dukkhakkhandhassa samudayo hoti

\begin{english}
  Such is the origin of this whole mass of suffering
\end{english}

Tattha katamā avijjā

\begin{english}
  Therein what is ignorance?
\end{english}

Dukkhe aññāṇaṁ dukkhasamudaye aññāṇaṁ dukkhanirodhe aññāṇaṁ dukkhanirodhagāminiyā paṭipadāya aññāṇaṁ

\begin{english}
  Not knowing suffering \breathmark\ not knowing the origin of suffering \breathmark\ not knowing the cessation of suffering \breathmark\ not knowing the way of practice leading to the cessation of suffering
\end{english}

Ayaṁ vuccati avijjā

\begin{english}
  This is called 'ignorance'
\end{english}

Tattha katame avijjā-paccayā saṅkhārā

\begin{english}
  Therein what are 'formations' \breathmark\ arising\pagenote{%
    Here and at other places of the English translation, the term “arises” is inserted, because the term “sambhavati” from “Jāti-paccayā jarāmaraṇaṁ; soka parideva dukkha domanassupāyāsā sambhavanti” applies to all 11 links (12 minus ignorance) and not only to ageing-and-death etc.}
  from ignorance as a condition?\pagenote{%
    To render as “Therein what is from ignorance as a condition arise formations.” would be misleading, because it is not the conditioned relationship between A and B (here: ignorance and formations) that gets elaborated upon in the following lines, but it is only the term B (here: formations), that gets defined. Hence the preferable translation: “Therein what are 'formations', arising from ignorance as a condition?”}
\end{english}

Puññābhisaṅkhāro apuññābhisaṅkhāro āneñjābhisaṅkhāro\pagenote{%
  SN 12.51 explains: “Bhikkhus, if a person immersed in ignorance generates a meritorious volitional formation, consciousness fares on to the meritorious; if he generates a demeritorious volitional formation, consciousness fares on to the demeritorious; if he generates an imperturbable volitional formation, consciousness fares on to the imperturbable.”}\\
Kāyasaṅkhāro vacīsaṅkhāro cittasaṅkhāro

\begin{english}
  Heightened formation of wholesomeness\\
  Heightened formation of unwholesomeness\\
  Heightened formation of imperturbability\\
  The bodily formation \breathmark\ the verbal formation \breathmark\ the mental formation
\end{english}

Tattha katamo puññābhisaṅkhāro

\begin{english}
  Therein what is 'heightened formation of wholesomeness'?
\end{english}

Kusalā cetanā kāmāvacarā rūpāvacarā dānamayā sīlamayā bhāvanāmayā

\begin{english}
  Skillful volition of the sense-sphere \breathmark\ of the form-sphere \breathmark\ connected with giving \breathmark\ connected with virtue \breathmark\ connected with meditation
\end{english}

Ayaṁ vuccati puññābhisaṅkhāro

\begin{english}
  This is called 'heightened formation of wholesomeness'
\end{english}

Tattha katamo apuññābhisaṅkhāro

\begin{english}
  Therein what is 'heightened formation of unwholesomeness'?
\end{english}

Akusalā cetanā kāmāvacarā

\begin{english}
  Unskillful volition of the sense-sphere
\end{english}

Ayaṁ vuccati apuññābhisaṅkhāro

\begin{english}
  This is called 'heightened formation of unwholesomeness'
\end{english}

Tattha katamo āneñjābhisaṅkhāro

\begin{english}
  Therein what is 'heightened formation of imperturbability'?
\end{english}

Kusalā cetanā arūpāvacarā

\begin{english}
  Skillful volition of the formless-sphere
\end{english}

Ayaṁ vuccati āneñjābhisaṅkhāro

\begin{english}
  This is called 'heightened formation of imperturbability'
\end{english}

Tattha katamo kāyasaṅkhāro

\begin{english}
  Therein what is 'the bodily formation'?
\end{english}

Kāyasañcetanā kāyasaṅkhāro vacīsañcetanā vacīsaṅkhāro manosañcetanā cittasaṅkhāro\pagenote{%
  Further explained in SN 12.25}

\begin{english}
  Volition associated with the body is the bodily formation\\
  Volition associated with speech is the verbal formation\\
  Volition associated with the mind is the mental\pagenote{%
    Manosañcetanā cittasaṅkhāro is translated here as “volition associated with the mind is the mental formation”. Despite of mano and citta having in certain contexts different shades of meaning, both were translated here as “mind”. Generally speaking mano refers more to the intellectual, whereas citta covers more the emotional/affective aspects of the mind. A detailed analysis of these terms can be found in Rune E. A. Johansson’s “Citta, Mano, Viññāṇa — a Psychosemantic Investigation”.}
  formation
\end{english}

Ime vuccanti avijjā-paccayā saṅkhārā

\begin{english}
  These are called 'formations' \breathmark\ arising from ignorance as a condition
\end{english}

Tattha katamaṁ saṅkhāra-paccayā viññāṇaṁ

\begin{english}
  Therein what is 'consciousness' \breathmark\ arising from formations as a condition?
\end{english}

Cakkhuviññāṇaṁ sotaviññāṇaṁ ghānaviññāṇaṁ jivhāviññāṇaṁ kāyaviññāṇaṁ manoviññāṇaṁ

\begin{english}
  Eye-consciousness ear-consciousness nose-consciousness tongue-consciousness body-consciousness mind-consciousness
\end{english}

Idaṁ vuccati saṅkhāra-paccayā viññāṇaṁ

\begin{english}
  This is called 'consciousness' \breathmark\ arising from formations as a condition
\end{english}

Tattha katamaṁ viññāṇa-paccayā nāmarūpaṁ

\begin{english}
  Therein what is 'name-and-form' \breathmark\ arising from consciousness as a condition?
\end{english}

Atthi nāmaṁ atthi rūpaṁ

\begin{english}
  There is name \breathmark\ there is form
\end{english}

Tattha katamaṁ nāmaṁ

\begin{english}
  Therein what is name?
\end{english}

Vedanā saññā cetanā phasso manasikāro\pagenote{%
  Vibh 130 defines nāma as “vedanākkhandho saññākkhandho saṅkhārakkhandho”. For consistency with the discourses, I substituted this for the standard definition as found in e.g. SN 12.2.}

\begin{english}
  Feeling perception volition contact and attention
\end{english}

Idaṁ vuccati nāmaṁ

\begin{english}
  This is called 'name'
\end{english}

Tattha katamaṁ rūpaṁ

\begin{english}
  Therein what is form?
\end{english}

Cattāro mahābhūtā catunnañca mahābhūtānaṁ upādāya rūpaṁ

\begin{english}
  The four great elements and form dependent on the four great elements
\end{english}

Idaṁ vuccati rūpaṁ

\begin{english}
  This is called 'form'
\end{english}

Iti idañca nāmaṁ idañca rūpaṁ

\begin{english}
  Thus is this name and this form
\end{english}

Idaṁ vuccati viññāṇa-paccayā nāmarūpaṁ

\begin{english}
  This is called 'name-and-form' \breathmark\ arising from consciousness as a condition
\end{english}

Tattha katamaṁ nāmarūpa-paccayā saḷāyatanaṁ

\begin{english}
  Therein what is 'the sixfold-sense-base' \breathmark\ arising from name-and-form as a condition?
\end{english}

Cakkhāyatanaṁ sotāyatanaṁ ghānāyatanaṁ jivhāyatanaṁ kāyāyatanaṁ manāyatanaṁ

\begin{english}
  The eye-base ear-base nose-base tongue-base body-base mind-base
\end{english}

Idaṁ vuccati nāmarūpa-paccayā saḷāyatanaṁ

\begin{english}
  This is called 'the sixfold-sense-base' \breathmark\ arising from name-and-form as a condition
\end{english}

Tattha katamo saḷāyatana-paccayā phasso

\begin{english}
  Therein what is 'contact' \breathmark\ arising from the sixfold-sense-base as a condition?
\end{english}

Cakkhusamphasso sotasamphasso ghānasamphasso jivhāsamphasso kāyasamphasso manosamphasso

\begin{english}
  Eye-contact ear-contact nose-contact tongue-contact body-contact mind-contact
\end{english}

Ayaṁ vuccati saḷāyatana-paccayā phasso

\begin{english}
  This is called 'contact' \breathmark\ arising from the sixfold-sense-base as a condition
\end{english}

Tattha katamā phassa-paccayā vedanā

\begin{english}
  Therein what is 'feeling' \breathmark\ arising from contact as a condition?
\end{english}

Cakkhusamphassajā vedanā sotasamphassajā vedanā ghānasamphassajā vedanā jivhāsamphassajā vedanā kāyasamphassajā vedanā manosamphassajā vedanā

\begin{english}
  Feeling born of eye-contact \breathmark\ feeling born of ear-contact  ̓
  feeling born of nose-contact \breathmark\ feeling born of tongue-contact \breathmark\ feeling born of body-contact \breathmark\ feeling born of mind-contact
\end{english}

Ayaṁ vuccati phassa-paccayā vedanā

\begin{english}
  This is called 'feeling' \breathmark\ arising from contact as a condition
\end{english}

Tattha katamā vedanā-paccayā taṇhā

\begin{english}
  Therein what is 'craving' \breathmark\ arising from feeling as a condition?
\end{english}

Rūpataṇhā saddataṇhā gandhataṇhā rasataṇhā phoṭṭhabbataṇhā dhammataṇhā

\begin{english}
  Craving for forms \breathmark\ craving for sounds \breathmark\ craving for odours \breathmark\ craving for flavours \breathmark\ craving for tangibles \breathmark\ craving for mind-objects
\end{english}

Ayaṁ vuccati vedanā-paccayā taṇhā

\begin{english}
  This is called 'craving' \breathmark\ arising from feeling as a condition
\end{english}

Tattha katamaṁ taṇhā-paccayā upādānaṁ

\begin{english}
  Therein what is 'clinging' \breathmark\ arising from craving as a condition?
\end{english}

Kāmupādānaṁ diṭṭhupādānaṁ sīlabbatupādānaṁ attavādupādānaṁ

\begin{english}
  Clinging to sensuality \breathmark\ clinging to views \breathmark\ clinging to rules and rituals \breathmark\ clinging to a sense of self\pagenote{%
    The term vāda in attavādupādāna does here not necessarily refer to “a doctrine” of self, but rather to a person’s sense of being someone; the sense of being or having a self. This is not exactly the same as personality view, which is destroyed already at the stage of stream-entry. The lingering sense of having a self may continue for a while, despite of having already intellectually and/or experientially understood that there is no self to be found in relation to the five aggregates; just as conceit (māna) is overcome only by the path to Arahantship, despite of having uprooted personality view already at the stage of stream-entry.}
\end{english}

Idaṁ vuccati taṇhā-paccayā upādānaṁ

\begin{english}
  This is called 'clinging' \breathmark\ arising from craving as a condition
\end{english}

Tattha katamo upādāna-paccayā bhavo

\begin{english}
  Therein what is 'becoming' \breathmark\ arising from clinging as a condition?
\end{english}

Kāmabhavo rūpabhavo arūpabhavo\pagenote{%
  Here the Vibh. differs substantially from the analysis found in the suttas. It introduces a distinction between action becoming (kammabhava) and rebirth becoming (upapattipbhava). Kammabhava is taken to refer to wholesome, unwholesome and imperturpable actions; upapattibhava is taken to refer to sense/form/formless-sphere becoming, percipient/non-percipient/neither-percipient-nor-non-percipient becoming, one/four/five-component becoming. But since AN 3.76 says that kamma (together with consciousness and craving) is a condition for bhava (if no kamma…then no becoming in the sense-sphere etc. would be discerned), it is not fit to say that kamma itself is a form of bhava (kammabhava). However, it would go too far for the purpose of this chanting book, to discuss further implications of the analysis found in Vibh. For the sake of simplicity and emphasis on the earliest strata of the Buddha’s teachings, we substituted the passage from Vibh. with the passage from SN 12.2, which defines becoming (bhava) simply as sense/form/formless-sphere becoming. Understood in this way, becoming (bhava) functions as an intermediary between clinging and birth, highlighting the gradual process of how rebirth in one of the three planes of existence takes place. This can even include an interim period between death and birth; thus it is called “becoming”, rather than instant birth straight after death.}

\begin{english}
  Sense-sphere becoming form-sphere becoming formless-sphere becoming
\end{english}

Ayaṁ vuccati upādāna-paccayā bhavo

\begin{english}
  This is called 'becoming' \breathmark\ arising from clinging as a condition
\end{english}

Tattha katamā bhava-paccayā jāti

\begin{english}
  Therein what is 'birth' \breathmark\ arising from becoming as a condition?
\end{english}

Yā tesaṁ tesaṁ sattānaṁ tamhi tamhi sattanikāye jāti \breathmark\ sañjāti okkanti abhinibbatti khandhānaṁ pātubhāvo āyatanānaṁ paṭilābho\pagenote{%
  While SN 12.2 does not contain a full elaboration on all types of dukkha, a similar analysis is found in DN 22.}

\begin{english}
  The birth of various beings among the various classes of beings \breathmark\ their being born \breathmark\ descent \breathmark\ production \breathmark\ appearance of the aggregates \breathmark\ obtaining of the sense-bases
\end{english}

Ayaṁ vuccati bhava-paccayā jāti

\begin{english}
  This is called 'birth' \breathmark\ arising from becoming as a condition
\end{english}

Tattha katamaṁ jāti-paccayā jarāmaraṇaṁ

\begin{english}
  This is called 'ageing-and-death' \breathmark\ arising from birth as a condition
\end{english}

Tattha katamo soko

\begin{english}
  Therein what is sorrow?
\end{english}

Ñātibyasanena vā phuṭṭhassa bhogabyasanena vā phuṭṭhassa rogabyasanena vā phuṭṭhassa sīlabyasanena vā phuṭṭhassa diṭṭhibyasanena vā phuṭṭhassa \breathmark\ aññataraññatarena byasanena samannāgatassa aññataraññatarena dukkhadhammena phuṭṭhassa \breathmark\ soko socanā socitattaṁ \breathmark\ antosoko antoparisoko cetaso parijjhāyanā domanassaṁ sokasallaṁ

\begin{english}
  Affected by the loss of relatives \breathmark\ or loss of wealth \breathmark\ or misfortune of sickness \breathmark\ or loss of virtue \breathmark\ or loss of right view\pagenote{%
    While SN 12.2 does not contain an elaboration on the different types of misfortune (vyasana), it is found in DN 33, AN 4.192, and AN 5.130. Diṭṭhi here does not mean just any view, but “right view”. The loss of other views would not be particularly unfortunate from a Buddhist perspective, especially the loss of wrong view could be regarded as a great blessing. The right view that is lost in this example is not the right view of a noble disciple, but the right view of a worldling (puthujjana), whose right view is not unshakeable and who may or may not change his view later on throughout the course of the present life or subsequent births.}
  \breathmark\ by whatever misfortune one encounters \breathmark\ by whatever painful thing one is affected \breathmark\ the sorrow \breathmark\ sorrowing \breathmark\ sorrowfulness \breathmark\ inner sorrow \breathmark\ extensive inner sorrow \breathmark\ the mind’s thorough burning \breathmark\ displeasure \breathmark\ the dart of sorrow
\end{english}

Ayaṁ vuccati soko

\begin{english}
  This is called 'sorrow'
\end{english}

Tattha katamo paridevo

\begin{english}
  Therein what is lamentation?
\end{english}

Ñātibyasanena vā phuṭṭhassa bhogabyasanena vā phuṭṭhassa rogabyasanena vā phuṭṭhassa sīlabyasanena vā phuṭṭhassa diṭṭhibyasanena vā phuṭṭhassa \breathmark\ aññataraññatarena byasanena samannāgatassa aññataraññatarena dukkhadhammena phuṭṭhassa \breathmark\ ādevo paridevo ādevanā paridevanā ādevitattaṁ paridevitattaṁ \breathmark\ vācā palāpo vippalāpo lālappo lālappanā lālappitattaṁ

\begin{english}
  Affected by the loss of relatives \breathmark\ or loss of wealth \breathmark\ or misfortune of sickness \breathmark\ or loss of virtue \breathmark\ or loss of right view \breathmark\ by whatever misfortune one encounters \breathmark\ by whatever painful thing one is affected \breathmark\ the wail and lament \breathmark\ wailing and lamenting \breathmark\ bewailing and lamentation \breathmark\ sorrowful talk \breathmark\ senseless \breathmark\ confused \breathmark\ sorrowful murmur \breathmark\ sorrowful murmuring \breathmark\ sorrowful murmuration
\end{english}

Ayaṁ vuccati paridevo

\begin{english}
  This is called 'lamentation'
\end{english}

Tattha katamaṁ dukkhaṁ?

\begin{english}
  Therein what is pain?
\end{english}

Yaṁ kāyikaṁ asātaṁ kāyikaṁ dukkhaṁ \breathmark\ kāyasamphassajaṁ asātaṁ dukkhaṁ vedayitaṁ \breathmark\ kāyasamphassajā asātā dukkhā vedanā

\begin{english}
  The bodily discomfort \breathmark\ bodily pain \breathmark\ what is felt as uncomfortable \breathmark\ painful \breathmark\ that is born of body-contact \breathmark\ the uncomfortable painful feeling that is born of body-contact
\end{english}

Idaṁ vuccati dukkhaṁ

\begin{english}
  This is called 'pain'
\end{english}

Tattha katamaṁ domanassaṁ

\begin{english}
  Therein what is displeasure?
\end{english}

Yaṁ cetasikaṁ asātaṁ cetasikaṁ dukkhaṁ \breathmark\ cetosamphassajaṁ asātaṁ dukkhaṁ vedayitaṁ \breathmark\ cetosamphassajā asātā dukkhā vedanā

\begin{english}
  The mental discomfort \breathmark\ mental pain \breathmark\ what is felt as uncomfortable \breathmark\ painful \breathmark\ that is born of mind-contact \breathmark\ the uncomfortable painful feeling that is born of mind-contact
\end{english}

Idaṁ vuccati domanassaṁ

\begin{english}
  This is called 'displeasure'
\end{english}

Tattha katamo upāyāso

\begin{english}
  Therein what is despair?
\end{english}

Ñātibyasanena vā phuṭṭhassa bhogabyasanena vā phuṭṭhassa rogabyasanena vā phuṭṭhassa sīlabyasanena vā phuṭṭhassa diṭṭhibyasanena vā phuṭṭhassa \breathmark\ aññataraññatarena byasanena samannāgatassa aññataraññatarena dukkhadhammena phuṭṭhassa \breathmark\ āyāso upāyāso āyāsitattaṁ upāyāsitattaṁ

\begin{english}
  Affected by the loss of relatives \breathmark\ or loss of wealth \breathmark\ or misfortune of sickness \breathmark\ or loss of virtue \breathmark\ or loss of right view \breathmark\ by whatever misfortune one encounters \breathmark\ by whatever painful thing one is affected \breathmark\ the trouble and despair \breathmark\ tribulation and desperation
\end{english}

Ayaṁ vuccati upāyāso

\begin{english}
  This is called 'despair'
\end{english}

Evametassa kevalassa dukkhakkhandhassa samudayo hotī’ti:

\begin{english}
  “Such is the origin of this whole mass of suffering” means this:
\end{english}

Evametassa kevalassa dukkhakkhandhassa saṅgati hoti \breathmark\ samāgamo hoti samodhānaṁ hoti pātubhāvo hoti

\begin{english}
  Such is the combination \breathmark\ composition \breathmark\ collocation \breathmark\ manifestation \breathmark\ of this whole mass of suffering
\end{english}

Tena vuccati evametassa kevalassa dukkhakkhandhassa samudayo hotī’ti

\begin{english}
  Therefore it is called\\
  “Such is the origin of this whole mass of suffering”
\end{english}

\suttaRef{Vibh 130 / SN 12.2}

\bottomNav{dhamma-in-brief}

\sectionPaliTitle{Saṅkhitta-dhamma}
\section{The Dhamma in Brief}
\label{dhamma-in-brief}

\begin{center}
  [Handa mayaṁ saṅkhitta-sutta-pāṭhaṁ bhaṇāmase]
\end{center}

Mahāpajāpatī Gotamī yena bhagavā tenupasaṅkami \breathmark\ upasaṅkamitvā bhagavantaṁ abhivādetvā ekamantaṁ aṭṭhāsi. Ekamantaṁ ṭhitā kho sā mahāpajāpatī gotamī bhagavantaṁ etadavoca:

\begin{english}
  Mahāpajāpatī Gotamī approached the Blessed One \breathmark\ paid homage to him \breathmark\ then standing to one side she said:
\end{english}

Sādhu me bhante bhagavā saṅkhittena dhammaṁ desetu

\begin{english}
  Bhante it would be good if the Blessed One\\
  Would teach me the Dhamma in brief
\end{english}

Yamahaṁ bhagavato dhammaṁ sutvā

\begin{english}
  Having heard the Dhamma from the Blessed One
\end{english}

Ekā vūpakaṭṭhā appamattā ātāpinī pahitattā vihareyyan’ti

\begin{english}
  I might dwell alone \breathmark\ withdrawn \breathmark\ heedful \breathmark\ ardent and resolute
\end{english}

Ye kho tvaṁ gotamī dhamme jāneyyāsi:

\begin{english}
  Gotamī those things of which you might know:
\end{english}

Ime dhammā virāgāya saṁvattanti no sarāgāya

\begin{english}
  ‘They lead to dispassion \breathmark\ not to passion
\end{english}

Visaṁyogāya saṁvattanti no saṁyogāya

\begin{english}
  To detachment \breathmark\ not to bondage
\end{english}

Apacayāya saṁvattanti no ācayāya

\begin{english}
  To dismantling \breathmark\ not to building up
\end{english}

Appicchatāya saṁvattanti no mahicchatāya

\begin{english}
  To fewness of desires \breathmark\ not to strong desires
\end{english}

Santuṭṭhiyā saṁvattanti no asantuṭṭhiyā

\begin{english}
  To contentment \breathmark\ not to discontent
\end{english}

Pavivekāya saṁvattanti no saṅgaṇikāya

\begin{english}
  To solitude \breathmark\ not to company
\end{english}

Vīriyārambhāya saṁvattanti no kosajjāya

\begin{english}
  To the arousing of energy \breathmark\ not to laziness
\end{english}

Subharatāya saṁvattanti no dubbharatāyā’ti

\begin{english}
  To being easy to support \breathmark\ not to being difficult to support’
\end{english}

Ekaṁsena gotami dhāreyyāsi:

\begin{english}
  Gotamī\pagenote{%
    Anglicising of Pāli words is quite arbitrary. For masc. and neuter we use the stem, but for fem. we use the nominative, wherefore Gotami (voc.) becomes Gotamī. }
  you should definitely recognize:
\end{english}

Eso dhammo eso vinayo etaṁ satthusāsanan’ti

\begin{english}
  ‘This is the Dhamma\\
  This is the Vinaya\\
  This is the Teacher’s teaching’
\end{english}

\suttaRef{[AN 8.53]}

\bottomNav{uddissanadhitthana}

\sectionPaliTitle{Cattāro mahāpadesā}
\section{The Four Great References}
\label{four-great-references}

\begin{center}
  [Handa mayaṁ mahāpadesa-sutta-pāṭhaṁ bhaṇāmase]
\end{center}

Katame bhikkhave cattāro mahāpadesā

\begin{english}
  What bhikkhus are the four great references?
\end{english}

Idha bhikkhave bhikkhu evaṁ vadeyya:

\begin{english}
  Here bhikkhus a bhikkhu might say:
\end{english}

Sammukhā metaṁ āvuso bhagavato sutaṁ

\begin{english}
  In the presence of the Blessed One I heard this
\end{english}

Sammukhā paṭiggahitaṁ

\begin{english}
  In his presence I learned this
\end{english}

Asukasmiṁ nāma āvāse

\begin{english}
  Or in such and such a residence
\end{english}

Saṅgho viharati sathero sapāmokkho

\begin{english}
  A Saṅgha is dwelling with elders and prominent monks
\end{english}

Tassa me saṅghassa sammukhā sutaṁ

\begin{english}
  In the presence of that Saṅgha I heard this
\end{english}

Sammukhā paṭiggahitaṁ

\begin{english}
  In its presence I learned this
\end{english}

Asukasmiṁ nāma āvāse

\begin{english}
  Or in such and such a residence
\end{english}

Sambahulā therā bhikkhū viharanti

\begin{english}
  Many elder bhikkhus are dwelling
\end{english}

Bahussutā āgatāgamā

\begin{english}
  Who are learned \breathmark\ heirs to the heritage
\end{english}

Dhammadharā vinayadharā mātikādharā

\begin{english}
  Experts on the Dhamma \breathmark\ experts on the Vinaya \breathmark\ experts on the outlines
\end{english}

Tesaṁ me therānaṁ sammukhā sutaṁ

\begin{english}
  In the presence of those elders I heard this
\end{english}

Sammukhā paṭiggahitaṁ

\begin{english}
  In their presence I learned this
\end{english}

Asukasmiṁ nāma āvāse

\begin{english}
  Or in such and such a residence
\end{english}

Eko thero bhikkhu viharati

\begin{english}
  One elder bhikkhu is dwelling
\end{english}

Bahussuto āgatāgamo

\begin{english}
  Who is learned \breathmark\ an heir to the heritage
\end{english}

Dhammadharo vinayadharo mātikādharo

\begin{english}
  An expert on the Dhamma \breathmark\ an expert on the Vinaya \breathmark\ an expert on the outlines
\end{english}

Tassa me therassa sammukhā sutaṁ

\begin{english}
  In the presence of that elder I heard this
\end{english}

Sammukhā paṭiggahitaṁ

\begin{english}
  In his presence I learned this
\end{english}

Ayaṁ dhammo ayaṁ vinayo idaṁ satthusāsanan’ti

\begin{english}
  “This is the Dhamma \breathmark\ this is the Vinaya
  This is the Teacher’s teaching!”
\end{english}

Tassa bhikkhave bhikkhuno bhāsitaṁ

\begin{english}
  That bhikkhu’s statement
\end{english}

Neva abhinanditabbaṁ nappaṭikkositabbaṁ

\begin{english}
  Should neither be approved nor rejected
\end{english}

Anabhinanditvā appaṭikkositvā

\begin{english}
  Without approving or rejecting it
\end{english}

Padabyañjanāni sādhukaṁ uggahetvā

\begin{english}
  Having thoroughly learned those words and phrases
\end{english}

Sutte otāretabbāni

\begin{english}
  They ought to be found in the suttas
\end{english}

Vinaye sandassetabbāni

\begin{english}
  And seen in the Vinaya
\end{english}

Na ceva sutte otaranti na vinaye sandissanti

\begin{english}
  If they are neither found in the suttas \breathmark\ nor seen in the Vinaya
\end{english}

Niṭṭhamettha gantabbaṁ:

\begin{english}
  You should draw the conclusion:
\end{english}

Addhā idaṁ na ceva tassa bhagavato vacanaṁ arahato sammāsambuddhassa

\begin{english}
  Surely this is not the word of the Blessed One\\
  The Worthy One \breathmark\ the Perfectly Enlightened One
\end{english}

Tassa ca therassa duggahitan’ti

\begin{english}
  It has been badly learned by that elder
\end{english}

Iti hetaṁ bhikkhave chaḍḍeyyātha

\begin{english}
  Thus you should discard it
\end{english}

Sutte ceva otaranti vinaye ca sandissanti

\begin{english}
  But if they are found in the suttas \breathmark\ and seen in the Vinaya
\end{english}

Niṭṭhamettha gantabbaṁ

\begin{english}
  You should draw the conclusion:
\end{english}

Addhā idaṁ tassa bhagavato vacanaṁ arahato sammāsambuddhassa

\begin{english}
  Surely this is the word of the Blessed One\\
  The Worthy One \breathmark\ the Perfectly Enlightened One
\end{english}

Imassa ca bhikkhuno suggahitaṁ

\begin{english}
  It has been well-learned by that bhikkhu
\end{english}

Tassa ca saṅghassa suggahitaṁ

\begin{english}
  It has been well-learned by that Saṅgha
\end{english}

Tesañca therānaṁ suggahitaṁ

\begin{english}
  It has been well-learned by those elders
\end{english}

Tassa ca therassa suggahitan’ti

\begin{english}
  It has been well-learned by that elder
\end{english}

Ime kho bhikkhave cattāro mahāpadesā’ti

\begin{english}
  Bhikkhus these are the four great references
\end{english}

\suttaRef{[AN 4.180]}

\bottomNav{patimokkha-exhortation}

\sectionPaliTitle{Cha sāraṇīya-dhammā}
\section{Principles of Cordiality}
\label{principles-of-cordiality}

\begin{center}
  [Handa mayaṁ sāraṇīyā-dhammā-pāṭhaṁ bhaṇāmase]
\end{center}

Chayime bhikkhave dhammā sāraṇīyā

\begin{english}
  Bhikkhus there are these six principles of cordiality
\end{english}

Piyakaraṇā garukaraṇā

\begin{english}
  That create endearment and respect
\end{english}

Saṅgahāya

\begin{english}
  And conduce to cohesion
\end{english}

Avivādāya

\begin{english}
  To non-dispute
\end{english}

Sāmaggiyā ekībhāvāya saṁvattanti

\begin{english}
  To concord and unity
\end{english}

Katame cha

\begin{english}
  What are the six?
\end{english}

Idha bhikkhave bhikkhuno

\begin{english}
  Here bhikkhus a bhikkhu
\end{english}

Mettaṁ kāyakammaṁ vacīkammaṁ manokammaṁ paccupaṭṭhitaṁ hoti

\begin{english}
  Maintains bodily \breathmark\ verbal \breathmark\ and mental acts of loving-kindness
\end{english}

Sabrahmacārīsu āvi ceva raho ca

\begin{english}
  Both in public and in private \breathmark\ towards his spiritual companions
\end{english}

Bhikkhu ye te lābhā

\begin{english}
  Whatever a bhikkhu gains
\end{english}

Dhammikā dhammaladdhā

\begin{english}
  That accords with the Dhamma \breathmark\ and has been righteously obtained
\end{english}

Antamaso patta-pariyāpanna-mattampi

\begin{english}
  Even including the mere contents of his bowl
\end{english}

Tathārūpehi lābhehi appaṭivibhatta-bhogī hoti

\begin{english}
  Such gains he does not use without sharing
\end{english}

Sīlavantehi sabrahmacārīhi sādhāraṇabhogī

\begin{english}
  But uses them in common \breathmark\ with his virtuous spiritual companions
\end{english}

Bhikkhu yāni tāni sīlāni

\begin{english}
  A bhikkhu dwells possessing the virtues
\end{english}

Akhaṇḍāni acchiddāni asabalāni akammāsāni bhujissāni

\begin{english}
  That are unbroken \breathmark\ untorn \breathmark\ unblotched \breathmark\ unmottled \breathmark\ liberating
\end{english}

Viññuppasatthāni aparāmaṭṭhāni samādhi-saṁvattanikāni

\begin{english}
  Commended by the wise \breathmark\ not misapprehended \breathmark\ and conducive to concentration
\end{english}

Tathārūpesu sīlesu sīlasāmaññagato viharati

\begin{english}
  Endowed with such virtues he dwells
\end{english}

Sabrahmacārīsu āvi ceva raho ca

\begin{english}
  Both in public and in private \breathmark\ towards his spiritual companions
\end{english}

Bhikkhu yāyaṁ diṭṭhi

\begin{english}
  A bhikkhu dwells possessing a view
\end{english}

Ariyā niyyānikā

\begin{english}
  That is noble and emancipating
\end{english}

Niyyāti takkarassa sammā dukkhakkhayāya

\begin{english}
  Acting it out \breathmark\ leads to the complete destruction of suffering
\end{english}

Tathārūpāya diṭṭhiyā diṭṭhisāmaññagato viharati

\begin{english}
  Endowed with such a view he dwells
\end{english}

Sabrahmacārīsu āvi ceva raho ca

\begin{english}
  Both in public and in private \breathmark\ towards his spiritual companions
\end{english}

Ime kho bhikkhave cha sāraṇīyā dhammā

\begin{english}
  Bhikkhus these are the six principles of cordiality
\end{english}

Piyakaraṇā garukaraṇā

\begin{english}
  That create endearment and respect
\end{english}

Saṅgahāya

\begin{english}
  And conduce to cohesion
\end{english}

Avivādāya

\begin{english}
  To non-dispute
\end{english}

Sāmaggiyā ekībhāvāya saṁvattanti

\begin{english}
  To concord and unity
\end{english}

\suttaRef{[MN 48]}

Ime ce tumhe cha sāraṇīye dhamme samādāya vatteyyātha

\begin{english}
  If you undertake and maintain \breathmark\ these six principles of cordiality
\end{english}

Passatha no tumhe taṁ vacana-pathaṁ

\begin{english}
  Do you see any course of speech
\end{english}

Aṇuṁ vā thūlaṁ vā yaṁ tumhe nādhivāseyyāthā’ti

\begin{english}
  Trivial or gross \breathmark\ that you could not endure?
\end{english}

No hetaṁ bhante

\begin{english}
  No venerable sir
\end{english}

Tasmātiha ime cha sāraṇīye dhamme samādāya vattatha

\begin{english}
  Therefore undertake and maintain \breathmark\ these six principles of cordiality
\end{english}

Taṁ vo bhavissati dīgharattaṁ hitāya sukhāyā’ti

\begin{english}
  That will lead to your welfare and happiness for a long time
\end{english}

\suttaRef{[MN 104]}

\bottomNav{highest-honour-aspirations}

\sectionPaliTitle{Aparihāniya-dhammā}
\section{Principles of Non-Decline}
\label{principles-of-non-decline}

\begin{center}
  [Handa mayaṁ aparihāniya-dhamma-pāṭhaṁ bhaṇāmase]
\end{center}

Katame bhikkhave satta aparihāniyā dhammā

\begin{english}
  What bhikkus are the seven principles of non-decline?
\end{english}

Yāvakīvañca bhikkhave bhikkhū

\begin{english}
  As long as the bhikkhus
\end{english}

Abhiṇhaṁ sannipātā bhavissanti sannipātabahulā

\begin{english}
  Assemble often and hold frequent assemblies
\end{english}

Vuddhiyeva bhikkhave bhikkhūnaṁ pāṭikaṅkhā no parihāni

\begin{english}
  Only growth is to be expected for the bhikkhus \breathmark\ not decline
\end{english}

Yāvakīvañca bhikkhave bhikkhū

\begin{english}
  As long as the bhikkhus
\end{english}

Samaggā sannipatissanti

\begin{english}
  Assemble in harmony
\end{english}

Samaggā vuṭṭhahissanti

\begin{english}
  Adjorn in harmony
\end{english}

Samaggā saṅghakaraṇīyāni karissanti

\begin{english}
  And conduct the affairs of the Saṅgha in harmony
\end{english}

Vuddhiyeva bhikkhave bhikkhūnaṁ pāṭikaṅkhā no parihāni

\begin{english}
  Only growth is to be expected for the bhikkhus \breathmark\ not decline
\end{english}

Yāvakīvañca bhikkhave bhikkhū

\begin{english}
  As long as the bhikkhus
\end{english}

Apaññattaṁ na paññāpessanti

\begin{english}
  Do not decree anything that has not been decreed
\end{english}

Paññattaṁ na samucchindissanti

\begin{english}
  Or abolish anything that has already been decreed
\end{english}

Yathāpaññattesu sikkhāpadesu samādāya vattissanti

\begin{english}
  But undertake and follow the training rules \breathmark\ as they have been decreed
\end{english}

Vuddhiyeva bhikkhave bhikkhūnaṁ pāṭikaṅkhā no parihāni

\begin{english}
  Only growth is to be expected for the bhikkhus \breathmark\ not decline
\end{english}

Yāvakīvañca bhikkhave bhikkhū

\begin{english}
  As long as the bhikkhus
\end{english}

Ye te bhikkhū therā rattaññū

\begin{english}
  Venerate those bhikkhus who are elders \breathmark\ of long standing
\end{english}

Cirapabbajitā

\begin{english}
  Long gone forth
\end{english}

Saṅghapitaro saṅghapariṇāyakā

\begin{english}
  Fathers and guides of the Saṅgha
\end{english}

Te sakkarissanti garuṁ karissanti mānessanti pūjessanti

\begin{english}
  Honour \breathmark\ respect \breathmark\ esteem them
\end{english}

Tesañca sotabbaṁ maññissanti

\begin{english}
  And think they should be heeded
\end{english}

Vuddhiyeva bhikkhave bhikkhūnaṁ pāṭikaṅkhā no parihāni

\begin{english}
  Only growth is to be expected for the bhikkhus \breathmark\ not decline
\end{english}

Yāvakīvañca bhikkhave bhikkhū

\begin{english}
  As long as the bhikkhus
\end{english}

Uppannāya taṇhāya ponobhavikāya na vasaṁ gacchissanti

\begin{english}
  Do not come under the control of arisen craving \breathmark\ that leads to renewed existence
\end{english}

Vuddhiyeva bhikkhave bhikkhūnaṁ pāṭikaṅkhā no parihāni

\begin{english}
  Only growth is to be expected for the bhikkhus \breathmark\ not decline
\end{english}

Yāvakīvañca bhikkhave bhikkhū

\begin{english}
  As long as the bhikkhus
\end{english}

Āraññakesu senāsanesu sāpekkhā bhavissanti

\begin{english}
  Are intent on forest lodgings
\end{english}

Vuddhiyeva bhikkhave bhikkhūnaṁ pāṭikaṅkhā no parihāni

\begin{english}
  Only growth is to be expected for the bhikkhus \breathmark\ not decline
\end{english}

Yāvakīvañca bhikkhave bhikkhū

\begin{english}
  As long as the bhikkhus
\end{english}

Paccattaññeva satiṁ upaṭṭhāpessanti:

\begin{english}
  Establish mindfulness within themselves \breathmark\ thinking thus:
\end{english}

‘Kinti anāgatā ca pesalā sabrahmacārī āgaccheyyuṁ

\begin{english}
  ‘How can well-behaved fellow monks come \breathmark\ who have not yet come
\end{english}

Āgatā ca pesalā sabrahmacārī phāsuṁ vihareyyun’ti

\begin{english}
  And how can well-behaved fellow monks who are here \breathmark\ dwell at ease?’
\end{english}

Vuddhiyeva bhikkhave bhikkhūnaṁ pāṭikaṅkhā no parihāni

\begin{english}
  Only growth is to be expected for the bhikkhus \breathmark\ not decline
\end{english}

Yāvakīvañca bhikkhave ime satta aparihāniyā dhammā bhikkhūsu ṭhassanti

\begin{english}
  Bhikkhus as long as these seven principles of non-decline \breathmark\ continue among the bhikkhus
\end{english}

Imesu ca sattasu aparihāniyesu dhammesu bhikkhū sandississanti

\begin{english}
  And the bhikkhus are seen established in them
\end{english}

Vuddhiyeva bhikkhave bhikkhūnaṁ pāṭikaṅkhā no parihāni

\begin{english}
  Only growth is to be expected for the bhikkhus \breathmark\ not decline
\end{english}

\suttaRef{[AN 7.23]}

Yāvakīvañca bhikkhave bhikkhū

\begin{english}
  As long as the bhikkhus
\end{english}

Aniccasaññaṁ bhāvessanti anattasaññaṁ bhāvessanti

\begin{english}
  Develop the perception of impermanence \breathmark\ the perception of not-self
\end{english}

Asubhasaññaṁ bhāvessanti ādīnavasaññaṁ bhāvessanti

\begin{english}
  The perception of unattractiveness \breathmark\ the perception of danger
\end{english}

Pahānasaññaṁ bhāvessanti virāgasaññaṁ bhāvessanti nirodhasaññaṁ bhāvessanti

\begin{english}
  The perception of abandoning \breathmark\ the perception of dispassion  ̓
  the perception of cessation\pagenote{%
    In Girimānandasutta (AN 10.60) the Buddha instruced Ven. Ānanda: “If, Ānanda, you visit the (severely sick) bhikkhu Girimānanda and speak to him about ten perceptions, it is possible that on hearing about them his affliction will immediately subside.” Then the same seven perceptions as above are mentioned, with the addition of the following three: 8. the perception of non-delight in the entire world; 9. the perception of impermanence in all conditioned phenomena, and 10. mindfulness of breathing. Girimānandasutta concludes: “Then, when the Venerable Ānanda had learned these ten perceptions from the Blessed One, he went to the Venerable Girimānanda and spoke to him about them. When the Venerable Girimānanda heard about these ten perceptions, his affliction immediately subsided.”}
\end{english}

Vuddhiyeva bhikkhave bhikkhūnaṁ pāṭikaṅkhā no parihāni

\begin{english}
  Only growth is to be expected for the bhikkhus \breathmark\ not decline
\end{english}

Yāvakīvañca bhikkhave bhikkhū

\begin{english}
  As long as the bhikkhus
\end{english}

Hirimanto bhavissanti ottappino bhavissanti bahussutā bhavissanti

\begin{english}
  Develop moral shame \breathmark\ moral dread \breathmark\ learnedness
\end{english}

Āraddhavīriyā bhavissanti satimanto bhavissanti paññavanto bhavissanti

\begin{english}
  Become energetic \breathmark\ mindful and wise
\end{english}

Na oramattakena visesādhigamena antarāvosānaṁ āpajjissanti

\begin{english}
  Do not stop midway on account of some minor achievement of distinction
\end{english}

Vuddhiyeva bhikkhave bhikkhūnaṁ pāṭikaṅkhā no parihāni

\begin{english}
  Only growth is to be expected for the bhikkhus \breathmark\ not decline
\end{english}

\suttaRef{[AN 7.23-27]}

Ime bhikkhave dhammā sekhassa bhikkhuno aparihānāya saṁvattanti

\begin{english}
  Bhikkhus these qualities lead to the non-decline of a bhikkhu who is a trainee
\end{english}

Na kammārāmatā na bhassārāmatā na niddārāmatā na saṅgaṇikārāmatā

\begin{english}
  Not taking delight in work \breathmark\ in talk \breathmark\ in sleep \breathmark\ in company
\end{english}

Indriyesu guttadvāratā bhojane mattaññutā

\begin{english}
  Guarding the doors of the sense faculties \breathmark\ moderation in eating
\end{english}

Asaṁsaggārāmatā nippapañcārāmatā

\begin{english}
  Not taking delight in bonding \breathmark\ not taking delight in proliferation
\end{english}

Sovacassatā kalyāṇamittatā

\begin{english}
  Being easy to correct and good friendship
\end{english}

Ime kho bhikkhave dhammā sekhassa bhikkhuno aparihānāya saṁvattantī"ti

\begin{english}
  Bhikkhus these qualities lead to the non-decline of a bhikkhu who is a trainee.
\end{english}

\suttaRef{[AN 6.22 \& 8.79]}

\bottomNav{protection}

\sectionPaliTitle{Dhamma-pahaṁsāna}
\section{Striving According to the Dhamma}
\label{striving-according-to-dhamma}

\begin{center}
  [Handa mayaṁ dhamma-pahaṁsāna-pāṭhaṁ bhaṇāmase]
\end{center}

Evaṁ svākkhāto bhikkhave mayā dhammo

\begin{english}
  Bhikkhus the Dhamma has thus been well-expounded by me
\end{english}

Uttāno

\begin{english}
  Elucidated
\end{english}

Vivaṭo

\begin{english}
  Disclosed
\end{english}

Pakāsito

\begin{english}
  Revealed
\end{english}

Chinna-pilotiko

\begin{english}
  And stripped of patchwork
\end{english}

Alam-eva saddhā-pabbajitena kula-puttena vīriyaṁ ārabhituṁ

\begin{english}
  This is enough for a clansman\\
  Who has gone forth out of faith\\
  To arouse his energy thus
\end{english}

Kāmaṁ taco ca nahāru ca aṭṭhi ca avasissatu

\begin{english}
  “Willingly let only my skin  sinews  and bones remain
\end{english}

Sarīre upasussatu maṁsa-lohitaṁ

\begin{english}
  And let the flesh and blood in this body wither away
\end{english}

Yaṁ taṁ purisa-thāmena purisa-vīriyena purisa-parakkamena pattabbaṁ\\
Na taṁ apāpuṇitvā\\
Vīriyassa saṇṭhānaṁ bhavissatī’ti

\begin{english}
  As long as whatever is to be attained\\
  By manly strength\\
  By manly energy\\
  By manly effort\pagenote{%
    Orig: “By human strength/energy/effort”; while purisa can also mean person, or human, in this context it probably refers to the feature of strength, which is typically associated with masculinity. A similar analogy occurs with the expression of a strong man (purisa) extending or contracting his arm (AN 7.61).}\\
  Has not been attained\\
  Let not my efforts stand still”
\end{english}

Dukkhaṁ bhikkhave kusīto viharati

\begin{english}
  Bhikkhus the lazy person dwells in suffering
\end{english}

Vokiṇṇo pāpakehi akusalehi dhammehi

\begin{english}
  Soiled by evil unwholesome states
\end{english}

Mahantañ-ca sadatthaṁ parihāpeti

\begin{english}
  And great is the personal good that he neglects
\end{english}

Āraddha-vīriyo ca kho bhikkhave sukhaṁ viharati

\begin{english}
  The energetic person though dwells happily
\end{english}

Pavivitto pāpakehi akusalehi dhammehi

\begin{english}
  Well withdrawn from evil\pagenote{%
    Orig: omitted “evil” (pāpa) in its translation.}
  unwholesome states
\end{english}

Mahantañ-ca sadatthaṁ paripūreti

\begin{english}
  And great is the personal good that he achieves
\end{english}

Na bhikkhave hīnena aggassa patti hoti

\begin{english}
  Bhikkhus it is not by lower means that the supreme is attained
\end{english}

Aggena ca kho bhikkhave aggassa patti hoti

\begin{english}
  But bhikkhus it is by the supreme that the supreme is attained
\end{english}

Maṇḍapeyyam-idaṁ bhikkhave brahmacariyaṁ

\begin{english}
  Bhikkhus this holy life is like a milkshake\pagenote{%
    Orig: “like the cream of the milk”; maṇḍapeyya is a Pāli idiom, meaning ‘of the best quality (lit. to be drunk like cream)’}
  % % TODO: choose a final translation and alter the footnote accordingly
\end{english}

Satthā sammukhī-bhūto

\begin{english}
  The Teacher is present
\end{english}

Tasmā’tiha bhikkhave vīriyaṁ ārabhatha

\begin{english}
  Therefore bhikkhus \breathmark\ start to arouse your energy
\end{english}

Appattassa pattiyā

\begin{english}
  For the attainment of the as yet unattained
\end{english}

Anadhigatassa adhigamāya

\begin{english}
  For the achievement of the as yet unachieved
\end{english}

Asacchikatassa sacchikiriyāya

\begin{english}
  For the realization of the as yet unrealized
\end{english}

‘Evaṁ no ayaṁ amhākaṁ pabbajjā\\
Avaṅkatā avañjhā\pagenote{%
  Orig: “avaññā”}
bhavissati

\begin{english}
  Thinking thus:\\
  “Our going forth will not be crooked and barren\pagenote{%
    Orig: “thinking in such a way: our going forth will not be barren”}
\end{english}

Saphalā sa-udrayā

\begin{english}
  But will become fruitful and fertile
\end{english}

Yesaṁ mayaṁ paribhuñjāma\\
Cīvara-piṇḍapāta\\
Senāsana-gilānappaccaya bhesajja-parikkhāraṁ\\
Tesaṁ te kārā amhesu

\begin{english}
  And all our use of robes\\
  Almsfood\\
  Lodgings\\
  Supports for the sick and medicinal requisites\pagenote{%
    Orig: “And supports for the sick”}\\
  Given by others for our support
  % % TODO: replace all „ to “
\end{english}

Mahapphalā bhavissanti mahā-nisaṁsā'ti

\begin{english}
  Will reward them with great fruit and great benefit”
\end{english}

Evaṁ hi vo bhikkhave sikkhitabbaṁ

\begin{english}
  Bhikkhus you should train yourselves thus
\end{english}

Att’atthaṁ vā hi bhikkhave sampassamānena

\begin{english}
  Considering your own good
\end{english}

Alam-eva appamādena sampādetuṁ

\begin{english}
  It is enough to strive for the goal without negligence
\end{english}

Par’atthaṁ vā hi bhikkhave sampassamānena

\begin{english}
  Bhikkhus considering the good of others
\end{english}

Alam-eva appamādena sampādetuṁ

\begin{english}
  It is enough to strive for the goal without negligence
\end{english}

Ubhaya’tthaṁ vā hi bhikkhave sampassamānena

\begin{english}
  Bhikkhus considering the good of both
\end{english}

Alam-eva appamādena sampādetun'ti\pagenote{%
  Orig: “sampādetun”}

\begin{english}
  It is enough to strive for the goal without negligence
\end{english}

\suttaRef{[SN 12.22]}

\bottomNav{divine-abidings}

\sectionPaliTitle{Buddha-pacchima-ovāda}
\section{The Buddha's Final Instruction}
\label{buddhas-final-instruction}

\begin{center}
  [Handa mayaṁ buddha-pacchima-ovāda bhaṇāmase]
\end{center}

Siyā kho tumhākaṁ evamassa

\begin{english}
  Now if it occurs to you
\end{english}

Atītasatthukaṁ pāvacanaṁ natthi no satthā’ti

\begin{english}
  “The Teacher’s word has passed \breathmark\ we are without a teacher”
\end{english}

Na kho panetaṁ evaṁ daṭṭhabbaṁ

\begin{english}
  You should not view it this way
\end{english}

Yo vo mayā dhammo ca vinayo ca desito paññatto\\
So vo mamaccayena satthā

\begin{english}
  Whatever Dhamma and Vinaya\\
  I have pointed out and formulated for you\\
  That will be your teacher when I am gone
\end{english}

Handa dāni bhikkhave āmantayāmi vo

\begin{english}
  Now bhikkhus I declare to you
\end{english}

Vaya-dhammā saṅkhārā

\begin{english}
  Conditioned things are of ceasing nature\pagenote{%
    Orig: “Change is the nature of conditioned things”}
\end{english}

Appamādena sampādetha

\begin{english}
  Perfect yourselves not being negligent
\end{english}

Ayaṁ tathāgatassa pacchimā vācā

\begin{english}
  These are the Tathāgata’s final words
\end{english}

\suttaRef{[DN 16]}

\bottomNav{uddissanadhitthana}
