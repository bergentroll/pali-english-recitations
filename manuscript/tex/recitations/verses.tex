\chapterOpeningPage{verses.pdf}

\chapter{Verses}

\section{The Buddha's First Exclamation}
\paliTitle{Buddha-paṭhama-bhāsita}

\begin{twochants}
  Aneka-jāti-saṁsāraṁ & Sandhāvissaṁ anibbisaṁ\\
  Gaha-kāraṁ gavesanto & Dukkhā jāti punappunaṁ\\
\end{twochants}

\begin{english}
  For many lifetimes in the round of birth\\
  Wandering on endlessly\\
  For the builder of this house I searched\\
  How painful is repeated birth.
\end{english}

\begin{twochants}
  Gaha-kāraka diṭṭho'si & Puna gehaṁ na kāhasi\\
  Sabbā te phāsukā bhaggā & Gaha-kūṭaṁ visaṅkhataṁ\\
  Visaṅkhāra-gataṁ cittaṁ & Taṇhānaṁ khayam-ajjhagā\\
\end{twochants}

\begin{english}
  House-builder you've been seen\\
  Another home you will not build\\
  All your rafters have been snapped\\
  Dismantled is your ridge-pole\\
  The non-constructing mind\\
  Has come to craving's end
\end{english}

\suttaRef{Dhp 153-154}

\clearpage

\section{Respect for the Dhamma}
\paliTitle{Dhamma-gārava}

\begin{twochants}
  Ye ca atītā sambuddhā & Ye ca buddhā anāgatā \\
  Yo c'etarahi sambuddho & Bahunnaṁ soka-nāsano \\
\end{twochants}

\begin{english}
  All the Buddhas of the past\\
  All the Buddhas yet to come\\
  The Buddha of this current age\\
  Dispellers of much sorrow
\end{english}

\begin{twochants}
  Sabbe saddhamma-garuno & Vihariṁsu viharanti ca\\
  Atho pi viharissanti & Esā buddhāna dhammatā\\
\end{twochants}

\begin{english}
  Those having lived or living now\\
  Those living in the future\\
  All do revere the True Dhamma\\
  That is the nature of all Buddhas
\end{english}

\begin{twochants}
  Tasmā hi atta-kāmena & Mahattam-abhikaṅkhatā\\
  Saddhammo garu-kātabbo & Saraṁ buddhāna sāsanaṁ\\
\end{twochants}

\begin{english}
  Therefore desiring one's own welfare\\
  Pursuing greatest aspirations\\
  One should revere the True Dhamma\\
  Recollecting the Buddha's teaching
\end{english}

\suttaRef{SN 6.2}

\begin{paritta}
  Na hi dhammo adhammo ca\\
  Ubho sama-vipākino\\
  Adhammo nirayaṁ neti\\
  Dhammo pāpeti suggatiṁ
\end{paritta}

\begin{english}
  What is true Dhamma and what's not\\
  Will never have the same results\\
  While wrong Dhamma leads to hell realms\\
  True Dhamma takes one on a good course
\end{english}

\begin{paritta}
  Dhammo have rakkhati dhamma-cāriṁ\\
  Dhammo suciṇṇo sukham-āvahāti\\
  Esānisaṁso dhamme suciṇṇe\\
  Na duggatiṁ gacchati dhamma-cārī
\end{paritta}

\clearpage

\begin{english}
  The Dhamma guards those who live in line with it\\
  And leads to happiness when practised well\\
  This is the blessing of well-practised Dhamma\\
  The Dhamma-farer does not go on a bad course
\end{english}

\suttaRef{Thag 4.10}

\clearpage

\section{Going to True and False Refuges}
\paliTitle{Khemākhema-saraṇa-gamana}

\begin{twochants}
  Bahuṁ ve saraṇaṁ yanti & Pabbatāni vanāni ca\\
  Ārāma-rukkha-cetyāni & Manussā bhaya-tajjitā\\
\end{twochants}

\begin{english}
  To many refuges they go\\
  To mountain slopes and forest glades\\
  To parkland shrines and sacred sites\\
  People overcome by fear
\end{english}

\begin{twochants}
  N'etaṁ kho saraṇaṁ khemaṁ & N'etaṁ saraṇam-uttamaṁ\\
  N'etaṁ saraṇam-āgamma & Sabba-dukkhā pamuccati\\
\end{twochants}

\begin{english}
  Such a refuge is not secure\\
  Such a refuge is not supreme\\
  Such a refuge does not bring\\
  Complete release from all suffering
\end{english}

\begin{twochants}
  Yo ca buddhañ-ca dhammañ-ca & Saṅghañ-ca saraṇaṁ gato\\
  Cattāri ariya-saccāni & Sammappaññāya passati\\
\end{twochants}

\begin{english}
  Whoever goes to refuge\\
  In the Triple Gem\\
  Sees with right discernment\\
  The Four Noble Truths
\end{english}

\begin{twochants}
  Dukkhaṁ dukkha-samuppādaṁ & Dukkhassa ca atikkamaṁ\\
  Ariyañ-c'aṭṭh'aṅgikaṁ maggaṁ & Dukkhūpasama-gāminaṁ\\
\end{twochants}

\begin{english}
  Suffering and its origin\\
  And that which lies beyond\\
  The Noble Eightfold Path\\
  That leads the way to suffering's end.
\end{english}

\begin{twochants}
  Etaṁ kho saraṇaṁ khemaṁ & Etaṁ saraṇam-uttamaṁ\\
  Etaṁ saraṇam-āgamma & Sabba-dukkhā pamuccatī'ti.
\end{twochants}

\begin{english}
  Such a refuge is secure\\
  Such a refuge is supreme\\
  Such a refuge truly brings\\
  Complete release from all suffering.
\end{english}

\suttaRef{Dhp 188-192}

\section{The Pāṭimokkha Exhortation}
\paliTitle{Ovāda-pāṭimokkha-gāthā}

\begin{leader}
  [Handa mayaṁ ovāda-pāṭimokkha-gāthāyo bhaṇāmase]
\end{leader}

Sabba-pāpassa akaraṇaṁ

\begin{cprenglish}
  Not doing any evil
\end{cprenglish}

Kusalassūpasampadā

\begin{cprenglish}
  To be committed to the good
\end{cprenglish}

Sacitta-pariyodapanaṁ

\begin{cprenglish}
  To purify one's mind
\end{cprenglish}

Etaṁ buddhāna sāsanaṁ

\begin{cprenglish}
  These are the teachings of all Buddhas
\end{cprenglish}

Khantī paramaṁ tapo tītikkhā

\begin{cprenglish}
  Patient endurance is the highest practice burning out defilements
\end{cprenglish}

Nibbānaṁ paramaṁ vadanti buddhā

\begin{cprenglish}
  The Buddhas say Nibbāna is supreme
\end{cprenglish}

Na hi pabbajito parūpaghātī

\begin{cprenglish}
  Not a renunciant is one who injures others
\end{cprenglish}

Samaṇo hoti paraṁ viheṭhayanto

\begin{cprenglish}
  Whoever troubles others can't be called a monk
\end{cprenglish}

Anūpavādo anūpaghāto

\begin{cprenglish}
  Not to insult and not to injure
\end{cprenglish}

Pāṭimokkhe ca saṁvaro

\begin{cprenglish}
  To live restrained by training rules
\end{cprenglish}

Mattaññutā ca bhattasmiṁ

\begin{cprenglish}
  Knowing one's measure at the meal
\end{cprenglish}

Pantañca sayan'āsanaṁ

\begin{cprenglish}
  Retreating to a lonely place
\end{cprenglish}

Adhicitte ca āyogo

\begin{cprenglish}
  Devotion to the higher mind
\end{cprenglish}

Etaṁ buddhāna sāsanaṁ

\begin{cprenglish}
  These are the teachings of all Buddhas
\end{cprenglish}

\suttaRef{Dhp 183-185}

\section{The Three Characteristics}
\paliTitle{Ti-lakkhaṇā}

\begin{leader}
  [Handa mayaṁ ti-lakkhaṇ’ādi-gāthāyo bhaṇāmase]
\end{leader}

\begin{twochants}
  Sabbe saṅkhārā aniccā’ti & Yadā paññāya passati\\
  Atha nibbindati dukkhe & Esa maggo visuddhiyā\\
\end{twochants}

\begin{english}
  “All conditioned things are impermanent”\\
  When with wisdom this is seen\\
  One feels weary of all dukkha\\
  This is the path to purity
\end{english}

\begin{twochants}
  Sabbe saṅkhārā dukkhā’ti & Yadā paññāya passati\\
  Atha nibbindati dukkhe & Esa maggo visuddhiyā\\
\end{twochants}

\begin{english}
  “All conditioned things are dukkha”\\
  When with wisdom this is seen\\
  One feels weary of all dukkha\\
  This is the path to purity
\end{english}

\begin{twochants}
  Sabbe dhammā anattā’ti & Yadā paññāya passati\\
  Atha nibbindati dukkhe & Esa maggo visuddhiyā\\
\end{twochants}

\begin{english}
  “All things are not-self”\\
  When with wisdom this is seen\\
  One feels weary of all dukkha\\
  This is the path to purity
\end{english}

\suttaRef{Dhp 183-185}

\begin{twochants}
  Appakā te manussesu & Ye janā pāra-gāmino\\
  Athāyaṁ itarā pajā & Tīram-evānudhāvati\\
\end{twochants}

\begin{english}
  Few amongst humankind\\
  Are those who go beyond\\
  Yet there are the many folks\\
  Ever wandering on this shore
\end{english}

\begin{twochants}
  Ye ca kho sammad-akkhāte & Dhamme dhammānuvattino\\
  Te janā pāram-essanti & Maccu-dheyyaṁ suduttaraṁ\\
\end{twochants}

\begin{english}
  Wherever Dhamma is well-taught\\
  Those who train in line with it\\
  Are the ones who will cross over\\
  The realm of death so hard to flee
\end{english}

\begin{twochants}
  Kaṇhaṁ dhammaṁ vippahāya & Sukkaṁ bhāvetha paṇḍito\\
  Okā anokam-āgamma & Viveke yattha dūramaṁ\\
  Tatrābhiratim-iccheyya & Hitvā kāme akiñcano
\end{twochants}

\begin{english}
  Abandoning the darker states\\
  The wise pursue the bright\\
  Gone from home to homelessness\\
  Living withdrawn so hard to enjoy\\
  Such rare delight one should desire\\
  Sense pleasures cast away\\
  Not having anything
\end{english}

\suttaRef{Dhp 85-87.5}

\clearpage

\section{The Burdens}
\paliTitle{Bhārā}

\begin{leader}
  [Handa mayaṁ bhāra-sutta-gāthāyo bhaṇāmase]
\end{leader}

\begin{twochants}
  Bhārā have pañcakkhandhā & Bhāra-hāro ca puggalo \\
  Bhār'ādānaṁ dukkhaṁ loke & Bhāra-nikkhepanaṁ sukhaṁ \\
\end{twochants}

\begin{english}
  The five aggregates indeed are burdens\\
  The beast of burden is the person\\
  In this world to take up burdens is dukkha\\
  Putting them down brings happiness
\end{english}

\begin{twochants}
  Nikkhipitvā garuṁ bhāraṁ & Aññaṁ bhāraṁ anādiya\\
  Samūlaṁ taṇhaṁ abbuyha & Nicchāto parinibbuto\\
\end{twochants}

\begin{english}
  A heavy burden cast away\\
  Not taking on another load\\
  With craving pulled out from the root\\
  Desires stilled, one is released
\end{english}

\suttaRef{SN 22.22}

\clearpage

\section{From the Elder Raṭṭhapāla}
\paliTitle{Raṭṭhapāla-thera-gāthā}

\begin{leader}
  [Handa mayaṁ raṭṭhapālatthera-gāthāyo bhaṇāmase]
\end{leader}

\begin{twochants}
  Passa cittakataṁ bimbaṁ & Arukāyaṁ samussitaṁ\\
  Āturaṁ bahusaṅkappaṁ & Yassa natthi dhuvaṁ ṭhiti\\
\end{twochants}

\begin{english}
  See this fancy puppet\\
  A body built of sores\\
  Diseased  ̓  obsessed over\\
  Which does not last at all
\end{english}

\begin{twochants}
  Passa cittakataṁ rūpaṁ & Maṇinā kuṇḍalena ca\\
  Aṭṭhiṁ tacena onaddhaṁ & Saha vatthehi sobhati\\
\end{twochants}

\begin{english}
  See this fancy figure\\
  With its gems and earrings\\
  It is bones wrapped in skin\\
  Made pretty by its clothes
\end{english}

\begin{twochants}
  Alattakakatā pādā & Mukhaṁ cuṇṇakamakkhitaṁ\\
  Alaṁ bālassa mohāya & No ca pāragavesino\\
\end{twochants}

\begin{english}
  Feet adorned with henna dye\\
  And powder smeared upon its face\\
  May be enough to beguile a fool\\
  But not a seeker of the far shore
\end{english}

\begin{twochants}
  Aṭṭhapadakatā kesā & Nettā añjanamakkhitā\\
  Alaṁ bālassa mohāya & No ca pāragavesino\\
\end{twochants}

\begin{english}
  Hair in eight braids\\
  And eyeliner\\
  May be enough to beguile a fool\\
  But not a seeker of the far shore
\end{english}

\begin{twochants}
  Añjanīva navā cittā & Pūtikāyo alaṅkato\\
  Alaṁ bālassa mohāya & No ca pāragavesino\\
\end{twochants}

\begin{english}
  A rotting body all adorned\\
  Like a freshly painted unguent pot\\
  May be enough to beguile a fool\\
  But not a seeker of the far shore
\end{english}

\begin{twochants}
  Passāmi loke sadhane manusse\\
  Laddhāna vittaṁ na dadanti mohā\\
  Luddhā dhanaṁ sannicayaṁ karonti\\
  Bhiyyova kāme abhipatthayanti
\end{twochants}

\begin{english}
  I see rich people in the world\\
  Who from delusion give not the wealth they’ve earned\\
  Greedily they hoard their riches\\
  Yearning for ever more sense pleasures
\end{english}

\begin{twochants}
  Rājā ca aññe ca bahū manussā\\
  Avītataṇhā maraṇaṁ upenti\\
  Ūnāva hutvāna jahanti dehaṁ\\
  Kāmehi lokamhi na hatthi titti
\end{twochants}

\begin{english}
  Not just the king but others too\\
  Reach death not rid of craving\\
  They leave the body still wanting\\
  For in this world sense pleasures never satisfy
\end{english}

\begin{twochants}
  Na dīghamāyuṁ labhate dhanena\\
  Na cāpi vittena jaraṁ vihanti\\
  Appaṁ hidaṁ jīvitamāhu dhīrā\\
  Asassataṁ vippariṇāma-dhammaṁ
\end{twochants}

\begin{english}
  Longevity is not gained by riches\\
  Nor does wealth banish ageing\\
  For the wise say this life is short\\
  Subject to change  ̓  and not eternal
\end{english}

\begin{twochants}
  Tasmā hi paññāva dhanena seyyā\\
  Yāya vosānamidhādhigacchati|\
  Abyositattā hi bhavābhavesu\\
  Pāpāni kammāni karoti mohā
\end{twochants}

\begin{english}
  Therefore wisdom is much better than wealth\\
  By which one reaches perfection in this life\\
  People through ignorance do evil deeds\\
  Failing to reach the goal  ̓  from life to life
\end{english}

\begin{twochants}
  Kāmā hi citrā madhurā manoramā\\
  Virūparūpena mathenti cittaṁ\\
  Ādīnavaṁ kāmaguṇesu disvā\\
  Tasmā ahaṁ pabbajitomhi rāja
\end{twochants}

\begin{english}
  Sense pleasures are diverse  ̓  sweet  ̓  delightful\\
  Appearing in disguise they disturb the mind\\
  Seeing danger in the cords of sense pleasure\\
  Therefore I went forth O King
\end{english}

\begin{twochants}
  Dumapphalānīva patanti māṇavā\\
  Daharā ca vuḍḍhā ca sarīrabhedā\\
  Etampi disvā pabbajitomhi rāja\\
  Apaṇṇakaṁ sāmaññameva seyyo
\end{twochants}

\begin{english}
  As fruits fall from a tree  ̓  so people fall\\
  Young and old  ̓  when the body breaks up\\
  Seeing this too I went forth O King\\
  Surely the ascetic life is better
\end{english}

\suttaRef{Thag 16.4 / MN 82}

\clearpage

\section{From the Elder Pārāpariya}
\paliTitle{Pārāpariya-thera-gāthā}

\begin{leader}
  [Handa mayaṁ pārāpariyatthera-gāthāyo bhaṇāmase]
\end{leader}

\begin{twochants}
  Aññathā lokanāthamhi & Tiṭṭhante purisuttame\\
  Iriyaṁ āsi bhikkhūnaṁ & Aññathā dāni dissati\\
\end{twochants}

\begin{english}
  The behavior of the bhikkhus\\
  These days seems different\\
  From when the protector of the world\\
  The best of men was still here
\end{english}

\begin{twochants}
  Sītavātaparittāṇaṁ & Hirikopīnachādanaṁ\\
  Mattaṭṭhiyaṁ abhuñjiṁsu & Santuṭṭhā itarītare\\
\end{twochants}

\begin{english}
  Their robes were just for modesty\\
  And protection from cold and wind\\
  They ate in moderation\\
  Content with whatever they were offered
\end{english}

\begin{twochants}
  Paṇītaṁ yadi vā lūkhaṁ & Appaṁ vā yadi vā bahuṁ\\
  Yāpanatthaṁ abhuñjiṁsu & Agiddhā nādhimucchitā\\
\end{twochants}

\begin{english}
  Whether food was refined or rough\\
  A little or a lot\\
  They ate only for sustenance\\
  Without greed or gluttony
\end{english}

\begin{twochants}
  Jīvitānaṁ parikkhāre & Bhesajje atha paccaye\\
  Na bāḷhaṁ ussukā āsuṁ & Yathā te āsavakkhaye\\
\end{twochants}

\begin{english}
  They were not so eager\\
  For the requisites of life\\
  Such as tonics and other supplies\\
  As they were for destructing the defilements
\end{english}

\begin{twochants}
  Araññe rukkhamūlesu & Kandarāsu guhāsu ca\\
  Vivekamanubrūhantā & Vihaṁsu tapparāyaṇā\\
\end{twochants}

\begin{english}
  In the wilderness  ̓  at the foot of a tree\\
  In caves and caverns\\
  Fostering seclusion\\
  They lived with that as their final goal
\end{english}

\begin{twochants}
  Nīcā niviṭṭhā subharā & Mudū atthaddhamānasā\\
  Abyāsekā amukharā & Atthacintā vasānugā\\
\end{twochants}

\begin{english}
  They were used to simple things  ̓  easy to look after\\
  Gentle  ̓  not stubborn at heart\\
  Unsullied  ̓  not gossipy\\
  Their thoughts were intent on the goal
\end{english}

\begin{twochants}
  Tato pāsādikaṁ āsi & Gataṁ bhuttaṁ nisevitaṁ\\
  Siniddhā teladhārāva & Ahosi iriyāpatho\\
\end{twochants}

\begin{english}
  That’s why they inspired confidence\\
  In their movements eating and practice\\
  Their deportment was as smooth\\
  As a stream of oil
\end{english}

\begin{twochants}
  Yathā kaṇṭakaṭṭhānamhi & Careyya anupāhano\\
  Satiṁ upaṭṭhapetvāna & Evaṁ gāme munī care\\
\end{twochants}

\begin{english}
  When barefoot on a thorny path\\
  One would walk\\
  Quite mindfully\\
  That’s how a sage should walk in the village
\end{english}

\begin{twochants}
  Saritvā pubbake yogī & Tesaṁ vattamanussaraṁ\\
  Kiñcāpi pacchimo kālo & Phuseyya amataṁ padaṁ\\
\end{twochants}

\begin{english}
  Remembering the meditators of old\\
  And recollecting their conduct\\
  Even in the latter days\\
  The Deathless can still be reached
\end{english}

\suttaRef{Thag 16.10}

\clearpage

\section{On Protection}
\paliTitle{Tāyana-gāthā}

\begin{leader}
  [Handa mayaṁ Tāyana-gāthāyo bhaṇāmase]
\end{leader}

\begin{twochants}
  Chinda sotaṁ parakkamma & Kāme panūda brāhmaṇa\\
  Nappahāya muni kāme & Nekattam-upapajjati\\
\end{twochants}

\begin{english}
  Exert yourself and cut the stream\\
  Discard sense pleasures holy man\\
  Not letting sensual pleasures go\\
  A sage will not reach unityi
\end{english}

\begin{twochants}
  Kayirā ce kayirāthenaṁ & Daḷham-enaṁ parakkame\\
  Sithilo hi paribbājo & Bhiyyo ākirate rajaṁ\\
\end{twochants}

\begin{english}
  Vigorously with all one’s strength\\
  It should be done what should be done\\
  A lax monastic life stirs up\\
  The dust of defilements all the more
\end{english}

\begin{twochants}
  Akataṁ dukkaṭaṁ seyyo & Pacchā tappati dukkaṭaṁ\\
  Katañ-ca sukataṁ seyyo & Yaṁ katvā nānutappati\\
\end{twochants}

\begin{english}
  Better is not to do bad deeds\\
  That afterwards would bring remorse\\
  It’s rather good deeds one should do\\
  Which having done one won’t regret
\end{english}

\begin{twochants}
  Kuso yathā duggahito & Hattham-evānukantati\\
  Sāmaññaṁ dupparāmaṭṭhaṁ & Nirayāy’ūpakaḍḍhati\\
\end{twochants}

\begin{english}
  As kusa grass when wrongly grasped\\
  Will only cut into one’s hand\\
  So does the monk’s life wrongly led\\
  Indeed drag one to hellish states
\end{english}

\begin{twochants}
  Yaṁ-kiñci sithilaṁ kammaṁ & Saṅkiliṭṭhañ-ca yaṁ vataṁ\\
  Saṅkassaraṁ brahma-cariyaṁ & Na taṁ hoti mahapphalan’ti\\
\end{twochants}

\begin{english}
  Whatever deed that’s slackly done\\
  Whatever vow corruptly kept\\
  The holy life led in doubtful ways\\
  All these will never bear great fruits
\end{english}

\suttaRef{SN 2.8}

\clearpage

\section{Misecellaneous Verses}
\paliTitle{Pakiṇṇaka-gāthā}

\begin{leader}
  [Handa mayaṁ pakiṇṇaka-gāthāyo bhaṇāmase]
\end{leader}

\begin{twochants}
  Attadīpā bhikkhave viharatha attasaraṇā anaññasaraṇā\\
  Dhammadīpā dhammasaraṇā anaññasaraṇā\\
\end{twochants}

\begin{english}
  Bhikkhus dwell with yourselves as an island\\
  With yourselves as a refuge  ̓  with no other refuge\\
  With the Dhamma as an island  ̓  with the Dhamma as a refuge\\
  With no other refuge
\end{english}

\suttaRef{SN 22.43}

\begin{twochants}
  Appassutāyaṁ puriso & Balibaddova jīrati\\
  Maṁsāni tassa vaḍḍhanti & Paññā tassa na vaḍḍhati\\
\end{twochants}

\begin{english}
  The man of little learning\\
  Grows old like an ox\\
  He grows only in bulk\\
  But his wisdom does not grow
\end{english}

\suttaRef{Dhp 152}

\begin{twochants}
  Uyyuñjanti satīmanto & Na nikete ramanti te\\
  Haṁsāva pallalaṁ hitvā & Okamokaṁ jahanti te\\
\end{twochants}

\begin{english}
  The mindful ones exert themselves\\
  They are not attached to any home\\
  Like swans that abandon the lake\\
  They leave home after home behind
\end{english}

\suttaRef{Dhp 91}

\begin{twochants}
  Yaṁ pubbe taṁ visosehi & Pacchā te māhu kiñcanaṁ\\
  Majjhe ce no gahessasi & Upasanto carissasi\\
\end{twochants}

\begin{english}
  Dry up what pertains to the past\\
  Let there be nothing afterward\\
  If you do not grasp in the middle\\
  You will live at peace
\end{english}

\suttaRef{Snp 949}

\begin{twochants}
  Uṭṭhahatha nisīdatha & Ko attho supitena vo\\
  Āturānañhi kā niddā & Sallaviddhāna ruppataṁ\\
\end{twochants}

\begin{english}
  Arouse yourselves  ̓  sit up!\\
  What good to you is sleeping?\\
  For what sleep can there be for the afflicted\\
  For those injured  ̓  pierced by the dart?
\end{english}

\begin{twochants}
  Uṭṭhahatha nisīdatha & Daḷhaṁ sikkhatha santiyā\\
  Mā vo pamatte viññāya & Maccurājā amohayittha vasānuge\\
\end{twochants}

\begin{english}
  Arouse yourselves  ̓  sit up!\\
  Train vigorously for the state of peace\\
  Let not the King of Death catch you heedless\\
  And delude you when under his control
\end{english}

\begin{twochants}
  Yāya devā manussā ca & Sitā tiṭṭhanti atthikā\\
  Tarathetaṁ visattikaṁ & Khaṇo vo mā upaccagā\\
  Khaṇātītā hi socanti & Nirayamhi samappitā\\
\end{twochants}

\begin{english}
  Cross over this attachment\\
  By which devas and human beings\\
  Full of need are held fast\\
  Don’t let the opportunity pass you by\\
  For those who have missed the opportunity\\
  Sorrow when they arrive in hell
\end{english}

\begin{twochants}
  Pamādo rajo pamādo & Pamādānupatito rajo\\
  Appamādena vijjāya & Abbahe sallamattanoti\\
\end{twochants}

\begin{english}
  Heedlessness is dust always\\
  Dust follows upon heedlessness\\
  By heedfulness by clear knowledge\\
  Draw out the dart from yourself
\end{english}

\suttaRef{Snp 333-336}

\begin{twochants}
  Piyato jāyatī soko & Piyato jāyatī bhayaṁ\\
  Piyato vippamuttassa & Natthi soko kuto bhayaṁ\\
\end{twochants}

\begin{english}
  From endearment springs sorrow\\
  From endearment springs fear\\
  For one who is free from endearment\\
  There is no sorrow  ̓  whence then fear?
\end{english}

\suttaRef{Dhp 212}

\begin{twochants}
  Tiṭṭhateva nibbānaṁ
\end{twochants}

\begin{english}
  Nibbāna exists
\end{english}

\begin{twochants}
  Tiṭṭhati nibbānagāmī maggo
\end{twochants}

\begin{english}
  The path leading to nibbāna exists
\end{english}

\begin{twochants}
  Maggakkhāyī tathāgato
\end{twochants}

\begin{english}
  A Tathāgata is one who shows the path
\end{english}

\suttaRef{MN 107}

\begin{twochants}
  Tumhehi kiccam-ātappaṁ
\end{twochants}

\begin{english}
  You yourselves must strive
\end{english}

\suttaRef{Dhp 276}

\begin{twochants}
  Yaṁ bhikkhave satthārā karaṇīyaṁ sāvakānaṁ\\
  Hitesinā anukampakena anukampaṁ upādāya\\
\end{twochants}

\begin{english}
  Bhikkhus what should be done for his disciples\\
  Out of compassion by a teacher\\
  Who seek their welfare and has compassion for them
\end{english}

\begin{twochants}
  Kataṁ vo taṁ mayā
\end{twochants}

\begin{english}
  That I have done for you
\end{english}

\begin{twochants}
  Etāni bhikkhave rukkhamūlāni
\end{twochants}

\begin{english}
  Bhikkhus these are roots of trees
\end{english}

\begin{twochants}
  Etāni suññāgārāni
\end{twochants}

\begin{english}
  These are empty huts
\end{english}

\begin{twochants}
  Jhāyatha bhikkhave mā pamādattha
\end{twochants}

\begin{english}
  Meditate bhikkhus  ̓  do not be negligent
\end{english}

\begin{twochants}
  Mā pacchā vippaṭisārino ahuvattha
\end{twochants}

\begin{english}
  Lest you regret it later
\end{english}

\begin{twochants}
  Ayaṁ vo amhākaṁ anusāsanī’ti
\end{twochants}

\begin{english}
  This is my instruction to you
\end{english}

\suttaRef{MN 19}

\clearpage

\section{A Single Excellent Night}
\paliTitle{Bhadd-eka-ratta}

\begin{leader}
  [Handa mayaṁ bhadd’eka-ratta-gāthāyo bhaṇāmase]
\end{leader}

\begin{twochants}
  Atītaṁ nānvāgameyya & Nappaṭikaṅkhe anāgataṁ\\
  Yad atītam-pahīnan-taṁ & Appattañ-ca anāgataṁ\\
\end{twochants}

\begin{english}
  One should not revive the past\\
  Nor speculate on what’s to come\\
  The past is left behind\\
  The future is unrealized
\end{english}

\begin{twochants}
  Paccuppannañ-ca yo dhammaṁ & Tattha tattha vipassati\\
  Asaṁhiraṁ asaṅkuppaṁ & Taṁ viddhām-anubrūhaye\\
\end{twochants}

\begin{english}
  In every presently arisen state\\
  There just there one clearly sees\\
  Unmoved unagitated\\
  That is what the wise would keep fostering
\end{english}

\begin{twochants}
  Ajjeva kiccam-ātappaṁ & Ko jaññā maraṇaṁ suve\\
  Na hi no saṅgaran-tena & Mahā-senena maccunā\\
\end{twochants}

\begin{english}
  Ardently doing one’s task today\\
  Tomorrow who knows death may come\\
  Facing the mighty hordes of death\\
  Indeed one cannot strike a deal
\end{english}

\begin{twochants}
  Evaṁ vihārim-ātāpiṁ & Aho-rattam-atanditaṁ\\
  Taṁ ve bhadd’eka-ratto’ti & Santo ācikkhate muni\\
\end{twochants}

\begin{english}
  To dwell with energy aroused\\
  Day and night relentlessly\\
  That is “a single excellent night”\\
  So it was taught by the Peaceful Sage
\end{english}

\suttaRef{MN 131}

\clearpage

\section{The Highest Blessings}
\paliTitle{Maṅgala-sutta}

\begin{leader}
  [Now let us recite the verses on the Highest Blessings]
\end{leader}

Thus have I heard that the Blessed One\\
Was dwelling at Sāvatthī\\
Residing at the Jeta’s Grove\\
In Anāthapiṇḍika’s Park

Then in the dark of the night\\
A radiant deva illuminated all Jeta’s Grove\\
She bowed down low before the Blessed One\\
Then standing to one side she said:

“Devas are concerned for happiness\\
And ever long for peace\\
The same is true for humankind\\
What then are the highest blessings?”

Avoiding those of foolish ways\\
Associating with the wise\\
And honouring those worthy of honour\\
These are the highest blessings

Living in places of suitable kinds\\
With the fruits of past good deeds\\
And guided by the rightful way\\
These are the highest blessings

Accomplished in learning and craftsman’s skills\\
With discipline highly trained\\
And speech that is true and pleasant to hear\\
These are the highest blessings

Providing for mother and father’s support\\
And cherishing family\\
And ways of work that harm no being\\
These are the highest blessings

Generosity and a righteous life\\
Offering help to relatives and kin\\
And acting in ways that leave no blame\\
These are the highest blessings

Steadfast in restraint and shunning evil ways\\
Avoiding intoxicants that dull the mind\\
And heedfulness in all things that arise\\
These are the highest blessings

Respectfulness and being of humble ways\\
Contentment and gratitude\\
And hearing the Dhamma frequently taught\\
These are the highest blessings

Patience and willingness to accept one’s faults\\
Seeing venerated seekers of the truth\\
And sharing often the words of Dhamma\\
These are the highest blessings

Ardent and committedii to the holy life\\
Seeing for oneself the Noble Truths\\
And the realization of Nibbāna\\
These are the highest blessings

Although in contact with the world\\
Unshaken the mind remains\\
Beyond all sorrow spotless secure\\
These are the highest blessings

They who live by following this path\\
Know victory wherever they go\\
And every place for them is safe\\
These are the highest blessings

\suttaRef{Snp 2.4}

\clearpage

\section{The Buddha's Words on Loving-Kindness}
\paliTitle{Karaṇīya-metta-sutta}

\begin{leader}
  [Now let us recite the Buddha’s words on loving-kindness]
\end{leader}

This is what should be done\\
By one who is skilled in goodness\\
And who knows the path of peace\\
Let them be able and upright\\
Straightforward and gentle in speech\\
Humble and not conceited\\
Contented and easily satisfied\\
Unburdened with duties  ̓  and frugal in their ways\\
Peaceful and calm and wise and skillful\\
Not proud and demanding in nature\\
Let them not do the slightest thing\\
That the wise would later reprove\\
Wishing in gladness and in safety\\
May all beings be at ease\\
Whatever living beings there may be\\
Whether they are weak or strong  ̓  omitting none\\
The great or the mighty  ̓  medium short or small\\
The seen and the unseen\\
Those living near and far away\\
Those born and to be born\\
May all beings be at ease\\
Let none deceive another\\
Or despise any being in any state\\
Let none through anger or ill-will\\
Wish harm upon another\\
Justi as a mother protects with her life\\
Her child her only child\\
So with a boundless heart\\
Should one cherish all living beings\\
Radiating kindness  ̓  over the entire world\\
Spreading upwards to the skies\\
And downwards to the depths\\
Outwards and unbounded\\
Freed from hatred and ill-will\\
Whether standing or walking\\
Seated or lying down free from drowsiness\\
One should sustain this recollection\\
This is said to be the sublime abiding\\
By not holding wrong views\\
The pure-hearted one having clarity of vision\\
Being freed from all sense-desires\\
Is not born again into this world

\suttaRef{Snp 1.8}

\clearpage
