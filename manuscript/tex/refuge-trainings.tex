\section{Undertaking the Refuges and Trainings}

\vspace{1em}

\begin{center}
  \textbf{\textsc{Undertaking the Three Refuges}}\hyperlink{endnote138-appendix}{\hypertarget{endnote138-body}{\pagenote{%
        \hypertarget{endnote138-appendix}{\hyperlink{endnote138-body}{The Thai tradition recites ``Ahaṁ bhante tisaraṇena saha pañca/aṭṭha sīlāni yācāmi'' three times which translates to ``I, Venerable Sir, request the Three Refuges and the Five/Eight Precepts'' before ``Namo tassa...''.}}}}}
\end{center}

\begin{center}
  (Repeat each phrase after the leader)
\end{center}

Namo tassa bhagavato arahato sammāsambuddhassa \hfill{[3x]}

\begin{english}
  Homage to the Blessed, Worthy, and Perfectly Enlightened One. \hfill{[3x]}
\end{english}

Buddhaṁ saraṇaṁ gacchāmi\\
Dhammaṁ saraṇaṁ gacchāmi\\
Saṅghaṁ saraṇaṁ gacchāmi

\begin{english-verses}
  To the Buddha I go for refuge.\\
  To the Dhamma I go for refuge.\\
  To the Saṅgha I go for refuge.
\end{english-verses}

Dutiyampi buddhaṁ saraṇaṁ gacchāmi\\
Dutiyampi dhammaṁ saraṇaṁ gacchāmi\\
Dutiyampi saṅghaṁ saraṇaṁ gacchāmi

\begin{english-verses}
  For the second time, to the Buddha I go for refuge.\\
  For the second time, to the Dhamma I go for refuge.\\
  For the second time, to the Saṅgha I go for refuge.
\end{english-verses}

Tatiyampi buddhaṁ saraṇaṁ gacchāmi\\
Tatiyampi dhammaṁ saraṇaṁ gacchāmi\\
Tatiyampi saṅghaṁ saraṇaṁ gacchāmi

\begin{english-verses}
  For the third time, to the Buddha I go for refuge.\\
  For the third time, to the Dhamma I go for refuge.\\
  For the third time, to the Saṅgha I go for refuge.
\end{english-verses}

\anglebracketleft\ \hspace{-0.5mm}Tisaraṇa-gamanaṁ niṭṭhitaṁ \hspace{-0.5mm}\anglebracketright\

\begin{english}
  This completes the going to the Three Refuges.
\end{english}

Āma bhante

\begin{english}
  Yes, Venerable Sir.
\end{english}

(Bow three times)

\clearpage

\begin{center}
  \textbf{\textsc{Undertaking the Five Trainings}}
\end{center}

\begin{center}
  (Repeat each phrase after the leader)
\end{center}

Pāṇātipātā veramaṇi-sikkhāpadaṁ samādiyāmi

\begin{english-hang}
  I undertake the training\hyperlink{endnote139-appendix}{\hypertarget{endnote139-body}{\pagenote{%
        \hypertarget{endnote139-appendix}{\hyperlink{endnote139-body}{Orig: “precept”. The same applies to the next four lines.}}}}}
  to refrain from taking the life of any living being.
\end{english-hang}

Adinnādānā veramaṇi-sikkhāpadaṁ samādiyāmi

\begin{english}
  I undertake the training to refrain from taking that which is not given.
\end{english}

Kāmesu micchācārā veramaṇi-sikkhāpadaṁ samādiyāmi

\begin{english}
  I undertake the training to refrain from sexual misconduct.
\end{english}

Musāvādā veramaṇi-sikkhāpadaṁ samādiyāmi

\begin{english}
  I undertake the training to refrain from lying.
\end{english}

\begin{pali-hang}
  Surāmeraya-majja-pamādaṭṭhānā veramaṇi-sikkhāpadaṁ samādiyāmi
\end{pali-hang}

\begin{english-hang}
  I undertake the training to refrain from consuming intoxicating\\
  drink and drugs that lead to carelessness.\hyperlink{endnote140-appendix}{\hypertarget{endnote140-body}{\pagenote{%
        \hypertarget{endnote140-appendix}{\hyperlink{endnote140-body}{Orig: “drugs which lead to carelessness”}}}}}
\end{english-hang}

\anglebracketleft\ \hspace{-0.5mm}Appamādena sampādetha \hspace{-0.5mm}\anglebracketright\

\begin{english}
  Perfect yourselves not being negligent.
\end{english}

Sādhu sādhu sādhu

\begin{english}
  Well done, well cone, well done.
\end{english}

(Bow three times)

\clearpage

\begin{center}
  \textbf{\textsc{Undertaking the Eight Trainings}}
\end{center}

\begin{center}
  (Repeat each phrase after the leader)
\end{center}

Pāṇātipātā veramaṇi-sikkhāpadaṁ samādiyāmi

\begin{english-hang}
  I undertake the training to refrain from taking the life of any living being.
\end{english-hang}

Adinnādānā veramaṇi-sikkhāpadaṁ samādiyāmi

\begin{english-hang}
  I undertake the training to refrain from taking that which is not given.
\end{english-hang}

Abrahmacariyā veramaṇi-sikkhāpadaṁ samādiyāmi

\begin{english-hang}
  I undertake the training to refrain from any intentional sexual activity.
\end{english-hang}

Musāvādā veramaṇi-sikkhāpadaṁ samādiyāmi

\begin{english}
  I undertake the training to refrain from lying.
\end{english}

\begin{pali-hang}
  Surāmeraya-majja-pamādaṭṭhānā veramaṇi-sikkhāpadaṁ samādiyāmi
\end{pali-hang}

\begin{english-hang}
  I undertake the training to refrain from consuming intoxicating drink and drugs that\hyperlink{endnote141-appendix}{\hypertarget{endnote141-body}{\pagenote{%
        \hypertarget{endnote141-appendix}{\hyperlink{endnote141-body}{Orig: “which”}}}}}
  lead to carelessness.
\end{english-hang}

Vikālabhojanā veramaṇi-sikkhāpadaṁ samādiyāmi

\begin{english}
  I undertake the training to refrain from eating after noon.\hyperlink{endnote142-appendix}{\hypertarget{endnote142-body}{\pagenote{%
        \hypertarget{endnote142-appendix}{\hyperlink{endnote142-body}{Orig: “at inappropriate times”.}}}}}
\end{english}

\begin{pali-hang}
  Nacca-gīta-vādita-visūkadassanā mālā-gandha-vilepana-dhāraṇa-maṇḍana-vibhūsanaṭṭhānā veramaṇi-sikkhāpadaṁ samādiyāmi
\end{pali-hang}

\begin{english-hang}
  I undertake the training to refrain from dancing, singing, music and going to entertainments; from perfumes, beautification and adornments.\hyperlink{endnote143-appendix}{\hypertarget{endnote143-body}{\pagenote{%
        \hypertarget{endnote143-appendix}{\hyperlink{endnote143-body}{Orig: “entertainment, beautification, and adornment”}}}}}
\end{english-hang}

Uccāsayana-mahāsayanā veramaṇi-sikkhāpadaṁ samādiyāmi

\begin{english-hang}
  I undertake the training to refrain from lying on a high or luxurious sleeping place.
\end{english-hang}

Imāni aṭṭha sikkhāpadāni samādiyāmi \hfill{[3x]}

\begin{english}
  I undertake these Eight Trainings.
\end{english}

\begin{pali-hang}
  \anglebracketleft\ \hspace{-0.5mm}Imāni aṭṭha sikkhāpadāni sīlena sugatiṁ yanti sīlena bhogasampadā sīlena nibbutiṁ yanti tasmā sīlaṁ visodhaye \hspace{-0.5mm}\anglebracketright\
\end{pali-hang}

\begin{english-verses}
  These Eight Trainings\\
  have virtue as a vehicle for happiness,\\
  have virtue as a vehicle for good fortune,\\
  have virtue as a vehicle for liberation.\\
  Therefore let virtue be purified.\hyperlink{endnote144-appendix}{\hypertarget{endnote144-body}{\pagenote{%
        \hypertarget{endnote144-appendix}{\hyperlink{endnote144-body}{Orig: “These are the Eight Precepts; Virtue is the source of happiness, Virtue is the source of true wealth, Virtue is the source of peacefulness. Therefore let virtue be purified.”}}}}}
\end{english-verses}

Sādhu sādhu sādhu

\begin{english}
  Well done, well done, well done.
\end{english}

(Bow three times)

\clearpage
