\section*{The Three Refuges \& Five/Eight Trainings}
TODO center these subsections
\subsection{Requesting the Three Refuges \& Five/Eight Trainings\pagenote{%
This request is recited only in the Thai tradition. At SBS we start straight with “Namo tassa…”}}


\begin{center}
(After bowing three times, with hands joined in añjali, recite as follows)\\
\end{center}

Ahaṁ bhante tisaraṇena saha pañca/aṭṭha sīlāni yācāmi\\
Dutiyampi ahaṁ bhante tisaraṇena saha pañca/aṭṭha sīlāni yācāmi\\
Tatiyampi ahaṁ bhante tisaraṇena saha pañca/aṭṭha sīlāni yācāmi\\

\begin{english}
I, Venerable Sir, request the Three Refuges and the Five/Eight Trainings.\pagenote{%
Orig: “Precepts”. The same applies to the next two lines.}\\
For the second time, I, Venerable Sir, request the Three Refuges and the Five/Eight Trainings.\\
For the third time, I, Venerable Sir, request the Three Refuges and the Five/Eight Trainings.
\end{english}

\subsection*{Undertaking the Three Refuges}

\begin{center}
(Repeat line by line after the leader)\\
\end{center}

Namo tassa bhagavato arahato sammāsambuddhassa \hfill{[3x]}\\

\begin{english}
Homage to the Blessed, Worthy, and Perfectly Enlightened One \hfill{[3x]}\\
\end{english}

Buddhaṁ saraṇaṁ gacchāmi\\
Dhammaṁ saraṇaṁ gacchāmi\\
Saṅghaṁ saraṇaṁ gacchāmi\\

\begin{english}
To the Buddha I go for refuge\\
To the Dhamma I go for refuge\\
To the Saṅgha I go for refuge.\\
\end{english}

Dutiyampi buddhaṁ saraṇaṁ gacchāmi\\
Dutiyampi dhammaṁ saraṇaṁ gacchāmi\\
Dutiyampi saṅghaṁ saraṇaṁ gacchāmi\\

\begin{english}
For the second time, to the Buddha I go for refuge\\
For the second time, to the Dhamma I go for refuge\\
For the second time, to the Saṅgha I go for refuge.\\
\end{english}

Tatiyampi buddhaṁ saraṇaṁ gacchāmi\\
Tatiyampi dhammaṁ saraṇaṁ gacchāmi\\
Tatiyampi saṅghaṁ saraṇaṁ gacchāmi\\

\begin{english}
For the third time, to the Buddha I go for refuge.\\
For the third time, to the Dhamma I go for refuge.\\
For the third time, to the Saṅgha I go for refuge.\\
\end{english}

Leader: Tisaraṇa-gamanaṁ niṭṭhitaṁ\\

\begin{english}
This completes the going to the Three Refuges.\\
\end{english}

Response: Āma bhante\\

\begin{english}
Yes, Venerable Sir.\\
\end{english}

\begin{center}
(Bow three times)\\
\end{center}

\subsection*{Undertaking the Five Trainings}

\begin{center}
  (To undertake the trainings, repeat each training after the leader)\\
\end{center}

Pāṇātipātā veramaṇi-sikkhāpadaṁ samādiyāmi.\\

\begin{english}
  I undertake the training\pagenote{%
    Orig: “precept”. The same applies to the next four lines.}
  to refrain from taking the life of any living being.\\
\end{english}

Adinnādānā veramaṇi-sikkhāpadaṁ samādiyāmi.\\

\begin{english}
  I undertake the training to refrain from taking that which is not given.\\
\end{english}

Kāmesu micchācārā veramaṇi-sikkhāpadaṁ samādiyāmi.\\

\begin{english}
  I undertake the training to refrain from sexual misconduct.\\
\end{english}

Musāvādā veramaṇi-sikkhāpadaṁ samādiyāmi.\\

\begin{english}
  I undertake the training to refrain from lying.\\
\end{english}

Surāmeraya-majja-pamādaṭṭhānā veramaṇi-sikkhāpadaṁ samādiyāmi.\\

\begin{english}
  I undertake the training to refrain from consuming intoxicating\\
  drink and drugs that lead to carelessness.\pagenote{%
  Orig: “drugs which lead to carelessness”}\\
  \end{english}

Leader: Appamādena sampādetha\\

\begin{english}
  Perfect yourselves not being negligent.\\
\end{english}

Response: Sādhu, sādhu, sādhu\\

\begin{center}
  (Bow three times)\\
\end{center}

\subsection*{Undertaking the Eight Trainings}

\begin{center}
  (To undertake the trainings, repeat each training after the leader)\\
\end{center}

Pāṇātipātā veramaṇi-sikkhāpadaṁ samādiyāmi.\\

\begin{english}
  I undertake the training to refrain from taking the life of any living being.\\
\end{english}

Adinnādānā veramaṇi-sikkhāpadaṁ samādiyāmi.\\

\begin{english}
  I undertake the training to refrain from taking that which is not given.\\
\end{english}

Abrahmacariyā veramaṇi-sikkhāpadaṁ samādiyāmi.\\

\begin{english}
  I undertake the training to refrain from any intentional sexual activity.\\
\end{english}

Musāvādā veramaṇi-sikkhāpadaṁ samādiyāmi.\\

\begin{english}
  I undertake the training to refrain from lying.\\
\end{english}

Surāmeraya-majja-pamādaṭṭhānā veramaṇi-sikkhāpadaṁ samādiyāmi.\\

\begin{english}
  I undertake the training to refrain from consuming intoxicating drink and drugs that\pagenote{%
    Orig: “which”}
  lead to carelessness.\\
\end{english}

Vikālabhojanā veramaṇi-sikkhāpadaṁ samādiyāmi.\\

\begin{english}
  I undertake the training to refrain from eating after noon.\pagenote{%
  Orig: “at inappropriate times”.}
\\
  \end{english}

Nacca-gīta-vādita-visūkadassanā mālā-gandha-vilepana-dhāraṇa-maṇḍana-vibhūsanaṭṭhānā veramaṇi-sikkhāpadaṁ samādiyāmi.\\

\begin{english}
  I undertake the training to refrain from dancing, singing, music and going to entertainments; from perfumes, beautification and adornments.\pagenote{%
  Orig: “entertainment, beautification, and adornment”}\\
  \end{english}

Uccāsayana-mahāsayanā veramaṇi-sikkhāpadaṁ samādiyāmi.\\

\begin{english}
  I undertake the training to refrain from lying on a high or luxurious sleeping place.\\
\end{english}

Imāni aṭṭha sikkhāpadāni samādiyāmi \hfill{[3x]}\\

\begin{english}
  I undertake these Eight Trainings.\\
\end{english}

Leader: Imāni aṭṭha sikkhāpadāni sīlena sugatiṁ yanti sīlena bhogasampadā sīlena nibbutiṁ yanti tasmā sīlaṁ visodhaye\\

\begin{english}
  These Eight Trainings\\
  have virtue as a vehicle for happiness,\\
  have virtue as a vehicle for good fortune,\\
  have virtue as a vehicle for liberation.\\
  Therefore let virtue be purified.\pagenote{%
  Orig: “These are the Eight Precepts; Virtue is the source of happiness, Virtue is the source of true wealth, Virtue is the source of peacefulness. Therefore let virtue be purified.”}\\
  \end{english}

Response: Sādhu, sādhu, sādhu\\

\begin{center}
  (Bow three times)\\
\end{center}
